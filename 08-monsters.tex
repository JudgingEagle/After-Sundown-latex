%%%%%%%%%%%%%%%%%%%%%%%%%%%%%%%%%%%%%%%%%%%%%%%%%%
%%%%%%%%%%%%%%%%%%%%%%%%%%%%%%%%%%%%%%%%%%%%%%%%%%
\chapter{Monsters}
%%%%%%%%%%%%%%%%%%%%%%%%%%%%%%%%%%%%%%%%%%%%%%%%%%
%%%%%%%%%%%%%%%%%%%%%%%%%%%%%%%%%%%%%%%%%%%%%%%%%%
\tagline{"Bullets! My only weakness!"}

Remember the 900+ page books about all these different creatures? You can find mythical sources which will back up pretty much any supernatural creature being weak or immune to pretty much anything you can imagine. Vampires who have to count seeds or are repulsed by sticky rice for example. There comes a time when you just have to buckle down and agree upon some basics or it's never going to generate stories. So up front we're going to talk about the basic things you use against supernaturals \textit{in general}, and which ones are specifically effective against which types. Remember that each supernatural character fits into two categories: they are a supernatural type (vampire, lycanthrope, etc.), and they also have a power source (Astral, Infernal, or Orphic). Every \textit{playable} supernatural creature is available with every power source, and each type/power source combination represents a unique subtype. For example: an Astral Powered Vampire is a Nosferatu and an Infernal Powered Vampire is a Daeva. The \textit{nonplayable} supernatural creatures have subtypes as well, but they literally come from one of the other worlds and every subtype has the same power source. The subtypes are roughly broken down by power -- a Triffid is more powerful than a Mantrap and a Pod is more powerful still. Note that while a good case can be made that the playable types are roughly equal to one another in overall utility, the same is not true at all for the nonplayable creatures. Indeed, an Ifrit or a Pod is not allowed for player characters in no small part because it's ridiculously powerful, while a Shambler is not up for grabs because it is weak even on the scale of competent normal humans.

%%%%%%%%%%%%%%%%%%%%%%%%%%%%%%%%%%%%%%%%%%%%%%%%%%
\section{Slaying Monsters (in general)}
%%%%%%%%%%%%%%%%%%%%%%%%%%%%%%%%%%%%%%%%%%%%%%%%%%

\hspace{\parindent} You \textit{can} hurt just about anything by just running it over with a car. However, supernatural creatures have supernatural defenses that make them incredibly tough. There are a couple of universal weapons that cut through crap like regeneration and magic force fields. They are effective  based on the type of creature you are trying to kill. When used on the appropriate foe, these weapons inflict aggravated damage and negate soak bonuses from disciplines. Note that simply \textit{having} a high Strength with the aid of a power like Giant Size would not constitute a soak bonus, but that a specific soak bonus (including bonus armor) such those provided by \linkpower{Force Field} or the passive benefit of Fortitude is negated.

\begin{description}
\item[Wood] In many songs and stories, only special wood counts against the forces of darkness. Maybe it is oak, or banyan or sacred ash. But seriously it doesn't even matter. Wood represents life even when it is dead, dry, and laminated. It is effective against Animates, Vampires, Zombies, and Ghosts.

\item[Iron] Seriously, \textit{Iron}. Like the stuff that your steel knives are already made out of. It's a symbol of modernity and industrialization and stuff and it classically drives away the old cthonic stuff. It is effective against Leviathan, Transhumans, Goblins, and Evil Plants.

\item[Silver] Shiny and inconstant like the moon, silver is hard enough to kill a man and easy enough to cast that you can do it before the invention of bronze. Silver is clearly magical, and is lethal to creatures as ephemeral and primal as it is. It is effective against Witches, Lycanthropes, Giant Animals, and Demons.
\end{description}

%%%%%%%%%%%%%%%%%%%%%%%%%%%%%%%%%%%%%%%%%%%%%%%%%%
\section{Weakening Supernatural Creatures}
%%%%%%%%%%%%%%%%%%%%%%%%%%%%%%%%%%%%%%%%%%%%%%%%%%

\hspace{\parindent} There are a lot of things which, while they won't burn the flesh of supernaturals, \textit{will} deplete their powers. They can be used to imprison some of these bad boys, or take their powers away long enough to beat them in a fight. When exposed to their particular kryptonite, a supernatural creature's powers are weakened in several ways:

\begin{itemize*}
\item Their Potency is considered zero, and any of their attributes that are raised past their normal maximum are considered to be their normal maximum (usually 6 for former humans).
\item They cannot spend Power Points, and any powers they activated this scene with Power Points already have no further effect for as long as the character's powers are suppressed.
\item They cannot spend an action required to activate any power.
\end{itemize*}

\begin{description}
\item[Alcohol]
It has to be very strong beer to count, but in general, if you get some good old fashioned spirits onto (or into) supernatural creatures they have a hard time using their powers. Consumed alcohol wears off in about an hour, alcohol spilled on a creature cleans up with club soda.\\
\textbf{Effective against:} Lycanthropes, Animates, Giant Animals, and Evil Plants

\item[Water]
Getting wet is a real problem for some evil beings. Unlike with alcohol, water vulnerabilities don't trigger in any way off of being \textit{drunk}. The creature's powers are dampened only so long as their exterior is wet. Moderate dampness doesn't count either, we're talking half-liter or more Wicked Witch of the West \textit{wet} (and they don't even melt, they just can't spend power points and such).\\
\textbf{Effective against:} Witches, Transhumans, Demons, and Goblins

\item[Sunlight]
The harsh light of the day star robs evil creatures of their strength. Creatures that are weak to sunlight can go out during the day as long as they avoid the sun's rays, such as during a heavy fog or storm, or just anywhere at all within Mictlan.\\
\textbf{Effective against:} Vampires, Leviathan, Ghosts, and Zombies
\end{description}

%%%%%%%%%%%%%%%%%%%%%%%%%%%%%%%%%%%%%%%%%%%%%%%%%%
\section{Power Schedules}
%%%%%%%%%%%%%%%%%%%%%%%%%%%%%%%%%%%%%%%%%%%%%%%%%%
\tagline{"It's five o'clock, time for your ass whupping."}

Power Points are regained on a \textit{schedule}\index{Power Schedule} that varies depending on the type of supernatural creature. This means that different characters and enemies will need to do different things to restore their powers, which will occasionally come up as a substantial advantage or disadvantage for one character or another, depending upon the circumstances. Sometimes this can be forced against a character due to the needs of the plot, and other times the characters can figure out how to use the intricacies of their power regaining system.

Common Power Schedules are as follows:

%%%%%%%%%%%%%%%%%%%%%%%%%
\subsection{Feeding}
%%%%%%%%%%%%%%%%%%%%%%%%%

\hspace{\parindent} Characters who must feed upon mortals to regain power points have obvious advantages and disadvantages. Firstly, they can often schedule their power gains whenever. People are all over the place and you can take time out of your schedule to devour them whenever you aren't pressed for time. Of course, when you're in polite company or you \textit{are} pressed for time, that may not be possible. Also, leaving a trail of victims is a dangerous thing to do, even if you have the ability to wipe their memories -- it angers people. It angers people who are luminaries. All vampire types are on the Feeding schedule. A character gains one Power Point for every lethal wound box inflicted with the character's idiom. You'll note that a victim always has 10 boxes, and even a starting character's Power Reserve goes up to 13. A character Feeding from a human (or supernal creature with living flesh) can take one Power Point for every wound inflicted, but mostly characters who want to go from empty to full will want to non-fatally feed from more than one mortal. Actual consumption of power points takes place at the rate of 1 power point every 12 seconds (1 turn during high resolution action scenes), and requires consumption during the entire period. The wound need not be inflicted in bite size pieces -- it is entirely possible to cause someone a single Serious wound (6 boxes) and then lick the blood up for 72 seconds to gain 6 power points. Feeding from Vampires (as well as Revenants and Akuma) is \textit{relatively} non-harmful, as their living blood isn't really \textit{theirs} to begin with. In that case, power points can be drained out at a rate of 1 per 12 seconds without actually injuring the victim.

%%%%%%%%%%%%%%%%%%%%%%%%%
\subsection{Lunar}
%%%%%%%%%%%%%%%%%%%%%%%%%

\hspace{\parindent} Characters who regain power points when the moon rises are able to pull fancy time shenanigans where their powers are restored fully in the middle of major scenes. Unfortunately, they also have to wait about an entire day between times when they get their powers back. A character on the Lunar power schedule fills their Power Reserve every time the Moon Rises. And yes, that can be a long wait at the North or South Poles. Try not to get imprisoned in those places.

%%%%%%%%%%%%%%%%%%%%%%%%%
\subsection{Ritual}
%%%%%%%%%%%%%%%%%%%%%%%%%

\hspace{\parindent} Characters on the ritual schedule have a specific and time consuming action they have to perform in order to regain their power points. This takes substantially longer than feeding on a mortal (usually about 2 hours), but hopefully entails less personal risk than actually victimizing someone. The ritual required varies depending upon the type of character, but usually requires special equipment. For example: an Android needs to hook themselves up to special equipment in order to literally recharge their batteries while one of the Fallen has to bathe in the magical glow from their artifact. Special circumstances may be available to reduce the ritual time below the 2 hour standard. For example, an Android or Frankenstein could recharge \textit{faster} if they were hooked up to a giant hydroelectric dam than they could with their equipment hooked up to the normal power grid in their home.

%%%%%%%%%%%%%%%%%%%%%%%%%
\subsection{Continuous}
%%%%%%%%%%%%%%%%%%%%%%%%%

\hspace{\parindent} Some creatures get their power on a continuous basis. These characters are at "full strength" scene after scene. They are narratively tireless, but have reduced power points to compensate. Creatures on the Continuous schedule enter every scene with half their Power Reserve filled. No playable creatures have a Continuous Power Schedule, and it's important to know that Trolls and the like \textit{cannot} be worn down. Creatures on a Continuous schedule do not gain power points while actually dead (essentially they do not get any more scenes), meaning that it may require the use of some Vampire Blood or something to revive a set of Troll bones.

%%%%%%%%%%%%%%%%%%%%%%%%%%%%%%%%%%%%%%%%%%%%%%%%%%
\section{Tragic Flaws}
%%%%%%%%%%%%%%%%%%%%%%%%%%%%%%%%%%%%%%%%%%%%%%%%%%
\tagline{"It's not all turning into a giant homicidal monster, there are also parts of the curse that are genuinely bad."}

Being supernatural is not without its drawbacks. As you can plainly see from watching pretty much any horror movie \textit{ever}, the life of a monster is filled with unfortunate conditions. The most painful of these are the crushing mental instabilities that come with the territory. Every supernatural being goes at least somewhat crazy, and is a danger to themselves and others. In game, this is represented by the Master Passions. Each character gets one if they are a supernatural creature. The precise way that any character goes mad is a deeply personal one, and players may (and should) select one that feels right for them. The suggested Master Passion for each supernatural creature type is just that -- a suggestion. While it may be the most common Master Passion that dominates the lives of that kind of creature, there is nothing inherently weird about a Werewolf or Troglodyte that happens to spiral into madness in some other way. Nothing weirder than the fact that they \textit{are} a murderous inhuman monster with claws, anyway.

But there are other limitations that affect supernatural creatures in a more universal and less personal fashion. Every member of a single class of supernatural creature has a Distinctive Flaw, which is a Disadvantage common to their type of supernatural creature. The player does not normally get any choice in the matter, \textit{every} Werewolf has the Temperamental Disadvantage. The character does not gain any compensatory Advantages for this Distinctive Flaw. If someone becomes a supernatural creature and already has the Distinctive Flaw for the type they are becoming, then they must either gain a new Disadvantage (chosen to be appropriate to the character), or abandon one of their Advantages.

Master Passions and Disadvantages are in the \hyperref[chapter:Character Options and Motivations]{Character Options \& Motivations} chapter.

%%%%%%%%%%%%%%%%%%%%%%%%%%%%%%%%%%%%%%%%%%%%%%%%%%
\section[Lycanthropy]{Lycanthropy: Ruled by Rage} \index{Lycanthrope}
%%%%%%%%%%%%%%%%%%%%%%%%%%%%%%%%%%%%%%%%%%%%%%%%%%
\tagline{A man may live for forty years, and a wolf only seven. But at the end of their lives, which one knows more of the tundra?\\
The man and the wolf know the same.}

Lycanthropy is a disease that passes from lycanthropes to people that are nearly killed at their jaws. All that is necessary to become a lycanthrope is to be bitten, nearly die, and yet survive the ordeal. It's like rabies. Only your heart has to stop at least for a little while.

Once afflicted, a newly created Lycanthrope will find their wounds closing rapidly. The essentially deadly injuries suffered in the initial attack heal without leaving a scar or discoloration to mark their passing. The victim will feel feverish to the touch and chilled in the chest -- a condition which will follow them until the end of their days.

\textbf{Lycanthropic Culture:} There really isn't any ancient lycanthrope culture. For the vast majority of time a lycanthrope would come into being only by surviving an attack by another lycanthrope. As such, most lycanthropes came into the world with their creator either defeated or hostile. The vast majority of lycanthropes either learned the ins and outs of their condition on their own or had it explained to them by someone else in the know (usually another supernatural). So it is quite common for lycanthropes to be adapted into the cultures of \textit{other} supernatural creatures. A werewolf who was taken in by gypsy witches would generally have the same traditions and prejudices as those gypsy witches, not those of whatever werewolf tore into him with its fangs and left him for dead.

An exception to that generalization can be found in small family groups. Lycanthropes often go all crazy with rage and are a severe danger to their families and friends. A loved one pushed nearly to death by the rampages of a wererat is rather likely to spurn the creature which transformed it, running off and ultimately forming a new "culture" of one. However it is not unheard of for such a victim to stay on and create a pack of lycanthropes. These groups tend to avoid contact with humans and supernaturals alike and have strange views.

\paragraph{Therianthropes? Fuck that noise!} It is important to note that the word "lycanthrope" literally comes from Greek words for "wolf" and "form" but that it is an English word which means a human who transforms into a wolf \textit{or other beast}. Many people will try to get you to use the word "Therianthrope" or "Zoanthrope" because of a misguided attempt to use Greek root words correctly. Those words are however \textit{not English}, and using them is not "technically correct" -- it is retarded. The plural of Octopus is "Octopuses" and not "Oktopodes" like it would be if we were speaking Greek, because if you are reading this document the chances are excellent that \textit{you are not an ancient Greek}.

Lycanthropes take aggravated damage from silver weapons. Every type of Lycanthrope has a Lunar power source. Most Lycanthropes are dominated by Master Passion Rage, a fact that leads them to sometimes go berserk when the moon rises. Lycanthropes of all kinds are well advised to avoid fragile things and people.

%%%%%%%%%%%%%%%%%%%%%%%%%
\subsection{Werewolves} \index{Lycanthrope!Werewolf}
%%%%%%%%%%%%%%%%%%%%%%%%%
\tagline{The better to eat you with, my dear.}

Somewhere in the howling wilderness of Scandinavia a "wolf warrior" of the North became darkly fused with a wolf pelt he was wearing while fighting the Huns -- an event which places the creation of the Werewolf at approximately 600 CE. Passed from warrior to enemy warrior and conquered victims, the curse spread throughout the lands of Europe and beyond along the warpaths of the Huns and later the Vikings. In later nights, it spread itself throughout the world on the backs of European conquistadors and imperial marines. By the twelve hundreds it had spread as far as Central Asia, and the Wolf Mother Asena led the Syndicate of the Bumin Horde of Ergenekon to rule much of Eurasia until her presumed death and the dissolution of the Syndicate in 1844.

Werewolves are instilled with a love of combat and destruction even in their human forms. In their monstrous forms they take the shape of awkwardly toothy man-wolves. Glistening with drool and usually fast covered with speckles of blood, the claws and fangs of a monstrous formed werewolf are a terror. Every Werewolf can transform into a "regular" canine as well as a huge canid beast. Not all Werewolves transform into an actual wolf when they become a mundane animal. For whatever reason, some become dogs, coyotes, foxes, or even hyenas.

Though they gain no magical power or sustenance from it, Werewolves are quite drawn to eating human flesh, and will often succumb to this temptation while in a frenzy. The curse of Lycanthropy comes with unusual and often unwanted hair growth on various parts of the body for Werewolves. In the era of waxing and depilatories, a Werewolf can keep this under control, but changing into the War Form causes hair to come back no matter what has been done. On the plus side: baldness is totally curable through infection with Lycanthropy.

Werewolves have an Astral power source and a Lunar power schedule.

\paragraph{The First Werewolf?}
The secret histories are fairly clear about the source of the canine form of Lycanthropy. There was a Norse warrior at the beginning of the 7th century who got cursed and every single Werewolf in the whole world traces their condition back to him. However, while this story is fairly well documented and has a lot of evidence behind it, there are still those who insist that there were earlier Werewolves or that canine Lycanthropy has occurred more than once. It is certainly true that the exact mechanics of the wolf pelt curse are unknown to modern sorcery, so it is within the realm of possibility that it could happen again. Or that it already has.

%%%
\subsubsection{Werewolf Starting Powers}
%%%

\hspace{\parindent} - Core Discipline: Call of the Wild -\\
\linkpower{Beast Form} (Basic Call of the Wild)\\
\linkpower{Tongue of Beasts} (Basic Call of the Wild)

- Basic Powers -\\
\linkpower{Revive the Flesh} (Basic Fortitude)\\
\linkpower{Vigor} (Basic Clout)\\
\linkpower{Quickness} (Basic Celerity)\\
\linkpower{Repel} (Basic Magnetism)

- Advanced Powers -\\
\linkpower{War Form} (Celerity / Clout Devotion)\\
\linkpower{The Beckoning} (Advanced Call of the Wild)

\textbf{Distinctive Flaw:} Temperamental

\paragraph{Story Inspiration:} Larry Talbot, Ginger Snaps, Dog Soldiers

%%%%%%%%%%%%%%%%%%%%%%%%%
\subsection[Nezumi]{The Nezumi: Plague on the world of Men} \index{Lycanthrope!Nezumi}
%%%%%%%%%%%%%%%%%%%%%%%%%
\tagline{"Tear him up."}

The most frequently told story of the origins of the plague of the Nezumi is that originally someone turned their back on the teachings of the Buddha and was cursed with reincarnating in a lower form during their own life. If even partially true, this would mean that the plague started no earlier than about 500 BCE. Once afflicted, a Nezumi Lycanthrope begins to twitch their nose like Elizabeth Montgomery and click their tongues nervously.

A Nezumi's animal form is that of a large rat. Nezumi are immune to the effects of diseases, but carry virtually every disease they come into contact with. Unlike other lycanthropes, Nezumi have no monstrous war form, though nothing is stopping them from eventually learning one and some certainly do. While most Nezumi lack the brutal jaws or rending claws of other lycanthropes, their bites are still often quite deadly because of the pestilence they carry. Still, it is important to note that even their animal form is of little help in a fight, being as it is just a large rat.

Most Nezumi are mastered by the same all-consuming rage of the other Lycanthropes. However, they \textit{don't} normally possess any great strength, a fact which frequently causes them to become quite cowardly and fearful of authority. Many Nezumi become quite resigned from the world: hiding in sewers, jumping at sounds; knowing that the heat of battle runs in their veins and feeling dreadfully afraid that it will eventually drive them to a suicidal battle over nothing. Interacting with the world of men seems to be a never ending river of frustrations, and many choose to simply not do it. Those who master their rage usually do so only by trading it for Master Passion Fear.

A Nezumi has an Infernal power source and a Lunar power schedule.

\paragraph{The First Nezumi?} Opinion is divided as to the historicity of the Buddha offending story in the secret histories. The oldest \textit{living} Nezumi is Hamamoto Yoshi, who was born in the 8th century CE in Nagaokyo, Japan and now lives in cryptic hermitude in Kyoto's sewers. Most Nezumi do in fact die of old age, and when Yoshi has deigned to answer such questions at all, he has denied being the first.

%%%
\subsubsection{Nezumi Starting Powers}
%%%

\hspace{\parindent} - Core Discipline: Call of the Wild -\\
\linkpower{Beast Form} (Basic Call of the Wild)\\
\linkpower{Tongue of Beasts} (Basic Call of the Wild)

- Basic Powers -\\
\linkpower{Abyss of the Body} (Basic Descent of Entropy)\\
\linkpower{Hide From Notice} (Basic Veil)\\
\linkpower{Learn the Heart's Pain} (Basic Names of the Blasphemies)\\
\linkpower{Revive the Flesh} (Basic Fortitude)

- Advanced Powers -\\
\linkpower{Hide in Plain Sight} (Advanced Veil)\\
\linkpower{The Beckoning} (Advanced Call of the Wild)

\textbf{Distinctive Flaw:} Red Taped

\paragraph{Story Inspiration:} Willard, Torment, Fruits Basket

%%%%%%%%%%%%%%%%%%%%%%%%%
\subsection[Bagheera]{The Bagheera: The Lady or the Tiger} \index{Lycanthrope!Bagheera}
%%%%%%%%%%%%%%%%%%%%%%%%%
\tagline{"You can fool everybody, but landie. Dearie me, you can't fool a cat. They seem to know who's not right."}

The origins of the Bagheera are confused. The legends say that the Bagheera were formed when a Bengali priestess refused to abandon her gods and goddesses when the Mughals took control of the province and banned idolatry, calling upon Shiva for a mighty boon. And yet the legends \textit{also} say that the Bagheera were formed first by a jaguar warrior pledged to Huitzilopochtli fighting against Cortez, an event which would necessarily push the formation of the line over fifty years back. When consulted on the subject directly, the Stellar Oracles enigmatically state that both stories are true. Regardless, once infected a Bagheera is a creature of Death and finds that their mere presence alarms and angers animals of all types, especially cats.

A Bagheera's War Form is that of a great and ghostly cat. Perhaps a tiger, jaguar, or leopard. Most Bagheera do not have the ability to transform into a remotely reasonably sized cat that can pass as something "not terrifying." It has been suggested that the beast inside a Bagheera isn't really an animal at all, since they seem to show less than no affinity for the mundane beasts they appear as -- a marked departure from other Lycanthropes.

A Bagheera has an Orphic power source and a Lunar power schedule.

\paragraph{The First Bagheera?} The origins of the Bagheera are clouded in mystery, but not in particularly \textit{ancient} mystery. Bagheera are not present in the secret histories before the 16th century, and they came on the scene quite rapidly with sightings all over the world in 1578 CE. Most stories appear to agree that the first Bagheera was female, but beyond that little is known for sure.

%%%
\subsubsection{Bagheera Starting Powers}
%%%

\hspace{\parindent} - Core Discipline: Celerity -\\
\linkpower{Quickness} (Basic Celerity)\\
\linkpower{Nimble Feet} (Basic Celerity)

- Basic Powers -\\
\linkpower{Vigor} (Basic Clout)\\
\linkpower{Hide From Notice} (Basic Veil)\\
\linkpower{Touch of Darkness} (Basic Lure of Destruction)\\
\linkpower{Revive the Flesh} (Basic Fortitude)

- Advanced Powers -\\
\linkpower{War Form} (Celerity / Clout Devotion)\\
\linkpower{Alacrity} (Advanced Celerity)

\textbf{Distinctive Flaw:} Offensive to Animals

\paragraph{Story Inspiration:} Cat People, The Cat and the Canary, Geobreeders

%%%%%%%%%%%%%%%%%%%%%%%%%%%%%%%%%%%%%%%%%%%%%%%%%%
\section{Vampirism} \index{Vampire}
%%%%%%%%%%%%%%%%%%%%%%%%%%%%%%%%%%%%%%%%%%%%%%%%%%
\tagline{Time is an abyss. Profound as a thousand nights\ldots{} Centuries come and go\ldots{} To be unable to grow old is terrible. Death is not the worst\ldots{} There are things more horrible than death. Can you imagine\ldots{} Enduring centuries\ldots{} experiencing each day with the same futile things?}

Vampirism is passed from one person to the next voluntarily on the part of the vampire passing on the trait. The vampire drinks the blood of the victim until it is gone and puts some of their own blood into the dying victim's mouth. Vampires drink the blood of the living in order to maintain their existence as the living dead night after night for as long as the Earth continues to turn.

How you can tell someone is a vampire varies from story to story. And it is this difference between vampires which we are recognizing in this game as the primary marker of bloodlines. Various games and book series have at times attempted to differentiate vampires on other criteria as well (such as the amount of sparkling they do or how long they can be played as characters before they go crazy), but this has not been overly successful. So we're recognizing a few common, and three variable ways to detect vampires. All vampires are the living dead, so their body temperature is below that of a living human even if they have recently eaten. Also, their blood doesn't move around their body at the behest of a pumping heart, so they have no detectable heart beat and their wounds don't bleed. Beyond that, they can be detectable by a number of means which vary by vampiric "type". The Nosferatu are hideous, the Daeva are beautiful but have distinct animal traits, and the Strigoi can't leave knots alone.

All vampires drink blood to regain power points. And all vampires can at least potentially live forever. In After Sundown, it is this facility with \textit{not dying} that most effectively defines a vampire. So long as they keep draining life from others, their own unlife need never end. Any vampire can sprout teeth sufficient to pierce the toughest arteries on a victim. A Daeva can produce fangs like a vampire bat, a Strigoi can exude serpent-like fangs, and the teeth of a Nosferatu are as ghastly and varied as imagination can allow.

Vampires suffer aggravated damage from Wood, and most of them have Master Passion Hunger. Every subtype of Vampire has a Feeding Power Schedule. Vampires are created by another vampire draining the life out of a Luminary and then passing the corpse a power point with Gift of Health before the body has become cool. They can \textit{attempt} this process with Extras, but the result is always disappointment and the creation of a near mindless Vampire Spawn. A victim whose lifeblood is drained to the point of death and then allowed to cool \textit{without} being fed a power point is simply dead.

%%%%%%%%%%%%%%%%%%%%%%%%%
\subsection[Nosferatu]{The Nosferatu: Grotesques of the Imagination} \index{Vampire!Nosferatu}
%%%%%%%%%%%%%%%%%%%%%%%%%
\tagline{That fate which condemns me to wallow in blood has also denied me the joys of the flesh. This face - the infection which poisons our love. This face which earned a mother's fear and loathing, a mask: my first unfeeling scrap of clothing. Pity comes too late, turn around and face your fate, an eternity of \textbf{this} before your eyes!}

Nosferatu are social pariahs. Many of them look monstrous, animalistic, or deformed. While others look reasonably human, there is a certain otherworldly air about them which frightens small children. Any human who sees a Nosferatu's face will be instantly convinced that they are dealing with something monstrous. Young women will recoil, priests will present crosses, that kind of thing.

The Nosferatu are considered great information brokers by other supernaturals, and this is not an unreasonable assumption, as Nosferatu can pass unseen in all but the most observant company. Upon transforming into a Nosferatu, a character gains a specific monstrous caste to their appearance. Some are lucky enough to have a face that can pass for a human that simply happens to be forbiddingly cruel looking, while others look like nothing so much as rabid beasts. Whether subtle or complete, the Nosferatu's appearance is morphed by the time they first open their eyes as a vampire. Nosferatu who persist for many nights sometimes find themselves gradually transforming into forms yet more monstrous.

A Nosferatu's feeding maw is unique to themselves. One might protrude a mosquito's proboscis while another might extend a lamprey's wheel of teeth or a leech's beak. There is no guaranty that a newly created Nosferatu will have a method of feeding that is the same or in any way less disgusting than their sire's.

A Nosferatu has an Astral power source and a Feeding power schedule.

\paragraph{The First Nosferatu?} The secret histories abound with unidentified bogie men and monsters of the night. Nosferatu are only first called out as such in the 6th century with Krampus the Child Eater of what is now Northern Austria. With their natural abilities to Hide From Notice and their proclivities for avoiding social situations, it is generally unknown how long Nosferatu have skulked in the night.

%%%
\subsubsection{Nosferatu Starting Powers}
%%%

\hspace{\parindent} - Core Discipline: Fortitude -\\
\linkpower{Patience of the Mountains} (Basic Fortitude)\\
\linkpower{Revive the Flesh} (Basic Fortitude)

- Basic Powers -\\
\linkpower{Vigor} (Basic Clout)\\
\linkpower{Tongue of Beasts} (Basic Call of the Wild)\\
\linkpower{Gift of Health} (Basic Path of Blood)\\
\linkpower{Hide From Notice} (Basic Veil)

- Advanced Powers -\\
\linkpower{Restoration} (Advanced Fortitude)\\
\linkpower{Hide in Plain Sight} (Advanced Veil)

\textbf{Distinctive Flaw:} Eerie Presence

\paragraph{Story Inspiration:} Shadow of the Vampire, Nosferatu, Buffy the Vampire Slayer

%%%%%%%%%%%%%%%%%%%%%%%%%
\subsection[Strigoi]{The Strigoi: Serpents of Death} \index{Vampire!Strigoi}
%%%%%%%%%%%%%%%%%%%%%%%%%
\tagline{"A weapon you don't have in your hand will not kill a snake."}

Elegant and sophisticated or possibly simply old fashioned and resistant to change, the Strigoi are the aristocratic vampire of legend. The Strigoi have a long tradition of being in control in Eastern Europe. Traditionally, the Strigoi only transform people who are already rich, powerful, or socially connected. The transformation preserves the Strigoi's physical bodies against the ravages of sickness, hunger, and time and even enhances their physical power. But immortality and power do not come without cost, for the Strigoi must feed on the living. Strigoi regain power by drinking the blood of humans, and if they don't consume blood on a regular basis they are driven mad with hunger and pain.

While they present themselves as genteel, the Strigoi are serpents. Not just in that they lie, but that they have retractable poisonous fangs and cold blood like a snake. Some Strigoi have their eyes change to snakelike yellow slitted affairs when their fangs come out. Strigoi are also inherently obsessive. Despite their healthy and timeless appearance, Strigoi are dead; and they will often display a strong aversion to certain things which remind them of their human lives, although such aversion is by no means consistent between individuals. Individual Strigoi usually display some symptoms of obsessive-compulsive disorder, such as counting, cleaning, and otherwise ritualizing their lives.

The Strigoi and Nosferatu have a checkered history, with the Strigoi usually holding the upper hand. Strigoi lands were usually farther to the South into what is now the Balkans, but in those areas where both existed the Strigoi recruited from the top of society and the Nosferatu from the bottom. These social class differences persisted in the night as they did in the day.

A Strigoi has an Orphic power source and a Feeding power schedule.

\paragraph{The First Strigoi?} By far the most famous Strigoi was (is?) Dracula. He is actually rather hated by most Strigoi because his obsessive grandstanding over the years is basically a giant thumb in the eye of the Vow of Silence. He's been killed repeatedly, but various fanatics keep figuring out ways to bring him back from the dead. Which, being a Strigoi, mostly involves finding some significant portion of his corpse and pouring a bunch of fresh human blood on it.

But Strigoi were running around being aristocratic blood drinkers for over two thousand years before Dracula was a major player in the Makhzen-Covenant wars of the 15th century. The first Strigoi in the secret histories is Zalmoxis, who was a god king of the Dacians in the 12th century BCE. He enslaved other supernatural creatures and taught humans about Orphic magic. The secret histories are a little unclear on what exactly he did, because shortly after he was defeated the Tradition of Misdirection was created to prevent as much actionable sorcerous knowledge from falling into the hands of mortal humans as Zalmoxis had allowed.

%%%
\subsubsection{Strigoi Starting Powers}
%%%

\hspace{\parindent} - Core Discipline: Fortitude -\\
\linkpower{Patience of the Mountains} (Basic Fortitude)\\
\linkpower{Revive the Flesh} (Basic Fortitude)

- Basic Powers -\\
\linkpower{Vigor} (Basic Clout)\\
\linkpower{Bite of the Serpent} (Basic Lure of Destruction)\\
\linkpower{Gift of Health} (Basic Path of Blood)\\
\linkpower{Mesmerism} (Basic Authority)

- Advanced Powers -\\
\linkpower{Restoration} (Advanced Fortitude)\\
\linkpower{Indominability} (Advanced Fortitude)

\textbf{Distinctive Flaw:} Compulsive Behavior

\paragraph{Story Inspiration:} Lair of the White Worm, Dracula, Ultraviolet

%%%%%%%%%%%%%%%%%%%%%%%%%
\subsection[Daeva]{The Daeva} \index{Vampire!Daeva}
%%%%%%%%%%%%%%%%%%%%%%%%%
\tagline{"It turns out that humans \textbf{in general} are a superstitious and cowardly lot. How marvelous."}

So thoroughly has the American vampire integrated itself into vampire lore that it is difficult for us to remember that before contact was established between the New World and the Old, that the bat vampire was a being only of the Americas (as indeed, Vampire Bats themselves are a New World creature). The Daeva were called Onaqui and Tlahuelpuchi by the Nahuatl speakers of Central America, but seemingly before 1492 not a single one of them set foot in Africa or Eurasia. And yet, in the most recent centuries they have spread throughout the world, and transformed Luminaries of every skin tone into more of themselves.

Daeva are vampires with an affinity for bats and fire. While the European name for the bloodline has long ago been claimed as Daeva by those within, they are named as demons in whispered tones by mortals and supernaturals alike. The Daeva do not burn, and small bat wings protrude from their backs. Some of them have spots like a jaguar on their arms or legs. A Daeva's distinguishing features can often be relatively easy to hide. Wearing a backpack or even a heavy jacket covers wings quite nicely. However, once discovered a Daeva's animalistic, even \textit{demonic} traits are difficult to explain away as anything short of an extreme breach of the Vow of Silence.

The eldest of the Daeva represented much of the power behind The False Face when it considered itself a Syndicate, before the war with the Covenant. Tonight, Daeva are members of every Syndicate. Looking at the demographics of Daeva tonight, it would never occur to an observer that just five hundred years ago every Daeva on the planet was Native American.

A Daeva has an Infernal power source and a Feeding power schedule.

\paragraph{The First Daeva?} The word "Daeva" is actually from South Asia, where it originally referred to those Asura who were not part of any infernal kingdoms. The term was applied to American Vampires by European conquistadors who at the time were still hoping against reason that Aztlan was a kingdom in Indochina. The first Daeva \textit{called} Daeva were thus the cannibalistic overlords of the Arawak people at the end of the 15th century CE.

The mythical origins of the False Face and the bloodline of the Daeva were written in an Incan book called \refwork{Inti Jiwana}, which means "The Panther That Swallows The Sun". This "book" was actually a series of strings upon which multiple knots were made that imparted information digitally -- like a stack of punch cards. As far as anyone knows, the last copy of \refwork{Inti Jiwana} was burned by zealous and uncomprehending Spaniards. And it is also far from certain that anyone remains who knows how to read it anyway.

%%%
\subsubsection{Daeva Starting Powers}
%%%

\hspace{\parindent} - Core Discipline: Fortitude -\\
\linkpower{Patience of the Mountains} (Basic Fortitude)\\
\linkpower{Revive the Flesh} (Basic Fortitude)

- Basic Powers -\\
\linkpower{Vigor} (Basic Clout)\\
\linkpower{Fire Walking} (Basic Walk of Flame)\\
\linkpower{Gift of Health} (Basic Path of Blood)\\
\linkpower{Attract} (Basic Magnetism)

- Advanced Powers -\\
\linkpower{Restoration} (Advanced Fortitude)\\
\linkpower{Flight} (Clout / Magnetism Devotion)

\textbf{Distinctive Flaw:} Blatantly Magical

\paragraph{Story Inspiration:} Lost Boys, Dark Stalkers, Underworld

%%%%%%%%%%%%%%%%%%%%%%%%%%%%%%%%%%%%%%%%%%%%%%%%%%
\section{Witchcraft} \index{Witch}
%%%%%%%%%%%%%%%%%%%%%%%%%%%%%%%%%%%%%%%%%%%%%%%%%%
\tagline{"It's not all hocus pocus and chicanery. There's also sleight of hand and deceit. And raw sorcerous power."}

The core of sorcery in a horror setting is that it's a series of secrets that an individual learns that gives them great powers and is a really bad idea. When Extras attempt to learn Magic they don't turn into monsters or anything, they just end up as cultists. Think Cthulhu or Bible Black cultists. Better yet, don't. Extras who attempt to learn magic usually end up getting sacrificed one way or another, and it's really no better of a deal than being a Vampire Spawn. But for those lucky few who can master the dark arts without succumbing to the whispers or tentacles of demons, a lifetime of sorcerous power and intrigue awaits. 

Using magic is, for a mortal, a harrowing, draining, and indeed life and sanity threatening thing. However, through deliberate study or horrid inspiration a human can manage to change themselves into something that \textit{can} use magic with relative safety and reliability. Once attuned to sorcery a soul can never be the same, and people who become sorcerers can be wicked difficult to get along with or even \textit{understand}.

Witches are generally fairly vulnerable to normal physical attacks, but Silver actually causes them aggravated damage. Most Witches have Master Passion Greed. In After Sundown, men and women can both be "Witches", although some men prefer the epithet "Warlocks" and a considerable number prefer to call themselves by some more culturally specific term for a magician such as Houngan or Rishi.

%%%%%%%%%%%%%%%%%%%%%%%%%
\subsection[Baali]{The Baali Tradition} \index{Witch!Baali}
%%%%%%%%%%%%%%%%%%%%%%%%%
\tagline{"You won't remember what I show you now, and yet I shall awaken memories of love\ldots{} and crime\ldots{} and death\ldots{}"}

Perhaps the most expedient manner to make one's human soul stop reacting poorly to magic is to set it on fire and be done with it. A Baali is someone who did exactly that. Some sell their soul off to demons, while others burn out their souls quickly or slowly with their own dabbling in dark power.

Baali tend to be emotionally distant and enraged by pictures of themselves (such as reflections, paintings, or photographs). This is probably to do with the fact that their souls are burned out husks or trapped far away in a mysterious hell dimension and reminders of their pasts as humans are infuriating. The thing where they usually fail at having normal relationships with people is probably much the same thing, albeit played out in a grinding process of ashy distance rather than inexplicable rage. 

Baali refresh their powers by hurting people. This is functionally equivalent to vampiric blood drinking, but they don't specifically put any blood into their mouths. They still have to spend the same amount of time absorbing the proceeds of their wickedness. The actual distance they can be from the victim is about 1 meter per Potency each round after the wound is inflicted to draw a power point from the fresh injury.

The Baali Witch has an Infernal power source and a Feeding power schedule.

\paragraph{The First Baali?} The word "Baali" is an old Mesopotamian word meaning someone worthy of respect or allegiance. It has been handed around the Middle East and North Africa for millennia, and even such famous figures as Hanni\textit{bal} have reference to precisely that in their names. But the title used to mean someone who used Sorcery. That changed with the reforms of Hammurabi in the 18th century BCE, where people who had no magic at all could be known as a Baal. The first person in the secret histories who used Infernal magic to burn out their soul to keep the use of magic from destroying them was Marduk. He slew Tiamat and began the Marduk Society in the 25th century BCE. It is interesting to note that in the nights since, the Marduk Society has moved to favoring Astral magic, and no longer has many Baali members.

%%%
\subsubsection{Baali Starting Powers}
%%%

\hspace{\parindent} - Core Discipline: Authority -\\
\linkpower{Command} (Basic Authority)\\
\linkpower{Mesmerism} (Basic Authority)

- Basic Powers -\\
\linkpower{Aura Perception} (Basic Discernment)\\
\linkpower{Hand of Flame} (Basic Walk of Flame)\\
\linkpower{Light of Ennui} (Basic Descent of Entropy)\\
\linkpower{Learn the Heart's Pain} (Basic Names of the Blasphemies)

- Advanced Powers -\\
\linkpower{Cloud Memory} (Advanced Authority)\\
\linkpower{Fire Starter} (Advanced Walk of Flame)

\textbf{Distinctive Flaw:} Disloyal

\paragraph{Story Inspiration:} The 1932 Mummy, Constantine, Bible Black, Devil's Advocate, Faust

%%%%%%%%%%%%%%%%%%%%%%%%%
\subsection[Dryad]{The Dryad Tradition} \index{Dryad}
%%%%%%%%%%%%%%%%%%%%%%%%%
\tagline{"You seek for knowledge and wisdom, as I once did; and I ardently hope that the gratification of your wishes may not be a serpent to sting you, as mine has been."}

You know who doesn't have emotionally crippling experiences when exposed to the horrors of true magic? Plants. A person who replaces their heart with a seed can channel magic through themselves like water goes through roots. Sometimes this is a deliberate process, and other times it is the providence of an evil seed taking control of someone in their sleep. Regardless, once a person's blood runs with sap they are forevermore a witch.

Dryads do not remember their dreams, even when woken in the middle of them. Their sweat smells like flowering trees. And they become plantish in their demeanor. Regaining their power requires that they put parts of their body into the ground and water themselves. When they draw their hands or feet out of the soil, there will momentarily be white  rootlets that are rapidly drawn back into their pores with a sickening hiss.

A Dryad has an Astral power source and a Ritual power schedule.

\paragraph{The First Dryad?} The Dryadic tradition of magic was probably codified in Greece, around the 3rd century BCE. Like the Maenads of the same period, Dryads were a conscious attempt by Greeks to emulate Egyptian cultic practices of the time. However, unlike the Chain of Coronis it is no longer part of the historical record what exactly they were emulating. In the early nights, Dryads were defined and divided by the type of seed used to anchor their heart. If an apple seed was used, the Witch was an Epimeliad; if a laurel seed was used, the Witch was an Oread; and so on. During the early centuries of the Common Era, Dryads were almost exterminated in Europe, but the tradition was picked up upon elsewhere in the world. These new Dryads used different seeds based on what was available in their climate, and the old clan distinctions are no longer used.

%%%
\subsubsection{Dryad Starting Powers}
%%%

\hspace{\parindent} - Core Discipline: Coil of Thorns -\\
\linkpower{Bitter Fruit} (Basic Coil of Thorns)\\
\linkpower{Grass Rope} (Basic Coil of Thorns)

- Basic Powers -\\
\linkpower{Aura Perception} (Basic Discernment)\\
\linkpower{Rising Mists} (Basic Chasing the Storm)\\
\linkpower{Enchanted Slumber} (Basic Veil of Morpheus)\\
\linkpower{Pain Drops} (Basic Trail of Tears)

- Advanced Powers -\\
\linkpower{Puppetry} (Advanced Coil of Thorns)\\
\linkpower{Telepathy} (Advanced Discernment)

\textbf{Distinctive Flaw:} Aimless

\paragraph{Story Inspiration:} The Craft, Kotetsu, Poison Ivy

%%%%%%%%%%%%%%%%%%%%%%%%%
\subsection[Khaibit]{The Khaibit Tradition} \index{Witch!Khaibit}
%%%%%%%%%%%%%%%%%%%%%%%%%
\tagline{"There are far worse things awaiting man than death."}

Fear of darkness and death leads men to do strange things. Perhaps the strangest of them is to murder one's self and then use sorcery or advanced medical techniques to make one's foray into the land of the dead a temporary one. This does not inherently prevent a person from dying in the future, indeed quite the opposite. Many who gain the intimate secrets of death in this manner are chased by the shadow of doom for the remainder of their existence.

The Khaibit gain venomous saliva, a fact that they cannot choose to turn off under any circumstances. However, each Khaibit's kiss is dangerous in a different manner. The player of a Khaibit may choose their venom of choice.

The Ritual to regain power requires \textit{either} a bunch of medical equipment \textit{or} a bunch of corpse pieces, depending upon whether they originally killed themselves with scientific equipment or ancient rituals. That declaration needs to be made right at the beginning of character creation. However, while the more mystical types need to have a skull to "Alas Poor Yorick" with, it doesn't need to be especially fresh and they don't need to have chopped it off themselves. Either way, Khaibit are incredibly bad tenants, as some sort of horrible luck rubs off onto any place they stay for long. Appliances break down, walls cave in, windows break. Many Khaibit become paranoid about these effects, and come to believe that a mysterious force is hunting them to take them back to The Gloom. They might even be right.

The Khaibit Witch has an Orphic power source and a Ritual power schedule.

\paragraph{The First Khaibit?} Khaibit Sorcery is available throughout the night all over the world. But the earliest recorded books of power written for the subject were in Egypt around 1550 BCE. The "Book of the Dead" was itself based on earlier sorcerous investigations, but Khaibit Witches date back in a currently recognizable form to the beginnings of the Middle Kingdom in Egypt.

%%%
\subsubsection{Khaibit Starting Powers}
%%%

\hspace{\parindent} - Core Discipline: Necromancy -\\
\linkpower{Summon Spirit} (Basic Necromancy)\\
\linkpower{Compel Spirits} (Basic Necromancy)

- Basic Powers -\\
\linkpower{Aura Perception} (Basic Discernment)\\
\linkpower{Eyes of the Night} (Basic Play of Shadows)\\
\linkpower{Bite of the Serpent} (Basic Lure of Destruction)\\
\linkpower{Thaumaturgical Forensics} (Basic Path of Blood)

- Advanced Powers -\\
\linkpower{Solid Darkness} (Advanced Play of Shadows)\\
\linkpower{Reanimate} (Advanced Necromancy)

\textbf{Distinctive Flaw:} Haunted

\paragraph{Story Inspiration:} Reanimator, Flatliners, Jason and the Argonauts

%%%%%%%%%%%%%%%%%%%%%%%%%%%%%%%%%%%%%%%%%%%%%%%%%%
\section[Animates]{Animates: Last of their Kind} \index{Animate}
%%%%%%%%%%%%%%%%%%%%%%%%%%%%%%%%%%%%%%%%%%%%%%%%%%
\tagline{Once upon a time, there was\ldots{} 'A king!' my little readers will say right away. No, children, you are wrong. Once upon a time there was a piece of wood\ldots{}}

Animates are different from the other types of supernaturals because they are not born as a human only to later achieve supernatural powers, but are instead created fully formed and fully grown with their powers already intrinsic to them. An Animate awakens for the first time already an outcast, already both more and less than a human. Animates are almost always rejected by humanity because regardless of how much they try, they are \textit{not} humans. Every Animate is not only the first of its kind, it is also the \textit{last} of its kind, every Animate creation event is unique and Animates tragically find that even other created life is intrinsically \textit{distinct} from them, different from humanity in ways which likewise reveal a difference that is just as unbridgeable.

Animates are capable of being Extras and Luminaries just as normal humans are. In general, a mad scientist's masterwork is a Luminary Animate (the kind that might be a player character), and a batch of killer robots is probably a bunch of Animate Spawn. Which means that yes, those Animates who actually have peers don't even care because they are just background characters and mooks in their own story.

All Animates use a Ritual Power Schedule. Their false life leaves them vulnerable to things that were once alive. Though they never were born and never died, they still suffer aggravated damage from wooden weapons as if they were undead. In most cases, Animates are dominated by Master Passion: Loneliness, a condition not unrelated to the fact that each Animate is literally without peers. They have more difficulty making emotional connections then virtually any other creature, and pine for the loss.

\paragraph{Special Note on Terminology:} It is important to note that Frankenstein is the name of the creator, and not the name of the creature. The creature's name was Adam. Similarly, in technical Greek an Android refers only to a \textit{male} human analog, while a female human analog would be called a \textit{Gynoid} (seriously). However it is also important to note that most of the people you talk to about this issue don't know this and don't care. Like Jello, Frankenstein is a brand name of such overwhelming market dominance that the stitched corpse creations of other mad scientists are \textit{called} Frankensteins (rather than calling them gelatin desserts or reanimated homunculi or whatever). Similarly, if you use the word Gynoid in every day conversation people will assume you are talking about some sort of sexually transmitted condition. Honestly, it is better that common terminology is used rather than historically correct terminology in this instance.

%%%%%%%%%%%%%%%%%%%%%%%%%
\subsection[Frankensteins]{Frankensteins: The Creature} \index{Animate!Frankenstein}
%%%%%%%%%%%%%%%%%%%%%%%%%
\tagline{"My heart was fashioned to be susceptible of love and sympathy, and when wrenched by misery to vice and hatred, it did not endure the violence of the change without torture such as you cannot even imagine."}

Frankensteins are crafted to be literally new living things. They are essentially people, albeit designed rather than evolved. Unfortunately, the creators of these new creatures almost never take to their assumed role as parent, because the Frankenstein is never truly a baby and never truly a human. Propelled into existence with a grown human's size and facilities, they nonetheless suffer from having missed the opportunities to be juvenile. Like a man who has been inducted as a child soldier or a woman married at first blood, a Frankenstein's life will always have deeply painful emotional issues. They will always be childlike in some aspects of their existence, even if they live to be hundreds of years old.

Each Frankenstein is created out of human flesh in some way. Sometimes it is as simple as cobbling together body parts and then harnessing lightning or mystic power to bring it to life. Sometimes they are grown in tanks like the Hank and Dean. And in yet other methods of creation they are crafted in some other medium such as wood or stone and then \textit{transformed} into human flesh. Regardless of the creation methodology, a Frankenstein is close enough to human in appearance to pass for one. 

Like all Animates, Frankensteins need power. For most Frankensteins, this is \textit{electrical} power, though there are examples that recharged by standing under waterfalls or exposing themselves to tidal forces. The days of making Pinocchios that have to be tossed into the sea or Osirians who need to camp in rivers to recharge are largely over, and almost all Frankensteins made in the last hundred and fifty years operate on the more exciting and portable system of electricity.

A Frankenstein has an Orphic power source and a Ritual power schedule.

%%%
\subsubsection{Frankenstein Starting Powers}
%%%

\hspace{\parindent} - Core Discipline: Celerity -\\
\linkpower{Nimble Feet} (Basic Celerity)\\
\linkpower{Quickness} (Basic Celerity)

- Basic Powers -\\
\linkpower{Vigor} (Basic Clout)\\
\linkpower{Repel} (Basic Magnetism)\\
\linkpower{Patience of the Mountains} (Basic Fortitude)\\
\linkpower{Supernatural Senses} (Basic Discernment)

- Advanced Powers -\\
\linkpower{Devastation} (Advanced Clout)\\
\linkpower{Quicken Sight} (Advanced Celerity)

\textbf{Distinctive Flaw:} Conspicuous Consumption

\paragraph{Story Inspiration:} Frankenstein, Pinocchio, Edward Scissorhands, Subject Two, Herbert West: Reanimator

%%%%%%%%%%%%%%%%%%%%%%%%%
\subsection[Golems]{Golems: The Servitor} \index{Animate!Golem}
%%%%%%%%%%%%%%%%%%%%%%%%%
\tagline{"Did I request thee, Maker from my clay\\
To mould Me man? Did I solicit thee\\
From darkness to promote me?"}

Created strong of body and unending in endurance, the Golem is made to do tasks that humans don't want to. Often these jobs are demeaning or dangerous, and it is no surprise that Golems rarely want to do the things that are asked of them. Golems appear crude and ogrish, having been initially created for physical functionality rather than social niceties. Generally a Golem is as capable of making human emotional connections and decisions as anyone, but their distinctly non-human, even non-\textit{living} appearance makes that difficult. 

Crafted as a servile, almost machine-like creature, Golems fade into the background as much as the most discrete of servants. No one notices a tool that is not in use or a man far beneath them in social status, and a Golem is both. Each Golem is empowered by mystic words and dreams, and it is \textit{these} that the Golem "is". The body one sees (if one sees it at all) is truly inanimate and irrelevant material that is given life by the mystic runes that represent the Golem. While a Golem may grow accustomed to its shell, the fact is that they can be transferred to a new body of clay or wood without particular harm (although the transfer process usually takes most of a day).

The Golem is powered by the meaning of the words, and they recharge themselves by repeating and emphasizing them over and over again like Tibetan monks. It is not unusual to find a Golem recharging itself by writing and rewriting phrases on a chalk board like Bart Simpson or intoning a mystic song again and again.

A Golem has an Astral power source and a Ritual power schedule.

%%%
\subsubsection{Golem Starting Powers}
%%%

\hspace{\parindent} - Core Discipline: Veil -\\
\linkpower{Hide From Notice} (Basic Veil)\\
\linkpower{Mask of a Thousand Faces} (Basic Veil)

- Basic Powers -\\
\linkpower{Fire Walking} (Basic Walk of Flame)\\
\linkpower{Vigor} (Basic Clout)\\
\linkpower{Patience of the Mountains} (Basic Fortitude)\\
\linkpower{Summon Spirit} (Basic Necromancy)

- Advanced Powers -\\
\linkpower{Indominability} (Advanced Fortitude)\\
\linkpower{Lost and Found} (Advanced Veil)

\textbf{Distinctive Flaw:} Either Anachronism \textit{or} Naive

\paragraph{Story Inspiration:} Rabbi Loew, R.U.R., Short Circuit, Talos

%%%%%%%%%%%%%%%%%%%%%%%%%
\subsection[Androids]{Androids: The Lover} \index{Animate!Android}
%%%%%%%%%%%%%%%%%%%%%%%%%
\tagline{"How fully functional are you?"}

Androids are made of hard and constructed material rather than the softness of flesh, and they are coldly perfect to behold. Whether man or woman, an android is beautiful beyond what normal humans are capable of. Unfortunately, the metallic perfection of their exterior belies a digital corsity that makes interpersonal relationships unsatisfactory. The emotions of an Android, though strongly felt, are simply too extreme and simple for others to relate to. The love of an Android is as off-putting as their hatred to a normal, nuanced individual. Androids do not, on initial inspection, look anything other than human. However there is something distinctly "not right" about them which will creep out any normal human who interacts with them socially for long. They are beyond autistic, their emotions are simply ones and zeros.

Androids can be made of circuitry, gears, or simply polished marble. The key is that at the end of the procedure they are crafted to look more human than a human could possibly be. The insides of an Android actually matter a fair amount for purposes of what they do to recharge themselves. An Android filled with gears needs to patiently wind themselves up (conservation of energy be damned), while an Android made of circuits and wires needs to plug themselves into a wall. Whichever it is, an Android recharging breaks the illusion of a perfect human, as the act of opening themselves to work on their clearly inorganic insides pushes them into the uncanny valley until they close themselves up again.

An Android has an Infernal power source and a Ritual power schedule.

%%%
\subsubsection{Android Starting Powers}
%%%

\hspace{\parindent} - Core Discipline: Clout -\\
\linkpower{Vigor} (Basic Clout)\\
\linkpower{Clinging} (Basic Clout)

- Basic Powers -\\
\linkpower{Light of Ennui} (Basic Descent of Entropy)\\
\linkpower{Attract} (Basic Magnetism)\\
\linkpower{Patience of the Mountains} (Basic Fortitude)\\
\linkpower{Howling Winds} (Basic Chasing the Storm)

- Advanced Powers -\\
\linkpower{Devastation} (Advanced Clout)\\
\linkpower{Contradiction} (Advanced Descent of Entropy)

\textbf{Distinctive Flaw:} Doomed Romance

\paragraph{Story Inspiration:} Pygmalion, Metropolis, RUR, Weird Science, Ifurita, Do Androids Dream of Electric Sheep?

%%%%%%%%%%%%%%%%%%%%%%%%%%%%%%%%%%%%%%%%%%%%%%%%%%
\section[Leviathan]{The Remnants of the Leviathan} \index{Leviathan}
%%%%%%%%%%%%%%%%%%%%%%%%%%%%%%%%%%%%%%%%%%%%%%%%%%
\tagline{Down and deep / dark and dank / below our feet / below they sank\\
Under waves and / under ground / no light we have / no hope for sound\\
Beneath us still / sea and mound / remember will / with silence found}

Long ago, Tiamat was a great and powerful dragon who ruled over the lands and exercised her rights of ownership with great cruelty. She demanded sacrifices from the frightened people of the soil, and she bore many monsters to expand her kingdom. The oppression of Tiamat did not forever last, for eventually a great champion of the human people arose with powerful magic from the sky to battle her. For days they fought, and ultimately it was Marduk and not Tiamat who was victorious.

But while the great Tiamat was slain, her children walked the Earth and swam the seas still. Each of her children bore children of their own. Many of these were able to pass for human and walk in the world of men, and still many more were monstrous beyond mortal comprehension. The taint in the blood of those who are heirs to the power of Tiamat gradually makes itself known. These new generations, called Leviathan, become less human in countenance and thought as they grow in age and power.

Tiamat's descendants are foul tasting as swamp mire, but they are prized as food regardless because whosoever devours the flesh of a Leviathan gains near immortality. However, the otherworldly and inhuman degeneration which happens to a natural born Leviathan happens twice as fast to those who attain their state by eating the flesh of mermaids.

Those born to Leviathan broods who are not luminaries will (if they are lucky) never betray their leviathan heritage, appearing for all the world as normal humans (though perhaps with a strange bestial cast to their features). These leviathan kin carry Tiamat's taint, and if they have children of their own they may revert to being full leviathans should they be luminaries themselves. Other children are not so lucky, and many of them are pathetic spawn monsters, often with little or no human intelligence. The mutated visages of humanity"s castoffs make ample fodder for nightmares, and little else.

Leviathans suffer aggravated damage from Iron. Most Leviathans suffer under Master Passion Fear and it is not unusual at all for them to flee from the world and hide themselves deep in the wilderness in fortresses of their own making. Some of these are simply Geinian farmsteads, whose only "real" defense from the outside world is obscurity.

%%%%%%%%%%%%%%%%%%%%%%%%%
\subsection[Deep Ones]{The Deep Ones: Scions of Dagon} \index{Leviathan!Deep One}
%%%%%%%%%%%%%%%%%%%%%%%%%
\tagline{"Those horrible eyes. Unblinking and inhuman, twas like looking at fish more than a man."}

According to legend, Dagon was spawned when Tiamat mated with the Tigris River. He was a powerful and fish-like creature, and his progeny all carry fish traits about their being. Hailing from the oceans, a Deep One gradually becomes scaled and gilled as he ages. His eyes flatten and rarely blink, and the lure of the sea becomes harder to ignore. There are few who can stomach long departures from the coast, for in their mind they can always listen for the strange whispered yet incessant call of the ocean.

Deep Ones can breathe water as easily as air (though they do not \textit{need} to breathe at all), and they have no problem speaking with lungs full of fluid. The strange whispers they hear in their own minds can also be shared with anyone foolish enough to look a Deep One in their strangely opaque eyes. Over time, the Deep One's teeth become sharp like some benthic terror and their skin becomes scaly and slimy like a muck dwelling fish. 

Deep Ones are in many ways closer to Tiamat than other Leviathans, and the corruption she caused is manifest in the way they seem to affect the world. Ordinary people who spend long periods of time in the company of Deep Ones will find their sanity frayed at the edges like cloth stretched too tight. A Deep One gripped by Master Passion Fear spreads terror amongst the mortal Extras they converse with, whether they want to or not. Deep Ones dominated by other Master Passions spread different madness by mere presence.

A Deep One has an Astral power source and a Lunar power schedule.

\paragraph{The First Deep One?} The secret histories record the first Deep One by name. He is Dagon, the first and largest of the Deep Ones. Accounts differ as to his appearance, but all attest to his great size and blue scales. While some historians doubt the veracity of the Dagon myth, the majority of those who study the Deep Ones regard it as true. The general opinion is that Dagon is still alive, lurking and dreaming deep beneath the waves. Sometimes one of the progeny will be contacted in their dreams by an Elder that claims to be Dagon, and the general opinion of Deep Ones is that those contacts are real. The requests that the putative Dagon makes are perplexing and do not seem to add up. Opinion is divided as to whether this entity is playing a far deeper game or is simply completely out of touch with reality.

If the Dagon story is true, he would have been born somewhere between 3 and 5 \textit{thousand} years BCE. That would make him ancient beyond reckoning even by the standards of Elders.

%%%
\subsubsection{Deep One Starting Powers}
%%%

\hspace{\parindent} - Core Discipline: Discernment -\\
\linkpower{Aura Perception} (Basic Discernment)\\
\linkpower{Supernatural Senses} (Basic Discernment)

- Basic Powers -\\
\linkpower{Dream Vision} (Basic Veil of Morpheus)\\
\linkpower{Patience of the Mountains} (Basic Fortitude)\\
\linkpower{Command} (Basic Authority)\\
\linkpower{Rising Mists} (Basic Chasing the Storm)

- Advanced Powers -\\
\linkpower{Telepathy} (Advanced Discernment)\\
\linkpower{Conditioning} (Advanced Authority)

\textbf{Distinctive Flaw:} Infectious Mood

\paragraph{Story Inspiration:} The Shadow Over Innsmouth, The Creature From the Black Lagoon, Mermaid's Scar

%%%%%%%%%%%%%%%%%%%%%%%%%
\subsection[Troglodytes]{The Troglodytes: Progeny of Drakaina} \index{Leviathan!Troglodyte}
%%%%%%%%%%%%%%%%%%%%%%%%%
\tagline{"Thousands of years we have toiled beneath the Earth. We will not stop just because shiftless Eloi tell us we must."}

The ancient tablets speak of Drakaina, Tiamat's heir by the Euphrates. A terrible worm creature living deep in the soil, her descendants carry always her pallor and disdain for the sun. The eyes of a Troglodyte darken until they are naught but obsidian orbs in their pallid sockets. The fingernails of a Troglodyte grow rapidly and hard, and without constant hygiene they form into claws.

Troglodytes see in darkness perfectly and can even read in the total absence of light. However, their blackened eyes are light sensitive, and bright flashes can easily blind a Troglodyte. The sharpened teeth of a Troglodyte seem destined to devour flesh, and indeed that is precisely how they gain Power Points. A Troglodyte needs to feed upon the flesh of sapient creatures (generally humans, but some Troglodytes say that Mirror Goblins are delicious), though these need not be particularly fresh. The corpse of a human has 15 power points in it, and when they have been consumed, the rest of the corpse is worthless (though possibly still delicious). Many Troglodytes dig up the interred dead for eating purposes, and this activity has earned them the nickname "ghouls".

A Troglodyte has an Orphic power source and a Feeding power schedule.

\paragraph{The First Troglodyte?} The lineage of the Troglodytes goes back to the days of Babylon and likely before. The ancient worm-mother of the Troglodytes is likely a historical person, and some say that she yet lives. The secret histories claim that she burrowed deep beneath the surface world to found herself a new kingdom where the sun would never shine. This land beneath the land is not in contact with any surface dwellers -- not even those Troglodytes who continue to see the stars.

%%%
\subsubsection{Troglodyte Starting Powers}
%%%

\hspace{\parindent} - Core Discipline: Discernment -\\
\linkpower{Aura Perception} (Basic Discernment)\\
\linkpower{Supernatural Senses} (Basic Discernment)

- Basic Powers -\\
\linkpower{Touch of Darkness} (Basic Lure of Destruction)\\
\linkpower{Clinging} (Basic Clout)\\
\linkpower{Patience of the Mountains} (Basic Fortitude)\\
\linkpower{Hide From Notice} (Basic Veil)

- Advanced Powers -\\
\linkpower{Psychometry} (Advanced Discernment)\\
\linkpower{Burrowing} (Veil / Clout Devotion)

\textbf{Distinctive Flaw:} Unattractive

\paragraph{Story Inspiration:} Mole People, Morlocks, The Hills Have Eyes

%%%%%%%%%%%%%%%%%%%%%%%%%
\subsection[Mi Go]{The Mi Go: The Larvae of Echidna} \index{Leviathan!Mi Go}
%%%%%%%%%%%%%%%%%%%%%%%%%
\tagline{"There can never be good for the bee which is bad for the hive."}

"Tiamat begat Echidna by the Indus." That is the beginning of the story that the Mi Go tell of their existence. And it is there that the story becomes confused, because the Mi Go \textit{do not know what they are}. This isn't some sort of regular existential angst caused by awakening to their supernatural nature, but a genuine question of philosophy and science. Each Mi Go is a telepathic symbiosis of human and insect, and it is entirely reasonable to question which (if either) is the actual person. Even the eldest and wisest of the Mi Go do not have incontrovertible evidence to determine whether they are a human that has insects growing within their body that they can control, or a colony of insects that lives within a man sized puppet host. Perhaps the truth is somewhere in between.

A Mi Go experiences their first years of life from the perspective of being a human child. A human child who sometimes feels painful gnawing sensations within their body as they drift to sleep and as they wake up in the morning -- as if they were being devoured from within by maggots. Which in truth, they are. If you were to drill a hole in one of these children, you would find larval insects in writhing clusters throughout their body. It is usually around their human puberty that the insects themselves mature to flying forms and it is a ghastly, painful, and terrifying spectacle when the insects burrow their way out of the awakening Mi Go's body and takes to the sky.

It is at this point that the Mi Go's powers truly become apparent, because during the metamorphosis, the young Mi Go feels not only their own flesh become torn asunder by uncounted scores of tiny mandibles, but also simultaneously see through the eyes of each of the hive's constituent members. From that moment on, "they" are able to see through the eyes of the bugs that grow within the human body \textit{and} from the eyes of the human body. They do not have a "feeling" of having their thoughts or point of view necessarily within their head, and many of them are convinced that they are solely the hive and have murdered and stolen the memories of the child that they walk around wearing the skin of. Getting insecticides onto a Mi Go has roughly the effects of mace -- it's \textit{incredibly} painful and distracting. The poisoning of the bugs inside their flesh \textit{feels} like the entirety of their body is on fire.

A Mi Go has an Infernal power source and a Lunar power schedule.

\paragraph{The First Mi Go?} The core question of identity is terribly important to many Mi Go, and finding the original being and asking it "Why" is one of the holy grails of Mi Go thought. According to the secret histories, Echidna flew into the Himalayan Mountains some four \textit{thousand} years ago and has not been seen or heard from since. Many believe that she (it?) died during that journey. But there are many Mi Go who would kill to see even a corpse representing what form represents their destiny.

%%%
\subsubsection{Mi Go Starting Powers}
%%%

\hspace{\parindent} - Core Discipline: Swarm Song -\\
\linkpower{Small Witness} (Basic Swarm Song)\\
\linkpower{Body Colony} (Basic Swarm Song)

- Basic Powers -\\
\linkpower{Abyss of the Body} (Basic Descent of Entropy)\\
\linkpower{Patience of the Mountains} (Basic Fortitude)\\
\linkpower{Attract} (Basic Magnetism)\\
\linkpower{Supernatural Senses} (Basic Discernment)

- Advanced Powers -\\
\linkpower{Telepathy} (Advanced Discernment)\\
\linkpower{Magnify the Swarm} (Advanced Swarm Song)

\textbf{Distinctive Flaw:} Flake

\paragraph{Story Inspiration:} Vampire Hunter D, Wrath of Khan, Starship Troopers, Whisperer in Darkness

%%%%%%%%%%%%%%%%%%%%%%%%%%%%%%%%%%%%%%%%%%%%%%%%%%
\section[Transhumans]{Transhumans: Point of No Return} \index{Transhuman}
%%%%%%%%%%%%%%%%%%%%%%%%%%%%%%%%%%%%%%%%%%%%%%%%%%
\tagline{"Think before you decide, I tell you! Do you want to be left as you are, or do you want your eyes and your soul to be blasted by a sight that would stagger the devil himself?"}

Knowledge and power do not come without a price, especially in the realm of horror. When one steps upon the path to power it is not long before the realization strikes that the person on that path is not the same person who took the first steps. Sometimes the point of no return is obvious and comes as a flash of insight. At other times it is only in retrospect and solemn reflection that one can see that there's little connection between the humanity of youth and the creature of the present. Transhumans do not have souls in the way that normal people do. 

Transhumans have the easiest time convincing themselves that they are still the person that they used to be -- still a mortal human with all the accompanying impetus and fragility. This is not actually true, and the sheer \textit{plausibility} of that kind of self delusion leads many into a constant rollercoaster of false hopes and crushing disappointments. On some level normal people can sense it, and vague misgivings surround all dealings between a Transhuman and mortal men. The moment the character attains their power they have become something more than a man, but they have also become something \textit{less}, and it is this truth that all Transhumans must confront.

Transhumans suffer aggravated damage from Iron weapons, the very human achievement that propels them beyond their biological origins also turns on them with cruel finality. Most Transhumans are dominated by Master Passion Despair. 

%%%%%%%%%%%%%%%%%%%%%%%%%
\subsection[Reborn]{The Reborn: Second Chances and Second Guesses} \index{Transhuman!Reborn}
%%%%%%%%%%%%%%%%%%%%%%%%%
\tagline{"Put your helmet on. I wouldn't want to scar your pretty face. Again."}

Reincarnation totally happens in After Sundown. Not to everyone, not even to most people. But to \textit{some} people. For whatever reason, \textit{only} Luminaries can access past lives, and even then past lives only become accessible when they are exposed to great stress in the presence of magic. Why this happens or what the consequences of this are is a matter of intense debate amongst the supernatural communities. It is possible that only Luminaries \textit{have} past lives, or even that Luminaries are Luminary \textit{because} they have lived before.

How many lives a particular Reborn remembers is quite variable. Sometimes it is as few as one, while other Reborn remember snippets of dozens. It is undeniably true that the chances of a particular Reborn running into specific people they knew in a past life (either other Reborn or immortals who happened to be alive back then) are extremely high. Past lovers and rivals both are \textit{very} likely to be encountered by a Reborn during their new life, leading as many to repeat their old mistakes as to fix their previous errors.

There are prophetic maps available that plot the likely eruptions of new Reborn, and there are more than a few supernatural creatures and organizations of supernatural creatures that take an active interest in such matters. The Makhzen and Marduk Society traditionally find the Reborn to be natural members and go to great lengths to recruit them. Reborn are interesting in that they can be successfully identified with prophetic writings and comparisons to ancient statues and paintings long before they actually develop any powers.

A Reborn has an Orphic power source and a Lunar power schedule.
\paragraph{The First Reborn?} It is apparently true that no Reborn is ever born (much less awakened) during the lifetime of one of their past lives. Furthermore, all past lives are truly \textit{in the past}, and there are no future incarnations remembered by any Reborn. So it stands to reason that at the earliest, the first Reborn had to have been born a couple generations after the first human. The earliest Reborn who is specifically listed in the secret histories is Krishna: a mighty Tamil warrior who lived around 670 BCE and who had at least one previous life as a sage.

%%%
\subsubsection{Reborn Starting Powers}
%%%

\hspace{\parindent} - Core Discipline: Celerity -\\
\linkpower{Quickness} (Basic Celerity)\\
\linkpower{Nimble Feet} (Basic Celerity)

- Basic Powers -\\
\linkpower{Supernatural Senses} (Basic Discernment)\\
\linkpower{Summon Spirit} (Basic Necromancy)\\
\linkpower{Attract} (Basic Magnetism)\\
\linkpower{Shadow Casting} (Basic Play of Shadows)

- Advanced Powers -\\
\linkpower{Shifting Sands} (Celerity / Magnetism Devotion)\\
\linkpower{Psychometry} (Advanced Discernment)

\textbf{Distinctive Flaw:} Distinctive Appearance

\paragraph{Story Inspiration:} She, The Mummy

%%%%%%%%%%%%%%%%%%%%%%%%%
\subsection[Fallen]{The Fallen: Rising from the Ashes} \index{Transhuman!Fallen}
%%%%%%%%%%%%%%%%%%%%%%%%%
\tagline{"There is only one thing worse in the world than to be talked about. And that is to not be talked about."}

Those luminaries whose souls are scoured out to beyond the possibility of recognition by the harsh infernos of Limbo are left ageless shells, they are The Fallen. Sometimes a man can become so by coming into possession of some demonic trinket that slowly or calamitously steals away their existence. But more commonly this is the result of a child being stolen away into the Dark Reflection by mirror goblins to toil away in the edifices of the dark king. These changelings usually are worked to death, and never see home. But sometimes a luminary child will grow hard and strong, and bitter like long steeped tea. And they will someday find way to escape their soot covered prison and return triumphantly to their parents, to be clasped again by loving hands and to walk again amongst mothers and friends. But this is perhaps the cruelest aspect of fairy captivity, for the children so taken are not long sought after for powerful magics are wrought to remove the Fallen from the thoughts of those who care for them. It is an alienating experience for those lucky enough to escape servitude back to the mortal world -- while the family they remember like as not still exists, it is rare indeed that this family has any familiarity with them in return.

The Fallen do not have magic power of their own, but their bodies have become attuned to the ghastly magic of the Dark Reflection, and they can charge themselves up by exposing themselves to more Infernal energies. The Ritual to regain Power Points requires exposure to something from Limbo. Those who were transformed by an Infernal artifact are well advised to keep their hands firmly upon it. Those who escape the demonic realm usually carry something with them to recharge themselves with, though of course simply \textit{returning} to the Dark Reflection is always an option.

The very fact that a Fallen's magic powers do not come back without exposure to more of the evil magic that caused their condition in the first place has led some to speculate that if they were to simply avoid Infernal power long enough that they would regain their human lives. If anyone has succeeded in that, there are no reliable records of it. This is in a sense unsurprising, in that there are very few reliable records of Fallen. Not only are Fallen simply unmemorable to Extras, but even documents pertaining to them just "get lost". And it isn't merely that their rental contracts get accidentally thrown out, though that does happen. Fallen find copies of old yearbooks that simply don't contain any pictures of them.

A Fallen has an Infernal power source and a Ritual power schedule.
\paragraph{The First Fallen?} The King with Three Shadows\index{King with Three Shadows} has only been kidnapping mortal children since "the project" began in the 6th century CE. While the Troll Kingdoms had human slaves a thousand years before that, that practice ended when they were banished to Limbo, and human slaves in the mortal world did not become Fallen. The earliest Fallen in the secret histories is a Central African Queen of the 3rd century BCE who went by "She Who Shall Not Be Named" -- likely because the process of becoming Fallen had stripped her of her original name. Supposedly she had a sacred flame that granted immortality. What happened to the Queen, her kingdom, or the presumably Limbo-powered flame is a matter of speculation even amongst those who know the secret histories. The last contact with any of them came with a trading expedition from Egypt in 476 CE.

%%%
\subsubsection{Fallen Starting Powers}
%%%

\hspace{\parindent} - Core Discipline: Magnetism -\\
\linkpower{Attract} (Basic Magnetism)\\
\linkpower{Repel} (Basic Magnetism)

- Basic Powers -\\
\linkpower{Deny the Gauntlet} (Basic Progress of Glass)\\
\linkpower{Mask of a Thousand Faces} (Basic Veil)\\
\linkpower{Patience of the Mountains} (Basic Fortitude)\\
\linkpower{Learn the Heart's Pain} (Basic Names of the Blasphemies)

- Advanced Powers -\\
\linkpower{Dismissal} (Advanced Magnetism)\\
\linkpower{Desire Reflection} (Veil / Magnetism Devotion)

\textbf{Distinctive Flaw:} Either Minor \textit{or} Feared By Children

\paragraph{Story Inspiration:} Portrait of Dorian Gray, She, Rip Van Winkle

%%%%%%%%%%%%%%%%%%%%%%%%%
\subsection[Icarids]{The Icarids: Children of Daedalus} \index{Transhuman!Icarids}
%%%%%%%%%%%%%%%%%%%%%%%%%
\tagline{"There are things man was not meant to know. It is interesting that we repeatedly seek to know as much about these as possible."}

Humanity's greatest survival trait, indeed the one which has ensured our ascent to power, is a willingness and ability to improve upon our environment -- the closest environment being our own flesh and bones. The desire to gain greater abilities overwhelms all else sometimes; even personal ethics, self-preservation, and basic sanity. Man is first and foremost a tool using creature, but when he treats himself as a tool, is he still a man?

The Icarids are born when a Luminary is subject to the (nominally successful) experiments of scientists and practitioners of medicine to improve the human form beyond what it already has -- and usually this Luminary is both subject and creator of the experiment. They rarely have any familiarity or experience with the supernatural beforehand, and simply stumble through the veil of normality by "accident". The transformational event bestows efficiency and power on the human form and mind that brings it beyond mortal limitations, including impressive strength and will -- at the cost of sanity and self. 

The drugs or procedures done to bring an Icarid to the point they now are aren't just unwise -- they're \textit{lethal}. If you did that to an Extra they'd just die. It is in no small part the strength of the Luminary's dreams that force them onward through certain death and into power and madness. The ritual to regain Power Points involves continuing to do more of the same to themselves. What's better than a lethal dose of mercury-based super serum? \textit{More} mercury-based super serum. It is not at all weird for an Icarid who has been at it for a while to bleed clear liquid, sand, or even tiny grubs.

An Icarid has an Astral power source and a Ritual power schedule.

\paragraph{The First Icarid?} According to the secret histories, the first identified Icharids were the natural philosopher Daedalus and his son Icarus. They stumbled upon a humoral treatment based on goose bile and hot wax that would allow them to transcend human limitations in approximately 630 BCE. Icarus improved upon the treatment considerably and attained the aptitude of Flight. Icarus' experiments ended in tragedy when he flew over the ocean and a sudden rainstorm left him powerless. When he plunged into the sea, the pitiless waves crashed his nearly mortal body against the unforgiving cliffs until he was dead. In his grief, Daedalus named the process in honor of his son, but he still followed the Tradition of Misdirection, and reported the events somewhat differently to the humans of his time.

%%%
\subsubsection{Icarid Starting Powers}
%%%

\hspace{\parindent} - Core Discipline: Veil -\\
\linkpower{Hide From Notice} (Basic Veil)\\
\linkpower{Mask of a Thousand Faces} (Basic Veil)

- Basic Powers -\\
\linkpower{Supernatural Senses} (Basic Discernment)\\
\linkpower{Curse of Failure} (Basic Trail of Tears)\\
\linkpower{Clinging} (Basic Clout)\\
\linkpower{Revive the Flesh} (Basic Fortitude)

- Advanced Powers -\\
\linkpower{Dark Night of the Soul} (Advanced Trail of Tears)\\
\linkpower{Hide in Plain Sight} (Advanced Veil)

\textbf{Distinctive Flaw:} Prideful

\paragraph{Story Inspiration:} The Invisible Man, Dr. Jekyll and Mr. Hyde, Norman Osbourne, Karl Ruprecht Kroenen, Bane, Icarus

%%%%%%%%%%%%%%%%%%%%%%%%%%%%%%%%%%%%%%%%%%%%%%%%%%
\section{Demons} \index{Demon}
%%%%%%%%%%%%%%%%%%%%%%%%%%%%%%%%%%%%%%%%%%%%%%%%%%
\tagline{"Match wits with a creature older than time? Match wits with a prince of the dark dominions?"}

The Demons were seemingly around a lot longer than other supernatural creatures: the oldest records left by vampires or the recollections of the Returned allow for the civilization of the Demons as having been around for unfathomable years even then. This makes their predicament something of a puzzle, because every Demon is imprisoned in Limbo to a greater degree than any of its other inhabitants. Who or \textit{what} imprisoned the Demons there is not particularly clear, and individual Demons advance distinct theories when asked. All Demons live forever, and many of them \textit{are} frightfully old. Many of them claim to have been around during "The Great Banishment" -- but in almost all cases this is certainly a lie. It's not even clear where the banishment was \textit{from}. Some claim that it was the Material World, while others claim that the original Demonic home world was Maya. Still others claim that it was some as yet unnamed world that they cannot return to. What is clear is that Demons have a tremendous difficulty leaving the Dark Reflection: their Potencies are considered four higher for purposes of overcoming the Gauntlet.

%%%%%%%%%%%%%%%%%%%%%%%%%
\subsection{Akuma} \index{Demon!Akuma}
%%%%%%%%%%%%%%%%%%%%%%%%%
\tagline{"I always get what I want because I take it."}

The Akuma were never human and this fact cannot easily escape onlookers. Standing about a meter taller than a man, Akuma are also blessed with extra\ldots{} parts. Rows of teeth, third eyes, sometimes even extra arms or mouths. Many come equipped with extra parts that are not analogous to any found in normal people such as horns, pincers, and tentacles. They come in colors like red and blue rather than the tan and slightly darker tan that humans are familiar with. Even the white ones are \textit{white}, rather than merely a slightly paler shade of tan. These are the demons you imagine when your imagination has a bottomless budget for costuming. 

Akuma need to feed on sapients in order to restore their powers. Left on their own in the Dark Reflection they foist mirror goblins into their maws with wild abandon, but they prefer the flesh of humans. While Akuma do not strictly speaking \textit{need} to eat, they love doing so and their gluttonous appetites cannot be sated easily or long. Akuma are a huge social problem for mortals and supernaturals alike when they appear in the material world, for they lack subtlety in any of their dealings. Akuma are lazy bullies whose only redeeming feature is that at least their short sighted avarice makes them easy to manipulate by those with sufficient power to not be devoured right off. An Akuma has no driving passions, and is dominated by Master Passions of Rage, Hunger, and Fear.

While it is \textit{possible} for an Akuma to gain power nonfatally from a victim, their love of livers and poor discipline ensure that this almost never happens. An Akuma cannot turn their Giant Size off. An Akuma's claws, horns, teeth, or whatever constitute a damage 3 weapon (including their monstrous size). Akuma have a nonstandard attribute array because they were never humans. Before their Potency modifier and their constant Giant Size, their attribute ranges are:

S: 6/11 A: 1/5 I: 1/3 L: 1/3 W: 2/7 C: 1/4

An Akuma has an Infernal power source and a Feeding power schedule.

%%%
\subsubsection{Akuma Starting Powers}
%%%

\hspace{\parindent} - Core Discipline: Clout -\\
\linkpower{Vigor} (Basic Clout)\\
\linkpower{Clinging} (Basic Clout)

- Basic Powers -\\
\linkpower{Command} (Basic Authority)\\
\linkpower{Patience of the Mountains} (Basic Fortitude)

- Advanced Powers -\\
\linkpower{Giant Size} (Advanced Clout)\\
\linkpower{Devastation} (Advanced Clout)

\paragraph{Story Inspiration:} Where the Wild Things Are, Legend, Urotsukidoji, Aka Oni

%%%%%%%%%%%%%%%%%%%%%%%%%
\subsection{Asura} \index{Demon!Asura}
%%%%%%%%%%%%%%%%%%%%%%%%%
\tagline{"And for revenge thou hast created this demon. Her domain is darkness - her purpose is wickedness"}

The Asura appear as relatively attractive, if distant and cruel humans. As the only type of Demon that is not constitutively required to eat or torture people to death, they are the most likely to have neutral or positive relations with residents of the mortal world. Nevertheless, Asura very rarely behave in any manner that could be even generously described as less than unnecessarily dickish. While they gain no literal sustenance from harming people, they are often pressed into the service of powerful Ifrit to do so and just plain seem to like doing it.

Some Asura have wings that resemble those of swans or bats growing from their backs. The ones that don't have either wheels of fire or tiny clouds appear under their feet when they fly. While technically a Asura was never a human, they are close enough in appearance and capabilities that their attribute ranges before Potency modifiers are human standard.

Asura are superficially similar to Daeva, and it is primarily for this reason that Daeva were often subject to persecution by Covenant forces about 400 years ago. More extensive investigations recognize many key differences. For example, while a Daeva is born as a human luminary and is converted into a Vampire upon death at another's hands, a Asura comes into being by coalescing out of ash in Limbo, already fully grown. Blood flows in the veins of a Daeva only when they have recently fed, while blood does not actually exist inside a Asura at all -- when their skin is broken it cracks like porcelain and a fine ash drifts out.

Asura easily insinuate themselves in leadership positions in the mortal world because of their tremendous and magically augmented presence. These qualities are held in no esteem whatsoever in Demonic culture, and mere likability is treated with extreme contempt. Virtually all Asura have been made to swear total vassalage to a more powerful Asura or Ifrit. Domination, whether magical in nature or simple brute force is the currency of Demonic relationships. An Asura's power ritual is a bitter ash-eating affair that leaves their throat parched and their eyes red and raw. Asura suffer from Master Passion: Loneliness, though demonic society is actually so unsympathetic to relationships that few of them actually understand that fact.

Some Asura have enough extra disciplines to put them on a roughly equal footing with starting player characters of other supernatural types. While still being totally inhuman, Asura are \textit{by far} the most player-friendly of the extra-terrestrials. It is still rather difficult roleplaying, because they were never humans, have no aspirations to ever \textit{be} humans, and were at no points in their un-Earthly lives ever confused on any of those points. But still, with some work, it can be done.

An Asura has an Infernal power source and a Ritual power schedule.

%%%
\subsubsection{Asura Starting Powers}
%%%

\hspace{\parindent} - Core Discipline: Magnetism -\\
\linkpower{Attract} (Basic Magnetism)\\
\linkpower{Repel} (Basic Magnetism)

- Basic Powers -\\
\linkpower{Supernatural Senses} (Basic Discernment)\\
\linkpower{Patience of the Mountains} (Basic Fortitude)\\
\linkpower{Clinging} (Basic Clout)

- Advanced Powers -\\
\linkpower{Flight} (Clout / Magnetism Devotion)\\
\linkpower{Dismissal} (Advanced Magnetism)\\
\linkpower{Summons} (Advanced Magnetism)

\paragraph{Story Inspiration:} Succubus, Disgaea, Hell Bent

%%%%%%%%%%%%%%%%%%%%%%%%%
\subsection{Ifrit} \index{Demon!Ifrit}
%%%%%%%%%%%%%%%%%%%%%%%%%
\tagline{"I don't need you dead, just for you to wish you were."}

By far the rarest and most terrifying denizens of Limbo are the Ifrit. They are not especially powerful physically, and indeed they are normally intangible when encountered. But they have incredible mystical powers, almost unmatched amongst supernatural creatures. Not a few of them have taken to passing themselves off as gods when they reach the material world, and yet they have never achieved much status within any of the major Earthly Syndicates. This is primarily because as a group they do not normally join supernatural Syndicates -- even the King with Three Shadows counts few Ifrit under his dominion. 

An Ifrit can look like anything and they do. But in their truest form they seem to be hideous humanoids of approximately 2 meters with craggy skin of vivid primary and secondary colors. Their vibrant hues appear like something more at home in a crayon box than a living being.

In order to feed, an Ifrit must be within a meter of someone in wracking agony. They do not have to actually consume any blood or viscera and indeed they generally cannot because they are incorporeal anywhere but the Depths of Limbo. While Ifrit were probably never living mortals, their attributes are normal for their Potency.  It's not entirely clear if Ifrit ever \textit{were} anything before being magical beings, no one knows of any of them coming into being, and the lowest Potency of any Ifrit seems to be 4.

An Ifrit has an Infernal power source and a Feeding power schedule.

%%%
\subsubsection{Ifrit Starting Powers}
%%%

\hspace{\parindent} - Core Powers: Progress of Glass and Trail of Tears -\\
\linkpower{Distant Reflection} (Basic Progress of Glass)\\
\linkpower{Deny the Gauntlet} (Basic Progress of Glass)\\
\linkpower{Curse of Failure} (Basic Trail of Tears)\\
\linkpower{Pain Drops} (Basic Trail of Tears)

- Basic Powers -\\
\linkpower{Aura Perception} (Basic Discernment)\\
\linkpower{Patience of the Mountains} (Basic Fortitude)\\
\linkpower{Clinging} (Basic Clout)\\
\linkpower{Learn the Heart's Pain} (Basic Names of the Blasphemies)\\
\linkpower{Mesmerism} (Basic Authority)\\
\linkpower{Mask of a Thousand Faces} (Basic Veil)

- Advanced Powers -\\
\linkpower{Empty Body} (Discernment / Fortitude Devotion)\\
\linkpower{Telekinesis} (Discernment / Clout Devotion)\\
\linkpower{Mirror Pocket} (Advanced Progress of Glass)\\
\linkpower{Dark Night of the Soul} (Advanced Trail of Tears)\\
\linkpower{Conditioning} (Advanced Authority)

- Elder Powers -\\
\linkpower{The Smoking Mirror} (Elder Progress of Glass)\\
\linkpower{Object of Envy} (Elder Trail of Tears)

\paragraph{Story Inspiration:} Wishmaster

%%%%%%%%%%%%%%%%%%%%%%%%%%%%%%%%%%%%%%%%%%%%%%%%%%
\section{Goblins} \index{Goblins}
%%%%%%%%%%%%%%%%%%%%%%%%%%%%%%%%%%%%%%%%%%%%%%%%%%
\tagline{"I'll not rest till I have me gold. Curse this well that me soul shall dwell, till I find me magic that breaks me spell."}

Covetous and frightening, the Goblins of After Sundown are a lot more like the Svartalfs of bloodthirsty Norse myth than the pixies of a Disney cartoon. If they \textit{were} from a Disney cartoon, it would probably be the Night on Bald Mountain segment from Fantasia. The hideous goblins and gnarled ogres that make Limbo their home are stupid and vile.

All Goblins have an Infernal Power Source and suffer aggravated wounds from iron. Before humanity had iron, there were Goblin outposts on Earth: cruel and barbarous slaveholding affairs that sculpted the land to be more like Limbo. When iron came to human hands, the Goblins were wiped from the Earth. This historical footnote has been obscured, since of course the obvious parallels possible with other supernatural creatures is enough to give any Covenant Prelate pause.

%%%%%%%%%%%%%%%%%%%%%%%%%
\subsection{Mirror Goblins} \index{Goblin!Mirror Goblin}
%%%%%%%%%%%%%%%%%%%%%%%%%
\tagline{"Everything is a scam to you, isn't it?"\\
"Damn right it is."}

The Mirror Goblins are hideous and seemingly subhuman. They are small and misshapen, and they have hideous hooked teeth and claws. Mirror Goblins are a lot like the Black Isz from The Maxx or the Mumblers from Silent Hill. Left to their own devices, they mostly wander around Limbo gibbering and periodically eating each other. However, they are also oppressed by more powerful residents of the Dark Reflection. Whipped into shape by stronger creatures and groups they are used as disposable fodder and monsters of the week by The King with Three Shadows and other Infernal groups. Since they always have a Potency of zero, a Mirror Goblin can pass through any portal to Limbo, which is where they get their name: literally goblins who come into the Mortal World through mirrors.

While Mirror Goblins have strange pacmanesque mouths and rarely stand much over a meter tall, their muppetlike visages seem to have little difficulty being understood in human languages. When combined with their ability to magically disguise themselves, Mirror Goblins can actually penetrate human society with tolerable ease. They dare not stay long in the Mortal World though, because their power ritual can only be performed in Limbo. Mirror Goblins are generally regarded as being less than fully trustworthy.

The best media to look through to get a handle on Mirror Goblins is The Maxx, as they pretty much look and act like the Isz.

A Mirror Goblin was never human and has a nonstandard attribute array. While they are magical and can develop magical powers, they don't have a Potency rating. Their stats are:

S: 1/4 A: 2/7 I: 1/6 L: 1/5 W: 1/4 C: 1/6

A Mirror Goblin has an Infernal power source and a Ritual power schedule.

%%%
\subsubsection{Mirror Goblin Starting Powers}
%%%

\hspace{\parindent} - Basic Powers -\\
\linkpower{Quickness} (Basic Celerity)\\
\linkpower{Mask of a Thousand Faces} (Basic Veil)

%%%%%%%%%%%%%%%%%%%%%%%%%
\subsection{Spriggans} \index{Goblin!Spriggan}
%%%%%%%%%%%%%%%%%%%%%%%%%
\tagline{"Why there's nothing under this mask but a neck and some tendons!"}

Spriggans are hideous worm eaten things scarcely larger than a Mirror Goblin who can draw upon Infernal power to become massive killing machines. These beings look like hunks of maggot infested meat in an only barely humanoid shape, and they are quite boneless until they invoke their Giant Size and War Form, which are always enacted together. At that point their worms and flesh are pulled tight over a scaffold of long wet bones that end in sharp points in many places.

When a Spriggan performs its power ritual, it covers itself in spoiled food and soiled goods. The worms crawl out of it and grasp filth and offal to pull back into their cavernous interiors. Spriggans are quite susceptible to their Master Passion: Despair.

A Spriggan was never human and has a nonstandard attribute array. Their stats are:

S: 1/4 A: 1/6 I: 2/7 L: 1/5 W: 1/6 C: 1/6

A Spriggan has an Infernal power source and a Ritual power schedule.

%%%
\subsubsection{Spriggan Starting Powers}
%%%

\hspace{\parindent} - Basic Powers -\\
\linkpower{Clinging} (Basic Clout)\\
\linkpower{Nimble Feet} (Basic Celerity)\\
\linkpower{Small Witness} (Basic Swarm Song)

- Advanced Powers -\\
\linkpower{Giant Size} (Advanced Clout)\\
\linkpower{War Form} Celerity / Clout Devotion) 

%%%%%%%%%%%%%%%%%%%%%%%%%
\subsection{Trolls} \index{Goblin!Troll}
%%%%%%%%%%%%%%%%%%%%%%%%%
\tagline{"Skin\ldots{} Graaaaah\ldots{} Tasty\ldots{}"}

Within the prison world that is the Deep Reflection, hideous ogres prowl and punish or even murder those unlucky enough to fall into their clutches. But while they are the jailers of this foul realm, they are also its prisoners. Trolls spend almost every moment of their existence in agony and dejection, and eagerly take out their pains on others.

Hulking brutes with bulging musculature and an inhuman appearance, Trolls cannot actually turn their Giant Size \textit{off}. These tortured giants of Limbo appear in literature as Tartarians and Pyramid Head. While they \textit{do} eat people, they don't actually get anything for it except a meal. They rip the skin from their victims simply because they find enjoyment in doing so, not for the advancement of any mystical agenda.

Trolls have a fairly multivarious appearance, varying from merely oversized humans to lumpy stone skinned oni with tremendous tusks. These changes are generally speaking purely cosmetic. Trolls have no difficulty recognizing different Trolls as being the same as themselves.

Trolls have a nonstandard attribute array because they were never humans. Before their Potency modifier and their constant Giant Size, their attribute ranges are:

S: 5/10 A: 1/5 I: 1/5 L: 1/3 W: 3/8 C: 1/6

A Troll has an Infernal power source and a Continuous power schedule.

%%%
\subsubsection{Troll Starting Powers}
%%%

\hspace{\parindent} - Core Discipline: Clout and Fortitude -\\
\linkpower{Vigor} (Basic Clout)\\
\linkpower{Clinging} (Basic Clout)\\
\linkpower{Revive the Flesh} (Basic Fortitude)\\
\linkpower{Patience of the Mountains} (Basic Fortitude)

- Basic Powers -\\
\linkpower{Poison Heart} (Basic Names of the Blasphemies)\\
\linkpower{Repel} (Basic Magnetism)

- Advanced Powers -\\
\linkpower{Giant Size} (Advanced Clout)\\
\linkpower{Devastation} (Advanced Clout)\\
\linkpower{Restoration} (Advanced Fortitude)

\paragraph{Story Inspiration:} Silent Hill, Billy Goats Gruff, Jack and the Giant

%%%%%%%%%%%%%%%%%%%%%%%%%%%%%%%%%%%%%%%%%%%%%%%%%%
\section{Evil Plants} \index{Evil Plant}
%%%%%%%%%%%%%%%%%%%%%%%%%%%%%%%%%%%%%%%%%%%%%%%%%%
\tagline{"There's no sense in getting killed by a plant."}

The lands of Maya are filled with strange foliage. But strangest of all is the foliage that harbors a deep hatred for people and animals. Possessed of an intelligence that is so wholly unlike those of humanity"s that they are difficult to measure, these plants are condemned universally as being "Evil" even by the jaded standards of supernatural society. Completely emotionless, Evil Plants are not motivated by passions and cannot be induced to frenzy nor convinced against a course of action with emotional argument of any kind.

%%%%%%%%%%%%%%%%%%%%%%%%%
\subsection{Mantraps} \index{Evil Plant!Mantrap}
%%%%%%%%%%%%%%%%%%%%%%%%%
\tagline{"Feed Me!"}

Mantraps are essentially Aubrey the monster from Little Shop of Horrors or the carnivorous plants from Mario Brothers. They are able to move their giant mouth quite quickly, but they are rooted to the ground and cannot effectively give chase. Brutally effective ambush hunters, Mantraps try to eat animals whenever the opportunities arise.

Mantraps can't readily transport themselves across the land, let alone the borders between the dreamlands and the material world. Nevertheless, they do appear on Earth, because there are those who for whatever reason dig them up and transplant them. They often appear as guard plants in Marduk Society compounds. Mantraps can learn to speak, and their muppet like countenance creates alien voices.

The huge woody maw of a Mantrap is a damage 3 weapon. Mantraps have a nonstandard attribute array because they were never humans. Before their Potency modifier their attribute ranges are:

S: 3/8 A: 1/4 I: 1/6 L: 1/3 W: 2/7 C: 1/4

A Mantrap has an Astral power source and a Feeding power schedule.

%%%
\subsubsection{Mantrap Starting Powers}
%%%

\hspace{\parindent} - Core Discipline: Clout -\\
\linkpower{Vigor} (Basic Clout)\\
\linkpower{Clinging} (Basic Clout)

- Basic Powers -\\
\linkpower{Revive the Flesh} (Basic Fortitude)\\
\linkpower{Hide From Notice} (Basic Veil)



%%%%%%%%%%%%%%%%%%%%%%%%%
\subsection{Triffids} \index{Evil Plant!Triffid}
%%%%%%%%%%%%%%%%%%%%%%%%%
\tagline{"I dunno what the hell's in there, but it's weird and pissed off, whatever it is."}

The Triffids are mobile plant monsters that are only vaguely describable as humanoid in shape. Like other Evil Plants, they need meat in their diet, but unlike the others they can get up and walk at a normal speed. As such, Triffids are not relegated to ambush hunting, they can seriously chase their prey down and eat them. And they do.

A Triffid has a poison barb on the end of each meter long tendril that extends from their maw. Some of them sprout leaves and/or flowers from various parts of their body. Some of them are instead smooth like an ivy vine or wrinkled like a redwood tree. It's not entirely clear what the significance of that distinction is.

The vicious barbed tendrils of a Triffid constitute a damage 3 weapon that can deliver the poison of the Triffid's "bite". Triffids have a nonstandard attribute array because they were never humans. Before their Potency modifier their attribute ranges are:

S: 3/8 A: 1/4 I: 1/6 L: 1/3 W: 2/7 C: 1/4

A Triffid has an Astral power source and a Lunar power schedule.

%%%
\subsubsection{Triffid Starting Powers}
%%%

\hspace{\parindent} - Core Discipline: Coil of Thorns -\\
\linkpower{Bitter Fruit} (Basic Coil of Thorns)\\
\linkpower{Grass Rope} (Basic Coil of Thorns)

- Basic Powers -\\
\linkpower{Bite of the Serpent} (Basic Lure of Destruction)\\
\linkpower{Revive the Flesh} (Basic Fortitude)

- Advanced Powers -\\
\linkpower{Puppetry} (Advanced Coil of Thorns)

\paragraph{Story Inspiration:} The Thing, Night of the Triffids, Dryad, The Creeping Terror

%%%%%%%%%%%%%%%%%%%%%%%%%
\subsection{Pods} \index{Evil Plant!Pod}
%%%%%%%%%%%%%%%%%%%%%%%%%
\tagline{"There are others\ldots{} they'll stop you!"\\
"In an hour, you won't want them to."}

The Pods are psychic hazards that take over people and use them to make more pods and take more people over. It's a rather disturbing pyramid scheme, and it is not uncommon for even the Marduk Society and the Communes to work together in order to unravel Pod cults. It's not especially clear what it is that Pods \textit{want}, but whatever it is, they appear to have no intention of discussing it with any political structures from the material world.

Each pod is roughly the size of a large watermelon and appears to be a massive green potato. Most pods have some milk white tendrils growing out of what is most probably the bottom. These look similar to the rootlets that grow out of a potato if you leave it in water for a long period of time. There are slight depressions all over the Pod that appear to be its "eyes" -- they are able to see things through them and use Authority on victims who look at them. The tendrils of a Pod can only move by growing over the course of minutes and hours, so for any practical purposes they are essentially immobile. Pods have a nonstandard attribute array because they were never humans. Before their Potency modifier their attribute ranges are:

S: 1/4 A: 0/0 I: 2/7 L: 2/7 W: 2/7 C: 1/6

A Pod has an Astral power source and a Lunar power schedule.

%%%
\subsubsection{Pod Starting Powers}
%%%

\hspace{\parindent} - Core Discipline: Authority and Discernment -\\
\linkpower{Supernatural Senses} (Basic Discernment)\\
\linkpower{Aura Perception} (Basic Discernment)\\
\linkpower{Command} (Basic Authority)\\
\linkpower{Mesmerism} (Basic Authority)

- Basic Powers -\\
\linkpower{Enchanted Slumber} (Basic Veil of Morpheus)\\
\linkpower{Grass Rope} (Basic Coil of Thorns)

- Advanced Powers -\\
\linkpower{Telepathy} (Advanced Discernment)\\
\linkpower{Conditioning} (Advanced Authority)\\
\linkpower{Mind Root} (Advanced Coil of Thorns)

-Elder Powers -\\
\linkpower{Possession} (Elder Authority)

\paragraph{Story Inspiration:} Invasion of the Body Snatchers, Pod People

%%%%%%%%%%%%%%%%%%%%%%%%%%%%%%%%%%%%%%%%%%%%%%%%%%
\section{Giant Animals} \index{Giant Animal}
%%%%%%%%%%%%%%%%%%%%%%%%%%%%%%%%%%%%%%%%%%%%%%%%%%
\tagline{"No chains will ever hold that\ldots{}"}

The wilderness of horror movies is a forbidding place full of horrible things. And in the realm of horror, a fair number of those things are merely \textit{dangerous} and horrible rather than \textit{malicious} and horrible. Giant Animals are likely to fall into that category, because they are \textit{animals} and generally of animal intelligence. However, it is important to note that in After Sundown, as in too many horror books and movies to mention, animals are entirely capable of being decidedly, premeditatedly, \textit{evil}. Animals in After Sundown really tend to be amazing jerks. So when animals are given great strength or amazing powers, they usually go on rampages pretty near to first thing. 

The wilderness of Maya gives birth to nightmarish and titanic beasts beyond number. And the magic of that untamed land bleeds into the material world in unpredictable ways. Where this happens, monstrosities are created and normal beasts are transformed into monstrosities. The wilderness in the After Sundowncan be a truly horrible place full of horrible things. And the horrible things it is full of are collectively known as the Giant Animals.

Giant Animals are vulnerable to silver.

%%%%%%%%%%%%%%%%%%%%%%%%%
\subsection{Behemoths} \index{Giant Animal!Behemoth}
%%%%%%%%%%%%%%%%%%%%%%%%%
\tagline{"Let's not overlook the fact that he didn't eat me."}

Magically enhanced monstrous beasts prowl the wilds of the astral plane, and sometimes their rampages take them through holes in reality to invade the mortal world. In areas unseen by man, animals will sometimes spontaneously grow to magically augmented size and then rampage from there. Behemoths look pretty much like mortal animals save that they are substantially larger than their natural kin and \textit{substantially} more aggressive and dangerous to humans.

A Behemoth's Giant Size is always on, and it is important to note that sometimes they are of a type of creature which is itself \textit{normally} 3.5 meters or more such as giant anacondas and the like. Giant Size in this case represents a proportional increase for such creatures, meaning a rough doubling in all dimensions and an increase in mass by approximately 10 times. A Behemoth was never human, but it is a real creature, and before considering its Giant Size and Potency, it uses the normal stats of an animal of its type.

A Behemoth has an Astral power source and a Continuous power schedule.

%%%
\subsubsection{Behemoth Starting Powers}
%%%

\hspace{\parindent} - Basic Powers -\\
\linkpower{Touch of Darkness} (Basic Lure of Destruction)\\
\linkpower{Vigor} (Basic Clout)

- Advanced Powers -\\
\linkpower{Giant Size} (Advanced Clout)

\paragraph{Story Inspiration:} Mighty Joe Young, Lake Placid, Boa vs. Python, Le Pacte de Loups, Them

%%%%%%%%%%%%%%%%%%%%%%%%%
\subsection{Swarms} \index{Giant Animal!Swarm}
%%%%%%%%%%%%%%%%%%%%%%%%%
\tagline{"Do you happen to have a pair of birds that are\ldots{} just friendly?"}

Whether it's the birds in \refwork{The Birds} or a horde of rats in \refwork{Willard}, life in horror has a propensity for having large groups of individually innocuous beasts get together in a big group and go on a rampage. It may seem like they can't possibly be fought with weapons, since there's about a fantastillion of them in each swarm. But the reality is that the Swarm itself is completely meaningless. They "are" an intangible energy field that whips mundane beasts into a frenzy of violence. As such, one simply has to hit the intangible force between the controlled beasts with silver until it stops moving.

Unfortunately, Swarms have a tendency to respawn within a few days unless something drastic is done.

A Swarm has an Astral power source and a Lunar power schedule.

%%%
\subsubsection{Swarm Starting Powers}
%%%

\hspace{\parindent} - Basic Powers -\\
\linkpower{Revive the Flesh} (Basic Fortitude)\\
\linkpower{Supernatural Senses} (Basic Discernment)\\
\linkpower{Tongue of Beasts} (Basic Call of the Wild)\\
\linkpower{Hide From Notice} (Basic Veil)

- Advanced Powers -\\
\linkpower{Empty Body} (Discernment / Fortitude Devotion)\\
\linkpower{Restoration} (Advanced Fortitude)\\
\linkpower{The Beckoning} (Advanced Call of the Wild)

\paragraph{Story Inspiration:} The Birds, The Swarm, Willard

%%%%%%%%%%%%%%%%%%%%%%%%%
\subsection{Chimera} \index{Giant Animal!Chimera}
%%%%%%%%%%%%%%%%%%%%%%%%%
\tagline{"That is simply unnatural."}

In the heart of Maya there are animals that don't really look like Earthly animals at all. Sometimes they look like two or more creatures melded together awkwardly, and sometimes they just look like something drawn up for \refwork{Heavy Metal}. These are the Chimera, and some of them look like mythic beasts such as unicorns and basilisks. And it is important to note that they actually \textit{behave} like those legendary monsters do \textit{in horror movies}. Which means that mostly what they do is wander around and use their powers to murder people. Chimeras are about the size of a horse and are not cuddly at all.

A Chimera has the mind of an animal, and the physique of a horrible man-eating monster. Since they aren't even reasonable facsimiles of humans, they not only have a non-standard attribute array but indeed fairly fixed attributes and skills. They \textit{do} have a Potency rating, and sometimes these values increase. Whether they have a spiral horn or a scorpion's tail, a Chimera has natural weaponry that inflict 3 Lethal damage.

S: 8 A: 3 I: 3 L: 1 W: 5 C: 1\\
\textbf{Skills:} Artisan 1; Athletics 5; Combat 3; Perception 2; Rigging 3 (Water); Stealth 2; Survival 4

A Chimera has an Astral power source and a Continuous power schedule.

%%%
\subsubsection{Chimera Starting Powers}
%%%

\hspace{\parindent} - Core Discipline: Trail of Tears -\\
\linkpower{Curse of Failure} (Trail of Tears)\\
\linkpower{Pain Drops} (Trail of Tears)

- Basic Powers -\\
\linkpower{Light of Ennui} (Basic Descent of Entropy)\\
\linkpower{Revive the Flesh} (Basic Fortitude)\\
\linkpower{Hide From Notice} (Basic Veil)\\
\linkpower{Tongue of Beasts} (Basic Call of the Wild)

- Advanced Powers -\\
\linkpower{Water Prison} (Advanced Trail of Tears)\\
\linkpower{Aura of Decay} (Advanced Descent of Entropy)\\
\linkpower{Hide in Plain Sight} (Advanced Veil)

\paragraph{Story Inspiration:} Basilisk, Orangopoid, Xenomorph, Spidron

%%%%%%%%%%%%%%%%%%%%%%%%%
\subsection{Kaiju} \index{Giant Animal!Kaiju}
%%%%%%%%%%%%%%%%%%%%%%%%%
\tagline{"What the hell is that?"\\
"We need bigger guns."}

There are things in the wilderness of horror that defy ready analysis. Some are really, \textit{really} big. And they stomp out of nightmares and crush everything beneath their feet. They are the Kaiju, and they have no place in the modern world. Standing at something over 15 meters tall, these prehistoric beasts are awe inspiringly terrifying. It's not even \textit{entirely} clear that these titanic monstrosities are even magical in nature, and indeed most of them don't seem to use any magical powers save for being really big.

A Kaiju is not a normal animal that uses magic to grow really big, it's just a magically spawned entity that is much larger than a land-bound mortal animal could possibly be without suffering from cube square law insufficiencies. They aren't using Clout effects, that's \textit{just how big they are}. Whether they are giant lizards or giant apes, their stats are pretty much the same. And yes, some of them are also magical and have magical disciplines like Fire Starter that allows them to breathe gouts of flame. But mostly they just rely on their size. They don't even have something that counts as special natural weaponry, their normal attack is a base zero normal damage fist -- and they are so big that attacking a human-sized target suffers a -5 penalty to the attack roll -- and they \textit{still} normally flatten things in one pound -- because that's how a creature with a 35 strength \textit{rolls}.

Kaiju can work as "boss monsters" in that they are so out of scale with even large and in-charge characters that finding something that can hurt them \textit{at all} can be an adventure in itself.

S: 35 A: 2 I: 4 L: 1 W: 6 C: 1\\
\textbf{Skills:} Animal Ken 3; Athletics 4; Combat 5; Perception 4; Stealth 2; Survival 4

\paragraph{Story Inspiration:} King Kong, T. Rex.

%%%%%%%%%%%%%%%%%%%%%%%%%%%%%%%%%%%%%%%%%%%%%%%%%%
\section{Ghosts} \index{Ghost}
%%%%%%%%%%%%%%%%%%%%%%%%%%%%%%%%%%%%%%%%%%%%%%%%%%
\tagline{"What we have here is what we call a non-repeating phantasm."}

The most numerous of supernatural creatures is the Ghost. One of them could be created every time someone dies, there are over seven billion people, and the number of deaths is one per person.

Every Ghost has tremendous difficulty interacting with the material world. They "live" in the Deep Gloom and return there whenever the sun rises even if they find a way to escape. When they \textit{do} manage to leave the Gloom, their Empty Body turns on and they are unable to touch any physical object that is not made of wood. 

Because of their extreme difficulty in interacting with things, Ghosts often make rather limited actors in a story, and are thus ill suited to be main characters.

%%%%%%%%%%%%%%%%%%%%%%%%%
\subsection{Wisps} \index{Ghost!Wisp}
%%%%%%%%%%%%%%%%%%%%%%%%%
\tagline{"Oooh\ldots{}"}

Wisps are little shadows or colored glows that flit about and scarcely remember that they used to be humans. Individual Wisps are everywhere, and usually fairly inconsequential. Some Wisps make their way to the material world and scare people or try to interact with the world as they remember it being -- it's a pretty frustrating experience because they can't actually touch anything.

Wisps primarily interact with an After Sundown chronicle by being utilized by necromancers as spies. Even without being compelled into service, Wisps are often quite willing to be helpful to physical beings that can interact with them. Wisps have a Strength score of 1, regardless of what they had in life.

A Wisp has an Orphic power source and a Lunar power schedule.

%%%
\subsubsection{Wisp Starting Powers}
%%%

\hspace{\parindent} - Basic Powers -\\
\linkpower{Aura Perception} (Basic Discernment)\\
\linkpower{Patience of the Mountains} (Basic Fortitude)

- Advanced Powers -\\
\linkpower{Empty Body} (Discernment / Fortitude Devotion)

\paragraph{Story Inspiration:} All the spooky background ghostly stuff in movies like Sleepy Hollow.

%%%%%%%%%%%%%%%%%%%%%%%%%
\subsection{Wraiths} \index{Ghost!Wraith}
%%%%%%%%%%%%%%%%%%%%%%%%%
\tagline{"You know, the best way to get rid of ghosts is to clean house."}

Wraiths appear as they did in life save with photographic filters applied to make them look washed out and see-through. Each Wraith is a specific person that died, and they tend to have goals that tie into their time while alive or the circumstances of their death. A Wraith that accomplishes its driving goal often dissolves into a Nirvanaish happy ending. But sometimes a Luminary Wraith will simply go on and make an unlife for themselves in supernatural society.

Like all Ghosts, a Wraith is only corporeal while within the Gloom, and can only be in the mortal world until the sun rises before being banished back to Mictlan. Each Wraith has a number of things that tie them to the real world called "fetters". These can be favored objects, their own corpse, or loved ones. In order to regain Power Points they have to be within a few meters of one of their fetters and interact with it in some obsessive way. If a Wraith loses track of their fetters, they cannot regain Power Points and are doomed to slowly fade away without them. If a Wraith attains its primary goal and persists afterwards, they can stop obsessing over the past and make a new life with new fetters.

Some Luminary Wraiths come into being with a lot of extra disciplines, putting them on close to the same footing as a normal player character. Some people might even be tempted to run Wraith chronicles, but that's really very difficult role playing.

A Wraith has an Orphic power source and a Ritual power schedule.

%%%
\subsubsection{Wraith Starting Powers}
%%%

\hspace{\parindent} - Basic Powers -\\
\linkpower{Aura Perception} (Basic Discernment)\\
\linkpower{Patience of the Mountains} (Basic Fortitude)\\
\linkpower{Vigor} (Basic Clout)

- Advanced Powers -\\
\linkpower{Empty Body} (Discernment / Fortitude Devotion)

\paragraph{Story Inspiration:} Ghost, Stir of Echoes, Sixth Sense, most ghost stories honestly.

%%%%%%%%%%%%%%%%%%%%%%%%%
\subsection{Poltergeists} \index{Ghost!Poltergeist}
%%%%%%%%%%%%%%%%%%%%%%%%%
\tagline{"They're here\ldots{}"}

Poltergeists are accumulations of untold numbers of wisps that have lost their individuality and become spiritual storms. Ruled by rage and quite destructive, Poltergeists can't even speak save to wail and scream. Sometimes they are tied to a location, sometimes they travel around Mictlan like thunderheads. In either case, they usually have some relation to a specific atrocity or another, and will calm down and disperse if someone can figure out how to appease them despite their inarticulate destruction.

Like all Ghosts, a Poltergeist has an Empty Body unless they are in Mictlan. They also have a Charisma and Logic of zero. They have no specific limits for other attributes, larger storms have more Strength and more Willpower\ldots{}

A Poltergeist has an Orphic power source and a Lunar power schedule.

%%%
\subsubsection{Poltergeist Starting Powers}
%%%

\hspace{\parindent} - Basic Powers -\\
\linkpower{Aura Perception} (Basic Discernment)\\
\linkpower{Patience of the Mountains} (Basic Fortitude)\\
\linkpower{Vigor} (Basic Clout)

- Advanced Powers -\\
\linkpower{Empty Body} (Discernment / Fortitude Devotion)\\
\linkpower{Telekinesis} (Discernment / Clout Devotion)\\
\linkpower{Devastation} (Advanced Clout)

\paragraph{Story Inspiration:} Poltergeist, It

%%%%%%%%%%%%%%%%%%%%%%%%%%%%%%%%%%%%%%%%%%%%%%%%%%
\section{Zombies} \index{Zombie}
%%%%%%%%%%%%%%%%%%%%%%%%%%%%%%%%%%%%%%%%%%%%%%%%%%
\tagline{"It is a truth universally acknowledged that a zombie in possession of brains must be in want of more brains."}

Zombies are physical bodies without the benefit of life or human spirit who nonetheless move about and hunger for the brains of the living. They are not respected members of the supernatural community, but are instead treated as servitors and disposable soldiers by those with necromantic powers and a social problem by most everyone else. Zombies are not traditionally considered playable, but are instead most likely to appear as tools or independent menaces in an After Sundown campaign.

All Zombies, regardless of strength have an Orphic power source. When an extra becomes a Zombie, they become a Shambler or a Soulless depending upon the circumstances of their transformation. When a Luminary becomes a Zombie, they become a Revenant in all cases. Unlike many unplayable types, a Zombie \textit{is} made out of a human being and they \textit{are} templated onto a normal human statline.

%%%%%%%%%%%%%%%%%%%%%%%%%
\subsection{Shamblers} \index{Zombie!Shambler}
%%%%%%%%%%%%%%%%%%%%%%%%%
\tagline{"When the dead rise, civilization will fall."}

Shamblers are the classic "slow zombies" from zombie movies from the eighties. They hunger for brains, but they are basically walking corpses who shamble around -- hence the name. Shamblers are \textit{not} individually particularly dangerous. But they \textit{are} implacable and they can come in fairly large numbers. Created by Orphic sorcery or by leakage of power from Mictlan, Shamblers will thoughtlessly move toward humans and attempt to eat their brains. The opening up of a major Well to Mictlan is often accompanied by the mass animation of large numbers of corpses as Shamblers, leading to potentially terrifying armies of the things even in the face of the relative incompetence of any solitary Shambler.

A Shambler has no Charisma or Logic score and automatically fails any test it would be called upon to make. Upon creation, a Shambler's Agility is reduced by one (to a minimum of 1), and their Strength is increased by 1. A Shambler loses all of their skills, even combat skills. When not controlled magically, they really will simply walk over and relentlessly and unskillfully claw and bite at potential victims with their substantial but undirected strength. Shamblers are not specifically slower than a normal person, but they \textit{always} move at the rate of a Careful Walk even when ordered to do otherwise. Only their literally endless endurance and willingness to travel unceasingly to find brains gives their aggregate daily travels a frightening total of up to 50 kilometers a day. 

Shamblers do not actually spread zombification, but they are often accompanied by evil magic that will reanimate all the corpses in the area, which will include their victims. The skin of a Shambler is quite resilient and hard, and their fists are Damage 1N weapons. Shamblers have no passions.

A Shambler has an Orphic power source but no power schedule, Potency, or Power attribute.

%%%
\subsubsection{Shambler Starting Powers}
%%%

\hspace{\parindent} - Basic Powers -\\
\linkpower{Patience of the Mountains} (Basic Fortitude)

- Advanced Powers -\\
\linkpower{Indominability} (Advanced Fortitude)


%%%%%%%%%%%%%%%%%%%%%%%%%
\subsection{Soulless} \index{Zombie!Soulless}
%%%%%%%%%%%%%%%%%%%%%%%%%
\tagline{"If you look at the whole life of the planet, we\ldots{} you know, man, has only been around for a few blinks of an eye. So if the infection wipes us all out, that is a return to normality."}

The Soulless are the lately fashionable "fast zombies" of more modern cinema such as 28 Days Later and the Dawn of the Dead remake. Clearly distinct from Shamblers by their bright red eyes and relatively speedy disposition, the Soulless are neither plodding nor tireless. Possessed of more humanlike speeds and faculties, the Soulless are individually much more terrifying than a Shambler. But while they are rarely spawned in the tremendous numbers of Shamblers, the fact that they can and do spread their affliction readily to the living means that their numbers can easily grow out of control if not checked by heroic intervention.

A Soulless loses their Charisma but not their Logic. Their Strength is increased by 1, but their Agility is unaffected. A Soulless is consumed by rage at all times to the point of complete irrationality, but they are not actually incapable of utilizing tools. While a Soulless will \textit{usually} opt to empower Vigor to tear down a door rather than attempting to turn the knob, this is a result of their all consuming hatred rather than actual ineptitude on their part. The bite or even \textit{spit} of the Soulless can corrupt and kill mortals, and the venom they are equipped with from their serpent's tongue ability is always a fatal one. The Soulless fighting style is simple and predictable, but usually effective enough. All Soulless have a Combat skill of 1.

Soulless, like Shamblers, have no special power to transform others into Zombies of any kind, but are often carriers of the Rage Virus which does. If they are sterilized by any method, any victims they take down will stay inert. Every Soulless is dominated by Master Passion Rage and will fly into a Rage Frenzy with basically no provocation at all. Triggers include \textit{seeing humans}.

A Soulless has an Orphic power source and a Lunar power schedule.

%%%
\subsubsection{Soulless Starting Powers}
%%%

\hspace{\parindent} - Basic Powers -\\
\linkpower{Vigor} (Basic Clout)\\
\linkpower{Nimble Feet} (Basic Celerity)\\
\linkpower{Bite of the Serpent} (Basic Lure of Destruction)

%%%%%%%%%%%%%%%%%%%%%%%%%
\subsection{Revenants} \index{Zombie!Revenant}
%%%%%%%%%%%%%%%%%%%%%%%%%
\tagline{"The Living Dead and the dying living are all the same. Cut from the same cloth. But disposing of dead people is a public service, whereas you're in all sorts of trouble if you kill someone while they're still alive."}

The Revenant is the talkative wight from virtually every piece of fiction where a Zombie is a major character. Whether the lovely and sensual She from \refwork{Cemetery Man}, the creepy Shelly Winters from \refwork{Scary Go Round} or the villainous Dark Ash and Sheila from \refwork{Army of Darkness}, \textit{every} Luminary who becomes a Zombie by \textit{whatever} means becomes a Revenant. Revenancy is curable, as can be plainly seen from both \refwork{Scary Go Round} and \refwork{Army of Darkness}, and it is the only form of Zombieness that is. A Revenant appears much as it did in life, save for pale white skin and darkness around the eyes. Basically they look like someone who is wearing heavy goth makeup, save that there is something \textit{obviously} unnatural about them that even the most casual observer can plainly see.

Revenants do not necessarily lose their reasoning faculties nor their personal moral compass. However, they \textit{do} hunger for the brains of the living, and are doomed to gradually weaken and lose power until they devour such. Eating the brain of a human refreshes their Power batteries, but they \textit{don't} have an unobtrusive or Vow of Silence upholding alternative at their disposal. Sooner or later, they \textit{will} be compelled to break open a human skull and feast on the morsels inside. It is for this reason that even the usually quite open minded supernatural societies generally want Revenants cured or destroyed -- their mere presence endangers the kindred more than most Syndicates are willing to condone.

A Revenant has their Strength, Intuition, and Willpower all increased by 1. Every Revenant is subject to Master Passion Hunger. Every Revenant carries the Z-Virus with their Abyss of the Body.

A Revenant has an Orphic power source and a Feeding power schedule.

%%%
\subsubsection{Revenant Starting Powers}
%%%

\hspace{\parindent} - Core Discipline: Fortitude -\\
\linkpower{Patience of the Mountains} (Basic Fortitude)\\
\linkpower{Revive the Flesh} (Basic Fortitude)

- Basic Powers -\\
\linkpower{Compel Spirits} (Basic Necromancy)\\
\linkpower{Nimble Feet} (Basic Celerity)\\
\linkpower{Abyss of the Body} (Basic Descent of Entropy)\\
\linkpower{Supernatural Senses} (Basic Discernment)\\
\linkpower{Vigor} (Basic Clout)

- Advanced Powers -\\
\linkpower{Indominability} (Advanced Fortitude)

%%%%%%%%%%%%%%%%%%%%%%%%%%%%%%%%%%%%%%%%%%%%%%%%%%
\section{Spawn Monsters} \index{Spawn}
%%%%%%%%%%%%%%%%%%%%%%%%%%%%%%%%%%%%%%%%%%%%%%%%%%
\tagline{That is a tragedy. You should probably kill it.}

Supernaturals who aren't Luminaries are sad, monstrous things that have short life expectancies once the title sequence rolls. Not only do they have very limited supernatural powers but they are dominated by their Master Passion even more strongly than the true creatures they emulate. They suffer a -2 dicepool penalty when resisting or attempting to end a Frenzy\index{Frenzy}. This penalty \textit{rises} over time. Spawn who are weak of will are simply in frenzy \textit{all the time}. As such, Spawn monsters don't have much success in society, and are frequently simply kept locked up to be released on main characters like guard dogs or bees (or twisted combinations of the two).

%%%%%%%%%%%%%%%%%%%%%%%%%
\subsection{Vampire Spawn: Thralls} \index{Spawn!Thrall}
%%%%%%%%%%%%%%%%%%%%%%%%%
\tagline{"Back, I tell you all! This man belongs to me!"}

Vampire spawn are monstrous and depraved, hungering for the blood of mortals and generally making a nuisance of themselves. All Vampire Spawn have Master Passion Hunger and like all spawn they have difficulty controlling it. Vampire Spawn have only \linkpower{Patience of the Mountains} and \linkpower{Revive the Flesh} for their disciplines. There is little tangible difference between a Nosferatu Spawn and a Daeva Spawn. Either could stand in for the nameless mook vampires that can be killed even by Xander in a Buffy episode. They still get the Feeding power schedule, but they have a Potency of zero that does not rise. Even hundreds of years later, taking down Dracula's clutch of bride thralls is not that much of a trick.

%%%%%%%%%%%%%%%%%%%%%%%%%
\subsection{Animate Spawn: Bots} \index{Spawn!Bot}
%%%%%%%%%%%%%%%%%%%%%%%%%
\tagline{"Five is alive."}

Not all attempts to create life are truly successful. Without the spark of individuality, an animate machine is simply a machine that moves. Not necessarily a machine that thinks or \textit{feels}. Depending upon its original purpose and the skill of construction, a non-luminary Animate has the stats of a Shambler or a Soulless. The only thing that really changes is the power source. When you open up a door and find a marching army of terracotta warriors, those have an Astral power source because they are technically Golem Spawn. Killer robots on the other hand, have an Infernal power source because they are Android Spawn. But in either case, the war thralls just use the stats of a Soulless. Golem and Android Spawn who are Soulless knock-offs have \linkpower{Patience of the Mountains}, \linkpower{Nimble Feet}, and \linkpower{Vigor} (effectively replacing \linkpower{Bite of the Serpent} with Patience of the Mountains). Frankensteins without the spark of a main character are \textit{literally} just a Shambler or a Soulless and don't replace any disciplines.

%%%%%%%%%%%%%%%%%%%%%%%%%
\subsection{Witch Spawn: Cultists} \index{Spawn!Cultist}
%%%%%%%%%%%%%%%%%%%%%%%%%
\tagline{"Uhluhtc! Uhluhtc!"}

Magic doesn't really work out very well for most people. Those who don't have the force of destiny and personality that comes with being a main character are frequently consumed by the power they attempt to wield. Becoming a cultist requires delving into the dark secrets and learning some magic. Maybe the Extra read a magic tome, maybe they got training from an actual spellcaster. Whatever the origin, they now have a basic Sorcery power or basic Authority power, and they start going completely insane. They are the proud recipient of Master Passion Greed, and they suffer the gradually increasing penalty to resisting its lure that afflicts spawn in general. Cultists do not inherently know how to get power points which, depending upon what magic they learn, may not even matter. Cultists have a Potency of zero.

%%%%%%%%%%%%%%%%%%%%%%%%%
\subsection{Leviathan Spawn: Mutants} \index{Spawn!Mutant}
%%%%%%%%%%%%%%%%%%%%%%%%%
\tagline{"It's breakfast time!"}

The taint of Tiamat is a curse more than a blessing to most who are affected at all. While luminaries turn into badasses like Robert Olmstead, Extras with the blood appear to be normal humans \textit{if they are lucky}. The less lucky ones become degenerate mutants who look all gross and want to eat people. They have a potency of zero and both \linkpower{Patience of the Mountains} and \linkpower{Supernatural Senses}. Unlike the paranoid true Leviathan, most Mutants end up with Master Passion Hunger or Master Passion Rage. This is also what happens to extras who eat the flesh of Leviathans. They become Leviathan spawn, gradually mutating into hideous monstrosities, and they are driven to eat people (especially other Leviathans). Some Leviathan communities keep these mutant throwbacks alive in locked up basements or attics.

%%%%%%%%%%%%%%%%%%%%%%%%%
\subsection{Transhuman Spawn: The Lost and the Damned}
%%%%%%%%%%%%%%%%%%%%%%%%%
\tagline{I don't believe in fate or destiny. I believe in various degrees of hatred, paranoia, and abandonment.}

Transhuman Spawn in many cases are not \textit{exactly} the same thing as regular Transhumans who are extras (and therefore doomed). But the concept is similar. A human Extra who is abandoned in the Dreamlands when their body dies becomes a Jalus\index{Spawn!Jalus}: what is effectively an Icarid Spawn. They live out the rest of their lives in Maya, surviving day to day in that ghastly jungle mostly by dint of no one noticing them. Indeed, they have \linkpower{Hide From Notice}. A human Extra who is adopted by spirits that aren't their own can be turned into a Seer\index{Spawn!Seer}. This is basically like being a Reborn Spawn, and while they get \linkpower{Summon Spirit}, they \textit{mostly} get to be picked on by ghosts and go crazy. However, breaking the pattern, a human Extra who is imprisoned for an extended period in the Dark Reflection or who draws upon the power of an Infernal artifact gradually becomes a Fallen Spawn. They gain access to \linkpower{Learn the Heart's Pain} and make up the majority of the human inhabitants of Limbo. In all cases, Transhuman Spawn are afflicted with Master Passion Despair, and for all the good it does them, they have the same power schedules as their associated Transhuman type.

%%%%%%%%%%%%%%%%%%%%%%%%%
\subsection{Lycanthrope Spawn: The Spawn that Die}
%%%%%%%%%%%%%%%%%%%%%%%%%
\tagline{"Some experiences are literally once in a life-time."}

You don't see armies of Werewolves in After Sundown because Extras who undergo the transformation procedures \textit{die}. And then they stay dead, and are dead. Sometimes they become ghosts, and wander around haunting the lycanthropes that killed them \refwork{American Werewolf in London} style.