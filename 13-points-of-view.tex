%%%%%%%%%%%%%%%%%%%%%%%%%%%%%%%%%%%%%%%%%%%%%%%%%%
%%%%%%%%%%%%%%%%%%%%%%%%%%%%%%%%%%%%%%%%%%%%%%%%%%
\chapter{Points of View}
%%%%%%%%%%%%%%%%%%%%%%%%%%%%%%%%%%%%%%%%%%%%%%%%%%
%%%%%%%%%%%%%%%%%%%%%%%%%%%%%%%%%%%%%%%%%%%%%%%%%%

%%%%%%%%%%%%%%%%%%%%%%%%%%%%%%%%%%%%%%%%%%%%%%%%%%
\section[First Person]{First Person: Summarizing Proust}
%%%%%%%%%%%%%%%%%%%%%%%%%%%%%%%%%%%%%%%%%%%%%%%%%%

\hspace{\parindent} The last thing anyone wants to be told is that their capabilities have diminished. Even if it's true. \textit{Especially} if it's true. That's not to say that people don't want to hear bad things about the present, or good things about the past -- people eat that shit \textit{right up}. Nostalgia is a powerful thing, and you can get a lot of drunken agreement by talking about how great the Reagan years were. The Reagan years! Back then we all thought we were going to die in a radioactive hellstorm, and unemployment was over ten percent, but no one \textit{remembers} that shit anymore. No one remembers Paul McCartney subjecting us to \textit{Take it Away} or John Mellencamp calling himself "Cougar" either. Every day, the past gets a little bit brighter, because we forget our failings. And we are left with nothing but \textit{Physical} and \textit{Eye of the Tiger}, and we wonder how kids these days can listen to such crap. But while everyone wants to hear about how \textit{those days} were better than \textit{these days}, it becomes downright insulting to mention the fact that possibly an actual person has lost their edge. At least if that person is within earshot. And yet, that is what happens to us. To all of them anyway. It doesn't to us any more, because we are still in 1982. To an extent we are there all the time, but especially now with Joan Jett playing -- on an \textit{oldies} station of all things, we are brought quite unwillingly to reminisce.

Back then I was in high school and I began noticing boys and hearing the voices. Because of my upbringing, I naturally thought that the voices represented goodness and my burgeoning sexuality was wickedness. This illusion was shattered for me when the voices crawled out of me -- revealing them to be not the whispers of a loving god but the humming of dreadful wasps. Panic was of course the first thing that filled me, and I lashed out with my World Civ textbook against the demonic creatures burrowing from beneath my skin. It burned like lightning, an analogy that I am able to stand by given my later encounters with the subject. But it was worse than that, because in striking them flat I was striking myself -- in a very tangible and literal fashion as I later came to understand. With each flattened bug I found parts of myself, my memory, my \textit{time} were gone. Perhaps this is what normal people go through when they simply forget things, when they allow the janitor of nostalgia to clean away the dull and the embarrassing -- leaving nothing but fondness for a time that cannot be recovered. But for someone recently a child, the sudden realization that history could be taken and lost was soul shattering.

Making peace with the bugs was not easy for me, but since they really were \textit{us}, doing so was inevitable. My abilities to open doors and carry books were valued, as were their own abilities to fly around and hear things. And in time we grew close, we became the same. It was precisely our ability to hear things that allowed us to come to terms with ourselves, to see that the desires we felt guilt for were quite normal. Not accepted, but certainly common. And yet, while we certainly accepted ourselves, we felt ostracized. We could hear peoples' \textit{thoughts}, and we could see easily the hypocrisy of those around us -- that even if we came to people who had thoughts like ours that we would still be condemned as a pervert, as a wasp-filled monster. It was a lonely time for us. We watched the boys playing together without us, and there was no place for us there.

But we weren't alone. Not forever. We met Robert, who at that time was scouring through a bunch of used LPs at a record store we frequented. It was immediately obvious that he was special, both because we could not hear the buzzing of his thoughts and also because he recognized us for being a hive of sorts immediately. He quickly abandoned his work and inquired as to the "Syndicate" that \textit{I} belonged to -- even though he could see the conflict and chorus within our body, he talked of us as a singular entity -- not out of ignorance but out of a friendship and respect that had never been offered us in sincerity before. And yet, we had no choice but to offer our own ignorance in return -- for we knew nothing of the world Robert was part of save that we could infer from our own existence. Robert took us to get some wine and pizza, we were underage, but Robert used his own powers to make sure that we were not carded. He told us about the Cauchemar Communes, and his reasons for coming to Iowa in the first place. Robert was a soldier of sorts, and he was here to fight a battle on behalf of the Communes. Not because he had been paid, not because he was threatened, but simply because he had been convinced that it was a good idea. We admired, \textit{still} admire Robert for his dedication to his ideals. That he would put himself in danger for things that he believed in, and for no other reason was incredible to us. That these Cauchemar would inspire that kind of loyalty while \textit{demanding} so little attracted me immediately. So did Robert of course, but alas it turned out that the "I" in us was the wrong sex for him. But this did not stop us from becoming friends.

When he was finished with his mission, we followed him back to Paris, to join up with society in the Syndicate. It was both more and less than we had hoped for. We had imagined firstly that in coming to the heart of the Communes that there would be an endless supply of creatures as strange to the human eye as we were, that we would be strolling through an endless cantina scene where we would be not a tenth part the strangest, where we could casually vanish into the shared camaraderie of inhuman origins. But it turns out that even there, the number of creatures strange and profane seemed scarcely enough to fill a small town. There were certainly places to go and parties to attend, and for the first time we felt we could be ourselves -- but even so we could not do so without being noticed. Worse still, there were many who felt that our type of creature -- which we now call "Mi Go" -- was the strangest of all. And there were only a hundred bodies filled with Mi Go. Though nothing could have prepared us for the majesty of the Dreamlands or the profanity of the Dark Reflection. 

There is a certain ennui that comes from having absolutely nothing demanded of one's self. And so it was that we must confess that the first impressions we made in Parisian society were of something resembling a dissipate. We viewed the art made by others, we tried our hand at flirting, and unfortunately produced nothing of value from that period. It wasn't until much later that we were convinced of our own ability to produce things of lasting or artistic value by relating our own matured reflections on past experiences. For now we hoped to find strength and purpose in others -- as Robert had done by joining the Vanguards and fighting the enemies of our way of life. But here, we hoped merely to make that way of life ours, and hoped that personal relationships with other Cauchemar would make for us a reason to take a stand. On anything.

And so we took up with Dor\'{e} Guermantes, who was a well liked vampire of considerable personal beauty. We looked to him for strength because he could lift and throw a car, he looked to us for strength because in having dollars we appeared to have a boundless source of financial independence. This was before the Euro, so perhaps our errors were understandable. We fell deeply in love with him, but his interest in us turned out to be a mere flighting fancy. Vampires it seems, cannot help trying to leech off of others -- even when those others want nothing but to share everything with them. We drifted apart, and we tried for a time to cultivate physical relationships. But it turns out that when you have little to offer but the warmth of your embrace, that there was little temptation for others to stay true when your embrace is absent. And the dating pool amongst supernatural creatures is\ldots{} sadly not large. We were particularly enamored of an orphan android named Albert, but he would have nothing to do with us romantically because of our basic flippant worthlessness.

Well later it was that I discovered the truth about the Communes. That it is hard indeed to live when there is nothing whatever that is demanded of you. It means that no one will find a place for you, and you must create a niche for yourself. Sometimes you might get lucky and be on hand when someone has an idea that needs extra bodies, and then your purpose is -- for a time -- given. But the reality is that when you follow a demagogue, you aren't \textit{really} offering any more than your warm embrace. If someone stronger or more talented, or even simply \textit{more proximate} comes while you are away, the project will be performed by others. But in any case, your contributions won't be remembered, it will forever be Bergotte's project, and never yours. And so it is that as we sit here in Iowa for perhaps the last time, we have determined to make something of ourselves. Which is why we are writing now. The art we create gives us purpose, provides us a place in society.

Once we have buried our grandmother, there will be nothing left to us. We will return to Paris. We will make writings for others in the Movement. And this will give us meaning. And possibly, just possibly, to make Albert notice us.

%%%%%%%%%%%%%%%%%%%%%%%%%%%%%%%%%%%%%%%%%%%%%%%%%%
\section[Second Person]{Second Person: Advice to Progeny}
%%%%%%%%%%%%%%%%%%%%%%%%%%%%%%%%%%%%%%%%%%%%%%%%%%

{
{\Large My Dearest,}\medskip

If you are reading this, then it appears that you have recently died and been brought back to a semblance of life as one of the living dead. You also presumably drank my blood and witnessed me hand you this very letter and then flee the premises with haste. This doubtless seemed odd to you, and I apologize for the necessity. You will receive further aid and instruction when I am able, but as of the writing of this letter, neither you nor I know when that might be.

The first thing you will do upon reading these words is to try out your powers. Doubtless you will find solace in climbing walls or even flying about. Also, you will be able to quickly heal from any injury, so perhaps you will wish to experiment with the sort of senselessly dangerous activities you previously gave only a slim moment's thought -- such as leaping off a building, stabbing yourself in the chest, or thrusting your hand into a fire. These are perfectly reasonable things for you to do, and you should probably get them out of your system sooner rather than later. However, even whilst abandoning yourself to the temporary insanity of genuine magical power you should still maintain a semblance of decorum and a thought towards self preservation and your future.

The first thing for you to keep in mind is that others of your kind take a gratuitously dim view of giving away the show. Supernatural creatures such as yourself maintain their existence in no small part through the expedient of secrecy, and you'll notice that even in your case the reality of magical beings was revealed relatively recently. You \uline{can} be hurt, and you \uline{can} die. Yes, even though you are essentially already dead, there are available bigger, more final deaths to be had. The first and foremost danger is \uline{wood}. Avoid that foul stuff like it was bathed in vipers, and if you \uline{must} experiment with cutting or bashing yourself, make for certain that you do so through the medium of comparatively harmless metal or stone. But also you will want at all times to keep your powers and nature a darkest secret -- other creatures may be helpful, but none will long hesitate before turning upon you should your actions jeopardize their safety or obscurity. Your ability to float about the room will doubtlessly seem magical to you -- in no small part because it \uline{is} in fact magical -- but mortal and vampire alike will feel threatened if you use it in public. Best not to tempt reprisals.

The second thing to keep in mind is the constant gnawing hunger which will drive you for the rest of time as a farmer drives an ox. Your experience of the cravings will doubtlessly be deeply personal and reminiscent of some craving you felt in life. However, this is no ordinary famine, you crave instead the blood of mortal humans. Actually acquiring this blood to drink is no easy task and will constantly threaten the secrecy of your true nature. Until you can master for yourself a feeding \uline{system}, you will wish to do it as infrequently as possible. Unfortunately, the beast inside you will not stay quiet forever, and you \uline{will} be driven to feed eventually, and possibly with wild and murderous abandon. Using your powers will make you grow more famished, making it harder to control yourself. Consuming large quantities of blood will likely be fatal for your [s]meals[/s] victims (and you \uline{will} have victims). And if you get hungry enough to lose control, you will very likely drink copious amounts. Try to not get hungry enough to have that happen.

In the meantime, it would probably be expedient to introduce yourself to the local Makhzen government. The Makhzen is an old organization, and by your very existence you are now a member of it. And by old, I do mean \uline{old}. You will, if you avoid the barbs of splinters and thorns, never die. And the leaders of the Makhzen never have. You will doubtlessly find their customs to be archaic and feudal. However, you should remind yourself that age is also a virtue and that these laws have outlasted every institution raised by mortal man. So presumably they have \uline{some} virtues, even if they are not immediately obvious to you. The address to go to is written on the envelope, you should ask for Florentine -- she is an ancient and accomplished mermaid -- and she can aid you in presenting yourself to the local Prince better than I can under the circumstances.

And yes, you did read correctly that you will be asking for aid from a mermaid. That is not the limits of the strangeness you will encounter before the night ends and you will long for sleep. The truth is that whilst \uline{most} stories you have been subjected to are indeed merely stories, there are dozens and dozens of legends that you have heard that -- are at least in part -- factual. Try to play it cool as much as possible.

\medskip
Your Sire and Mentor,

\fontsize{17}{17} \selectfont \textit{X} \normalfont
}

%%%%%%%%%%%%%%%%%%%%%%%%%%%%%%%%%%%%%%%%%%%%%%%%%%
\section[Third Person]{Third Person: Missing Persons}
%%%%%%%%%%%%%%%%%%%%%%%%%%%%%%%%%%%%%%%%%%%%%%%%%%

\hspace{\parindent} Clich\'{e} as it may be, it \textit{was} a dark and stormy night. It could not have been elsewise, for the shadowy figures climbing the exterior wall of the office had waited for precisely such an occasion in order to make their ingress. It was their intension to take files that they believed were stored therein, and for this they needed to be protected from being overseen or overheard.

It had been two weeks back that Freddy had gotten the tip that these offices kept strange hours and were probably smuggling something, and after he'd gotten his gang to scope the place out, he was pretty sure that smuggling was the least of the issues. Men and women coming in and out at odd times of the day and night, rolled carpets going in thick in the night and coming out thin in the morning. People walking out who had not been spotted walking in since the surveillance started. There was a mystery going on, and Freddy was pretty sure the answers were inside this building.

Freddy wrestled the window open with more noise than he had hoped for, but behind the constant hiss of wheels on wet asphalt and the incessant patter of rain no one in the city seemed to notice or care. Once the aperture was enough for his muscular frame, he pulled himself in, with his assistant Velma close behind. The room inside was filthy and rank. It smelled like a refrigerator that hadn't been opened in some time, and piles of clothing and discarded packaging littered the floor. As his eyes accustomed themselves to the gloom, he began to make out\ldots{} squash? The faint yellow light reflected and refracted from street lights through a million tiny prisms of water gave the soft outlines of some kind of growing operation. Pots, vines, and some kind of big fruit lined every shelf. He leaned over to get a closer look\ldots{}

As Velma squeezed herself through the window, she felt a slight tug on the lodestone she wore tied to a string bracelet. That was odd, and she started to comment on it "Jinkrrrk-!" was all that came out as Freddy suddenly grabbed her by the neck. She could feel his powerful hands clamp down on her throat and began flailing her fists into his arms and torso with what strength she had left. The pounding made a muffled sound like dropping a steak on a counter, but he showed no predilection for loosening his grip to protect himself from the counter assault.

He looked at her purpling face with genuine interest, but what came out of his mouth was the most surprising part of the ghastly situation yet. "Now, huu maiyt \textit{you} beee?" he inquired as if the words themselves were in an alien tongue. The wounds being inflicted on his body might have been of just passing interest and he genuinely seemed to not recognize Velma at all. And then realization dawned on her as she saw that the lodestone was pointing directly to Freddy.

She hoped the rest of the gang was having better luck, because the prospect of being made a puppet for some alien thing was not part of her plan. But even then she did not give up hope for herself, because there was still a chance she could reach her pocket\ldots{}

%%%%%%%%%%%%%%%%%%%%%%%%%%%%%%%%%%%%%%%%%%%%%%%%%%
\section[The Fourth Wall]{The Fourth Wall: Starting a Game}
%%%%%%%%%%%%%%%%%%%%%%%%%%%%%%%%%%%%%%%%%%%%%%%%%%

\hspace{\parindent} A story in After Sundown is like a story you would read in a book or watch in a movie, save that it is written cooperatively by the people playing the game. In order for people to tell a story together, the story has to be something that everyone wants to tell. And that means getting everyone on the same first page as regards basic story themes and content. If one player \textit{really} doesn't like the inclusion of something (whether it be "graphic descriptions of violence", "sexual situations", or "discussions of economic theory"), that probably has to take precedence over any number of other people at the table wanting to include it.

This sort of compromise is not as difficult as it might at first seem. After all, if someone didn't want to tell multi-author stories involving modern supernatural horror with an emphasis on adventure and mystery would probably have said "No" when you asked them the question "Do you want to play a game of After Sundown?". So before you even begin to hash out the basic disagreements about what kinds of stories you intend to tell, you already understand that you \textit{do} have broad agreements on genre and story structure. So be prepared to make some concessions to what the other players want to do, whether you're the MC or not. Nevertheless, players will need to know where the game is going to take place and whether the chronicle will be an origin story or an in media res story before they even start to craft themselves characters, so hash that out in the beginning.

%%%%%%%%%%%%%%%%%%%%%%%%%%%%%%%%%%%%%%%%%%%%%%%%%%
\section{Making a Character} \index{Character Creation}
%%%%%%%%%%%%%%%%%%%%%%%%%%%%%%%%%%%%%%%%%%%%%%%%%%

\hspace{\parindent} Probably the most important thing you can do while making a character is to discuss with the other players what kinds of characters you'd like to play. There are a lot of characters that would fit into \textit{some} After Sundown games that won't fit with the specific chronicle you are actually playing. The number one thing you want to look out for is intra-party conflict. And in turn, the number one cause of intra-party conflict is moral disagreements. That is, if you want to be playing a particularly dark game, you might want to make a character who is a serial killer, or a rapist, or whatever. If on the other hand, the other players want to play Syndicate troubleshooters who go on Scooby Doo hunts for supernatural criminals, that kind of character concept isn't going to fly. Heck, if the players want to bewail their loss of humanity and try to fit in with mortal society and listen to \refwork{The Cure}, that kind of character concept probably isn't going to fly. And so on.

Once you've established that your character concept is going to work with the other players, find out about the chronicle and work it in there. If the game is taking place in Paris, it's probably a pretty good idea for your character to speak some French. If the game revolves around police procedurals in gritty New York, make sure your character has \textit{some} relevant skills and something in their backstory to justify being in that situation. Maybe it should go without saying, but it really doesn't: make sure you know what the actual character generation guidelines being used \textbf{are}. It's no good making the whole backstory for a 150 year old Austro-Hungarian Vampire only to get to the table where everyone is making contemporary humans in order to play out the origin story in the 21st century. Even above and beyond the basic category of game (Origin Story, In Media Res, or Power Fantasy), the expected direction of the chronicle may place additional constraints on the PCs. Maybe everyone has to be able to swim, or no one is to be or speak Romanian, or whatever. Talk this all out before you commit too hard to any particular character traits. Ideally, these constraints won't just be the MC jerking you around, but will instead be actual well thought out guidelines to make sure that the chronicle can and will be completed.

%%%%%%%%%%%%%%%%%%%%%%%%%
\subsection[Barbara Stanwick]{The Curious Tail of Barbara Stanwick}
%%%%%%%%%%%%%%%%%%%%%%%%%

\hspace{\parindent} Barbara Stanwick is a sample character, who is played by a sample player who we will call Jenny. Jenny wants to kick ass and take names, but she \textit{also} wants to identify with her character on a physical basis. To balance these two desires, she chooses to have her alter-ego Barbara be a Werewolf. This allows her to make a character who is a slightly sporty liberal arts type, while still being able to turn into a giant monster and rip a car in half. Fortunately, she is making an In Media Res character and she can totally start with her character already being a Werewolf.

\textbf{The Backstory:} Jenny can write whatever backstory she wants, subject to the story \textit{ending} with Barbara Stanwick being wherever the chronicle is going to take place and being interested in interacting with the chronicle rather than ignoring it or getting on a plane to Boston or something. So ironically the first thing she finds out is where her backstory has to end up. And after discussing it with the MC, she finds that the initial chronicle is going to take place in the San Francisco area, hunting a rogue Werewolf under the auspices of The World Crime League. That sparks some inspiration with Jenny -- and her first question to the MC is if it would be disruptive for her to suspect the main villain Werewolf of being the one that mauled Barbara (making her a Werewolf in the first place). The MC lauds as how that sounds like a great idea (as it personalizes the chronicle for Barbara and also motivates her to participate), and so Jenny begins writing.

Barbara is the daughter of the District Attorney of Santa Clara County. She attends the University of California at Berkeley, where she studies literature. Her father wants this to be a pre-law degree, but she has been strongly considering going into teaching. However, things became both simpler and more complicated when she and a couple of acquaintances were attacked and killed by a wild animal after a party in Oakland. She ended up lying in the torn up shreds of a couple of students for several hours, before waking up and making a full recovery. She has since been contacted by Black Hand agents, and has joined that organization as an enforcer from beyond the grave. Also, having been associated with supernatural society for some time, she has presented herself to the local Syndicates and has citizenship in the World Crime League.

Jenny actually writes a backstory that is several pages long, and includes reasons why and how she knows the other player characters (which she worked out \textit{with} the other players), but that is neither here nor there.

\textbf{Attributes:} Because Jenny is making a character for an In Media Res game, she can assign 2 points to her least stat category, 4 points to the middle pair, and 5 points to the top selection. However, she also has 3 points she can assign however she likes, so if she \textit{wanted}, she could bring her "lowest" attribute category to parity with her highest. What she decides to do is to emphasize Barbara's sport and dance background, with a later emphasis on academic development. So she makes Social her highest priority, Physical her next highest, and Mental her lowest. Then she applies her three bonus points to Logic and Charisma. This ends up with attributes of:

\medskip
\begin{tabular}{lc lc lc}
\textbf{Strength} & 3 & \textbf{Intuition} & 2 & \textbf{Willpower} & 3\\
\textbf{Agility} & 3 & \textbf{Logic} & 4 & \textbf{Charisma} & 5\\
\textbf{Edge} & 3 & \textbf{Potency} & 1 & \textbf{Power} & 13\\
\end{tabular}
\medskip

Technically of course, she has not applied the supernaturalness yet, so at this stage she wouldn't have a Potency or Power rating, but it saves time to write it down with the rest of the attributes if you know you're making a Werewolf, which Jenny does. You'll also note that Jenny has cast her net pretty wide -- Barbara doesn't have really high physical attributes for a combat oriented character, or really any super high attributes at all. That's deliberate on her part, and actually OK given her magical abilities, which augment the mediocre physicals into deadly ones and still leave her enough Logic and Charisma to investigate well.

\textbf{Active Skills:} It's very difficult to make a combat monster without the combat \textit{skill}, but fortunately Jenny has written combat training with The Black Hand into her character's backstory. She'll get 14 points to one category, 19 to another, and 24 to the third. Then she can distribute 6 points wherever. Again, she decides to focus Social, Physical, then Mental. After all, Barbara is still a student. For thematic reasons, she decides to put all six bonus skill points in Combat. She also gets 4 specializations (each Technical skill has one of its own specializations for free).

\medskip
{\center
\begin{tabular}{lc lc lc}
\textbf{Athletics} & 4 & \textbf{Animal Ken} & 2 & \textbf{Artisan} & 4 (Painting)\\
\textbf{Combat} & 6 (Pencak Silat) & \textbf{Bureaucracy} & 6 (Law) & \textbf{Electronics} & 4 (Software)\\
\textbf{Drive} & 1 & \textbf{Empathy} & 4 & \textbf{Rigging} & -\\
\textbf{Larceny} & 4 & \textbf{Expression} & 4 (Dance) & \textbf{Medicine} & -\\
\textbf{Perception} & 6 & \textbf{Intimiation} & 1 & \textbf{Operations} & -\\
\textbf{Stealth} & 4 & \textbf{Persuasion} & 6 (Pedantics) & \textbf{Research} & 6 (Library)\\
\textbf{Survival} & - & \textbf{Tactics} & 1 & \textbf{Sabotage} & -\\
\end{tabular}
}
\medskip

All in all, Jenny made a character who could do well as a lawyer or a teacher\ldots{} or a murderous enforcer. Which is appropriate, because Barbara is still uncertain which direction she'll end up going with her life.

\textbf{Backgrounds:} As a character in an In Media Res story, Barbara Stanwick can have 35 points of Backgrounds. She decides to split them up amongst her interests:

\begin{itemize*}
\item Classical Literature 4
\item Modern Fantasy Books 4
\item Student Life 3
\item Binge Drinking 3
\item Fine Cuisine 4
\item Secret History of Supernatural Conflict 5
\item California Legal System 4
\item American Education System 4
\item World Crime League Rules and Etiquette 4
\end{itemize*}

The best criminals are also the best lawyers.

\textbf{Resources:} Jenny can assign one 3 point Resource, two 2 point Resources, an one 1 point Resource. Also, she has to select an Obligation, but gets a Resource for that too. She assigns them as follows:

\begin{itemize*}
\item Barbara's backstory includes ending up with a magical mirror that shows whatever is on the other side of a wall it is placed against -- like a portable window that isn't transparent on the other side. This requires Jenny to discuss it with the MC, and they agree that this item is a Rating 3 Destiny.
\item Barbara Stanwick is still on reasonably good terms with her father, and can get records and information from the Santa Clara County DA's Office. This is a Rating 2 Contact.
\item Barbara's parents set her up with a trust fund that is quite substantial. This is a Rating 2 Finances Resource.
\item As a student of classical literature, Barbara knows Greek. This is a Rating 1 Languages.
\item After being literally pulled out of a pile of bodies where she had been left for dead and then trained in seven deadly Indonesian martial disciplines, Barbara feels obligated to assist The Black Hand in their efforts. This manifests as a Rating 2 Duty.
\item Barbara has been allowed into the "Rare Books" archive at UCB -- which is a magical library. A Rating 2 Destiny.
\end{itemize*}

\textbf{Advantages and Disadvantages:} As a Werewolf, she gets to have Temperamental for free. But Jenny feels that her character's divided loyalties towards her various available lifepaths would be appropriate for the Disadvantage Flake. Having taken an extra Disadvantage, she can select an Advantage as well. She chooses Attractive.

\textbf{Magical Transformation:} Barbara Stanwick is a Werewolf in an In Media Res game. She gets all of the inherent Powers of a Werewolf, and also gets to select 2 Basic Powers and an Advanced Power of her choice. She then gets one Basic or Advanced Power that can be either Universal or Lure of Destruction (the Cultic Sorcery of the Black Hand). She chooses to go for the Sorcery option, and her ability list looks like this:

\begin{itemize*}
\item Tongue of Beasts (Basic Call of the Wild)
\item Beast Form (Basic Call of the Wild)
\item Revive the Flesh (Basic Fortitude) 
\item Vigor (Basic Clout) 
\item Quickness (Basic Celerity) 
\item Repel (Basic Magnetism) 
\item War Form (Celerity / Clout Devotion) 
\item The Beckoning (Advanced Call of the Wild)
\item Touch of Darkness (Basic Lure of Destruction)
\item Glimpse of the Abyss (Advanced Lure of Destruction)
\item Attract (Basic Magnetism)
\item Summons (Advanced Magnetism)
\end{itemize*}

It is worth noting that she gains +2 to Socialization tests (Advanced Magnetism), +4 to avoid death and healing (Advance Lure of Destruction), +1 Strength (Basic Clout), +2 to Physical Resistance checks (Basic Fortitude), +2 to initiative tests (Basic Celerity), and a +4 to Animal Ken checks that increases to +6 for canines (Advanced Call of the Wild, and she can personally turn into a wolf). However, while many of those are situational, the Clout bonus really is just an always on \textit{thing}. So Jenny chooses to just add it in to her stat line, like this:

\medskip
\begin{tabular}{lc lc lc}
\textbf{Strength} & 4 & \textbf{Intuition} & 2 & \textbf{Willpower} & 3\\
\textbf{Agility} & 3 & \textbf{Logic} & 4 & \textbf{Charisma} & 5\\
\textbf{Edge} & 3 & \textbf{Potency} & 1 & \textbf{Power} & 13\\
\end{tabular}
\medskip

\textbf{Place in the World:} Jenny has a pretty good idea of what Barbara is about. She lives in a loft in Berkeley and is a citizen in good standing with the World Crime League. She is also an enforcer for The Black Hand. Lycanthropy is driving a wedge between her and the aspirations she has to become a teacher or a lawyer. But she has some good friends amongst the supernaturals, some of whom are the other player characters in her band.
