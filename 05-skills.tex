%%%%%%%%%%%%%%%%%%%%%%%%%%%%%%%%%%%%%%%%%%%%%%%%%%
%%%%%%%%%%%%%%%%%%%%%%%%%%%%%%%%%%%%%%%%%%%%%%%%%%
\chapter{Skills}
%%%%%%%%%%%%%%%%%%%%%%%%%%%%%%%%%%%%%%%%%%%%%%%%%%
%%%%%%%%%%%%%%%%%%%%%%%%%%%%%%%%%%%%%%%%%%%%%%%%%%
\tagline{No one may ever know if what you did was good or bad, but you did it well.}

A key portion of the die roll for any test is the \textit{Skill}\index{Skill}. It represents \textit{specific} training that helps a character perform a task. And because of that, it can lead to a far amount of confusion among some people, because linguistically we refer both to people who are very good at something and also people who are good at a wide variety of things as being "skilled". In After Sundown, skills are basically confined to the former interpretation. Jacks of All Trades are represented game mechanically by people who have relatively high \textit{Attributes} and low skills. People with high skills are specialists by definition.

%%%%%%%%%%%%%%%%%%%%%%%%%%%%%%%%%%%%%%%%%%%%%%%%%%
\section{Using Skills}
%%%%%%%%%%%%%%%%%%%%%%%%%%%%%%%%%%%%%%%%%%%%%%%%%%

\begin{table}[htb] \rowcolors{1}{white}{tan} \center \vspace{-1cm}
\caption[Awesomeness per Hit (Skills)]{Awesomeness per Hit}\index{Awesomeness}
\begin{tabular}{c l}
\textbf{Hits} & \textbf{Awesomeness} \\
\textbf{0} & Not Awesome. Tying shoes, climbing stairs.\\
\textbf{1} & Completely Pedestrian. Driving a car, Throwing Darts.\\
\textbf{2} & Professional. Don't try this at home.\\
\textbf{3} & Hard. Don't try this at all.\\
\textbf{4} & Extreme.\\
\textbf{5} & Crazy Extreme.\\
\textbf{6} & Super Human. Does not need disclaimers because it is clearly impossible. \\
\end{tabular}
\end{table}

\hspace{\parindent} When you use a skill, describe what it is that you intend to do to the MC, and then between the two of you determine an acceptable Skill and Attribute to use. Remember that the attribute being used in an action determines what kind of people are naturally talented at that kind of action, not on what kind of character is generally good at a skill. For example, in the general running of a power plant one might expect that a "smart" hero would be the man for the job, and thus a good standby check to make for actions from shunting power away from the financial district to increasing power yield might be Logic + Operations. But in the specific case of getting the emergency valves opened during an overheating event you might expect a "strong" hero to be the man for the job, and Strength + Operations might be called for instead (those valves can be hard to turn).

The next thing you do is roll your dice, counting every 5 or 6 as a 'Hit'. The number of hits you get determines how awesome you did, with this representing overall success or not depending on how awesome the specific thing you were attempting to do was. Doing something incredibly awesome when the task at hand is something like "bake a cake" is potentially \textit{delicious}, but often fairly inconsequential. On the flip side, if the goal is to do something of awesome difficulty such as leap into an open window on a moving train, the results will be unfortunate if the level of success attained is merely normal.

\paragraph{Buying Hits:} \index{Buying Hits}When a character is not under any particular threat or pressure, they may elect to forgo the process of actually rolling dice and simply get one hit for every 4 full dice in their dice pool. This process of "phoning it in" gets a character less awesomeness than had they legitimately tried, but it has a strong tendency to work if that's all that is required.

\begin{wraptable}[18]{R}{5cm} \rowcolors{1}{white}{tan} \vspace{-.25cm}
\caption{Timeframe Chart}\index{Timeframe} \centering
\begin{tabular}{l}
Century\\
Decade\\
Year\\
Season\\
Month\\
Week\\
3 Days\\
1 Day\\
5 Hours\\
1 Hour\\
20 minutes\\
5 Minutes\\
1 Minute\\
1 Round\\
Simple Action\\
Free Action\\
\end{tabular}
\end{wraptable}

It is important to note that normal humans often have dice pools of 4 dice or less on tasks they do frequently. So when a supernatural critter throws down on a task with 12 dice or more that really is an incredible thing to watch. Such characters can literally phone in a TV quality performance and the like. MCs should not become jaded and allow success inflation to cheapen the actions of characters with super human dice pools. Characters who can lift and throw motorcycles genuinely can expect to casually kick in locked doors. The fact that success is practically automatic for these tasks should not be resisted, but rather embraced as a fact that is itself impressive and magical.

\paragraph{Predictable Failure:} Sometimes a character will be struggling under enough penalties that they don't have a dice pool at all. In these instances, the character is going to get zero hits, which means that absolutely nothing they do will be awesome. They can still stagger down the corridor or open a door, but as soon as a stunt requires even one hit they are \textit{going} to fail unless they are a Luminary who can spend Edge on the problem to get some dice and a chance.

\paragraph{Extended Tests:} \index{Extended Tests}Some actions take an expected amount of time. If a character gets the requisite number of hits, they succeed in the expected amount of time. If they get more than the requisite number of hits, they may complete the task well ahead of schedule. For every hit made in excess of the minimum, move to the next lower amount of time on the time chart. If a character fails to succeed, they may retry, but only after having put in the normal time into the first shot. So for example: Mina is attempting to paint a house (Strength + Artisan, 1, 2 days) and gets 3 hits. Since she got 2 more hits than she needed, she can go to the next lower time period twice, bringing the time frame down to five hours.

%%%%%%%%%%%%%%%%%%%%%%%%%
\subsection{Team Work} \index{Teamwork}
%%%%%%%%%%%%%%%%%%%%%%%%%
\tagline{If you're about to launch a friendship speech, please don't.}

When more than one character throws their weight into a project they can achieve results that are more awesome and in less time than what either character could achieve alone. However, the game mechanics completely break down if you just add the dicepool of one character to another. What is done instead is that whichever character has the best dicepool is considered the main acting character, and the other characters are considered the assisting characters. Each assisting character makes their check, and each hit is added as a bonus die on the main character's test. Since characters get about 1 hit per three dice, on average improving the awesomeness of a task is "hard" (threshold 3). In many cases an MC will allow a character to assist with a tangential but vaguely related skill (and in such cases it is entirely possible for one of the assisting characters to roll more dice than the main acting character).

\paragraph{Maximum Characters:} Too many cooks spoil the broth. How many characters qualify as "too many" is unfortunately a very fluid concept that depends a lot on what you're doing. Sometimes there are real physical limits to how many people can literally fit around a project, and other times it's procedural. In general, most teamwork projects should be handled with five or less people. A project larger than that should probably be split into multiple tests, although at the MC's discretion there may be exceptions. A good set of management protocols is essential for most group projects to move forward. Most of the time, no more assisting characters can work on a project than the highest Tactics skill of the characters. The character providing the tactics skill allowing multiple characters to work on the project need not be the main acting character and often will not be.

%%%%%%%%%%%%%%%%%%%%%%%%%%%%%%%%%%%%%%%%%%%%%%%%%%
\section{Using Attributes Without Skills}
%%%%%%%%%%%%%%%%%%%%%%%%%%%%%%%%%%%%%%%%%%%%%%%%%%
\tagline{"Granted, but I'm still huge."}

Characters in After Sundown may be called upon to use skills when they don't actually have training in that area. In this case, the character is called upon to \textit{Default}\index{Default} on the skill. This allows the character to roll a dicepool of their appropriate Attribute (plus zero dice for not having the skill). When defaulting on a Social or Technical Skill, the character suffers a -1 die penalty for being untrained. When using Technical Skills, that same -1 die penalty applies whenever the character doesn't have an appropriate specialization (even if they \textit{do} have the appropriate skill). There are a number of times when you will want to do something for which \textit{no} skill applies. In that case a mere attribute roll may suffice (obviously with no -1 die penalty), but remember that dice pools without skills are substantially \textit{smaller} than dicepools with skills attached -- so in most cases the MC should try to figure out a way to fit a skill in.

%%%%%%%%%%%%%%%%%%%%%%%%%
\subsection{Resistance Tests} \index{Resistance Tests}
%%%%%%%%%%%%%%%%%%%%%%%%%
\tagline{"No one could have survived that."}

Characters who are attacked or endangered are often entitled to a Resistance Test to soak the effects of whatever they are threatened with, whether its the power of a magical assault or a bullet to the stomach. In general, a Physical Resistance Test will usually be just Strength (no skill), a Mental Resistance Test will usually be just Intuition (again, no skill), and a Social Resistance Test will be just Willpower (likewise). Luminaries get a special bonus, where they can add their Edge to Resistance Tests, almost like Edge was the "take less damage from bullets" skill, if that makes things any easier to conceptualize.

%%%%%%%%%%%%%%%%%%%%%%%%%
\subsection{Sure Things: Heavy Lifting} \index{Carrying Capacity}
%%%%%%%%%%%%%%%%%%%%%%%%%
\tagline{"Sure, sometimes you can do all kinds of stuff. But I can \textbf{always} lift a car."}

There are things you don't have to roll because they simply \textit{are}. A character with a high Charisma \textit{is} charming, a character with a high Logic \textit{is} smart. Even if they offend someone or fail to solve a problem, they will do so in a charming or intelligent fashion. But probably the thing you will run into most frequently as far as automatic uses of Attributes is Strength. People who have a high Strength \textit{are strong}, and they can lift heavy things. So to help out with that, here's a table of how much a character might be able to push themselves to lift up, and how much they might be able to carry home without hurting themselves.

\begin{table}[htb] \rowcolors{1}{white}{tan} \center
\caption{Carrying Capacity}
\begin{tabular}{c | r r | r r}
\textbf{Strength} & \textbf{Lift (kg)} & \textbf{Carry (kg)} & \textbf{Lift (lb)} & \textbf{Carry (lb)}\\
1 & 30 & 10 & 65 & 22 \\
2 & 50 & 20 & 110 & 44 \\
3 & 100 & 30 & 220 & 66 \\
4 & 150 & 50 & 330 & 110 \\
5 & 250 & 70 & 550 & 155 \\
6 & 450 & 100 & 990 & 220 \\
7 & 750 & 200 & 1,650 & 440 \\
8 & 1,250 & 500 & 2,750 & 1,100 \\
9 & 2,500 & 1,000 & 5,500 & 2,200 \\
10 & 5,000 & 2,000 & 11,000 & 4,400 \\
11 & 7,000 & 3,000 & 15,500 & 6,600 \\
12 & 10,000 & 4,500 & 22,000 & 9,900 \\
13 & 14,000 & 6,000 & 31,000 & 13,200 \\
14 & 20,000 & 8,000 & 44,000 & 17,600 \\
15 & 28,000 & 10,000 & 62,000 & 22,000 \\
20 & 60,000 & 24,000 & 132,000 & 53,000 \\
25 & 100,000 & 40,000 & 220,500 & 88,000 \\
30 & 150,000 & 60,000 & 331,000 & 132,000 \\
35 & 200,000 & 80,000 & 441,000 & 176,000 \\
\end{tabular}
\end{table}

And yes, things that are really strong are \textit{really strong}. A creature with a strength of 35 can lift a \textit{train} right off the track. Although they can only do this by lifting one car at a time and can't really walk off with it. Consider the scene in \refwork{King Kong} where the giant ape (who in After Sundown would be a Kaiju) pulls a train off the tracks by lifting a car and dropping the whole thing. That's not an exaggeration, in After Sundown the giant apes can actually do those things.


%%%%%%%%%%%%%%%%%%%%%%%%%%%%%%%%%%%%%%%%%%%%%%%%%%
\section{Physical Skills}
%%%%%%%%%%%%%%%%%%%%%%%%%%%%%%%%%%%%%%%%%%%%%%%%%%

\hspace{\parindent} Physical skills involve doing stuff with your body. Since everyone has a body, the defaulting penalty for using physical skills untrained is zero. Without specialized training you can always still make a raw attribute test. A very strong person can grapple fairly effectively on that basis alone, an intuitive person can easily notice things, and so on and so forth.

%%%%%%%%%%%%%%%%%%%%%%%%%
\skillentry{Athletics}
%%%%%%%%%%%%%%%%%%%%%%%%%

\hspace{\parindent} Athletics is the skill you use to run, jump, climb, swim, throw things, and generally do everything that you're vaguely expected to do in PE except sneak into the girl's locker room and get shamed by your peers. Most stunts are athletics based, and it can even be useful in attacking enemies in combat by dint of throwing things at specific targets.

\paragraph{Specializations:} Climbing, Parkour, Swimming, Throwing

%%%%%%%%%%%%%%%%%%%%%%%%%
\skillentry{Combat}
%%%%%%%%%%%%%%%%%%%%%%%%%

\hspace{\parindent} Combat is the training needed to fight. Usually other people. More than any other skill, people will ask to divide up Combat into smaller fragments. It is not immediately clear why, but we strongly suggest that you do not do this. While there are truly a vast array of differences between stabbing someone with a sword and shooting them with a gun -- the simple fact that the combat simulationist player can \textit{name} them all draws attention to how simply not that different they are. The basic truths of combat skill include making rapid life and death decisions while avoiding threats and putting the pointy end of your weapon into the other man. There's a world of differences from combat situation to combat situation, and that's why the skill goes all the way to six. But please remember that this is a game where flying a plane and managing a nuclear power plant can be the same skill (Operations).

\paragraph{Specializations:} By weapon or martial style

%%%%%%%%%%%%%%%%%%%%%%%%%
\skillentry{Drive}
%%%%%%%%%%%%%%%%%%%%%%%%%

\hspace{\parindent} Drive allows people to drive culturally appropriate vehicles. For people in the west, that's mostly just cars. But for people in river areas or fishing communities, that's often small boats as well. Driving under safe conditions is such a banal and non-awesome thing, that characters do not need to actually make rolls to do it, so players may not even need the skill. But when it comes to dangerous driving conditions, car chases, or even just cutting commute times, rolls are generally required, and having the skill is helpful. This dichotomy is generally why in movies cars wipe out spectacularly the moment someone uses magic to create icy road conditions -- a lot of people on the road are legitimately terrible drivers, and the moment things get harsh they become one of the 18,000 car accidents that happen every day in the US. 

A character can drive a new non-standard type of vehicle for every rating point and specialization of the skill they have. Common choices are emergency vehicles, motor boats, construction equipment, armored vehicles, light aircraft, and oversized (semis and buses), but really players can pretty much go nuts. Any plane or ship which is piloted with dials and knobs rather than a wheel or stick is the domain of Operations rather than Drive.

\paragraph{Specializations:} Bad Weather, Aggressive Driving, Cross Town Traffic, Navigation

%%%%%%%%%%%%%%%%%%%%%%%%%
\skillentry{Larceny}
%%%%%%%%%%%%%%%%%%%%%%%%%

\hspace{\parindent} Characters skilled at Larceny are adept at working outside the law. It is a broad skill that covers lots of dubious activities, from identifying and bypassing security systems to picking other peoples' pockets. There is some overlap between Larceny and Rigging when dealing with locks. Locks are both geared puzzles and a basic hindrance to breaking into places. This is a good skill to have for security workers in addition to criminals. You gotta know your enemy if you're gonna win the war. 

\paragraph{Specializations:} Concealing Goods, Legerdemain, Lockpicking, Security Systems

%%%%%%%%%%%%%%%%%%%%%%%%%
\skillentry{Perception}
%%%%%%%%%%%%%%%%%%%%%%%%%

\hspace{\parindent} Perception is the skill by which a character perceives the world around themselves. It is used to spot clues, notice subtle noises, and smell unfortunate smells. Characters with very low Perceptions are the characters who do not notice monsters sneaking up on them or have to have the meaning or import of subtle clues explained to them by other protagonists.

\paragraph{Specializations:} By sense, Investigation, Noticing sneaking

%%%%%%%%%%%%%%%%%%%%%%%%%
\skillentry{Stealth}
%%%%%%%%%%%%%%%%%%%%%%%%%

\hspace{\parindent} The Stealth skill is what one uses to avoid being noticed, either by moving quietly, becoming unseen against the background, or simply blending into the crowd. Whenever a character is being searched for, a Stealth check can be used to make that searching more difficult. Stealth involves using what is available, so there is almost no circumstance in which it cannot be used to at least postpone the moment that a character is noticed.

\paragraph{Specializations:} Hiding, Innocuity, Shadowing, Sneaking

%%%%%%%%%%%%%%%%%%%%%%%%%
\skillentry{Survival}
%%%%%%%%%%%%%%%%%%%%%%%%%

\hspace{\parindent} 100\% of the creatures alive today are the descendants of an unbroken line of ancestors who were all able to survive in their environment long enough to have had children that extends back to when single celled organisms had only two different nucleotides in their DNA. So persisting in the face of adversity is something that creatures have a birthright to. And yet, adversity has also kept up with the times. Survival is the skill of keeping up with the elements.

One can also make Survival checks to scavenge things of a more modern nature. A Survival check could be called for to loot useful things out of a junk yard or to track the layout of a sewer system.

\paragraph{Specializations:} Tracking, Gathering, Shelter, by Environment

%%%%%%%%%%%%%%%%%%%%%%%%%%%%%%%%%%%%%%%%%%%%%%%%%%
\section{Social Skills}
%%%%%%%%%%%%%%%%%%%%%%%%%%%%%%%%%%%%%%%%%%%%%%%%%%

\hspace{\parindent} Regular socialization is performed with Backgrounds, rather than social skills. If you want to ingratiate yourself with others, track down the word on the street, or otherwise perform social legwork, you probably want to use a background like \textit{High Society} or \textit{Barrio}. Social skills apply a -1 dicepool penalty when defaulting.

%%%%%%%%%%%%%%%%%%%%%%%%%
\skillentry{Animal Ken}
%%%%%%%%%%%%%%%%%%%%%%%%%

\hspace{\parindent} Dealing with inhuman beasts is a skill in and of itself. Neither lions nor sheep really have any backgrounds, and the Animal Ken skill is used in their place. Animal Ken is used to read the emotions of an animal as well as to calm one down or train it to do stuff. Animal Ken is thus your one-stop-shop for all socialization with dogs, which considering how much less a dog knows than any human or supernatural about important plot points, is not nearly as overpowered as you might think.

\paragraph{Specializations:} Domestic Animals, Training, Wild Animals, Riding

%%%%%%%%%%%%%%%%%%%%%%%%%
\skillentry{Bureaucracy}
%%%%%%%%%%%%%%%%%%%%%%%%%

\hspace{\parindent} Managing logistics and patiently untangling skeins of red tape is the focus of this skill. Characters can understand and manipulate laws, navigate management systems, and correctly formulate formal requests. Bureaucracy is of use whether the character is attempting to perform bureaucratic tasks and of equal utility when confronted by the implacable edifice of a Kafkaesque course. It is not unusual for people to resent bureaucracy, because it is annoying. But as anyone who has done logistics under any circumstances can tell you, \textit{not} having rules, management, and records in place is \textit{even worse}.

\paragraph{Specializations:} Business, Government, Logistics

%%%%%%%%%%%%%%%%%%%%%%%%%
\skillentry{Empathy}
%%%%%%%%%%%%%%%%%%%%%%%%%

\hspace{\parindent} Empathy is our primary means of interpreting the meaning of actions and inactions of other people. It is a trainable sense of how others are feeling given how they look, what they say, and what they do. Empathy is of obvious use to people like lawyers and police, but it is also an important skill for batters in baseball. It is not merely about figuring out whether someone is being truthful when they are talking, but also about determining what someone is about to do in the physical world.

\paragraph{Specializations:} Motivation Determination, Detecting Lies, Action Anticipation

%%%%%%%%%%%%%%%%%%%%%%%%%
\skillentry{Expression}
%%%%%%%%%%%%%%%%%%%%%%%%%

\hspace{\parindent} Expression is the art of entertaining and changing peoples' minds through art. Lots of people think that this can only be accomplished by making movies about gay cowboys eating pudding, but the truth is that \textit{any} art that provokes the audience to even acknowledge it is on some level influencing the audience. 

\paragraph{Specializations:} Writing, Dance, Music, Oratory

%%%%%%%%%%%%%%%%%%%%%%%%%
\skillentry{Intimidation}
%%%%%%%%%%%%%%%%%%%%%%%%%

\hspace{\parindent} Intimidation is the art of using fear to get other people to believe or do things desired of them. Intimidation can be explicit ("If you don't do X, I will stab you. In the face.") or implied ("Did you hear that the feds caught Ted for his tax non-payment? He's going to be doing \textit{time}.") and the threats can be to the target's person, finances, or reputation. And some of the best Intimidation is actually phrased in a manner that implies that some third party will do some thing to the target and the Intimidating character is willing to help \textit{the target}.

\paragraph{Specializations:} Interrogation, Fear Mongering, Skulduggery, Blackmail

%%%%%%%%%%%%%%%%%%%%%%%%%
\skillentry{Persuasion}
%%%%%%%%%%%%%%%%%%%%%%%%%

\hspace{\parindent} Persuasion is the art of manipulating people in such a manner that it isn't immediately obvious that is what you're doing. People who are skilled at Persuasion are essentially good at \textit{lying}, although many of them get offended if you call it that. They may prefer the term \textit{acting} or \textit{sales}.

\paragraph{Specializations:} Acting, Insinuation, Fast Talk

%%%%%%%%%%%%%%%%%%%%%%%%%
\skillentry{Tactics}
%%%%%%%%%%%%%%%%%%%%%%%%%

\hspace{\parindent} Tactics is the skill that governs leadership in both the military and corporate sense of the term. Characters can inspire others to give 100\% or produce a battle plan. The dragon crawls on its belly, and Tactics dovetails closely with Bureaucracy in the plotting of war, whether genuine or metaphorical. The tactical aspect involves actually maneuvering and the orders necessary to get others to do that -- in contrast to the simple appeals to rules or potentially complex logistical management of Bureaucracy.

\paragraph{Specializations:} Inspiration, Maneuvers, Naval, Siege

%%%%%%%%%%%%%%%%%%%%%%%%%%%%%%%%%%%%%%%%%%%%%%%%%%
\section{Technical Skills}
%%%%%%%%%%%%%%%%%%%%%%%%%%%%%%%%%%%%%%%%%%%%%%%%%%

\hspace{\parindent} Technical skills apply the -1 penalty for defaulting if a character doesn't have an appropriate specialization. That is, a character may have Artisan (Painting), but they will still have to default when welding. A character who becomes trained in any Technical skill gains a specialization for free.

%%%%%%%%%%%%%%%%%%%%%%%%%
\skillentry{Artisan}
%%%%%%%%%%%%%%%%%%%%%%%%%

\hspace{\parindent} The Artisan skill is used when you want to produce a physical object of some level of workmanship -- whether you're going for aesthetic quality or simple utility. 

There are a few more specializations in Artisan than in most Technical skills, in no small part because there are many materials that involve wildly different skills. It is recommended that these specializations are taken as applying to Artisan uses that are "close enough" -- so a calligrapher might use the Painting specialization since in both cases they're applying pigments to surfaces.

\paragraph{Specializations:} By Medium (Painting, Sculpture, Metalwork, Carpentry, etc.)

%%%%%%%%%%%%%%%%%%%%%%%%%
\skillentry{Electronics}
%%%%%%%%%%%%%%%%%%%%%%%%%

\hspace{\parindent} Electronics is the skill used to make the tools of modernity go. Everything from toasters to computers uses electronics to function. And a character with the electronics \textit{skill} can figure out how it functions and alter it.

\paragraph{Specializations:} Wiring, Software, Repair, Hacking

%%%%%%%%%%%%%%%%%%%%%%%%%
\skillentry{Medicine}
%%%%%%%%%%%%%%%%%%%%%%%%%

\hspace{\parindent} Medicine is the art of treating injury and illness to promote good health. Characters use this skill to patch injuries in their pets and team mates. Remember that the realm of horror runs on movie physics, meaning that characters who receive proper medical care are able to make impressive and full recoveries from amazing injuries.

\paragraph{Specializations:} Veterinary, First Aid, Long Term Care, Psychiatric

%%%%%%%%%%%%%%%%%%%%%%%%%
\skillentry{Operations}
%%%%%%%%%%%%%%%%%%%%%%%%%

\hspace{\parindent} Operations is the skill of making machines and systems go. One part mechanical engineering and one part heavy machinery operation. This is distinct from \textit{making} machines (generally artisan), running computer programs that make systems go for you (generally electronics), or driving (generally driving). You make an Operations test when there is \textit{not} a 1:1 correspondence between your muscle movement and the action of the machine. So it's Operations to pilot a boat and Driving to walk a mech around.

\textbf{Specializations:}Piloting, Industry, Repair

%%%%%%%%%%%%%%%%%%%%%%%%%
\skillentry{Research}
%%%%%%%%%%%%%%%%%%%%%%%%%

\hspace{\parindent} Knowing things is important, but the fact is that your brain probably can't hold all the information you might possibly want to have available -- and doesn't always keep the things you do know readily accessible. When you need information that you don't actually have in your head, you can use the Research skill to go look it up. 

Researching things overall is fairly uniform, but there are particular methods of looking things up that might not be obvious to people who don't use that system in particular. Specialization in Archives indicates an ability to look up information in data logs, newspaper histories, and other chronological information stores. The Library specialization involves looking up information in stores classified by content, and Datamining covers sifting through internet searches, wikis, and highly disorganized information for something useful. 

\paragraph{Specializations:} Archives, Library, Datamining, Interrogation

%%%%%%%%%%%%%%%%%%%%%%%%%
\skillentry{Rigging}
%%%%%%%%%%%%%%%%%%%%%%%%%

\hspace{\parindent} Rigging is the skill of MacGyvering and Rube Goldberging things. It is the skill of practical and impromptu engineering. Including lockpicking, plumbing, and clockwork. Rigging is used for most non-electric jury rigging as well as the creation, operation, and repair of most steam punk technologies. 

\paragraph{Specializations:} Fluids, Gears, Ropes and Pulleys

%%%%%%%%%%%%%%%%%%%%%%%%%
\skillentry{Sabotage}
%%%%%%%%%%%%%%%%%%%%%%%%%

\hspace{\parindent} Sabotage is the art of breaking stuff in a manner which will be most effective. Sabotage can be used for "rigging things to explode" rather than the actual Rigging skill. Sabotage can be used to break things in such a way as to make them look not broken, to not break things in such a manner as they do look broken, and to make things break in such a manner as to explode. Remember that events in After Sundown have a pyrotechnics budget, so things tend to explode big.

\paragraph{Specializations:} Explosives, Disabling Stuff, Structural Weaknesses, Traps

%%%%%%%%%%%%%%%%%%%%%%%%%%%%%%%%%%%%%%%%%%%%%%%%%%
\section{Specializations}
%%%%%%%%%%%%%%%%%%%%%%%%%%%%%%%%%%%%%%%%%%%%%%%%%%
\tagline{This is what I'm good at. And I'm the best. You might ask: 'what good is that'? And the answer is that being the best at anything makes you the best at something.}

A specialization\index{Specialization} is a subset of a skill that a character is especially proficient in. When they make tests using the skill in a manner that is relevant to their interests, they gain 2 extra dice in their dice pool. Technical skills are an inherently specialized field, so in addition to getting 2 extra dice within a character's specialties, Technical Skill dicepools are \textit{penalized} by 1 die if they are being used outside a relevant specialization.

The sample specializations are by no means comprehensive, and players should work out with their MC to find or create specializations that are right for them. A character might have their Sabotage specialized in Eco-Terrorism covering both spiking trees (that might more frequently go under "traps") \textit{and} breaking a bulldozer (that might more frequently go under "Disabling Stuff"). Another character might have their Animal Ken specialized in Horses, covering the training, breeding, and calming of wild and domestic horses. 

The MC should take care to make sure that no specialization is universally useful. Specializing a skill in something that would apply in all cases is basically the same as just getting 2 points in the skill, and that's unfair. MCs must be expected to reject specializing Combat in "fighting" or specializing Bureaucracy in "paperwork". A character can have more than one specialization in the same skill, and this is often very important for Technical skills. If more than one Specialization would apply, the character still only gets 2 bonus dice.

%%%%%%%%%%%%%%%%%%%%%%%%%%%%%%%%%%%%%%%%%%%%%%%%%%
\section{Background Skills}
%%%%%%%%%%%%%%%%%%%%%%%%%%%%%%%%%%%%%%%%%%%%%%%%%%
\tagline{You need to have knowledge to get knowledge.}

A character's background skills can be literally anything. They represent areas of the game that a character can potentially do legwork in. That is to say that during a chronicle a character may find a clue (such as a strange shape on a video feed from outside a crime scene, a discarded heroin needle, or a tuft of fur), and background skills are methods a character could have to research that clue and gain more information. Background skills are ways for the players to transform story seeds into additional exposition.

Characters can personally know \textit{any} isolated fact or individual person within the context of the story without there needing to be a notation on the character sheet or die roll involved. You don't need to have a background in evolutionary biology to know that humans are closest related to chimps of all the other great apes -- you just need to be told that fact directly or indirectly by someone who \textit{does} have such a background. But to actually evaluate the research, you need to understand the power and limitations of the methods and the kinds of results that have also been achieved. 

Backgrounds are divided into how one interacts with them. \textbf{Academic} Backgrounds are ones in which the character can "go look something up". They often dovetail nicely with the Research, Perception, or Bureaucracy skills. Sciences, ancient languages, classical art, and so forth make good Academic Backgrounds. \textbf{Social} Backgrounds are ones where the character "goes to talk to some people". They often dovetail nicely with Social skills like Empathy or Persuasion. Any social group can and does represent a potential Social Background. \textbf{Occult} Backgrounds are ones that involve the character "going to do something secret". What skills are helpful for this kind of legwork are highly variable, because Occult Backgrounds are a very variable category. Many Occult Backgrounds are literally magical in origin (such as \textit{Marduk Society Histories} or \textit{Tarot Readings}), while others are simply secret for a variety of other reasons (legality, morality, or whatever). The defining point of Occult Backgrounds is that telling other people that you have them jeopardizes your ability to use them. The first rule about ghost cartels is the same as the first rule about fight club.

It is frequently important for purposes of socialization whether or not characters have "the same" background or not. Characters who have the same Background automatically have shared interests that they can talk about. However, game mechanically, Backgrounds \textit{do not} have to have exactly the same name to be "the same." And two Backgrounds that are "the same" in one instance may be "different" in another. A Background is "the same" if in the current instance it covers essentially the same stuff. If one character had \textit{Triads} and another character had \textit{San Francisco Crime} as a Background, the two characters would be on wholly common ground when discussing San Francisco's triad operations, and would be speaking Greek or Martian to each other if the conversation changed to Hong Kong triads or San Francisco's IRA network.

It is up to the MC to determine what constitutes an acceptable Background for the campaign. In general however, it is better to err on the side of Backgrounds that are too useful than ones that are too narrow. While it \textit{is} overpowered for a character to have a Background that applies in virtually (or actually) all circumstances like "Stuff" or "Trivia", the worst thing that's going to happen under such a circumstance is that the player is going to roleplay a lot and move the plot forward. That's not fair to the other characters (unless they are doing the same), but that's still better than the players feeling powerless and having the story stagnate.

%%%%%%%%%%%%%%%%%%%%%%%%%%%%%%%%%%%%%%%%%%%%%%%%%%
\section{Sample Backgrounds}
%%%%%%%%%%%%%%%%%%%%%%%%%%%%%%%%%%%%%%%%%%%%%%%%%%
\tagline{Trust me, I've seen stuff like this before.}

The completely open ended nature of Backgrounds can be paralyzing when it comes to actual character creation. A blank page can be filled up with \textit{anything}, but frequently it isn't. So to help with that, here are some examples of backgrounds that some of the characters from the Persona Non Grata chapter have and some descriptions of things they use them for in their chronicles.

%%%%%%%%%%%%%%%%%%%%%%%%%
\subsection{Social Backgrounds}
%%%%%%%%%%%%%%%%%%%%%%%%%

\hspace{\parindent} \textbf{Circus Life:} Marionette is a former trapeze artist, and this facet of her life is represented with the Circus Life Background. After hearing the kids talking about the mountain lion they saw, Marionette's player points out that she has actually heard a lot of kids describe lions and tigers, and wishes to reconstruct the real appearance from them. After some leading questions (Intuition + Circus Life), she thinks she knows which Bagheera it was. When she goes to the Cannibal Mimes, she falls back on her circus experiences for Friendly Banter ammunition (Charisma + Circus Life).

\textbf{Truckin:} Jack spends a lot of time on the road and on the radio talkin to other truckers, and this is represented by his Truckin Background. This means that he has an encyclopedic knowledge of rest stops all up and down the 101. So when he gets a time frame for when the van presumably lightened its load, he can make a very accurate guess (Logic + Truckin) about where to start looking for the bodies. When he is listening to someone's description of their journey to Walnut Creek, he notices (Intuition + Truckin) a discrepancy in their story.

\textbf{Bar Scene}: Dean goes out drinking frequently to attempt to forget the hole in his soul, and this justifies his Bar Scene Background. The team needs a decoy, so Dean goes off to the SK8R | to go pick up an extra woman for that purpose (Willpower + Bar Scene). Later on, they need to track down a fishy poker game, so he asks around (Charisma + Bar Scene).

%%%%%%%%%%%%%%%%%%%%%%%%%
\subsection{Academic Backgrounds}
%%%%%%%%%%%%%%%%%%%%%%%%%

\hspace{\parindent} \textbf{Chemistry:} Marionette was an accomplished chemist even before her transformation, and thus it is reasonable for her to have the Chemistry background. Looking at the man who died from a Soulless bite, they need to throw the police off the trail. So Marionette throws out a plausible sounding Chemical explanation for the scene cops to eat up and spread as rumor (Willpower + Chemistry). Then when she's looking up the actual poison to produce an antidote, this is very easy for her, because her Background in Chemistry makes her (Logic + Research) test have a very low Threshold.

\textbf{Cars and Trucks:} Jack knows all about things on the roads. So when it comes time to research up the DMV registration on the kidnappers' van, Jack grabs it and gets an answer for certain, because \textit{he} knows that he is looking for a 2005 Charcoal Dodge Sprinter. But later on, he reaches for something to talk to the guys at Pizza Hog about, and cars seems like as good a topic as any. And since there are some other gear heads, it works out and he uses it as a Friendly Banter (Charisma + Cars and Trucks) platform.

\textbf{Ballistics:} Dean may not seem that bright, but he \textit{does} know his way around firearms, and this interest is represented by his Ballistics Background. Not only can he talk for hours to gun enthusiasts about caliber and grains, but he can perform the kind of scientific forensic investigation that a ballistics expert might be called upon for. When he picks up the supposed murder weapon, he immediately notices (Intuition + Ballistics) that something is wrong because the armor piercing bullets in the magazine should have exit wounds on the corpse. Later he measures out the probable point of origin from the bullets and gets a fix on the shooter's location (Logic + Ballistics). And when they have the specs on the enchanted rifle and Dean wants to look up what kind of weapon they are dealing with, he just \textit{does} it because his Background knowledge pushes the Research threshold down to zero.

%%%%%%%%%%%%%%%%%%%%%%%%%
\subsection{Occult Backgrounds}
%%%%%%%%%%%%%%%%%%%%%%%%%

\hspace{\parindent} \textbf{ETA:} Marionette spent some time working with the Basque Separatists, as is reflected in her ETA Background. She calls upon a friend of a friend to get some explosives (Charisma + ETA), and when the report of the "terrorist attack" on the clinic comes out, she is able to go through the details and identify it as a feint (Logic + ETA) because she knows how terrorist attacks work.

\textbf{Chinese Monsters:} Jack has put up with a lot of bullshit from the Eastern wing of the Shattered Empire, and he knows his long haired ghosts from his thundering witches. During the investigations, he meets up with The Peach Lady, and needing something to talk about, he reaches for supernatural stuff from her homeland. This is an acceptable form of Friendly Banter (Charisma + Chinese Monsters), and seemingly gets the immortal beauty to open up to him. Later on, he comes face to face with an Asian Leviathan and identifies it as such (Logic + Chinese Monsters).

\textbf{Hell Mouths:} Dean has been in and out of the Dark Reflection \textit{many} times, and this is reflected in his Hell Mouths Background. As they are searching the house, Dean notices (Intuition + Hell Mouths) that the bathroom mirror has been used as a portal, and fairly recently by the ashen smell. When the team finds itself outside the Iron Tower, Dean falls on his knowledge of hell mouth locations to plan a route back to the mortal world (Logic + Hell Mouths).
