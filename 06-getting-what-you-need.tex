%%%%%%%%%%%%%%%%%%%%%%%%%%%%%%%%%%%%%%%%%%%%%%%%%%
%%%%%%%%%%%%%%%%%%%%%%%%%%%%%%%%%%%%%%%%%%%%%%%%%%
\chapter{Getting What You Need}
%%%%%%%%%%%%%%%%%%%%%%%%%%%%%%%%%%%%%%%%%%%%%%%%%%
%%%%%%%%%%%%%%%%%%%%%%%%%%%%%%%%%%%%%%%%%%%%%%%%%%
\tagline{"Where would I get a gun like that?"}

Society does all kinds of things. It guarantees revenge against those who threaten personal safety and property, it provides inherent services like transportation and communication. And so on. And beyond the things it does generally, society allows specialization of labor, meaning that people can fungibly transform the products of their labor into goods and services produced by people with radically different skill sets. It's pretty awesome. But when characters are out there doing things on the far side of the Vow of Silence, they may not be able to count on any of that. It's not like you can report a werewolf attack (at least, not and keep your end of the Vow of Silence deal), so many of the guarantees of mortal society are rather difficult for players in After Sundown to take advantage of. Intrigue being what it is, the players may not be able to trust analogous structures in supernatural Syndicates. Further, characters are going to want access to goods that are highly restricted in sane society, and they are going to want these things without having the police chat them up about why they need heavy explosives or a gun that shoots silver swords.

Very often players will be in situations where the information, goods, or services they want are not available on e-Bay (at least, not in any recognizable form). And in such cases it often falls to going and getting it out of society yourself. Depending on the needs of the story, this is either done through a \textit{montage} scene or through an \textit{interview} scene. In an interview, the character speaks directly to an NPC and tries to get something they want (for example: consider the scene where Rorschach is interrogating Moloch or the scene where Indiana Jones is asking for help from his friend Sallah). The interview format is appropriate when you want to do sentence by sentence roleplay and/or when there is a single pivotal NPC who is the source of the needed information or goods. On the flip side, the montage is a scene made up of several short linked scenes where the characters are talking to different people or doing different things. It is the equivalent of the Training Sequence for getting a character access to new knowledge or equipment.


%%%%%%%%%%%%%%%%%%%%%%%%%%%%%%%%%%%%%%%%%%%%%%%%%%
\section{Keeping Things Quiet}
%%%%%%%%%%%%%%%%%%%%%%%%%%%%%%%%%%%%%%%%%%%%%%%%%%
\tagline{What happens in Vegas, Stays in Vegas.\\
Welcome to Compton. No Snitches.}

It is entirely reasonable that one might want to suppress information about something. Maybe pass around some hush money or threats, or simply hunt down the relevant information and delete it from archives. This makes things more difficult for people trying to find the information that you are trying to conceal -- at least, it does if you do it right. The number of ways one could potentially go about suppressing things is innumerate, and from the standpoint of investigators it often doesn't \textit{really} matter that much. As soon as you mention \textit{Mr. Jones}, the bar tender just clams up, and you don't really \textit{care} whether the head of the criminal organization threatened peoples' families or offered a standing bounty on "not snitching" or whatever.

From a practical standpoint, suppressing information is a threshold 2 check. Net hits increase the threshold to find things out about the suppressed topic. What test is actually made to do that varies based on what is being suppressed and how one is going about doing it. \dicepool{Logic + Bureaucracy} might be used to forge civic records, while \dicepool{Willpower + Expression} might be used to propagandize people to not talk about something with outsiders. A character can even psyche themselves up to resist an interrogation, generally with a \dicepool{Willpower + Intimidate} check.

These sorts of things can backfire spectacularly. If the character fails to get 2 hits, their shortfall is actually applied as a \textit{reduction} in the threshold of future investigators. If you just yoink the tax records of the people you are trying to make vanish, it actually makes them stick out like a sore thumb because they have obvious shenanigans going on with their tax records.

%%%%%%%%%%%%%%%%%%%%%%%%%%%%%%%%%%%%%%%%%%%%%%%%%%
\section{Asking Around: Montages} \index{Montage}
%%%%%%%%%%%%%%%%%%%%%%%%%%%%%%%%%%%%%%%%%%%%%%%%%%
\tagline{20 minutes later\ldots{}}

Montages are used for situations where the relevant action takes place over a time period that is longer than is interesting. They are therefore a concern primarily of \textit{pacing}. If the action would take more time at the table to resolve than it is worth for how important or interesting it is, use a montage. If the it is interesting, important, or just plain \textit{short} enough skip the montage and roleplay out the activities and dialogue.

The difficulty Threshold of a Montage is based on how difficult it is to get a piece of information or item by the route you are attempting in the montage. So for example, the Threshold to find out what gang wears purple bandanas is going to be pretty low (1 or 2) if you're asking a bunch of hooligans at  SK8R |, and it's going to be pretty high (3 or 4) if you are asking around the country club. In most cases, a montage will take place with a brief description of the endeavor by the player with possibly some interjection by the MC, followed by rolling dice and subsequently the MC divulging what is learned and/or gained. Roleplaying the consequences should usually begin from that point, having at least a small scene played out in full before jumping into another montage.

%%%%%%%%%%%%%%%%%%%%%%%%%
\subsection{Formal Request Montage}
%%%%%%%%%%%%%%%%%%%%%%%%%
\tagline{"Be sure to attach your TPS form to the front of that."}

When a character wants to get something out of an organization it can seriously take a long time. A lot of forms may need to be filled out, appointments made, plans explicated, needs justified, and who knows what all else. This can be done as a montage, the potentially vast stretches of time between one form submission and the next appointment can be wiped away with literal screen wipes, possibly cutting to the exasperated faces of the characters or a time/date subtitle. But for all the time involved, the legal systems of mortal government or supernatural Syndicates both are incredibly useful sources of information, resources, and action. In the case that the story is \textit{about} navigating through bureaucracy (perhaps you are redoing \refwork{The Castle} or some other story set in Czech Republic), it is probably better to handle these situations as a series of interviews that are tied together with Bureaucracy checks.

Dicepools of these kinds of requests are usually \dicepool{Logic + Bureaucracy}. Organizations are usually limited in what they are capable of delivering through these methods. It's also very useful at times to use a Formal Request as an entrance requirement to a scene that will be roleplayed in more detail. For example, the military won't just \textit{give} you old school flame throwers, but you can get yourself a closed door meeting with an inventory officer who you could attempt to persuade or coerce into arranging for the goods to be delivered.  A formal request is likely to be poorly received if the character delivering it is not familiar with the topic or organization. If someone lacks an appropriate background, raise the threshold by a point or two.

%%%%%%%%%%%%%%%%%%%%%%%%%
\subsection{Burglary Montage}
%%%%%%%%%%%%%%%%%%%%%%%%%
\tagline{"I would like to triangle button a car."}

Civilization in general, and big cities especially, are full of stuff that people "own". And we use quotation marks around that concept because there's generally no magical markings on objects that tie them to their recognized owner. The recognition of ownership is only \textit{really} acknowledged by society and the social contract. People are only able to put things down with the expectation that those objects will still be there waiting for them in the future because promises of retribution have been made on each person's behalf by the nation. And you know what? All that really doesn't even apply when the city is being invaded by alien plants or you happen to be a monster who doesn't give a rat's ass anyway. Very often a player will be in the situation of wanting to use some property that is defined as belonging to someone else. In this case it's often useful to just have the character come from offscreen with the appropriate object that has been taken from a home, car, or storefront. 

Dicepools of these kinds of requests are usually \dicepool{Agility + Larceny}. You can't actually steal stuff that isn't nearby. And if it isn't just sort of "around" but available only in specific limited quantity possessed by people who matter to the story, then you should probably play it out as an action scene. So while a character can go "hotwire a car" offscreen, it's usually inappropriate for a character to steal "Fangorz's Bentley" without devoting genuine story scenes to the action of finding it, breaking in, and hotwiring it up. It is also worth noting that going off and stealing shit is almost by definition "illegal" and that may matter depending upon where you are.

It is important to note that the difficulty of stealing crap is based on how difficult it is to get to the stuff, \textit{not} on how valuable that stuff is. For example, stealing blankets is really difficult because they are inside locked houses with people actually sleeping on them, while stealing cars is comparatively easy (at least if you aren't after the expensive cars that people keep in guarded locked facilities) because people park their cars \textit{outside} in \textit{plain view}. It's also important to note that just because you stole something fair and square doesn't mean that the universe now recognizes your ownership of it.

%%%%%%%%%%%%%%%%%%%%%%%%%
\subsection{Social Interaction Montages}
%%%%%%%%%%%%%%%%%%%%%%%%%
\tagline{"Honestly, I just want the rifle. I don't really care who it comes from."}

Sometimes the character is going to be doing some combination of skulduggery and schmoozing, but the direction of the story really doesn't call for a scene to be extensively roleplayed. This is often the case in a situation where many NPCs are going to need to be talked to, the actual effects of the scene are relatively minor, and/or negotiations involved will simply take a long time. Players and MCs alike can be inclined to "get on with it" rather than haggling through the location and procurement of an unlicensed firearm or finding a gang member who has seen the Red Ghost. In these cases it is often best to just skip to the montage and roll dice.

Dicepools of these kinds of requests are usually \dicepool{Charisma + Background}. Even propelled by narrative imperative as they are, the player characters are not going to be able to get information or objects that literally do not exist. None of the ruffians at the pool hall have a magic sword or know the true name of the Mask of Envy. They just\ldots{} don't.

%%%%%%%%%%%%%%%%%%%%%%%%%
\subsection{Research Montage}
%%%%%%%%%%%%%%%%%%%%%%%%%
\tagline{"Is the world ending? I have to research a paper on Bosnia for tomorrow, but if the world's ending, I'm not gonna bother."}

It is often important for a character to go look something up. This may simply be something that is sufficiently arcane, convoluted, or obscure that even experts have to look it up, or it may be something that the character simply does not happen to know. A montage of this sort will generally appear as just a few frames of a computer screen reflected in the character's glasses or even skip to the character walking out with a relevant book open to the correct page.

Dicepools of this type of request are usually \dicepool{Logic + Research}. A character's Background skills are absolutely vital in these types of requests, and their relevant Background is subtracted directly from the threshold. Someone with a strong background in physics is never going to \textit{fail} to look up the mass gravitational constant even though they probably have to look it up in the first place. Remember also that just because you have the background knowledge and the research chops needed to hone in on the information you want, it doesn't follow that the information you want is actually available. A local library may well have the genealogical data needed to show that Old Man Withers is actually an immortal, but it probably doesn't have any book anywhere in it that will tell you how to get to the Oaken Abyss or the Eye of Despair. 

%%%%%%%%%%%%%%%%%%%%%%%%%%%%%%%%%%%%%%%%%%%%%%%%%%
\section{Interviewing People} \index{Interviews}
%%%%%%%%%%%%%%%%%%%%%%%%%%%%%%%%%%%%%%%%%%%%%%%%%%
\tagline{"One does not just walk into Mictlan, Mr. Anderson."}

Characters are going to be in situations where they are talking to NPCs. It's actually most of the game when you consider actual devoted screen time. In many cases, the interaction with the NPCs can be mostly freeform roleplaying because what you're \textit{mostly} after is exploring the personality of the character and elucidating their connections to the people around them. But there come times when your character will want to get specific information or assistance from the NPC that they are talking to. And because After Sundown \textit{is} a game, it falls here to roll some dice. What dice a character rolls depends largely on what they are doing and what role they are playing in the conversation. The Threshold depends upon how much the NPC in question wants to give it to you. If the NPC really wants to give it to you, such as trying to buy drugs from a drug dealer or trying to get information about someone who has wronged them recently, is not especially difficult (Threshold 1). If the NPC wants to keep it to themselves, for example the information might get them in trouble, it very much is (Threshold 3 or higher).

%%%%%%%%%%%%%%%%%%%%%%%%%
\subsection{Questions of Interrogation}
%%%%%%%%%%%%%%%%%%%%%%%%%
\tagline{"Where were you on the night of the 29th?"}

Sometimes a character is in a situation where they have a fair amount of power over the NPC and can ask the really rude questions. In these situations, the character can indeed do that. This is no more likely to get a coherent, accurate, or helpful answer than doing it in some round-about-polite manner, but it is generally \textit{short}. Also, if you ask someone a simple question it is generally much easier to see if someone is evading it than if you ask a complex one.

Dicepools of these kinds of questions are usually \dicepool{Logic + Intimidate}. Even if you don't get the success you needed to get the NPC to spill the beans, you can derive substantial information from the questions that they didn't answer or gave vague/contradictory answers to. If asked about something that the NPC is trying to keep secret and the character's number of hits is insufficient by 1 or 2 they will become alerted that the NPC is specifically holding out on them. If the character lacks familiarity with the relevant subject, it becomes easier for information to be withheld, and they only become alerted to deception if the test falls short by 1.

%%%%%%%%%%%%%%%%%%%%%%%%%
\subsection{Questions of Subterfuge}
%%%%%%%%%%%%%%%%%%%%%%%%%
\tagline{"That's fascinating. How did the boss respond to that?"}

It is often the case that the thing a character is looking for is for one reason or another not something that they wish to give away. If a character is engaged in conversation and attempting to lead it to the revelation of key secrets or whatever, they may attempt subterfuge. The character drops hints like a trail of breadcrumbs, and \textit{ideally}, the target responds by crawling along to the conversation's destination and makes the big reveal. Obviously, this only works if for some reason the other person is voluntarily talking already. Normally this is because the character and the target share some Background and are having an actual conversation about something inane while the subterfuge is taking place. But in some cases there can be an otherwise unconnected business deal going on. For example, if a character is opening an account at a bank, the manager pretty much has to talk to them even if they share no interests whatsoever.

Dicepools of these kinds of questions are usually \dicepool{Charisma + Persuasion}. Subterfuge lays a verbal minefield for the user as well as the target, and it is entirely possible that the character will give themselves away. The target is entitled to an Intuition + Empathy test. If they get more hits than the manipulating character got on their test -- they will catch on to the fact that they are being played. Whether they care or not (and how they respond if they do), depends entirely on circumstances. But having the discussion terminated is a pretty common reaction.

%%%%%%%%%%%%%%%%%%%%%%%%%
\subsection{Friendly Banter}
%%%%%%%%%%%%%%%%%%%%%%%%%
\tagline{"You wouldn't happen to know where I could score some meth, do you?"}

When you are in a regular social situation, you can actually just ask people stuff without having it get all weird. On the plus side, people just tell you stuff and there is no chance of being "found out" and you didn't step on anyone's feelings. Of course, if someone is asking for something that the person doesn't approve of, that lack of approval may well get transferred to the asking character. And of course, blunt questions can be easily overheard.

Dicepools of these kinds of questions are usually \dicepool{Charisma + Background}. You can only use a Background that is appropriate to the situation and shared by the target. Friendly Banter can also be used to just plain \textit{make friends}, and is quite invaluable in that respect.

%%%%%%%%%%%%%%%%%%%%%%%%%
\subsection{Testing the Waters}
%%%%%%%%%%%%%%%%%%%%%%%%%
\tagline{"Some people were pretty excited last night, any idea what that was about?"}

For reconnaissance, it is sometimes merely important to find out \textit{if} someone knows about something rather than specifically what they know. This can be especially true if the character already knows a piece of information and merely wants to know how far the information has spread. In any case when the character is trying to passively identify who knows about a subject without literally broaching it, the character is said to be Testing the Waters.

Dicepools of these kinds of questions are usually \dicepool{Intuition + Empathy}. The primary disadvantage of Testing the Waters is that it doesn't usually give you the answers you are looking for -- it just tells you who \textit{has} those answers. On the plus side, if you do it professionally (Threshold 2) you don't give away what you were looking for, and if you do it crazy extremely (Threshold 4), people don't even realize you were inquiring about anything.

%%%%%%%%%%%%%%%%%%%%%%%%%
\subsection{Impersonation}
%%%%%%%%%%%%%%%%%%%%%%%%%
\tagline{Getting in is easy. Getting it done is hard.}

In the old days, pictures were hard to come by of even the most important people. With the advent of the digital camera and internet pornography that is no longer true. Nevertheless, there exists a number of ways in After Sundown to appear to be a different person in a way that will pass even the most thorough visual inspections.

Dicepools of these kinds of questions are either \dicepool{Charisma + Persuasion} or \dicepool{Willpower + Persuasion}, depending upon the character's chosen demeanor. Impersonation is opposed by the target's \dicepool{Intuition + Empathy}. If the target has an appropriate Background skill and the character does not, the target can add their Background skill to their test. Only a target who has some reason to know what the impersonating character should be behaving like gets to make a test at all. If a character disguises themselves as former Australian Prime Minister John Howard, a group of American or Asian targets will probably get no test to pierce the impersonation, because they have no idea what kind of mannerisms or opinions the real John Howard has.

%%%%%%%%%%%%%%%%%%%%%%%%%%%%%%%%%%%%%%%%%%%%%%%%%%
\section{Persuasive Argumentation} \index{Argumentation}
%%%%%%%%%%%%%%%%%%%%%%%%%%%%%%%%%%%%%%%%%%%%%%%%%%
\tagline{"Those dogs and you make a very compelling point."}

A persuasive argument is one where the speaker has a goal in mind for something that he wants to convince his intended audience of. The intended audience may be the person being spoken to, or in the case of a debate may be one or more people \textit{observing} the discussion. Classically speaking there are three branches of discourse (Grammar, Logic, and Rhetoric), but in modern era basic grammar is \textit{assumed} and one's argumentation is made from some combination of Logic and Rhetoric. It is common in modern discourse to pretend that one's arguments are founded entirely in "Logic" but this is horse shit -- literally nothing more than a Rhetorical technique. People on the Internet who don't have a great argument themselves will \textit{often} spout off about how this or that is a "fallacy," but all that means is that they have found (or claimed to have found) a part of an argument that is Rhetorical rather than Logical -- or even just that it uses Inductive Logic rather than Deductive Logic. And even if that's true, it doesn't actually say anything one way or the other about whether the argument in question is good or if its conclusions are solid.

Regardless of the method used to argue a subject, the difficulty of convincing an audience depends entirely on how much the audience \textit{wants} to believe what is being said. Many holy men and talk-show hosts are actually rather incompetent public speakers, but they don't know that because their usual audience is so predisposed to believe whatever they say that they frequently don't need to say anything to get applause and agreement. It is also entirely possible for different members of the audience to have different predispositions, meaning that the same argument can persuade some onlookers and not others.

\begin{table}[htb]\center
\caption{Audience Difficulty} \rowcolors{1}{white}{tan}
\begin{tabular}{c l}
\textbf{Threshold} & \textbf{Audience Predisposition} \\
\textbf{1} & Already Believes \\
\textbf{2} & Wants to Believe \\
\textbf{3} & Receptive \\
\textbf{4} & Skeptical \\
\textbf{5} & Hostile \\
\textbf{6} & Uninterested \\
\end{tabular}
\end{table}

When two or more people make an argument that successfully persuades audience members of different -- even contradictory -- positions, audience members are persuaded in both directions. People are entirely capable, even seemingly eager, to believe completely incompatible things simultaneously. What individual audience members will end up doing in such a case will vary depending on their personality and goals. Some will flip a coin, be paralyzed by indecision, or go do their own research. But if one of the presentations got more net hits, that position has a noticeable advantage.

%%%%%%%%%%%%%%%%%%%%%%%%%%%%%%%%%%%%%%%%%%%%%%%%%%
\section{Arguments From Reason}
%%%%%%%%%%%%%%%%%%%%%%%%%%%%%%%%%%%%%%%%%%%%%%%%%%
\tagline{"Given what we know about mass acceleration, your figures seem\ldots{} unlikely."}

A deductively logical argument is one in which the premises contain the conclusions. In short, any deductively logical reasoned argument is a form of \textbf{circular reasoning}. Deductive Logic is helpful only in showing the implications of ideas. Inductive Logic can tell us more about the world, but it carries with it the possibility for error -- black swans bite as hard as geese. While it is technically correct to label arguments of Inductive Logic as "fallacies," the fact is that most peoples' personal epistemology does not distinguish between "True" and "Almost Certainly True". I mean seriously, you can't deductively prove that you aren't in The Matrix right now, but how many readers take Solipsism seriously enough to allow themselves to be beaten with a chair?

Argumentation from Reason requires a set of common ground at some point. For people who are coming to the table with radically different precepts, a Reasoned Argument must be scaled back several layers until commonalties in acknowledged premises can be found. This can be quite a shock to characters dealing with radically different cultures and creatures. Imagine an Imam attempting to make a Reasoned Argument to a Buddhist who won't concede that Allah even \textit{exists}. Now take it a step farther imagine an environmentalist trying to make a Reasoned Argument to a Makhzen Vampire who does not even breathe air and cannot be poisoned. Game mechanically, if a test uses a Background skill, the test is wholly ineffective as a Reasoned Argument on any audience member who does not have that Background themselves. An argument of this sort may still be appreciated as theater, and at the MC's discretion may also be rolled as a Rhetorical Appeal for the laymen in the audience. An argument from Reason is called a Contention.

%%%%%%%%%%%%%%%%%%%%%%%%%
\subsection{Contention of Details}
%%%%%%%%%%%%%%%%%%%%%%%%%
\tagline{"\ldots{}which is naturally why in early March of 1649 there were already calls by Winstanley to repeal the property restriction on voting for parliamentarians\ldots{}"}

A Contention of Details is essentially an attempt to establish one's own credibility as an authority on the subject of dispute. The character presents a large and intricately linked body of facts in or related to the relevant subject that are demonstrably or apparently true. By listing off a large number of facts, the character presents themselves as a reliable source of information on the subject, such that their declaration of their disputed point becomes reliable by association. This is a form of Inductive Logic with the format "Everything I have said about sewer maintenance in general is consistent with me knowing what I am talking about with regards to sewer maintenance. Therefore what I say regarding the maintenance of this particular sewer is likely to be correct." This is essentially the same format as the argument that the sun will rise tomorrow, so properly constructed it can be pretty persuasive.

Dicepools of these kinds of Contentions are usually \dicepool{Logic + Research}. The character must have a relevant Background skill to attempt such a Contention. A Contention of Details takes a significant amount of time to get to the point, and is confronted with increases in threshold to convince audience members who are in a hurry. Also, the entire strategy is weak to attacks on the person's character. An opposing debater gains bonus dice if they choose to go the low road.

%%%%%%%%%%%%%%%%%%%%%%%%%
\subsection{Contention of Disagreement}
%%%%%%%%%%%%%%%%%%%%%%%%%
\tagline{"That's preposterous! Everyone knows sea turtles sink when tampered with."}

A Contention of Disagreement is an attempt to steal the spotlight from another speaker or an assumed paradigm of thought in the audience by finding fault with premises or connections made between premises to invalidate the conclusions. While technically this only deductively proves that the original argument being attacked is \textit{invalid}, it inductively suggests that it is also wrong. And more tenuously (but still persuasively) suggests that the character's ideas are better.

Dicepools of these kinds of Contentions are usually \dicepool{Intuition + Background}. The Contention of Disagreement literally \textit{requires} an opposing argument to disagree \textit{with}. In the absence of real or known opposition, the character can construct an opposing argument (called a "straw man") -- but this is regarded as dirty pool in many circles and may make skeptical observers hostile. It is also weak if the original argument was crafted especially well, following up a strong oration may well leave the character with a dicepool penalty.

%%%%%%%%%%%%%%%%%%%%%%%%%
\subsection{Contention of Reference}
%%%%%%%%%%%%%%%%%%%%%%%%%
\tagline{"The market exists when information exists for producer and consumer, and both producer and consumer can choose to buy or sell a given product at a given price. And when no such choice exists?"}

A Contention of Reference is one of the few lines of argumentation that are \textit{not} technically fallacious. Unfortunately, the internet being what it is, people will accuse you of fallacy anyway. It's not a whole lot different in the forums of the Covenant. Such is life. And unlife. The idea is that you take a set of premises that the audience believes that they agree with, and then you extract implications from them with a seemingly acceptable argument. Ideally, the implication is one that the audience hasn't thought of or disagrees with. But hey, sometimes it's for whatever reason important or useful to convince people that something they \textit{think} is true \textit{is} true. Referential Contentions are good for that.

Dicepools of these kinds of Contentions are usually \dicepool{Logic + Background}. A Contention of Reference is very weak if the audience has wildly different acceptable premises from the speaking character, and may well be saddled with increases in threshold.

%%%%%%%%%%%%%%%%%%%%%%%%%
\subsection{Contention of Validity}
%%%%%%%%%%%%%%%%%%%%%%%%%
\tagline{"That's a good question. The short answer is yes, but that brings me to my next point\ldots{}"}

A Contention of Validity is an attempt to demonstrate the character's mastery of the subject and reliability as a source by responding quickly, effectively, and verifiably truthfully; thereby inductively giving support to their other ideas. Fallacy hunters will note that this frequently counts as a "non sequitur" in that a correct response to one question does not necessarily imply a correct statement next time. But seriously, it works pretty well. Some speakers like to spike the audience with shills to throw out softball questions, and others don't. 

Dicepools of these kinds of Contentions are usually \dicepool{Charisma + Background}. A Contention of Validity is incredibly weak in the absence of viewer participation. While the character could write themselves a dialogue where they were asked questions by an imaginary audience, the persuasive effect is much dimmed, provoking all but the most devoted into skepticism. Having people ask tougher questions is a double edged sword. If the character can field them (which may, at the MC's discretion require an \dicepool{Intuition + Background} test), a small bonus may be in order. If the character fails to field them, their Contention pretty much falls apart right there. If audience members notice shills offering softball questions, dicepool penalties or threshold increases should be awarded.

%%%%%%%%%%%%%%%%%%%%%%%%%%%%%%%%%%%%%%%%%%%%%%%%%%
\section{Arguments From Rhetoric}
%%%%%%%%%%%%%%%%%%%%%%%%%%%%%%%%%%%%%%%%%%%%%%%%%%
\tagline{"I'm sorry, I can't hear you over the sound of how awesome I am."}

When you want to talk to someone and you don't have something to talk \textit{about}, you still have someone to talk \textit{at}. Even if you or your audience is too ignorant to even understand relevant premises to agree or disagree with, you can still persuade people that your conclusions are correct. The techniques generally used to persuade in such situations are completely "illogical" in that the desired conclusions do not follow deductively \textit{or} inductively from any shared premises, because there \textit{are} no relevant shared premises. It falls to the user therefore to generate approval by some other means.

A solid example for After Sundown is getting people to evacuate in the face of an incoming Zombie uprising. Sure, as a supernatural creature your character \textit{knows} that Zombie uprisings happen, and they \textit{know} that the humans in the small sleepy town in its path are little more than food for the walking dead. Were your characters explaining the situation to another supernatural creature, it would be entirely reasonable to have a reasoned discussion about the best way to get everyone out before the Soulless arrive and how to deal with the Zombies themselves. But that's not going to fly for the humans on the other side of the Vow of Silence. They don't know that Zombies exist, they don't share any of the premises required to even \textit{begin} that debate of Reason. Your argument needs to be something else, some kind of tangent or lie that will hopefully convince people to leave in the absence of understanding or believing the true reasons involved.

A severe disadvantage of Rhetorical argumentation is that it is generally ineffective on people who "know better". When a character has a relevant background skill, they are unlikely to be persuaded by Appeals of any kind. Braveheart style speeches are great for the peasants, but people with a background in logistics would rather just see the numbers.

%%%%%%%%%%%%%%%%%%%%%%%%%
\subsection{Appeal to Authority}
%%%%%%%%%%%%%%%%%%%%%%%%%
\tagline{"\textbf{Who} is Prince of this city?!"}

Drawing upon one's own gravitas and imparting ideas as truth is a powerful way to inspire belief in those ideas. By giving orders or information directly, the audience can be made to believe. This is how people lead troops, give commands, or even just teach new subjects. These are good ways to motivate people to do things, but require a substantial amount of trust from the audience to work at all.

Dicepools of these kinds of Appeals are usually \dicepool{Willpower + Tactics}. An Appeal to Authority is basically useless if the character \textit{has} no recognized authority. Having a lot of recognized status is a good start, but it can be made when the character is merely in a position where they are circumstantially supposed to be listened to (such as a teacher in a class or even a team leader in a group project). As with most Rhetorical Appeals, an Appeal to Authority is poor at convincing people who are familiar with the topic if you are not. If someone has an appropriate background, and the character does not, raise the threshold by a point or two. An Appeal to Authority plays to the character's strengths, if the character is in a position of weakness, they suffer dicepool penalties.

An Appeal to Authority can be used to boost morale. Net Hits can cancel morale penalties or provide bonus dice to overcome fear.

%%%%%%%%%%%%%%%%%%%%%%%%%
\subsection{Appeal to Babble}
%%%%%%%%%%%%%%%%%%%%%%%%%
\tagline{"Iknowwhatyou'rethinking. Moreevil?Morepower?WheredoIsign?! Butwait. There'smore\ldots{}"}

Putting a lot of ideas out there in an avalanche of vaguely supported theses is a great way to at least temporarily convince someone of the veracity of an idea. With one concept following another in brutally rapid succession it can be difficult for onlookers to see the gaps in slippery slope arguments, disjointed statements and inadequately linked proof. This avalanche of text can easily overwhelm an onlooker, and take some amount of time to pick apart. A person with a strong personality can fast talk through any subject with no preparation whatsoever simply by letting loose the text avalanche.

Dicepools of these kinds of Appeals are usually \dicepool{Willpower + Persuasion}. An Appeal to Babble is inherently vulnerable to time. If someone is given time to prepare a reply, they get bonus dice on any Contention or Appeal they make, and a large pile of bonus dice should be awarded to a Contention of Disagreement. Even audience members convinced by babble may well have their affects wear off over time as they have time to think about it. Successfully fast talking someone is generally only good for a few minutes, though Net Hits increase this timeframe. As with most Rhetorical Appeals, an Appeal to Babble is poor at convincing people who are familiar with the topic if you are not. If someone has an appropriate background, and the character does not, raise the threshold by a point or two.

%%%%%%%%%%%%%%%%%%%%%%%%%
\subsection{Appeal to Emotion}
%%%%%%%%%%%%%%%%%%%%%%%%%
\tagline{"Think of the children!"}

An Appeal to Emotion is where you attempt to provoke a visceral response from your audience. Usually the most motivating ones are fear, sadness, and anger, but there is a lot to work with. The idea is that the character does something, shows something, or describes something that provokes an immediate emotional response, and then presents an option to do something about it. Like a kettle under pressure, the audience hopefully rushes towards the presented option, creating and filling a need.

Dicepools of these kinds of Appeals are usually \dicepool{Willpower + Empathy}. While generally speaking, most people respond to the same strings on their heart (children, country, the locally appropriate gods, food), the fact is that is just \textit{most} people. If you choose to make a case about something that your audience generally doesn't care about, you straight up fail. Furthermore, people who are informed on the subject matter are unlikely to be swayed by purely emotional appeals on the grounds that they probably already have opinions about what should be done. If someone has an appropriate background, raise the threshold by a point or two. If someone has an appropriate background and the character making the Appeal does not, do that twice.

%%%%%%%%%%%%%%%%%%%%%%%%%
\subsection{Appeal to Force}
%%%%%%%%%%%%%%%%%%%%%%%%%
\tagline{"We have normal rooms, and for a bit more we have luxury suites\ldots{}"\\
"We would like 'free rooms' because the zombies are going to destroy this entire city and we're the only ones who can do dick about it."}

An Appeal to Force is the suggestion that something horrible will happen to the people being addressed if they don't accept the point or accede to the demands. An Appeal to Force can be given as a threat or a warning, because those are really the same thing with very slightly different emphasis. Threats are \textit{more} insulting than warnings, and in many cases actually illegal. Appeals to force usually work better if the audience has actually seen evidence of whatever is being threatened. But they can be delivered quite successfully with pure innuendo. Some of the greatest scare tactics in history have been based on terrorists, communists, or other foreign combatants that may well have not even existed.

Dicepools for Force based arguments are generally either \dicepool{Strength + Intimidate} (for personal threats) or \dicepool{Willpower + Intimidate} (for impersonal threats). Yes, while strictly speaking a gun is pretty much exactly as dangerous in the hands of a small man as it is held by a giant ogre, the fact remains that bigger people are \textit{always} scarier (all other things being equal). It's not fair and it doesn't make sense, but that is how people react. The threshold to convince someone with an Appeal to Force is generally the higher of their Strength or Willpower. And yeah, that means that it's basically impossible to threaten Trolls into doing anything even though there are a lot of things in the realm of horror that \textit{do} pose a real threat to them. Again, that's how psychology tends to work out. An Appeal to Force is a Rhetorical Argument, but it actually doesn't usually get penalized from audience members having relevant backgrounds. The exception is if you're basically bluffing: if you're threatening the audience with something that is essentially not threatening and your audience knows (or believes) that, expect threshold increases.

At the MC's option, a genuine demonstration of the power of the threat, such as using mighty magic to burn a Troll into fine white ash in front of the other Trolls, may convert the threshold of your Appeal to Force to a normal value based on the apparent danger to each of the onlookers. It is important to remember that if you \textit{try} to intimidate someone and fail, you've pretty much made yourself an enemy. You may make them your enemy even if you \textit{do} succeed in pushing them around -- once they get enough backup that for good or ill they aren't afraid anymore, you're still the asshole who threatened them.

%%%%%%%%%%%%%%%%%%%%%%%%%
\subsection{Appeal to Style}
%%%%%%%%%%%%%%%%%%%%%%%%%
\tagline{"Of course you should go with me, have you seen my awesome hat?"}

An Appeal to Style is a persuasive argument that consists of a series of catchy slogans or verbal barbs. They are generally considered to be non sequitur arguments (at best) when viewed through the lens of logic, but that doesn't mean that they aren't persuasive or worth making. Presentation is everything, but in Appeal to Style it is also the \textit{only} thing.

Dicepools for style based arguments are generally \dicepool{Charisma + Expression}. An Appeal to Style is culturally specific. What presents as some catchy "Yo Mama Jokes" in a back alley confrontation is nothing more than a severe breach of etiquette in a court room. A character without an appropriate Background fails automatically. Appeals to Style also tend to fall apart under close examination, and if someone is given the chance to mount a counter argument with substantial time they can gain a bonus of 1-3 dice if they use an Argument From Reason. However, since a Stylistic Appeal is devoid of facts, there is no penalty whatsoever for arguing a case that is counter-factual.

%%%%%%%%%%%%%%%%%%%%%%%%%
\subsection{Appeal to Insults}
%%%%%%%%%%%%%%%%%%%%%%%%%
\tagline{"Ad Hominem? I'm surprised you can add small integers!"}

An Insulting Appeal is a persuasive argument where you simply verbally attack your opponent or people who disagree with your position, rather than necessarily making a case for your own position. Belittling people in public doesn't just feel great, it also makes you appear to be socially dominant. And by extension, in the right. Sure, people will \textit{say} that making fun of people is no reason to think your position is the better one -- and that's true. But it's not a Reasoned Argument, so who cares? The entire \textit{point} of an emotional appeal is to bypass reason entirely, and nothing does that faster than an attack -- even one made out of wit.

Dicepools of these kinds of Appeals are usually \dicepool{Charisma + Intimidate}. One of the great advantages to insults is that there is absolutely no need to know what you are talking about, since you aren't attempting to abide by culturally relevant politeness, nor are you engaging in meaningful dialog. As such, not having an appropriate Background is no hindrance \textit{at all} when engaging such an Appeal. What \textit{is} a problem is using such an Appeal against someone who is respected. Any audience member who respects the status of the targeted opponent will react as if you had scored less hits. And if they respect that status a lot, the amount less hits will be likewise a lot. It doesn't really matter how funny it is, you aren't getting anywhere mocking the Prince in front of most Mehtar Councils.

Insults also need someone to target. You can't just insult ambiently, you need to insult \textit{someone}. And while it can be an effective rhetorical tool to insult people who \textit{aren't there} (on account of them not being able to defend themselves), you still need to target someone or something every time. And whenever you do that, you risk alienating the subjects of your scorn and everyone who likes them. Of perhaps more importance, is that by engaging in Insults yourself, you leave yourself open to insults in return. Even passive-aggressive insults. Other characters can Insult you in return without it being readily apparent that they are doing so. They can go on about how you're lowering the standard of discourse and so on, and in all ways make an emotional appeal of their own. But despite the fact that they aren't engaging in rational discourse themselves or displaying any knowledge of any relevant topics, there are no repercussions save for the character who took the low road first.
