%%%%%%%%%%%%%%%%%%%%%%%%%%%%%%%%%%%%%%%%%%%%%%%%%%
%%%%%%%%%%%%%%%%%%%%%%%%%%%%%%%%%%%%%%%%%%%%%%%%%%
\chapter{Persona non Grata}
%%%%%%%%%%%%%%%%%%%%%%%%%%%%%%%%%%%%%%%%%%%%%%%%%%
%%%%%%%%%%%%%%%%%%%%%%%%%%%%%%%%%%%%%%%%%%%%%%%%%%
\tagline{Some things you're better off not knowing about. Some objects, too.}

The realm of horror is a pretty big place, and it's full of crazy crap. And lots of that stuff is unique. So while the rules make a good starting point for the creation of characters, they are only that: a \textit{starting} point. Each individual person and twisted tale of tormented tools is going to be -- at least ideally -- crafted \textit{individually}. This chapter contains some samples that have been made ready to use in the game, as well as some pointers on how to go about making your own.

It is important to note that the various nouns selected here are merely \textit{examples}, and while it is fine to grab them for use in your own campaign, it is well within the prerogative of the MC to modify them or write them out entirely. People in the world of After Sundown believe in these people, places, and things in roughly the same way that people believe in Madison Wisconsin or the President of the United States: they probably have never been there or met them \textit{personally}, but their existence doesn't seem implausible and other people talk about them as if they were real. However, on the far side of the Vow of Silence it is notably difficult to go check C-Span or the equivalent. The normal clues that someone is \textit{enlarging} upon a tale or a description of a person simply don't apply when discussing a supernatural creature -- when they got angry maybe they \textit{literally} caught fire. So it's pretty easy for a story to get exaggerated after multiple repetitions. The Wolf Mother of Ergenekon \textit{probably} wasn't really 6 meters long (although she may have been). 

%%%%%%%%%%%%%%%%%%%%%%%%%%%%%%%%%%%%%%%%%%%%%%%%%%
\section{People}
%%%%%%%%%%%%%%%%%%%%%%%%%%%%%%%%%%%%%%%%%%%%%%%%%%
\tagline{No you should've stayed out of my way. Do not test me, 'cause I'm the fucking king of the world!}

Making NPCs is likely the part of storytelling that the MC will be called upon to do the most. Many times, players will take the action to some place unexpected. Maybe they will have a throwdown in a bowling alley, or decide to follow up on some minor rumors about the docks. It is a good idea to have a cast of characters who can be used with modest adjustments in lieu of other characters who suddenly become important. Some of this can -- and should -- be done in the form of having a script of random bullshit for non-player characters to make small talk with. The use and re-use of common small talk phrases by random people in various parts of the chronicle can quickly become a running gag, which frees the PCs to explore in a more sandbox fashion.

What follows are some sample characters. It is entirely intentional that these characters are entirely recognizable rips from popular culture. Ultimately when creating NPC cast members of your own, you will want to transform ideas in your mind into the numbers and abilities that represent a character in After Sundown. Therefore it is more useful as an example for the sample characters to be ones which you could plausibly be familiar with the \textit{idea} in addition to the stats themselves.

%%%%%%%%%%%%%%%%%%%%%%%%%
\subsection{Dean}
%%%%%%%%%%%%%%%%%%%%%%%%%
\hspace{\parindent} Dean spent several years as a slave soldier to the King with Three Shadows after he sold his soul to get a family member raised from the dead. Since escaping from Limbo, he spends most of his time traveling the country and killing monsters, which has earned him a place as a respected hunter in the Stellar Oracles, but a noted name on the shit list of several Makhzen Princes.

Dean is a Fallen. He is rather brash and tends to leap into danger as soon as he sees the barest opportunity to do so. In combat he will activate Quickness immediately, and when investigating things he will generally mimic a governmental authority figure. Dean's Powers grant him a +2 bonus to Socialization and a similar bonus to people asking about him later, and he almost never shies from this, grabbing the center of attention with Atract any time he enters a bar. He gains a +2 bonus to soak damage against weapons that are not iron, and personally has a true name that he does not use (Taxiarch {\fontfamily{cmr} \selectfont {M{\textiota\textchi$\alpha\eta$\textlambda}}}) and knows the true name of everyone he runs into. Dean deals with a Master Passion of Rage, though admittedly, not very well.

S: 4; A: 5; I: 3; L: 1; W: 2; C: 5; Edge: 3; Potency: 1.

\textbf{Skills:} Athletics 4; Combat 6; Drive 3 [+2]; Larceny (Palming) 5; Perception 3; Stealth 5; Survival 3; 
Bureaucracy (Government) 2; Empathy 3; Expression 1; Intimidation (Fear Mongering) 5; Persuasion (Fast Talk) 3; Tactics 6
Electronics (Wiring) 2; Medicine (First Aid) 1; Operations (Repair) 2; Rigging (Lockpicking) 4; Research 1 (Archives); Sabotage (Explosives) 4

\textbf{Backgrounds:} Bar Scene 6, Ballistics 6, Credit Fraud 5, Hell Mouths 4, Infernal Politics 4, Christian Splinter Groups 3, Cars 3, Sports Teams 2, Chemistry 2

\textbf{Powers:} [Fallen] Attract, Repel, Deny the Gauntlet, Mask of a Thousand Faces, Patience of the Mountain, Learn the Heart's Pain, Suggestion, Quickness, Dismissal, Desire Reflection, Fa\c{c}ade of Nonchalance, Bind the Name

\textbf{Advantages and Disadvantages:} Attractive, Diplomatic Incident, Feared by Children

\textbf{Equipment:} Shotgun [4], Colt 1911 (silver bullets) [3], Badass Jacket, Numerous hidden knives [1], sand (including sand cartridges for the shotgun), salt (including salt cartridges for the shotgun), Lockpick set, Chevy Impala.

%%%%%%%%%%%%%%%%%%%%%%%%%
\subsection{Jack}
%%%%%%%%%%%%%%%%%%%%%%%%%
\hspace{\parindent} Jack is a trucker who has been married many times and lived many different lives in many different cities. And that's even before you factor in the fact that he has begun to remember past incarnations of himself, where he apparently used to be a warrior in ancient China. He is openly skeptical that these "memories" are anything other than particularly vivid and weird dreams, but nonetheless finds himself embroiled in some seriously weird shit. Driven by a constant feeling that things are not right, Jack seeks out and gets new relationships frequently. And while he frequently just drifts off from his previous lives and interests (and has a legacy of failed marriage to show for it), he doesn't specifically do anything \textit{bad}, and maintains a friendship with all of his ex-wives.

Jack is a Reborn. Jack has a Master Passion of Despair. Jack's Powers grant him a +2 bonus to Socialization and a similar bonus to people asking about him later, an ability he calls upon only if the subtle approach fails. Jack often tries to get through confrontation without drawing upon magical powers at all, but if he feels in danger he will reset things and go into full speed mode. After all, as Jack always says "It's all in the reflexes."

S: 3[+1]; A: 4; I: 4; L: 2; W: 5; C: 2; Edge: 3; Potency: 1.

\textbf{Skills:} Athletics 2; Combat (Knives) 4; Drive 6; Larceny (Security Systems) 3; Perception 4 [+4]; Stealth 3 [+1]; Survival 2; Animal Ken 1; Bureaucracy 2; Empathy (Other Peoples' Troubles) 4; Expression  3; Intimidation 4 [+1]; Persuasion (Acting) 6; Tactics 3; Artisan (Carpentry) 2; Electronics (Radio) 1; Rigging (Plumbing) 6; Operations (Repair) 4; Research (Databases) 1; Sabotage (Thorough Destruction) 2

\textbf{Backgrounds:} Truckin 6; Cars and Trucks 5; Chinese Monsters 5; Gambling 4; Construction 4; World Crime League Finances 3; Small Town Life 3, Corporate Jobs 3, Food Service 2

\textbf{Powers:} [Reborn] Quickness, Nimble Feet, Supernatural Senses, Sensory Damper, Summon Spirits, Vigor, Attract, Shadow Casting, Shifting Sands, Retrocognition, Rapid Thought, Blur

\textbf{Advantages and Disadvantages:} Loyalty, Haunted, Distinctive Appearance

\textbf{Equipment:} A six pack of Pabst, Machine Pistol (wooden bullets) [2\textsuperscript{a}], A very nice knife [2], Crowbar [2], Baseball Bat [2N], MAC Truck, a spare suit.

%%%%%%%%%%%%%%%%%%%%%%%%%
\subsection{Marionette}
%%%%%%%%%%%%%%%%%%%%%%%%%
\hspace{\parindent} Marionette is a harlequin clown from Southern France who made her own elaborate performance enhancing drugs that eventually drove her quite mad. She isn't exactly certain that she isn't a human anymore, but she rarely takes the makeup off and continues to inject herself with "performance enhancements" of various sorts. She is a hanger-on to supernatural politics, preferring to promote the world as some kind of massive clown stage where people apparently get fed to hyenas.

Marionette is an Icarid. A natural contrarion, Marionette gains a +4 bonus to argue against any course of action, and a +4 bonus to soak damage against weapons that are not iron.

S: 2[+1]; A: 6; I: 3; L: 4; W: 4; C: 1; Edge: 3; Potency: 1.

\textbf{Skills:} Athletics (Acrobatics) 6; Combat 4; Larceny (Disguise) 6; Perception 2[+2]; Stealth 6
Animal Ken 6 [+2]; Bureaucracy 3; Empathy 1; Expression  (Dance) 6; Intimidation (Tormenting) 2; Persuasion 1;
Artisan (Metalworking) 3; Rigging (Ropes) 3; Medicine (Poison) 6; Research (Old Books) 2; Sabotage (Explosives) 6[+2]

\textbf{Backgrounds:} Circus Life 6; Chemistry 6;  Surreal Humorists 5; ETA 4; Gambling 4; Gloomy Poetry 4; Children's Programming 4; Wine 2

\textbf{Powers:} [Icarid] Hide From Notice, Mask of a Thousand Faces, Supernatural Senses, Curse of Failure, Clinging, Revive the Flesh, Walk of Flame, Tongue of the Beast, Dark Night of the Soul, Hide in Plain Sight, Holistic Ventriloquism, Indominability

\textbf{Advantages and Disadvantages:} Experimenter, Weapon Finesse, Delusional (Nihilism), Frivolous, Prideful

\textbf{Equipment:} Silver Rapier [3], Flash Bombs, Smoke Bombs, Playing Cards, Silver Bells, Fishing Line, Squeaky Toy, Lighter, Can of Gasoline.

%%%%%%%%%%%%%%%%%%%%%%%%%
\subsection{Dante}
%%%%%%%%%%%%%%%%%%%%%%%%%
\hspace{\parindent} Dante is not supernatural, and doesn't actually believe that the supernatural exists at all. He is the manager of a Quick Stop, and would be almost wholly uninvolved in the machinations of monsters save for the fact that he \textit{is} a Luminary.

S: 3; A: 2; I: 4; L: 4; W: 1; C: 4; Edge: 3

\textbf{Skills:}Athletics 2; Drive 2; Larceny 1; Perception 3; Stealth (Inconspicuousness) 3; 
Animal Ken 2; Bureaucracy (Corporate) 5; Empathy 4; Expression 3; Intimidation 2; Persuasion (Whining) 5;
Artisan 5; Electronics (Computers) 4; Rigging (Refrigeration) 2; Research (Culture) 5

\textbf{Backgrounds:} Science Fiction 5, Food Service 5, Hockey 4, Periodicals 4, New Jersey 3, Pornography 2, Town Savages 2, Recreational Drugs 2

\textbf{Equipment:} Hockey Stick [2N], Car, Today's Newspaper, Pornographic Video, Beef Jerky.

%%%%%%%%%%%%%%%%%%%%%%%%%
\subsection{Chris}
%%%%%%%%%%%%%%%%%%%%%%%%%
\hspace{\parindent} Chris is a \textit{bystander}. He works at the SK8R|, and he pours beers. Chris is an Extra, and is of primary note in that his stat line can be used with minor adjustments for all kinds of extras who work in different parts of the realms of horror.

S: 3; A: 2; I: 2; L: 2; W: 1; C: 3

\textbf{Skills:} Athletics 3; Combat 1; Drive 1; Larceny 1; Perception 2; Stealth 1;
Bureaucracy 4; Empathy 2; Expression (Guitar) 3; Intimidation 4; Persuasion 3;
Rigging (Pressurized Fluids) 4; Operations (Mechanical Bull) 3;

\textbf{Backgrounds:} Skating 6, Food Service 4, Drinking 4, Sports 3, Politics 3, Organic Food 2, Recreational Drugs 2

%%%%%%%%%%%%%%%%%%%%%%%%%
\subsection{The Black Isz}
%%%%%%%%%%%%%%%%%%%%%%%%%
\hspace{\parindent} There are a cadre of Mirror Goblins who work for a vicious killer: a dangerous Baali named Artemis Pender. The player characters may never meet Artemis Pender or learn what it is that he is doing -- but his Mirror Goblin agents are sufficiently numerous that they may be encountered many times. A typical group will be between 4 and 6 of the Isz, and be dispatched with some identifiable (if baroque) mission. Abilities vary slightly within an Isz pack, but one of them might look like this:

S: 3; A: 4; I: 2; L: 2; W: 1; C: 3

\textbf{Skills:} Artisan 1 (Painting); Athletics 2; Combat 3; Larceny 4; Perception 2; Sabotage 1 (Traps); Stealth 2; Survival 4; Tactics 2

\textbf{Backgrounds:} Gibbering 4, Infernal Politics 3, Rare Art 3, Human Laws 2

\textbf{Powers:} Quickness, Mask of a Thousand Faces, Touch of Darkness

\textbf{Advantages and Disadvantages:} Swarming, Fiercely Competitive

\textbf{Equipment:} Trench Coat, Spray Paint.

%%%%%%%%%%%%%%%%%%%%%%%%%%%%%%%%%%%%%%%%%%%%%%%%%%
\section{Animals}
%%%%%%%%%%%%%%%%%%%%%%%%%%%%%%%%%%%%%%%%%%%%%%%%%%
\tagline{Lions, tigers \textbf{and} bears?}

People are a form of animal. Vampires are humans. Humans are apes. Apes are monkeys. Monkeys are mammals. And mammals are animals. However, not all animals are people, and indeed most of them are not. Animals get a separate entry precisely because their capabilities, while generally much more limited than a human's, are incredibly varied. It's just not particularly helpful to attempt to use the same system to generate a hippo and a hamster.

Animals have wildly different speeds than that of a human. And so what is given is simply the creature's speed during a draining sprint. For simplicity, animals move proportionally less fast than that when doing other kinds of movement. Remember also that real life 70 kilogram humans often only have 1 or 2 in an attribute, so don't be terribly surprised if many animals have values that are very low. Inherent in the dicepool system is an assumption of basically human norms, and creatures significantly outside those norms have numbers that are significantly outside those values. The game simply does not have the granularity to differentiate the strength of animals that are not as strong as a person -- and that's most of them (people are pretty strong).

%%%%%%%%%%%%%%%%%%%%%%%%%
\subsection{Rat}
%%%%%%%%%%%%%%%%%%%%%%%%%
\hspace{\parindent} Game mechanically not particularly different from a hamster or a mouse. Or pretty much any rodent. Or a small lagomorph like a bunny. These creatures are not individually threatening unless they happen to carry disease, and even then can be "defeated" by an old woman with a broom. Small animals like this suffer wounds as if any attack had inflicted 3 more wound levels. Even a small firearm is generally overkill. They are very small though, so the threshold to hit them from any range past Near is increased by 1. Rats are a  frequently used magical spy because they are so ubiquitous.

S: 0; A: 3; I: 2; L: 1; W: 3; C: 1

\textbf{Skills:} Athletics 2; Combat 1; Perception 2; Stealth 4; Survival 3

\textbf{Speed:} 45m

%%%%%%%%%%%%%%%%%%%%%%%%%
\subsection{Bat}
%%%%%%%%%%%%%%%%%%%%%%%%%
\hspace{\parindent} Bats are cute little nocturnal insectivores. Or sometimes hematophages. Tiny animals like this suffer wounds as if any attack had inflicted 3 more wound levels. They are very small though, so the threshold to hit them from any range past Near is increased by 1.

S: 0; A: 3; I: 3; L: 1; W: 1; C: 2

\textbf{Skills:} Athletics 4; Combat 1; Perception 4; Stealth 2; Survival 2

\textbf{Speed:} 200m (Flying)

%%%%%%%%%%%%%%%%%%%%%%%%%
\subsection{Raven}
%%%%%%%%%%%%%%%%%%%%%%%%%
\hspace{\parindent} Ravens are creatures of ill omen, mainly because they eat carrion, but also because a raven is always thinking about how to peck your eyes out. Ravens are small, and suffer wounds as if any attack had inflicted 2 more wound levels than its base value. Attacks against them from farther than Short range have their thresholds increased by 1. Ravens are popular as familiars for witches and the like, because they can be taught to speak simple human words. Also they can fly.

S: 0; A: 3; I: 3; L: 1; W: 3; C: 1

\textbf{Skills:} Athletics 3; Combat 3; Perception 2; Stealth 1; Survival 2

\textbf{Speed:} 250m (Flying)

%%%%%%%%%%%%%%%%%%%%%%%%%
\subsection{Cat}
%%%%%%%%%%%%%%%%%%%%%%%%%
\hspace{\parindent} House cats are important to many magical beings as a symbol of status. Cats are small, and suffer wounds as if any attack had inflicted 2 more wound levels than its base value. Attacks against them from farther than Short range have their thresholds increased by 1.

S: 0; A: 3; I: 3; L: 1; W: 2; C: 1

\textbf{Skills:} Athletics 3; Combat 3; Perception 2; Stealth 2; Survival 2

\textbf{Speed:} 120m

%%%%%%%%%%%%%%%%%%%%%%%%%
\subsection{Cobra}
%%%%%%%%%%%%%%%%%%%%%%%%%
\hspace{\parindent} Various deadly snakes are a mainstay of horror and adventure fiction. Why does it have to be snakes? Because they are creepy looking and don't make footfall noises when they move. Cobras and vipers and rattlesnakes are poisonous, and largely interchangeable from a storytelling standpoint. Poisonous snakes are small, and suffer wounds as if any attack had inflicted 2 more wound levels than its base value. Attacks against them from farther than Short range have their thresholds increased by 1.

S: 1; A: 3; I: 1; L: 1; W: 2; C: 1

\textbf{Skills:} Athletics 2; Combat 3; Perception 2; Stealth 4; Survival 2

\textbf{Advantages and Disadvantages:} Double Jointed

\textbf{Speed:} 40m

%%%%%%%%%%%%%%%%%%%%%%%%%
\subsection{Large Dog}
%%%%%%%%%%%%%%%%%%%%%%%%%
\hspace{\parindent} It is important to note that a large and scary dog is still genuinely on the small side for an actual human being, and that while a dog's teeth are frightening, they are not actually as deadly as a baseball bat or a crowbar in the hands of an angry dude. A large dog's teeth and claws amount to a 1N weapon. While a dog can indeed perform a fatal mauling, this is generally not accomplished without many seconds of worrying (that's the thing dogs do where they grab something in their mouth and shake it).

S: 2; A: 3; I: 3; L: 1; W: 2; C: 1

\textbf{Skills:} Athletics 3; Combat 4; Perception 4; Stealth 2; Survival (tracking) 2

\textbf{Speed:} 150m

%%%%%%%%%%%%%%%%%%%%%%%%%
\subsection{Alligator}
%%%%%%%%%%%%%%%%%%%%%%%%%
\hspace{\parindent} Alligators are the pit monster of choice, because while they aren't super fast and have profound difficulties getting out of a pit, they \textit{are} pretty deadly and terrifying. The jaws of an alligator are a 2 damage weapon, and alligators have a point of armor. A couple of hungry alligators in a pit can make short work of most victims, making them an ideal death trap for any insane cultists or world conquering madmen.

S: 5; A: 1; I: 1; L: 1; W: 3; C: 1

\textbf{Skills:} Athletics 2; Combat 4; Perception 2; Stealth 2; Survival 2

\textbf{Speed:} 50m

\textbf{Advantages and Disadvantages:} Combat Paralysis

%%%%%%%%%%%%%%%%%%%%%%%%%%%%%%%%%%%%%%%%%%%%%%%%%%
\section{Places}
%%%%%%%%%%%%%%%%%%%%%%%%%%%%%%%%%%%%%%%%%%%%%%%%%%
\textit{I am the master of all I see, so long as all I see is that which I have mastered.}

Mystical locations are amongst the \textit{least} destabilizing of all possible special magics. The fact that they are large compared to a person and completely non-transportable means that they can be worked into one story without necessarily having a strong impact on any later stories.

%%%%%%%%%%%%%%%%%%%%%%%%%
\subsection{The Pet Cemetery}
%%%%%%%%%%%%%%%%%%%%%%%%%
\hspace{\parindent} You all know the legend of the pet cemetery where if you bury someone "just right"they come back from the dead. And maybe (like, almost every time) it goes horribly wrong and the person or animal comes back as some sort of hideous monster that probably eats people or something. That sort of thing \textit{exists} in After Sundown. It's a kind of Shadow Gate, and the thing that makes it of interest is that they aren't continuously open, so they aren't spraying a horde of poltergeists all over everything, nor does it \textit{necessarily} come with a giant zombie apocalypse. The idea is that there are things you have to do in order to activate the Shadow Gate, and even then only for brief periods. So while it \textit{can} be used to transform the corpse of a Luminary into a Revenant, and by extension can be used to start a zombie outbreak deliberately, it doesn't have to be. And that's \textit{great} news for supernaturals who want to send people and things back and forth to Mictlan and \textit{don't} want a never ending ghost storm or zombie tide. These "locked gates" are, while certainly dangerous, highly prized.

A Locked Shadow Gate requires some "key" to "open". While open, it can function as a two way passage to The Gloom, and it can manifest Necromancies like a normal Shadowgate would. The one from the King novel of the same name opened for one hour the midnight following someone making a stack of rocks in front of it. This was used to raise the dead, with different corpses coming back as per Reanimation, and others coming back as per Resurrection. The very unpredictability of it (and the fact that zombies created in this manner are completely uncontrolled) is why those in the know about supernatural events generally suggest \textit{not} trying to bring loved ones back with a Locked Gate.

A Locked or Open Shadowgate\index{Shadowgate} both constitute a rating 3 Destiny.

%%%%%%%%%%%%%%%%%%%%%%%%%
\subsection{The Goblin Market}
%%%%%%%%%%%%%%%%%%%%%%%%%
\hspace{\parindent} It's a wretched hive of scum and villainy to be sure, but an invaluable source of information and rare goods. Step through the mirror and you find yourself in a grungy and ash strewn marketplace where foul abominations sell horrid things to one another. For the right number of kittens, you could get yourself rare spices, dubious information, or a group of shady mercenaries from all over the world. There are a number of egresses from the Goblin Market, and while all of their Limbo sides are within the marketplace, their human world sides are all over the place.

The Goblin Market is a \textit{Mirror Nexus}, a small area of the Dark Reflection that is connected to several mirrors in different parts of the mortal world. Most of those gate mirrors are actually quite immobile (such as the lake surface), or transient (such as the newly washed window of a skyscraper), but there are a few entrances that are portable enough for someone to own one. Like any Mirror Nexus, the Goblin Market can be used to travel around the world at tremendous speed (albeit only to specific places linked by the mirrors in it). Some Mirror Nexuses are basically unknown, which means that unlike the Goblin Market, a character cannot go there in order to do shady commerce or perform some sort of information gathering montage. However, these have the added advantage that the characters can travel through them without being seen by dozens of Trolls and hundreds of Mirror Goblins who are all of dubious and negotiable loyalties.

Any of the handful of Mirror Nexuses\index{Mirror Nexus}, whether currently occupied or not, constitutes a Rating 4 Destiny.

%%%%%%%%%%%%%%%%%%%%%%%%%
\subsection{Yggdrasil}
%%%%%%%%%%%%%%%%%%%%%%%%%
\hspace{\parindent} It is the greatest tree, from whence dreams of power come. And it has roots in the dreams of all who wield authority. Yggdrasil is a towering tree of specifically indeterminate type. It grows somewhere in the Deep Maya, and it is taller than the limits of vision in that strange place. Supposedly it is somehow watered by the \textit{idea} of authority, and drinking its sap or eating its seeds gives one delusions of might. The Deep Maya is pretty locationally unstable, and when in the presence of the mighty arboreal construct this property is exacerbated -- returning to the world of mortals could leave one anywhere. At least, anywhere that there are people sleeping who hold power over others.

Climbing its branches you can end up in the location of any source of temporal power where people are sleeping nearby. So you can pop into the White House in the middle of the night or into the Indian Parliament during a particularly boring speech. By climbing the tree with deliberate intention, one can arrange to end up near a specific hall of power (such as the Tienanmen Square Parliament Building or the Covenant Grand Chapel in Rome), but not to the extent of showing up in a specific person's bedroom. The accuracy is to within a couple dozen meters, and when the character appears in the mortal world they seem to simply mist in. The seeds and sap of the tree are poisonous and cause delusions of grandeur. But they are also extremely valuable in a number of special Bitter Fruit recipes, and can be used to make potions that sap the will, inspire loyalty, or make someone appear physically attractive.

Yggdrasil\index{Yggdrasil} is a Rating 6 Destiny.

%%%%%%%%%%%%%%%%%%%%%%%%%%%%%%%%%%%%%%%%%%%%%%%%%%
\section{Things}
%%%%%%%%%%%%%%%%%%%%%%%%%%%%%%%%%%%%%%%%%%%%%%%%%%
\tagline{Is that what I think it is? I thought it was only a legend!}

After Sundown uses roughly the same system of magic item creation as Dwarf Fortress. People periodically get overtaken by strange moods (or macabre or fey moods as the case may be), and then they gather weird materials together, lock themselves in a workshop, and make a perplexing masterpiece. This masterpiece is generally of surpassing quality and thus considerable value as an object of art or history, and also frequently cursed.

%%%%%%%%%%%%%%%%%%%%%%%%%
\subsection{The Necronomicon}
%%%%%%%%%%%%%%%%%%%%%%%%%
\hspace{\parindent} It's a creepy tome that is filled with demonic magic. Each page is velum with a brown ink that could seriously be blood forming its writings in a mad and tangled hand. Those who read from its dread pages gain dark insight and go mad. Sometimes in that order.

Magic tomes are pretty common as rare and unique magical objects go. What they basically \textit{do} is let someone who studies them for extended periods of time learn some sort of evil sorcery. This has the clearest effect of giving a character an excuse to learn a couple of sorcerous disciplines, so that if they are afforded the opportunity to learn a sorcerous discipline, they can choose to pursue one of the sorcerous disciplines described in the pages. Magic tomes have additional effects when they are read by humans. Luminaries can turn into Witches, and Extras become cultists. Generally, a major magic tome will cover two or three sorcerous paths, and convert Luminaries into the most closely associated kind of Witch. The Necronomicon itself covers the Descent of Entropy, the Progress of Glass, and the Song of Swarms. Luminaries who read it become Baali. Other Infernal grimoires include \refwork{De Vermis Mysteriis}, \refwork{The Bible Black}, \refwork{The Book of Nod}, and \refwork{Unaussprechlichen Kulten}; while Orphic grimoires that can change a man into a Khaibit include \refwork{Secrets of Life and Death}, \refwork{Cultes de Ghoules}, \refwork{Beyond the Setting Sun}, and \refwork{The Book of the Dead}; and Astral grimoires that can transform a Luminary into a Dryad include \refwork{The King in Yellow}, \refwork{Allessehen Auge}, \refwork{The Endless Nightmare}, or \refwork{The Voynich Manuscript}.

Any of these books is a Rating 1 Destiny\index{Magic Book}, as is any similar book you make for your campaign.

%%%%%%%%%%%%%%%%%%%%%%%%%
\subsection{Shamshir-e Zomorrodnegar}
%%%%%%%%%%%%%%%%%%%%%%%%%
\hspace{\parindent} It's a scimitar whose hilt is encrusted with emeralds. It has been prophesied that it alone can fell the great Efreet Fulad-zereh, and that it will inevitably do so. Needless to say, pretty much everyone who takes an especial interest in the wellbeing of Fulad-zereh (for good or ill) wants to have control this scimitar. There are of course \textit{numerous} weapons out there that are the fated killers of various powerful supernatural monsters. Carnwennan the dagger, Clarent the sword, and Luin the spear were all famously used to slay various specific beings. But only the  Shamshir-e Zomorrodnegar is fated to kill Fulad-zereh.

These weapons have several properties that are of especial interest to the creature they are targeted against. The first is that they cause aggravated damage to the creature in question and ignore the soak bonuses of the target's Powers \textit{and Edge}. The second is that if they are used to attack their destined foe, the attacker gains +3 dice on all attacks. The third is that when a character holds the weapon, they know where the targeted foe is as if they were being \textit{summoned} whenever they speak the creature's name. No creature has more than one fated bane weapon targeted against them at a time, and they are usually reticent to destroy it because if they do so outside of the auspices of a specific mighty ritual of vast power, another weapon \textit{somewhere in the worlds} instantly gains that property.

Shamshir-e Zomorrodnegar, or any bane weapon\index{Bane Weapon}, is a Rating 1 Destiny.

%%%%%%%%%%%%%%%%%%%%%%%%%
\subsection{The Legacy of Thomas Craine}
%%%%%%%%%%%%%%%%%%%%%%%%%
\hspace{\parindent} The room is full of books. Dusty old tomes filled with mildewed pages and cryptic writings from generations of goetic researches. There are numerous falsehoods in their crinkled pages, and yet there are some quite poignant discussions on the land of the dead. Many of the books are penned by Thomas Craine himself -- some of them when he was still alive.

A magical library generally contains the source material for five sorcerous disciplines. In the case of Thomas Craine's Legacy, it contains all paths of Orphic Sorcery. But beyond that, one can do library research in such an environment, giving it a substantial edge in identifying things over any single enchanted tome. The Legacy is a good place to research locations and inhabitants of Mictlan.

A Magical Library\index{Magical Library} is a rating 2 Destiny.

%%%%%%%%%%%%%%%%%%%%%%%%%
\subsection{The Wings of Needless Sorrow}
%%%%%%%%%%%%%%%%%%%%%%%%%
\hspace{\parindent} It is an amulet made of obsidian that has been cut to resemble a stylized bird in flight. The Wings are filled with hatred, and are drawn to suffering. The bearer of the amulet will see calamity after calamity, with the sick and dying plaguing their existence like a literal disease of which they cannot be cured.

The amulet senses the future, is evil, and subtly adjusts the bearer's behavior so that the amulet is brought into contact with as many crimes, accidents, and human misery as possible. If the bearer has the opportunity to take two different routes to the convenience store, they will \textit{always} end up choosing the one where they witness a traffic accident. This sort of thing can actually be immensely useful to a Scooby Gang, since they can just have the bearer point to a road map at random and they'll end up driving to a place their skills are needed.

The Wings of Needless Sorrow\index{Wings of Needless Sorrow} are a rating 2 Destiny.

%%%%%%%%%%%%%%%%%%%%%%%%%
\subsection{The Bull of Despondent Glory}
%%%%%%%%%%%%%%%%%%%%%%%%%
\hspace{\parindent} It is a scroll of vellum made from human skin and written in a dense Latin script. Supposedly it was written up by the Covenant Anti-Pope Typhus II in the 11th century. Much of the second half is completely illegible, but as time goes by it \textit{sharpens}, and words can be discerned.

The Bull of Despondent Glory is a True Prophesy. That means that a bunch of things that are written in it \textit{will} come true, and a bunch of things \textit{have} come true. Entries appear in it irregularly, but when important stuff is going on, a number of entries may come to focus all at once. While it is a True Prophesy, some of the events described did not (at least apparently) actually happen as predicted. Still, with an accuracy above 80\%, it's pretty uncanny. And not a few supernatural creatures believe that it is \textit{always} right and that history is merely wrong (or perhaps deliberately obfuscated) in those instances where history and the Bull conflict.

The Bull of Despondent Glory\index{Bull of Despondent Glory} is a rating 3 Destiny.

%%%%%%%%%%%%%%%%%%%%%%%%%
\subsection{Preah Khan}
%%%%%%%%%%%%%%%%%%%%%%%%%
\hspace{\parindent} It's a jewel and pearl encrusted broadsword of clear quality. The blade is a dull and dark gray as heavily tarnished silver, but the blade is sharp enough to shave with and does not appear to need sharpening. Legend has it that the metal was forged out of naga venom somehow and that it is still toxic. But the big thing is that owning it makes you the rightful king of Cambodia. Whoever holds the sword to the banks of the Mekong can make the river shrivel, swell, or even flow backwards. This sort of grotesque displays of magic are generally frowned upon by the World Crime League, and not a few supernatural creatures want the Preah Khan to stay missing.

Preah Khan is an excellent sword, inflicting a base 3 damage, and every wound it inflicts is aggravated (even to mortal people). Khmer people will treat the holder as if they had the status of a king even if they do not know why. Even if they aren't even Cambodia, the wielder of Preah Khan can get the royal treatment in donut shops all over the Bay Area. And then there's the whole "controlling the Mekong River" thing. If word gets out that someone has Preah Khan, the owner can expect numerous enemies.

Preah Khan\index{Preah Khan} is a Rating 4 Destiny \textit{and} comes with Rating 4 Stalkers.

%%%%%%%%%%%%%%%%%%%%%%%%%%%%%%%%%%%%%%%%%%%%%%%%%%
\section{Standardizing Nonstandard Magic}
%%%%%%%%%%%%%%%%%%%%%%%%%%%%%%%%%%%%%%%%%%%%%%%%%%
\tagline{No seriously, that's something you can't do that you \textbf{actually} can't do.}

Cursed knives, ancient curses, stellar alignments, and mighty rituals of vast power are all an integral part of the Modern Gothic Horror genre, but they are also generally specific enough that they do not readily translate into hard and fast rules. Nonstandard magics should be different enough to drive stories and provoke interest, but not different enough to undermine the consistency of the setting. Powerful enough to be worth chasing after, but not so powerful that they invalidate other life choices.

%%%%%%%%%%%%%%%%%%%%%%%%%
\subsection{Ritual Magic}
%%%%%%%%%%%%%%%%%%%%%%%%%
\tagline{More magic can do more things. If you draw upon more magic, you can do more things. But more things can happen when more magic is used.}

What Ritual Magic \textit{does} is allow characters to use powers that they never actually wrote on their character sheet. As such it is almost by definition \textit{unbalanced}, and needs to be kept under a strict leash. A mighty ritual that allows a character to use an Advanced Power when they don't have it is problematic, and a mighty ritual that allows a character to draw upon an Elder Power they don't actually know is even more so.

The provisions for a mighty ritual of vast power can be basically anything, and it is up to the MC to determine what they specifically are in any specific case. Within the context of the story, the characters read through a bunch of mystical books in a library and find the formula for a relevant and possible sounding ritual, but in a very real way any possible ritual is essentially custom placed by the MC with (ideally) the chronicle going on \textit{in mind}. Here are some guidelines:

\begin{itemize*}
\item A Ritual whose effects are of sufficient importance to be \textit{worth} questing an entire story for should have requirements that are sufficiently difficult that they actually take up the whole story to put together.

\item A Ritual whose effects are merely a stepping stone towards completing a story should have requirements that the players can plausibly get together as part of a story.

\item Any Ritual should have requirements that for whatever reason the players are unlikely to be able to repeat for credit. At least, not often enough to become routine or get annoying. Stellar alignments are great for this, because by definition they won't come again for some arbitrary (and probably large) amount of time.
\end{itemize*}

%%%%%%%%%%%%%%%%%%%%%%%%%
\subsection{Items of Great Power}
%%%%%%%%%%%%%%%%%%%%%%%%%
\tagline{The ring \textbf{wants} to be found.}

Magical objects, like tools of science, could do practically anything. They are limited primarily by your imagination and the suspension of disbelief of the audience. That second one is important, and it is good to remember that no item has ever been powerful enough that the world would not have turned out the way it did. There is no gong that cures all the whooping cough on Earth, because pertussis still exists. There's no pearl that sinks Japan, because Japan is still demonstrably above the sea. There is no cauldron that makes an unbeatable world-conquering army, because no one conquered the whole world. No matter how impressive any magic item is, it is in some very important way less world changing, less \textit{powerful} than a hydroelectric dam or an atomic bomb. And yes, that leaves a great deal of wiggle room, but it is important to keep perspective that there are no magic items that could allow their owner to overwhelm a powerful country like the Russian Federation by force of arms.

Which does not mean of course that any particular item of power is not a big deal. There is no everfull purse that produces enough gold to destabilize world metal prices, but there may very well be a goose that lays golden eggs of sufficient quantity to make the owner spectacularly filthy rich. And while no magical weapon reaches the heights of city destruction achievable by nuclear fusion, there are many magical weapons that are exceedingly impressive for their size, and capable of feats of murder all out of proportion to their probable police response. You could probably get a license to own, and in Texas \textit{concealed carry} the masterpiece of Daniel Colt.

Magical objects should be valuable, which means that the things they do should in general be things that a comparably sized electronic object would not do with the introduction of a few AA batteries. While there are stones that make light \textit{by magic}, those are not really worth talking about in a modern context unless you are showing the paucity of ancient wonders and superstitions (in which case, go nuts). Magic items are not reproducible (or at least, not \textit{mass} reproducible). If you find a magical perpetual motion machine, you \textit{can't} make tonnes of ostensibly identical copies that collectively generate limitless power and change the world. If you are given such an item you may well be able to keep a secret base operating "completely off the grid", but magic is not going to be an answer to fossil fuel dependency. An important thing to consider when making a magical object is that in general it should be doing something that is not replicable by  store bought materials and thus \textit{worth} searching for and fighting over. Magical objects transcend what a normal tool is capable of, and since tools can already do some amazing things, that's a pretty big deal. Here are some short and pithy guidelines for making a magical object:

\begin{itemize*}
\item Magical objects should do something that is clearly magical, not simply have enhanced properties that might be achieved by making an object out of better steel or burning through batteries faster.
\item Magical objects may in fact be cursed, and a lot of them are. But remember that any curse that is more impressive than the nominal effect will be used for whatever it is the curse does (turn people into murderers, summon chimerae, whatever), if at all. An item where the curse exceeds its nominal utility is essentially useless for its nominal utility and you shouldn't make it like that.
\item Magical objects in general should do something different from normal Powers. This is so that people who got actual Powers don't feel like \textit{suckers} when someone else finds a magic mirror that does the same thing. For that matter, any item that is as impressive or more so than a genuine Power should be presented as being very rare and impressive -- and they should not be easily acquired or retained. 
\item Bonuses are boring. It isn't that people don't \textit{like} having a +2 bonus to Athletics because they are wearing magic shoes, but those sorts of effects are basically indistinguishable from simply having larger numbers in those dicepools in an entirely mundane fashion. In general, magic shoes should do something \textit{different} than merely doing the thing you normally do, but better.
\end{itemize*}

%%%%%%%%%%%%%%%%%%%%%%%%%
\subsection{Getting Items of Power}
%%%%%%%%%%%%%%%%%%%%%%%%%
\tagline{See an evil penny and pick it up, all the day\ldots{} something something.}

An item of power can serve as one of two main roles in a chronicle: either as a plot point or as a resource. In either case, the amount of effort it takes to acquire them should be roughly commensurate with how much effort would be required to get to a similarly useful resource or plot twist by other means. A minor artifact is a simple Rating 1 Destiny Resource, and could easily be a toss-off to show that "something was up" in the same way that handing out a Rating 1 Financial Resource might be. For example, if the coterie breaks into a loft and the villains aren't there, they might find a simple magic dagger or cursed wand to show that -- indeed -- they had the right loft (and also to make it not seem like the players had wasted their time only to find that the enemy had already been on the move). The MC could just as easily use a stolen painting or a pile of bloody watches to show the same thing, but sometimes a monkey's paw that mysteriously curdles milk and sours juice it is pointed at is just \textit{more interesting}.

A plot device can be basically anything. Even a tattered scarf could plausibly be the clue that reveals the true murderer or the fetter that binds an important ghost. So a magic item whose purpose in the story is to advance the Chronicle should be restricted merely by how important the plot point is, and what stage the characters are in the story. The special mirrored surface out of which an ancient and powerful Asura can walk might take much heartache, legwork, and sacrifice to get to if doing so is the culmination of a multi-story chronicle. On the other hand, if said Asura is just someone who is going to give a clue as to where the coterie needs to go fly to in order to find the ruins of the Troll city that the Shattered Empire hugs have taken Caitlin, then the player characters might just need to break into a museum at night and touch the thing.

Most magic items are in fact built. At least, at some point. And they are created through mighty rituals of vast power. And the mighty ritual of vast power should be roughly as difficult to research and pull off as going through the story and finding an artifact of similar utility. Items gained without a story just don't have much of a story to them -- and the story is after all what people are actually there for. Whether the story is that you went into the bog and listened to the old hag and took the silver pentagram that can be used to break a shadow gate after you proved your worth; or that you went into the bog and listened to the old hag, and got told how to make the silver pentagram that breaks the shadowgate after proving your worth is rather meaningless of a metagame distinction. But it \textit{feels} different, so the MC should mix it up a bit and have players end up conducting the mighty rituals of power that make these things at least sometimes. 

%%%%%%%%%%%%%%%%%%%%%%%%%
\subsection{Additional Powers Within Groups}
%%%%%%%%%%%%%%%%%%%%%%%%%
\tagline{Didn't think I could do that, did ya?}

There is no special rule that each power group has to have precisely two powers available at Basic, Advanced, and Elder levels. And indeed, some groups have a third power listed in one of the mastery levels in the basic rules. Furthermore, future material may well come with alternate abilities for characters to have instead of the normal features of each group. Or, as a MC you could write some of your own.

Adding new powers to the groups can be a cool way to spice things up. But it can also dilute the setting, and make characters more powerful. All things being equal, the more powers a player has to sort through the more work it is going to be \textit{and} the more powerful a result they are likely to get by cherry picking the right combination of abilities. New powers should probably not use radically different dicepools than other powers in the same group, because otherwise you are going to be moving towards a world where someone can max one attribute and one skill and get a very high dicepool in a wide variety of different powers. Similarly, a new power should not do something radically different from other power in the same group.

But above and beyond the simple balance issues that arise from increasing the set size of potential abilities, there's also the concept of \textit{setting strain} to worry about. There is, for example, no ability that currently allows a character to travel to the dark side of the moon. And while there's nothing inherently overpowered about collecting moon rocks or building a secret base there, it would still be a very large problem to add such an option to the game. So long as there's no way to get to Luna, the moon is outside the playspace entirely, and players do not have to discuss their enemies having gone to the moon -- nor do they have to invest in powers to potentially go there. One of the primary things that make games fall apart is the simple proliferation of other places that all need their own powers to reach. If one were to add the distant planet of the Pods, the mythical homeland of the Demons, and a few garden variety Alternate Earths, the game would become very cluttered and confusing. 

\textbf{Example:} Building on the huge strength facet of Clout, an extra Elder ability can be stuck in that makes the user bigger than Giant Size does. This is the signature power of Nabau, the enormous Makhzen Mehtar of Kuala Lumpur (and yeah, he has the "can't turn it off" version). This power is a reasonable extension of Clout because it is different enough from previously published powers to notice, and yet it does not overly increase the utility of any skill, nor does it bring something into the conceptual space that is completely unprecedented (Kaiju are already this huge).

\powerentry{Titanic Size} The character gets super huge. Like 8-12 meters in raw hugeness. While in Titanic Size, the character's Strength is increased by 20 and they have 3 points of armor. Transforming takes a Complex Action and 8 Power Points, and lasts until the end of the scene. Titanic Size is a Protean Power, and does not stack with Giant Size. Some creatures get Titanic Size "always on", where they never shrink down to human size.

%%%%%%%%%%%%%%%%%%%%%%%%%
\subsection{New Power Groups}
%%%%%%%%%%%%%%%%%%%%%%%%%
\tagline{I have mastered the art of Obscurica, and I can do things you doubtlessly think are impossible.}

I strongly suggest \textit{not} making new universal or sorcerous groups. It's not that you can't make a new group that is balanced, because you totally can. It's that new groups undermine the foundation of the game as a \textit{shared} storytelling medium. The coherency of the world comes apart a little bit every time a new group is introduced. At the limit of adding infinite groups, no action that any character takes with magic has any \textit{context} -- even if none of the new groups is substantially over powered or deceptively worthless, the game still becomes essentially unplayable. However, I am equally aware that a substantial number of players will, for reasons base or noble, choose to ignore that advice and write new Universal or Sorcerous power groups into the story. This can actually be fine. A 23rd group is \textit{not} the same thing as the 101st group, and the conceptual coverage of the magical groups in After Sundown is essentially arbitrary. It's entirely possible to add a new group or two without breaking anything. But remember: it \textit{is} a slippery slope and you seriously \textit{can't} just keep adding new groups forever without breaking the world. Don't be afraid to put your foot down. Just because you let one player bring a new power group into the game and it would be "fair" to allow another player the same opportunity doesn't mean that placing more straw on your camel is a good idea.

But if you \textit{are} going to make a new group of powers, keep some things in mind:

\begin{itemize*}
\item Sorceries are still Astral, Infernal, or Orphic. You may think you have a cool idea for some fourth power source, but dowsing and preparing counterspells for 3 flavors of magic is already hard to keep track of. A fourth power source means that monster hunters will have a whole new set of equipment in addition to all the crap they have to carry around with themselves. And that's bad.

\item A new group should have 2 Basic, 2 Advanced, and 2 Elder Abilities in it. If you can't think of that many powers for it, you should \textit{seriously} consider the idea that you don't have a power group worth writing up or disturbing the status quo for.

\item You already have a set of a powers that you can draw upon for ideas in the form of the groups already printed. If you have a Basic Ability in a group, go ahead and compare it to the Basic Abilities of groups already printed.

\item Each new group of powers will be taken by probably one character at most in your game. So you should make sure that the dicepool choices for the powers are pretty static through the whole writeup. 

\item Groups that are collections of heterogeneous abilities are in general hard to keep track of and you shouldn't make them. It might be tempting to make a group that is simply the list of abilities displayed by some sorcerer in a book you liked, but such "stuff from the attic" groups confuse players. A new group should, if anything, be more clearly themed than one of the ones in the basic book -- the players won't have it in the book to go back to so it needs to be more memorable on its own merits.
\end{itemize*}

%%%%%%%%%%%%%%%%%%%%%%%%%
\subsection[Subtypes]{Subtypes: Bloodlines, Strains, and Schools of Thought}
%%%%%%%%%%%%%%%%%%%%%%%%%
\tagline{We are defined by our similarities as well as our differences.}

It is sometimes useful to a story to have a bloodline of Vampires or a family of Leviathan who represent a recognizable clade. It is tempting to write additional powers for such groups or to trade basic powers of their type for other powers in order to make them stand out and "feel unique". This is a terrible plan, and you shouldn't do it. A subgroup doesn't need to feel \textit{unique}, because every character is by definition a unique individual to begin with. A group actually needs something to promote a feeling of \textit{group identity}, because that doesn't just happen. An easy and effective way to do this is to give everyone in the defined group one or more of the same selections of optional Powers. An entire family where everyone is super strong, or every member can see ghosts has obvious traction, and relates members of the family one to another.

These kinds of subgroups can be pretty small -- often appearing in only one city and being of merely sufficient size to be interesting actors in a single storyline. And they have different names depending on what they are a subgroup \textit{of}. A line of Vampires where each of the children ends up with specific "optional" Powers that the progenitor also had is called a "bloodline". A line of Lycanthropes where each newly risen victim has a specific and recognizable power shared by their attacker is called a "strain". Related Leviathan whose abilities manifest in a similar way are called a "house" if you want to be fancy and a "tribe" if you don't. A group of Witches who learn their similar powers in a similar way is called a "school". Animates who are built by the same technique are rare indeed, but are called a "version". Transhumans don't really \textit{have} a name for this sort of thing, because they experience it very differently. The Reborn only happen like this in groups of 2-4 people whose past lives intersected in all kinds of ways (usually as lovers or enemies -- or both), and the \textit{process} is called "fate linking," but there is no special name for the people whose fates have been so linked. People who become Fallen by being cast into the Dark Reflection together or having had their soul yanked by the same demon or artifact may well get similar powers -- and they are called a "chain". The Icarids have pretensions to science and do not have consistent nomenclature for the phenomenon. Each competing mad scientist has their own theory of how it works, and the fact that a process repeated on another mad luminary produces the same result is in no way surprising or noteworthy to the luminaries who did it -- and they end up calling themselves whatever it is that they call the results of their process according to their personal nomenclature.

So the Sawyer Tribe is a group of closely related Troglodytes who are all fearfully strong. Every one of them has Devastation as one of their Powers. And if a player wished to make a character who was born into the Sawyer Tribe, they would select Devastation as one of their additional Powers upon applying the transformation. On a less hoboriffic note, within the Ulmi there is a core of dedicated immortal necromancers who teach the original Venetian necromancer's secrets of power. This is the Ulmi School, and those luminaries who are trained in it become Khaibit who have Patience of the Mountains. The Sisters of Cacophony is the name given to a lesbian Strigoi who calls herself Cacophony and the exclusively female bloodline she has founded, with each new inductee manifesting with Missing Voice and Death Note.

%%%%%%%%%%%%%%%%%%%%%%%%%%%%%%%%%%%%%%%%%%%%%%%%%%
\section{I fought the Law\ldots{}}
%%%%%%%%%%%%%%%%%%%%%%%%%%%%%%%%%%%%%%%%%%%%%%%%%%
\tagline{Twinkies are the best friend I ever had.}

Actions that characters take have consequences. And not just in the direct Aristotelean sense of how when you push on an object it moves. Both directly, and indirectly in the form of reactions and reprisals from others, actions will set other actions in motion. And perhaps no reaction gets as much justifiable consideration as the reactions of Justice and Revenge.

%%%%%%%%%%%%%%%%%%%%%%%%%
\subsection{Human Law}
%%%%%%%%%%%%%%%%%%%%%%%%%

\hspace{\parindent} The militaries of the world have literally millions of soldiers under arms, and it is essentially impossible for any man to fight them alone. And yes, even paramilitary forces such as police and investigative units have so may members that no one could plausibly face them alone. The 123 core precincts of New York have over 40,000 police officers and train hundreds of new cadets every year. Even a highly effective serial killer simply could not take out police as fast as the state hired more. And yet it is manifestly true that people get away with committing crimes every day. Your chances of getting brought to justice for killing a man in cold blood are about 2 in 3, and in ages past the rates of case solving were much less.

So how can it be that a man can get away with a crime if they can't actually fight the state and win? Well, mostly by dint of the fact that the the number of people involved in actually enforcing the law is very much less than the number of people the law is being enforced \textit{on}. While 40,000 police officers sounds like a lot (and it is), those same 123 precincts contain 8 million total residents and an equal number of visitors every day. Law enforcement simply cannot spare limitless resources to a single crime, nor can they reasonably expect to punish all crimes or even \textit{know} about all crimes. Crime in human societies is defined and punished in order to hold society together. Murder is a crime in every nation because it tears society apart to attack individual members of it. Treason is an even more serious crime because it is an attack on society itself. And a good thing to keep in mind is that a whole lot of things are criminalized not because they are actions that harm society in any measurable way, but because the act of criminalizing them purchases the loyalty of people who want people who do them to be punished. That is to say that so called "victimless crimes" don't really tear society \textit{apart}, but that the act of \textit{persecuting} people who do them can bind the rest of society together. That kind of logic has driven the creation of laws for thousands of years -- if you crack open a bible you can find the law against wearing the kinds of shirts that the neighboring tribes wear (Deuteronomy 22:11).

But in any case, as a supernatural creature, most player characters are going to at various times break laws. So not getting caught is going to be a pretty important thing to consider for a lot of characters. The actual laws of individual municipalities are available for you to look up, and are often a surreal trip.

%%%
\subsubsection{No report, no crime}
%%%
\hspace{\parindent} Remember that the police are not omniscient. They only investigate crimes that they know about. This does not mean that murdering every person you steal from is some kind of magic talisman against police interference -- far from it. People generally have \textit{schedules}, so even if the body is never found, people still get reported when they are merely \textit{missing}. However, it's important to note that if you steal something in a manner that the owner does not know that it is gone, or is afraid to \textit{report} that it is gone (for any of a number of reasons), then the cops will not get called, and from the standpoint of the government it is just like no crime occurred.

%%%
\subsubsection{The Police have shit to do}
%%%
\hspace{\parindent} Crimes that are "minor" compared to the amount of work needed to do anything about them are likely to be ignored. That means that if you are caught shoplifting and you flee to Indonesia, the police are \textit{unlikely} to follow you. But hey, Javert chased Jean Valjean for 17 years for stealing a loaf of bread, so there is no guaranty that the cops will lay off just because it would be totally ludicrous for them not to.

%%%
\subsubsection{The Police try not to rock the boat}
%%%
\hspace{\parindent} The purpose of the police force is to maintain society, not to tear it down. Investigations that seem like they will cause more damage to society than they will fix will usually not even get \textit{started}. Major pillars of the community can be \textit{suspected} of some pretty heinous things without any serious police inquiry. This is how priests can get away with molesting children for years without the authorities actually doing anything. However, it's important to remember that political power is not the same as untouchability. Powerful people often have powerful enemies, and law enforcement may feel itself forced to act if they actually have overwhelming evidence given to them. 

%%%
\subsubsection{The Police cannot punish everyone}
%%%
\hspace{\parindent} There are a lot of people in the world, and basically all of them did \textit{something} that they wouldn't want their neighbors to know about. And the police can't cover it all. What this means is that they will generally only act when they are \textit{sure} that a specific person did a crime. This means that ambiguity is your friend. Even if there is a question of whether one of two different suspects perpetrated a crime, most human justice systems will allow both to walk free. However, remember that police are people too, and often get totally irrational "hunches" that one person or another is a criminal and will do whatever it takes to make something stick.

%%%
\subsubsection{Human Law In Practice}
%%%
To make a long story short: basically when you go out into the wilderness and blow the crap out of a bunch of zombies and then murder the necromancer who raised them, you've committed like a dozen felonies, but neither you nor anyone else is going to jail for it. None of those people were in any database as being \textit{alive}, so their \textit{deaths} won't get reported anywhere either. No witnesses are around who will talk to the police (squirrels do not count), so even if they eventually found a bunch of bodies that were dug up and filled with shotgun pellets, the authorities wouldn't have any leads to follow and the case would be as cold as the cadavers.


%%%%%%%%%%%%%%%%%%%%%%%%%
\subsection{Syndicate Law}
%%%%%%%%%%%%%%%%%%%%%%%%%

\hspace{\parindent} Supernatural law differs little in basic intention from its mortal counterpart. Essentially it is there to keep society and those within it safe and to perpetuate itself as a social organization. Where the Syndicates differ from most modern concepts of legal systems is that they are actually just there to preserve a very small clique. The World Crime League isn't particularly concerned with whether Thailand continues to exist, or even how many people die in Kuala Lumpur (their capital). All they care about is preserving the organization of the World Crime League and their own membership -- which is only about 150,000 \textit{world wide}. Which means that the World Crime League seriously does not have rules against many of the traditionally thought of "natural" crimes. They don't care if you steal, or rape, or murder. They only care if you endanger the apple cart. It's exactly the kind of system you'd think would be invented by pragmatic, immortal, man-eating \textit{monsters}. Which of course it was.

So what is it that the bogeyman fears? What things could you do that would threaten the existence of supernatural society? Well, \textit{lots}

%%%
\subsubsection{The Vow of Silence}\index{Vow of Silence}
%%%
\hspace{\parindent} Back in the "old nights" the "Tradition of Misdirection" was simply a set of informal rules that you couldn't tell normal people what worked and what didn't work as regards fighting supernatural creatures. Because even the ancient vampires understood that they would have a hard time fighting a hundred mortal humans with wooden spears in the daylight. Back then it was perfectly acceptable to openly be a Vampire Queen or whatever, but spoiling the mystery of how vampires worked to the peasantry was considered an attack on every single other vampire, and would be met with reprisals and concerted disinformation campaigns to reconfuse the issue.

The Vow of Silence has generally replaced the Tradition of Misdirection in modern nights. Hiding the weaknesses of monsters has become a bigger and bigger deal in most parts of the world as human populations and human technology have expanded so much in the last two centuries. These nights, giving away to the "general public" that vampires are real \textit{at all} is generally considered to be as bad as telling people the specific types and weaknesses of vampires was in ages past. The 18th and especially 19th centuries were marked by some awe inspiring blood baths of supernatural creatures at human hands -- the Wolf Khans are apparently \textit{all dead}. And the reaction of most Syndicates has been to hide more than just their specific Achilles Heels, and in modern nights the general assumption is that human scientists could figure out the weaknesses of Werewolves or Strigoi quite rapidly if they ever started investigating the matter. So if someone were to leave strong evidence of the supernatural, most Syndicate responses are going to be to discredit that evidence (by destroying it, making it look faked, suppressing it in the news, or whatever), and to punish those involved to the point that it encourages others to not do that kind of thing in the future. And yes, if a Makhzen Prince has to do a lot of work to suppress some Vow of Silence Breech, they are well within their rights to have the perpetrator killed.

But not everyone sees it that way. The Shattered Empire and the Covenant Domain of Ciudad de Mexico hold that the "good old nights" where a Witch could have their own Witch Tower and have a fearful populace come groveling to them when they wanted some magic done can be achieved again in the here and now. These groups hold to the old ways of Misdirection where freaking the mundanes is acceptable and even in cases \textit{encouraged}, but this does not mean that the wearing of masks is not practiced -- just that the masks worn are those of prophets, gods, and demons rather than masks of mortal men. Needless to say, this is quite a sticking point between Syndicates and domains -- with the proponents of the "new" Silence claiming that the proponents of the "old" Misdirection are inviting the downfall of everyone by opening themselves up to scientific inquiry, while the proponents of the "old" Misdirection counter back that the "new" Silence endangers everyone because the big secret can't be kept forever and in the absence of a body of misinformation the truth will become weaponized in mortal hands.

%%%
\subsubsection{The Peace}
%%%
\hspace{\parindent} Wild West style combat and intimidation doesn't really work to keep society together -- it drives people away and it drives supernatural creatures away too. And while there have been Syndicates in the past based on the "might is right" principle where the strongest were allowed to eat the smaller at any time, those Syndicates are simply not \textit{around} any more. The fact is that for any Syndicate to hold together it has to offer a better shot at surviving to the end of one's immortality than simply hiding under a rock in the wilderness. Otherwise, rational supernatural creatures are just going to run for the hills. And so it is that Syndicates find themselves charged with protecting their members -- and quite often protecting their members from \textit{other members}. This means firstly that murdering other members is highly discouraged, but it also means that every Syndicate has a forum for handling grievances such that creatures will feel properly (or at least minimally) satisfied without chopping anyone's head of.

Rules in any Syndicate tend to be pretty draconian, since they are designed by and to appeal to literal ancient monsters from before anyone had written \refwork{A Theory of Justice}. But it is important to remember that even these rules are not as kill crazy as unfettered mob justice.

%%%
\subsubsection{Respect}
%%%
\hspace{\parindent} Being a member of a Syndicate isn't just a list of "Thou Shalt Nots". It's also a set of perks. First of all, it lets you hang out with creatures that are actually in your peer group, which is \textit{awesome}, but membership also straight up has \textit{privileges}. Members of the World Crime League can call upon the organization to give them legal counsel and they can use the exclusive Syndicate pool. 

But perhaps the biggest perk that any Syndicate can offer is the respect and obedience of other members of the Syndicate. And that in turn becomes one of the most important concessions that one makes by joining a Syndicate. The Makhzen promises its members that if they work their way up to Prince of a domain that they'll be able to make the rules for that domain, and that other members of the Makhzen \textit{will follow those rules}. And thus, every member of the Syndicate is expected actually honor the perks that other members have earned within the Syndicate. You have to act like everyone else has their carrots for you to receive yours. Telling the Quartermaster of a World Crime League territory to go fuck themselves can get your pool privileges revoked -- or even get you booted from the Syndicate's protection altogether.

%%%
\subsubsection{Syndicate Law In Practice}
%%%
\hspace{\parindent} Rules governing these concepts are expressed differently in different Syndicates. For example, in the Makhzen, each of those three concepts is expressed as two separate "traditions". The Vow of Silence is the "Tradition of Lies" (basically: "Don't talk about fight club") and the "Tradition of Truth" (basically that you tell the creatures in the Syndicate -- and \textit{only} them -- what's going on and how things work). The concept of The Peace is the "Tradition of Hospitality" (that you let other supernatural creatures into your city and social circle) and the "Tradition of Hostility" (that killing supernatural creatures is a right and duty reserved to the Syndicate to be used against -- and \textit{only} against -- creatures that break the Traditions). And the concept of Respect gets broken up into the "Tradition of Accounting" (that every member is due the respect owed their status in the Syndicate) and the "Tradition of Domain" (which is basically the same thing, but includes the idea that the Princeps defines the rules and status within their domain). Other Syndicates use different formulations, but all of them cover those three concepts on way or another, because it's the essential glue that keeps supernatural society together.
