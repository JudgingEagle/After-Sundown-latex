%%%%%%%%%%%%%%%%%%%%%%%%%%%%%%%%%%%%%%%%%%%%%%%%%%
%%%%%%%%%%%%%%%%%%%%%%%%%%%%%%%%%%%%%%%%%%%%%%%%%%
\chapter{After Sundown}
%%%%%%%%%%%%%%%%%%%%%%%%%%%%%%%%%%%%%%%%%%%%%%%%%%
%%%%%%%%%%%%%%%%%%%%%%%%%%%%%%%%%%%%%%%%%%%%%%%%%%

\tagline{"I hope you like nightmare worlds!"}

After Sundown is a cooperative storytelling game that tells stories in the realm of horror. Players take on the roles of monsters out of horror movies or the humans who oppose them, while one of the players takes on the role of the MC -- a combination referee, narrator, and roleplayer of last resort for antagonists and minor characters in the story. 

Cooperative storytelling can be done without any products at all, as with collaborative writing or \refwork{Cops and Robbers}. After Sundown provides structure and conflict resolution in the form of an established world and story, as well as with a set of mechanics to determine the results of actions with the help of six sided dice. In this way, players of After Sundown can bypass many of the hangups of both collaborative fiction and Cops and Robbers: most notably the "I shot you/ No you did not" problem. It is hoped that the backstory and established characters of After Sundown will be sufficiently evocative as to give players of protagonists and MCs ample launching points for stories of their own. 

The materials in this book are open content. You can print copies, trade electronic copies with friends, modify the files, or produce derivative work. If you like After Sundown enough, go ahead and "buy" a pdf. But if you'd rather trade it around as a torrent, that is fine too.

%%%%%%%%%%%%%%%%%%%%%%%%%%%%%%%%%%%%%%%%%%%%%%%%%%
\section{After Sundown: An Introduction}
%%%%%%%%%%%%%%%%%%%%%%%%%%%%%%%%%%%%%%%%%%%%%%%%%%
\tagline{To write a story together, everyone must be on the same page.}

One of the primary purposes of a cooperative storytelling \textit{game} is to provide a foundation upon which stories can be told. The other is to provide a framework by which disagreements about how a story should progress can be worked out in an acceptably impartial fashion. 

The setting of After Sundown is a world like our own would be if horror fiction had an element of truth to it. There really are monsters in the night and other worlds full of nightmarish horrors that bleed into the mortal world. But it is also set in a world which is decidedly modern, and that means modern sensibilities. The game's backstory sees history and mythology through a modern interpretation, and adopts horror tropes that resonate with modern audiences. Many horror tropes are timeless -- blood speckled claws in the dark is pretty much always going to be scary -- but many other horror elements are merely puzzling, and are going to be downplayed. The modern audience is not particularly worried about miscegenation or communist invasion, and those elements of old horror fiction are deliberately excluded from their appropriation into After Sundown.

%%%%%%%%%%%%%%%%%%%%%%%%%%%%%%%%%%%%%%%%%%%%%%%%%%
\section{Monster Means Many Things}
%%%%%%%%%%%%%%%%%%%%%%%%%%%%%%%%%%%%%%%%%%%%%%%%%%
\tagline{A story is finite in length.\\
To have anything in it, an infinite number of things must be excluded.}

Ask a dozen people to describe vampires or witches, and you'll get a dozen different answers. And that is a tremendous problem for cooperative storytelling, because everyone is supposedly trying to add to the \textit{same} story. Stories told in After Sundown may have vampires in them, but these are \textit{not} the vampires written about by Stoker or Rice, they are the vampires in the stories told by \textit{your gaming group} set in the realm of horror described in this book. These vampires have an aesthetic that is informed by horror movies, comic books, and both punk and goth subcultures, but they are necessarily different from the monsters described in any particular other work of fiction, and they absolutely do not sparkle.

It is important to note that you can't take everything from myth and legend and cram it into a story. I'm not saying that your story will be completely incoherent, although of course it will be, I'm saying that you are literally incapable of doing that. \refwork{The Vampire Book} is an encyclopedia of just vampire lore from various cultures and it is literally \textit{over nine hundred pages} long. And we're not talking about character backgrounds or rules text or any of the other crap that we know eats up word count like you wouldn't believe. We're talking about just a bare list of facts by mythical origin. And it is still nine hundred pages. And while it is quite comprehensive, there are still vampire facts it does not contain. So it is imperative not only that you acknowledge that you're going to have to cut things down to a manageable amount, but also that you establish specifically what is off limits and what's fair game before you start telling a cooperative story. It is unreasonable to expect that other people sitting down at the table with you think to the same mythic source material when you mention even something as specific as Frankenstein's Monster -- the creature in the book was wicked fast but the Boris Karloff rendition was a lumbering brute.

So we're paring things down. A lot. We don't have, need, or even want a bajillion clans of vampires, or fifteen tribes of werewolves. There \textit{should} be few enough  flavors of things that all the players can remember what the differences between them are. Ideally, people should be able to play whatever supernatural guys they want, sort of like the \refwork{League of Extraordinary Gentlemen}; but in practice you have to put explicit limitations on what is part of the story or things get all weird. Like with Martian invasions and stuff, what was up with that? A story that doesn't have specific exclusions does not truly have any specific inclusions. It's not really a story at all at that point, it's a mess.

The base concept for After Sundown is that you are roleplaying a classic Universal Studios Monster and you engage in narrative driven dramatic role playing of both horror and intrigue. The Universal Horror Films were, if not documentaries, at least "dramatic reenactments" of real events in the shared world you will be telling stories in. The Invisible Man, The Wolfman, The Creature from the Black Lagoon, and of course Dracula were all real people, and the player characters can be creatures like them. And the players in After Sundown can use monster movies old and new for inspiration. But remember that the monsters in every piece of fiction are different, and that while you are telling stories After Sundown that it is the descriptions of monsters \textit{in this book} that break ties. Werewolves can transform voluntarily when the moon isn't full, Golems resist fire, and Vampires do not sparkle. Not because these creatures are like this in every movie, but because that is how they are in the stories told with the After Sundown cooperative storytelling game.

%%%%%%%%%%%%%%%%%%%%%%%%%%%%%%%%%%%%%%%%%%%%%%%%%%
\section{Things You Need To Play}
%%%%%%%%%%%%%%%%%%%%%%%%%%%%%%%%%%%%%%%%%%%%%%%%%%
\tagline{"Assuming flippant things like 'food, water, and shelter' are out of the way."}

After Sundown has one or more players roleplaying as protagonists in the story, and a single player acting as MC. The bare minimum number of players is therefore two, and there is no specific upper limit to the number of players a game of After Sundown can accommodate. It really does seem to work best with between 3 and 6 total players though.

During the game, every single player is going to want some method of keeping track of things. This can be done with pencil and paper, post-it notes, or electronically using laptops or smartphones. Different players prefer note taking and result tallying in different ways, and I won't tell you what format your character has to be kept in. A character sheet has been provided in this book, so if you want to print it out and write on it like we did in the old days, that's fine. Speaking of this book: it is distributed in electronic format, and you'll probably want to reference it (or at least have the capability to reference it) during play. That means that someone will need to bring their laptop or e-reader to the game, or print it out and bring a physical copy to the game instead. You have the right to make as many printed versions of this book as you want for non-commercial purposes, so do what you feel is best.

The actual mechanics of the game require rolling several six sided dice at once to resolve actions. I suggest having at least twelve at the table. But honestly, the more the merrier. Game play is sped up noticeably if each player has their own dice pile in front of them, and some people even like making pre-made piles of different sizes so that they don't have to count dice before rolling. So I wouldn't hazard a number of six sided dice that would serve as a maximum. 

Further, character advancement normally uses draws from a deck of cards (either poker or tarot). Since this is character advancement, it is entirely acceptable to have the card draws happen between sessions or even the beginning of next session. So the game is entirely playable for an entire evening even if no one brought a deck of cards. But \textit{eventually}, you'll want a deck of cards as well. If cards are for whatever reason unavailable, remember that virtually any computer has some sort of solitaire game loaded on it, which will among other things provide an acceptable substitute for any number of card draws.

Since you're going to be telling cooperative stories, you'll want some way to get diagrams across to the other players. This can be done with an erasable mat, or a dry erase board, or a pad of paper in the middle of a table. But it can also be done by sketching things on a computer, provided the screen is large enough for everyone to see.

After Sundown can also be played over the internet by players who each have a computer. You can roll real dice and report the results over internet chat (text or video), or you can use various automatic dice rollers. You'll still want an equal number of players, but they need not be in the same city or even country as one another.

%%%%%%%%%%%%%%%%%%%%%%%%%%%%%%%%%%%%%%%%%%%%%%%%%%
\section{The Role of the MC}
%%%%%%%%%%%%%%%%%%%%%%%%%%%%%%%%%%%%%%%%%%%%%%%%%%
\tagline{"Someone has to have the last word."}

Every player except the MC plays a specific character who is a protagonist in the ensemble fiction that the cooperative storytelling game generates. This allows deeper roleplaying and lends itself to the use of first person pronouns when describing action. However many protagonists there are, there will still be more characters that need to be written into the story, and the actions of these characters are determined by the MC. Note that I say "determined" rather than "played" by, because none of the antagonists, allies, or indifferent background characters are literally the alter ego of the MC. The MC has a very big job in telling the story, but is also the least partisan position, and so it falls to the MC to make snap judgments about what rules apply or what numbers to use for different situations. No rule system can ever be completely comprehensive (there was a comprehensive mathematical proof to that effect by G\"{o}del), and when you run into a situation that the rules \textit{cannot} resolve, the MC should act as an impartial arbitrator.

How much plots for stories are generated by the actions of the protagonists and how much are generated by external events will vary from group to group, but in either case the MC is heavily involved. Even when the other players are producing action and dialog sufficient to drive storytelling, it still falls to the MC to determine what responses are engendered in non player characters (NPCs) or the world around them. Playing the MC is demanding, and some groups rotate the responsibility.
