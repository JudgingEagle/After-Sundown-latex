%%%%%%%%%%%%%%%%%%%%%%%%%%%%%%%%%%%%%%%%%%%%%%%%%%
%%%%%%%%%%%%%%%%%%%%%%%%%%%%%%%%%%%%%%%%%%%%%%%%%%
\chapter{Running the Game}
%%%%%%%%%%%%%%%%%%%%%%%%%%%%%%%%%%%%%%%%%%%%%%%%%%
%%%%%%%%%%%%%%%%%%%%%%%%%%%%%%%%%%%%%%%%%%%%%%%%%%

\hspace{\parindent} After Sundown is a collaborative storytelling game, where all the players except one take the roles of the protagonists in an ensemble story set in the realm of horror. The odd player out is the MC, who acts as narrator, director, and actor of last resort for all the other characters in the story. The MC has substantial leeway in interpreting how events unfold and is responsible for much of the writing of the backstory. Nevertheless, the story is still \textit{about} the player characters, and no one should forget that.

Player characters will have numbers and abilities written on a character sheet that demonstrate what they are capable of, but in most cases the players themselves will determine what their character actually does. When actions are declared, dice are often rolled to determine the results of the action. But when that is not enough (or would be too tedious as is the case for many minor actions), the MC can deterministically assign results.

Actual stories are told cooperatively, with players bringing up events relevant to their character's backstory and possibilities for the MC to consider weaving into the ongoing description of the world and having real solid first person narrative control over the actions and dialog of their player character. The MC is sometimes said to be "running" the game, because they provide the narrative control on the stage where the protagonists act. 

%%%%%%%%%%%%%%%%%%%%%%%%%%%%%%%%%%%%%%%%%%%%%%%%%%
\section{Basic Dice Mechanics}
%%%%%%%%%%%%%%%%%%%%%%%%%%%%%%%%%%%%%%%%%%%%%%%%%%

\hspace{\parindent} When you perform an action, you roll a pile of d6s called a \index{Dicepool}\textit{dicepool}. Dice which come up as a 5 or 6 are \index{Hits}\textit{hits}. A task will normally require a number of hits to succeed equal to the \index{Threshold}\textit{Threshold}, and any hits gained in addition to that are \index{Net Hits}\textit{Net Hits}. If you get 4 or more net hits, you get a \index{Critical Success}\textit{Critical Success}. If a die roll generates an insufficient number of hits, it is a \index{Failure}\textit{failure}.

\paragraph{Dicepools:} Your dicepool is generally speaking a pile of d6s with dice equal to your character's Attribute + Skill, and circumstantial modifiers increase or decrease the number of dice rolled. A human's attributes and skills go up to 6. A supernatural creature's go up higher than that, both in that their Potency increases their Attribute maximums and that some of their disciplines (magic power groups are called disciplines) further increase their attributes or skills. As such, it is expected that supernatural critters will roll more dice on actions that their powers apply to than normal humans do.

The effects one can expect out of getting a number of Hits are proportionately more awesome as the number of hits increases:

\begin{table}[htb] \rowcolors{1}{white}{tan} \center
\caption[Awesomeness per Hit (Running the Game)]{Awesomeness per Hit}\index{Awesomeness} \hspace{3cm}
\begin{tabular}{c l}
\textbf{Hits:} & \textbf{Awesomeness} \\
\textbf{0:} & Not Awesome. Tying shoes, climbing stairs.\\
\textbf{1:} & Completely Pedestrian. Driving a car, Throwing Darts.\\
\textbf{2:} & Professional. Don't try this at home.\\
\textbf{3:} & Hard. Don't try this at all.\\
\textbf{4:} & Extreme.\\
\textbf{5:} & Crazy Extreme.\\
\textbf{6:} & Super Human. Does not need disclaimers because it is clearly impossible. \\
\end{tabular}
\end{table}

\paragraph{Resistance Tests:} \index{Resistance Tests}Sometimes a character will be allowed to \textit{resist} something being done to them. This is done by rolling dice like normal, save that rather than generating an awesome result, the character is merely reducing the number of hits against them, making whatever is being done to them take less or even no effect. In general, a Physical Resistance Test is simply a Strength roll, a Mental Resistance Test is an Intuition roll, and a Social Resistance Test is a Willpower roll. If a character has an Edge score, it is added to these Resistance tests. When a character is struck with a weapon, they may be called upon to make a Soak\index{Soak Test} roll, which is a special kind of Physical Resistance test. In most cases, if the Resistance Test reduces the number of hits on a test to zero or less, the original attempt has failed.

%%%%%%%%%%%%%%%%%%%%%%%%%%%%%%%%%%%%%%%%%%%%%%%%%%
\section[Basic Attributes]{Basic Attributes: Physical, Mental, and Social}
%%%%%%%%%%%%%%%%%%%%%%%%%%%%%%%%%%%%%%%%%%%%%%%%%%

\hspace{\parindent} Characters in After Sundown have six basic attributes that are divided into Physical, Mental, and Social attributes. These basic attributes range in value between 1 and 6 for normal humans.

\textbf{Physical Attributes}

\hspace{\parindent}\textbf{Strength:} \index{Strength}Strength determines how physically strong \textit{and} tough you are.

\hspace{\parindent}\textbf{Agility:} \index{Agility}Agility is a combination of precision and speed.

\textbf{Mental Attributes}

\hspace{\parindent}\textbf{Intuition:} \index{Intuition}Intuition is a combination of empathic and physical perception.

\hspace{\parindent}\textbf{Logic:} \index{Logic}Logic is a combination of scientific know-how and logical intelligence.

\textbf{Social Attributes}

\hspace{\parindent}\textbf{Willpower:} \index{Willpower}Willpower is a combination of determination and domination.

\hspace{\parindent}\textbf{Charisma:} \index{Charisma}Charisma is one's ability to convince and ingratiate.

The primary purpose of basic attributes is to set the dicepools for actions. A character with a high Agility will have a bunch of extra dice to roll for every skill test that invokes Agility. A character with a high Charisma will have extra dice on every test that uses Charisma, and so on. A character's basic attributes represent a broad aptitude in a wide variety of actions.

%%%%%%%%%%%%%%%%%%%%%%%%%%%%%%%%%%%%%%%%%%%%%%%%%%
\section[Special Attributes]{Special Attributes: Edge, Power, and Potency}
%%%%%%%%%%%%%%%%%%%%%%%%%%%%%%%%%%%%%%%%%%%%%%%%%%

\hspace{\parindent} \textbf{Edge}\index{Edge} is a measure of narrative importance that Luminaries have. You might say it's how much of a protagonist, antagonist, or other sort of main character someone is. All Extras have an Edge of zero. Your Edge stat gives you a number of \textbf{Action Points}\index{Action Points} per chronicle, and also adds to all of your Resistance Tests. Whenever an ability says that the target resists with their Strength or Willpower (or whatever) you can assume that it means "plus Edge" if the target is a Luminary and actually has an Edge stat.

Action Points can be spent in one of several ways:
\begin{itemize*}
\item You can purchase a number of bonus dice on any test equal to your Edge score.
\item You can re-roll all of the failed dice on a test.
\item You can gain an extra Initiative Pass during a combat round.
\item You can avoid a seemingly certain death.
\end{itemize*}

Action Points refresh between chronicles, and you can also regain an Action Point if you do something that puts you closer to achieving your Driving Passion (see Chapter 10).

When you want to spend an Action Point to gain a benefit on a single test, all you really have to do is say that you are doing so. Mark it off and you'll see it again when a new chronicle starts. When spending an Action Point for an extra combat action, you announce that you are doing so at the end of the Initiative Pass that you last acted in. But when you are buying your way out of a deadly situation it is more complicated. At this point you are having fate intervene to save you in a narratively plausible fashion. This will require a negotiation of some kind with the MC to determine what is plausible under the circumstances. Maybe it's the police showing up before the Trolls have the opportunity to put the boot in, maybe it's the floor collapsing and dumping your comatose character into the basement before the flames have an opportunity to burn them to death. It is advisable that a character being saved in such a manner be inconvenienced in some fashion. It's intended as a last ditch save-your-ass moment, not something to be smugly relied upon.

Finally, every two points of Edge a character has also increases the absolute maximum they can raise their skills to during play. While characters start with skill ratings of 6 or less, a character with an Edge of 2 or 3 can eventually raise them to 7, while a character with an Edge of 4 or 5 could raise their skills to 8. A character with an Edge of 6 can potentially get a skill of 9. Six is the maximum Edge for any character (regardless of Potency, as it is a special attribute).

\textbf{Power Reserve}\index{Power Reserve}\index{Power Points} in After Sundown is a parallel attribute similar to Edge. Rather than being spent on any test, Power Points are spent to activate specific supernatural abilities that a character might have. Additionally, you can spend a Power Point to get +2 on a single test to resist magic being used against you. Non-supernatural creatures don't have Power Points at all, and different supernatural creatures refresh their power points in different ways according to their \textit{Power Schedule}\index{Power Schedule}. Characters have a \textbf{maximum} Power at any given time of 10 points plus 3 points per point of Potency. This means that most player characters (who have a Potency of 1) will have their Power Reserve fill up at 13. Mortals and other Potency 0 creatures have a Power Point maximum of 10, but likely have no ability or reason to acquire Power Points.

\medskip

\textit{For example: Genevra is a Vampire with the Quickness ability and the Vigor ability. She can spend 3 Power Points to take extra actions during a scene with her Quickness power. In addition,  She can spend 1 or more Power Points to increase her Strength for a scene with her Vigor power. Because she is a vampire, she can refresh her Power Points by drinking blood from other people.}

\medskip

\textbf{Potency}\index{Potency} When a character's powers increase they may get a special attribute called \textit{Potency}. A character's Potency is added to the \textit{limits} (but not necessarily the actual value) of each of their Physical, Mental, and Social attributes, and every point of Potency increases their maximum Power Reserve by 3. Normal humans and some weak supernatural creatures (such as Mirror Goblins) have a Potency of zero. Starting supernatural creatures of a playable type such as Vampires or Witches have a Potency of 1. This rating gradually rises as the creature ages and grows in eldritch power, and can rise quite abruptly by slaying powerful elders \refwork{Highlander} style or by attuning to powerful artifacts, completing mighty rituals of vast power, or otherwise reaching the kinds of breakpoints in a story in which a monster might become nearly unstoppable. The most powerful named characters in the setting (such as The King with Three Shadows or Echidna) have a Potency of 10, and player characters can expect to have a Potency much, much lower than that.

\medskip

\textit{Genevra is a Vampire with Potency 1. This means that her \textbf{maximum} value for raising her Agility, Willpower, and other attributes is 7. Six maximum for having started as a human, plus her Potency of 1. Her maximum Power Reserve is 13.}

%%%%%%%%%%%%%%%%%%%%%%%%%%%%%%%%%%%%%%%%%%%%%%%%%%
\section{Character Generation}
%%%%%%%%%%%%%%%%%%%%%%%%%%%%%%%%%%%%%%%%%%%%%%%%%%
\tagline{Scene opens on a ringing cellular phone. Camera pans back to reveal\ldots{}}

\index{Character Creation}Characters in After Sundown run the gamut of power. Normal human extras are slaughtered in groups by monsters run amok in the fine tradition of slasher movies the world over.  And yet even those monsters live in terror of even larger monsters from the ancient past. It can best be thought of in terms of regular horror movies going on simultaneously with the events from \refwork{Hellboy} or \refwork{Queen of the Damned}. The Characters are created at a point just before the story starts. And thus, the point where the characters begin is dependent upon what kind of story is being told:

\begin{description}
\item[The Origin Story:] \index{Origin Story}The characters begin with little or no knowledge of the supernatural as human Luminaries. Over the course of the story they discover magical powers within themselves, attain magical powers, or simply come into conflict with supernatural agents.

\item[The In Media Res Adventure:] \index{In Media Res}The characters begin having already been supernatural creatures for some time. They have come to terms with what they are, they have made social connections amongst other supernatural creatures, and they are already members of supernatural organizations.

\item[The Power Fantasy:] \index{Power Fantasy}The characters begin already powerful within the context of supernatural society. Each character is a thing to be feared, whose name is uttered in whispered tones by creatures who themselves inspire fear in mortal hearts.
\end{description}

Regardless of what kind of story is being told, remember that it is going to be an ensemble cast with each player's character being one of the story driving protagonists. Each character thus should have interests and goals that drive the plot forward, not just passive interests or reactive interests. It's fine for a character to be a doctor or a police officer, but they can't just be a passive observer waiting for people to get sick or cars to be reported stolen. A character worth playing thus needs to have a goal that they can work towards when there \textit{isn't} a fire to be put out. It's fine to play a character who can heal others, but the character \textit{needs} a hook that can get them motivated to pursue goals and advance the plot when there isn't anyone injured in their presence.

%%%%%%%%%%%%%%%%%%%%%%%%%
\subsection[Origin Story]{Characters for an Origin Story}
%%%%%%%%%%%%%%%%%%%%%%%%%
\tagline{Don't be silly, there's no such thing as vampires.}

In an Origin Story\index{Origin Story}, the characters are, or at least \textit{believe themselves} to be mortal humans. As such, the player creates their character as if they were a mortal human Luminary. Generally over the course of the story, the player characters will be embraced by vampires, mauled by werewolves, or even discover that their memories of growing up in Indiana are digital imprints and they've been a robot the entire time. But becoming cursed with magic powers or discovering that they have had them all along is something that doesn't happen until after character generation. This makes characters who are substantially better than normal humans and makes sure that they have a diverse set of competencies. They \textit{are} the Protagonists, after all.

\begin{description}
\item[Attributes]
All of a character's attributes start at 1. The player then prioritizes their Physical, Mental, and Social attributes, distributing 1 point to one pair, 3 points to another pair and 5 points to the last pair. Then they get 2 additional points that they can place anywhere they want. An individual attribute cannot be higher than 6 on character creation.
Luminaries begin the game with an Edge of 3.

\item[Active Skills]
A character's Active Skills start at zero. The player then prioritizes their Physical, Social, and Technical skills, distributing 11 points to one set, 16 points to the next set, and 21 points to the last set. Then they get 6 points they can place anywhere they want. An individual skill cannot be higher than 6 on character creation. The character then chooses three skill specializations. Remember that they also gain a specialization for each Technical skill they have trained.

\item[Backgrounds]
A character starts with 27 points of Backgrounds. No Background can start higher than rating 6. An important thing to note is that a character in an Origin Story has a job, some social networks and maybe a family. They almost certainly \textit{don't} have Backgrounds like "Makhzen Society" or "Black Spot Ethos." 

\item[Resources]
The player chooses one 3-point Resource, one 2-point Resource, and one 1-point Resource. At the MC's discretion, a player may be able to buy more Resources with Obligations. Mortal humans are not normally able to take Destiny or Secrets.

\item[Motivations]
Human characters do not normally have Master Passions. However, they still do have Driving Passions and Ethical Taboos. So the player should define some for their character. The player should really think about what their character wants, and what their character is willing to do to get what they want.

\item[Advantages and Disadvantages]
Characters can have Advantages and Disadvantages, but the number of the one should equal the number of the other.
\end{description}

Sometimes players will want to skip the portion of body horror that goes with actually transforming into a supernatural creature. In these cases the characters can have their transformations already applied, and then tell the origin story of the characters being introduced to supernatural society.

%%%%%%%%%%%%%%%%%%%%%%%%%
\subsection[In Media Res]{Characters for an In Media Res story}
%%%%%%%%%%%%%%%%%%%%%%%%%
\tagline{Whether you like it or not, you're in the middle of a war that has been raging for the better part of a thousand years.}

In Media Res is the storytelling technique of beginning the narration in the middle of the action. It can be exciting and engrossing, and can lead to greater audience attention, especially if the origin story is somewhat tangential to the primary events of the story. Sometimes the story will go back in flashbacks to produce the past events that led to the circumstances with which the story began like in \refwork{Pulp Fiction} or \refwork{Memento}, and other times the action will simply continue towards conclusion like \refwork{Ocean's Eleven} or \refwork{Three Kings}. In any case, those are all solid pieces of In Media Res storytelling and you should watch those movies if you haven't already.

Characters for an In Media Res\index{In Media Res} story begin with all the introductions to supernatural society well out of the way. They are already known by and cognizant of the major Syndicates and they already have membership in a cult (if they want one) and have come to an understanding of the basic score of the horror-inspired world they live (or at least persist) in. They are supernatural creatures themselves, and have been that way for long enough that neither using their powers nor seeing others use their powers actually surprises them any more. An In Media Res story is a good place to start for groups that want to tell stories about events \textit{in} the realm of horror rather than ones exploring their characters' reactions \textit{to} the realm of horror, and as such it is considered the default game.

\begin{description}
\item[Attributes]
All of a character's attributes start at 1. The player then prioritizes their Physical, Mental, and Social attributes, distributing 2 points to one pair, 4 points to another pair and 5 points to the last pair. Then they get 3 additional points that they can place anywhere they want. An individual attribute cannot be higher than 7 on character creation.
Luminaries begin the game with an Edge of 3.

\item[Active Skills]
A character's Active Skills start at zero. The player then prioritizes their Physical, Social, and Technical skills, distributing 14 points to one set, 19 points to the second set, and 24 points to the last set. Then they get 6 points they can place anywhere they want. An individual skill cannot be higher than 6 on character creation. The character then chooses four skill specializations. Remember that they also gain a specialization for each Technical skill they have trained.

\item[Backgrounds]
A character starts with 35 points of Backgrounds. No Background can start higher than rating 6. It is entirely reasonable for a character to have been out of mortal society long enough that they don't have any "Corporate Culture" or "Service Work" type Backgrounds. On the other hand, it's equally plausible for such a character to have been keeping up appearances in the mortal world (or even being involved in mortal affairs when the story begins) and thus have such Backgrounds. Mixing characters who don't know how to handle themselves in a 7-11 with streetwise modernists in the same Band can be a good roleplaying hook.

\item[Resources]
The player chooses one 3-point Resource, two 2-point Resources, and one 1-point Resource. The player must take an Obligation of rating 3 or less, but they get to buy an extra Resource for taking that Obligation as normal. At the MC's discretion, a player may be able to buy more Resources with more Obligations. Some characters will have retreated entirely from the mortal world or have been out of circulation long enough as to make no difference. As such, it is entirely possible that the character has no access to Resources in "mortal life" at all. It's not weird for characters to begin an In Media Res story living in a cardboard box in an alley with no job or registration in government documents.

\item[Motivations]
The character in an In Media Res story presumably has a Master Passion. If it's not the same as the default one for their supernatural type, there should be a good in-character reason for that. While they are supernatural and non-human, they should presumably still have Driving Passions and Ethical Taboos. So the player should define some for their character. The player should really think about what their character wants, and what their character is willing to do to get what they want.

\item[Advantages and Disadvantages]
Characters can have Advantages and Disadvantages, but the number of the one should equal the number of the other.

\item[Magical Transformation]
Characters in an In Media Res game have been supernatural for some time and have developed some tricks that are their own in addition to having a mastery of their form and the basic powers that come with it. The character has the 6 Basic and 2 Advanced disciplines common to their type, and have developed 2 Basic Disciplines and an Advanced Discipline that are theirs. In addition, the character knows one Basic or Advanced Discipline that must be from a Universal Discipline or the Sorcery that their Cult (if any) specializes in. The character has a Potency of 1 and therefore a Power Reserve of 13.

\item[Place in the Worlds]
The character is \textit{involved} with Supernatural Society. They already are a citizen of one Syndicate or another (this need not be the dominant Syndicate in the city the story begins in, the character could easily have spent formative years elsewhere or under the tutelage of another who was). If the player wishes their character to be a member of one Cult or another, they can simply declare that on character generation.
\end{description}

Sometimes players will want to play "monster hunters" rather than supernatural creatures. For an In Media Res story about Van Helsings, Watchers, and Whistlers, it is plausible to begin the story where the characters are \textit{not} magical creatures and are instead badass human Luminaries fighting \textit{against} the supernatural monsters of the week. In such a case the characters don't start with a supernatural type or any disciplines, but they still may well have Resources and Backgrounds dealing with the occult because in an In Media Res story, the hunters have already done this sort of thing at some point in the past. Hunters get an Edge of 4 and an increase of two other Attributes of their choice. However, they have no Potency stat, so their attribute maximums are still 6.

%%%%%%%%%%%%%%%%%%%%%%%%%
\subsection[Power Fantasy]{The Power Fantasy}
%%%%%%%%%%%%%%%%%%%%%%%%%
\tagline{I will crush you.}

Characters in a Power Fantasy\index{Power Fantasy} game have the strength to throw their weight around and challenge other powerful creatures in personal conflicts. There are unfortunately no firm guidelines that can be given to what marks a good Power Fantasy Character. A chronicle where the characters are expected to ultimately square off against some wicked Ifrit may well have characters come in with a Potency of 2, while a chronicle where the characters were ultimately going to be up against The King with Three Shadows would expect characters with a Potency of 6 or even 7.

The important consideration is that characters in a Power Fantasy game should have roughly \textit{equal} bonuses. Although it is important to note that characters in such a game will be expected to be more divergent in capabilities. The nature of the dicepool system means that specialists in any field will roll more extra dice in their specialty than the other characters do than in an Origin Story or In Media Res game. It can lend the air of a game of rocket launcher tag in Doom or Unreal. Which for a game where things are supposed to feel powerful is fine.

%%%%%%%%%%%%%%%%%%%%%%%%%
\subsection{Placing Opposition}
%%%%%%%%%%%%%%%%%%%%%%%%%
\tagline{In the third series, one of the main antagonists is her own daughter from an alternate timeline where she has become evil and resentful. There weren't a lot of things left in the current world that could challenge her after the shenanigans of the second series.}

The key thing to remember as an MC is that you don't \textit{get} anything for thwarting the protagonists. While it is generally a good idea to make the progression of the chronicle feel somewhat challenging, there is no special benefit to be had by terminating the chronicle early due to failure or character death. When designing opposition for the player characters, the goal should be to maintain the verisimilitude of the world first, and to provide interesting and exciting challenges to the players second. "Defeating" the players is not a meaningful or acceptable goal for the MC to have.

After Sundown has some seriously terrifying stuff in it, but that doesn't mean that the protagonists have to be challenged by it all the time. Indeed, if the players face nothing but enemies scaled to their abilities, the power of the characters won't ever be apparent. A Werewolf on a rampage in war form is a very scary thing, and a couple of police officers are very unlikely to be able to effectively contain it. Having the players periodically overpower challenges of this sort in a casual manner is good for the game. At the same time, if the protagonists never take on enemies that are near their power level, their adventurers will gradually lose interest.

Players should work with the MC to direct their adventurers in a manner where they come into direct conflict with opposition that is in their ecological niche. The MC should take aims to make the world feel like it is persistent and continues to have events and conflicts while the player characters are not around. This means among other things that the powerful elders in the worlds should be available to be communicated with and opposed. The MC should also frankly discuss political and military realities such that players can pick battles they can win. While the choice between two unmarked doors leading to victory or defeat may be fair, it does not \textit{feel} fair and it doesn't really make for an interesting story.

%%%%%%%%%%%%%%%%%%%%%%%%%%%%%%%%%%%%%%%%%%%%%%%%%%
\section{Advancing Goals}
%%%%%%%%%%%%%%%%%%%%%%%%%%%%%%%%%%%%%%%%%%%%%%%%%%
\tagline{Ultimately, in every situation, everyone does what they want to do.}

The realm of horror has pretty much everything in it that our world does, plus several additional planets of essentially equal size full of strange and poorly explored places of mystery. So it is entirely reasonable to have players who simply want to play After Sundown like a four planet sized sandbox. This can be a rewarding experience, but you should take care to make sure that players have some \textit{direction} so that the game doesn't end up with the characters simply whiling away their immortality (or simple inhuman power in the case of supernatural characters who are not literally immortal) in a booth at Der Wafflehouse wondering what all is going on in the world. When the game direction is player driven in this manner, the core impetus comes not from antagonists or world events but from the different goals that the player characters have.

These goal driven games require a delicate negotiation before they even begin, because if the characters have goals that are not compatible or at odds then the game can quickly devolve into a "Let's go this way!" argument or worse: actual in-party combat. That's pretty much the end of the game, and is thus only suited to games of pre-defined limited length for which the end of the game is pretty much already a foregone conclusion.

So it falls to the MC to dangle some plot hooks that player characters might jump on. But it also falls on the MC \textit{and} the players to make sure that the goals set by the different characters can coexist in the same band and work towards the completion of the same stories. Directly antagonistic goals being advanced by different characters may sound like a cool source of in-game tension and roleplaying, but it actually just sucks. In order to keep things from falling apart into recrimination immediately, the characters all have to be constantly distracted by "bigger problems" like threats to the world and shit, and that basically means that the characters never get to meaningfully interact with their character goals and the entire chronicle is frustratingly on rails the entire time. Fuck that. Players have a mandate to discuss their character's goals with the other players before the chronicles even begin, and even to make adjustments if necessary to make sure that unity within the band remains a possibility.

It is important that players realize that they do not have the "right" to play any character concept that they want. They have a right and a responsibility to play a character who is capable of being involved in the stories that the other characters are participating in. Characters who don't want to do interesting things, who want to be alone, or who simply do not want to do the things that the other characters want to do are not acceptable character concepts. But also remember that this is a \textit{cooperative} storytelling game, and that an incompatibility between two players' character concepts is not a problem of only one of the players. The players should reconcile their characters together so there is a reason that they would be involved in stories together. Sometimes this can be as simple as one of the players choosing to play a different character; but most of the time it involves both players compromising their characters somewhat.

For very short games it can be OK to have characters of wildly different ideologies thrown together by circumstances and a shared need for survival. Many haunted house movies rely on this conceit as do some seriously excellent pieces of storytelling like the movie \refwork{Lifeboat}. But it is important to note that these situations inherently have an end. Once the PCs have defeated the dream assassin stalking them all or escaped from Mindtrap Manor or whatever, it rather stretches believability for them to not go their separate ways. Villain/hero teamups can make great \textit{stories}, but rarely make any sense as a \textit{series}.

%%%%%%%%%%%%%%%%%%%%%%%%%
\subsection{Accumulating Power}
%%%%%%%%%%%%%%%%%%%%%%%%%
\tagline{In 2009, the mayor of Prague 5 gave his mistress a seat on the European Parliament as a romantic gift, putting to shame any gift you or I will ever give to any woman. Diamonds are friendly and all, but \textbf{nothing} replaces Power.}

There is a manner of looking at things where everything is just a means to an end. And to that extent, everything is measurable as to how many ends it can deliver. And the units of that measure are Power. Some people desire power because they have lost sight of the importance of their original goals: spending long periods making concessions to achieve the power to achieve their dreams has burnt out any passion they had for anything but the empty accumulation of power for as long as they live. But for others, the gaining of power is an entirely reasoned goal based in their inherent uncertainty of what needs the future will have and the certainty that greater power to respond to the future must be a good thing. Still others desire specific or general powers for no other reason than that power is \textit{fucking awesome}. While the villain driven by an unexamined goal to accumulate more power at any cost until the cost of sanity and self are long paid is indeed a reality, there is nothing inherent about the goal of power that leaves anyone any less sane.

Power comes to our world in many forms, and in After Sundown it comes in several additional ones. Gaining resources of any kind can be thought of as power, as can status in any group or any attributes or disciplines. The game system being what it is, the character will be rewarded with \textit{some} kind of power no matter what stories they participate in. And while that fact can be enough to get the \textit{player} involved in practically any story, technically the \textit{character} doesn't know that. The character should probably be uninterested in any potential adventures that don't seem like they have any payoff.

However, it is important to note that actually very few missions that After Sundown characters would be offered are devoid of obvious payment in power. Anytime a character does something that other people want them to do, they are doing a favor. And doing a favor for someone else is a lot like lending them money. It makes them owe you favors in return. It gives you power over them, it gives you\ldots{} \textit{power}. So while a character motivated wholly or mostly by power can be expected to be kind of a dick about taking on individual tasks, it's not like they won't \textit{do} it.

%%%%%%%%%%%%%%%%%%%%%%%%%
\subsection{Changing the World}
%%%%%%%%%%%%%%%%%%%%%%%%%
\tagline{That would look better over there.}

Everybody has ideas on how to change the world for the better. What constitutes "better" for these purposes is incredibly varied. Maybe they want to change the economic or political structures of the mortal or supernatural societies they are immersed in. Maybe they want to change physical realities or fight wars against the Zombie uprisings to conclusion. Whatever their goals, a goal driven character can  generate their own missions based on who their opponents and allies are. This can be a major time saver for the MC, because the goal driven character will simply create plot hooks out of nothing at all. It can also be a major headache for the MC at times when they drop carefully constructed plot hooks in favor of running off GTA style. Without keeping the goals of a goal oriented character in mind, the MC may be forced to "think fast" and run the game by top-of-head or seat-of-pants fairly often.

Change oriented characters are inherently resistant to going on a lot of missions. They won't go on missions that appear to hurt their cause or benefit their enemies. They simply \textit{will not do them}. And that's a problem if you have multiple goal-driven characters in a team. Incompatible goals between players at the table will make the game grind to a halt. It is important to remember that it is the responsibility of the players to make sure their character's goals do not place an undue burden on the other characters.

%%%%%%%%%%%%%%%%%%%%%%%%%
\subsection[Fame]{Fame and Acknowledgement by Strangers}
%%%%%%%%%%%%%%%%%%%%%%%%%
\tagline{Everyone likes doctors, but they aren't famous.}

Being recognized is considered by some to be "creepy" or even an attack on their person. Fame is not for everyone. And yet, for many people it's virtually the only thing that matters. People will eat bugs for less money than they make at their job just to get on TV being visibly upset. Fame, even \textit{stupid} fame, is a powerful draw. If you're reading this at any point close to its original publication date, you probably know who Paris Hilton is. And you probably also realize that there are people who would literally kill someone to get the recognition she has as of this writing. Even though most people have a negative opinion of her. In fact, \textit{because} most people have a negative opinion of her -- it means that most people have an opinion about her at all. And that's something the vast majority of people live and die without ever achieving. There is no such thing as bad publicity.

Fame driven characters are well likely to jump on any story hook you dangle in front of them, because accomplishing "stuff" is perhaps one of the best ways to get fame. Especially if they're even a little bit concerned about the relative positivity of their fame. The main struggle is not as much to get them to \textit{go} on missions, but to get them to drop them afterward. Asking the character to walk away from publishing their successes in full to keep the Vow of Silence and their own safety is much more of a struggle than getting them to go explore a haunted mansion or take on a wicked Troll in the first place.

And you'd think that the pull between people who want a life of quiet privacy and the people who want to be a known face would tear the game in half. And in some cases, you'd be right. It's seriously a strain on the group if some characters want fame and others do not. Not every character wants to solve supernatural mysteries while playing in a rock band. \refwork{Josie and the Pussycats} is not for everyone. But it doesn't have to tank the story. Indeed, having a character who \textit{wants} to take credit for everything can actually be a \textit{boon} to the characters who want to be left alone. Having a preening camwhore on the team is a godsend for the team Nosferatu who wants to keep their very existence a secret.

%%%%%%%%%%%%%%%%%%%%%%%%%
\subsection{Hedonism}
%%%%%%%%%%%%%%%%%%%%%%%%%
\tagline{Honestly, I have rhythm, I have music, who could ask for anything more?}

Some characters just want to have fun. Once you live forever, what's left to \textit{aspire} to but to dress in frilly shirts and practice your bored expressions? On a more serious note, a lot of people put "self actualization" or something like it at the top of their hierarchy of needs. And when you can get anyone to do pretty much anything with fucking mind control, it's not like fulfilling your lesser needs is really all that hard. So pure hedonism makes a lot of sense under the circumstances. That being said, hedonism \textit{rarely} actually entails spending all your days in an opium fog while fondling the breasts of prostitutes or mind-controlled cheerleaders. Sure, that's one of the roads it can follow, but most flavors of hedonism seek out variety. And that's important for having actual stories to tell.

A hedonistic character can actually make a very reasonable addition to any team, because hanging out with friends and having exciting adventures is a good start towards having good stories to tell to attract the attention of any mind-controlled cheerleaders you happen to meet in an opium den. Such characters are very likely to just say "fuck it" and follow up on whatever plot hook interests the rest of the group.

As an MC it is important to remember that while a hedonistic character offers little resistance to jumping into an adventure, they also aren't very invested in seeing them to completion. If things get too shitty, they'll \textit{leave}. And that's not the player being a douche and sabotaging "your" story, that's entirely understandable from the perspective of their character. If they find themselves in a no-win situation or everyone in town suddenly wants to kill them or whatever, they're going to advocate for grabbing a bus ticket out of town. It's important therefore to remember that laying it on too thick does not constitute a motivation for characters motivated by "yucks" to complete the adventure at any cost. As the MC, you have to temper sticks with carrots. They are not going to respond to Chicago becoming far too dangerous for them to stay by \textit{staying}.

For the other players, it is similarly important to not put too heavy a burden on such a character. While they are going along with whatever the other characters want to do, this should not be taken as a license to roll all over them. They don't put up much resistance to helping other characters with their goals because they are not heavily invested in doing one thing or the other, but that equally means that they are not heavily invested in accomplishing the other character's goals either. The number one imposition that a hedonistic character is unlikely to accept is being sidelined. They are here to \textit{do stuff}, so if you ask them to sit out while your character does stuff alone they will wander off and find adventure of \textit{their} own. It's easy to think that these players are being disruptive, but in many cases they are simply responding rationally and in-character to abandonment.

%%%%%%%%%%%%%%%%%%%%%%%%%
\subsection{Recognition of your Peers}
%%%%%%%%%%%%%%%%%%%%%%%%%
\tagline{The worst prison is not the one where the other inmates rape you, it's the one where there are no other inmates.}

Friendship, acceptance, and status within one's peer group is a major motivating factor for all but the most anti-social of humans. And supernatural creatures rarely escape that particular need. Yet, supernatural creatures are inherently relegated to the status of "the other" in many ways by their tremendously different physical and mystical characteristics. From the standpoint of the supernatural creature, perhaps the bitterest aspect of their emergence from the world of humanity is the loss of all so many hard-won relationships in the mortal world. Even though they may now be able to control minds and rip a car door off its hinges, in the rat race of life they are truly back to square one. It is entirely fitting thus for a character to make as their goal the accumulation of accolades from their peers.

Who counts as one's "peers" is an entirely arbitrary, and deeply personal, concern at the best of times. And it is a very strange question to ask of supernatural creatures because they lack many of the commonalties that might seem to link humanity"s state one to another. But in a general sense, most supernatural creatures recognize other supernatural creatures as being their "peers", a decision that relegates their options for socialization to a number that is limited beyond what mortal humans have had to contend with for tens of thousands of years. And so it is that this goal frequently amounts to little more than "I'm going to go wherever my band goes and make sure to get invited to all their parties." But for the more ambitious it often entails garnering status within their cult and Syndicate.

In any case, the goal is most effectively actualized by going places and doing "stuff," which means that they should be fairly amenable to altering their plans to include the goals of other characters.

%%%%%%%%%%%%%%%%%%%%%%%%%
\subsection{Saving People}
%%%%%%%%%%%%%%%%%%%%%%%%%
\tagline{So let the trumpet players play. Because I am on the way!}

Being the hero is oftentimes reward enough to be an agenda worth pursuing. And considering that people tend to \textit{like} heroes and shower them with favors (both material and sexual), it is by no means an "irrational" life goal. And there is lots of heroing to do in After Sundown. There is a lot of crime, people frequently can't trust the police, and there are secret magical threats that could hurt or kill many people. Trying to save people involves asking big questions. For example, how do you reconcile trying to save people when you or your allies may in fact periodically eat people. It's an \textit{answerable} question, but it's one you have to ask.

Believe it or not, some of the most effective heroes in the realm of horror \textit{don't} have a lot of taboos with regards to hurting people, killing fools, and generally being a dick. If you were trying to make some sort of larger point about how it's never OK to bash someone's head open and eat their delicious brains, you wouldn't even \textit{be} a vampire. You'd go public, hope that humanity would win the immediate war with the supernaturals, and figure it would all be OK at some point. But since you're \textit{not} doing that (as defined by the fact that you're still playing the game), you're pulling some sort of justification based on the fact that currently the \textit{only} known ways to actually end Zombie uprisings or meaningfully stop Pod invasions are magic based. So if humanity goes to war with the supernaturals and \textit{wins} (already at horrendous cost), it also pretty much \textit{loses}, because the Z-War starts up shortly after that, and there's no guarantees that any humans will survive. So being a hero in After Sundown pretty much means that you've resigned yourself to keeping humans in the dark about actual cannibalistic monsters in their midst to forestall even bigger problems should those beasts be destroyed en masse. 

So there we have it: you're a hero in After Sundown and \textit{nonetheless} you're spending a certain amount of your time covering up brutal monstrous bastards who are actual monsters. That's a difficult head space for a lot of people to get into, but it's probably best to think of \refwork{Angel} or \refwork{Men In Black}. You fight monsters that overstep the lines, but you seriously have monsters on your team. And you \textit{help} those monsters who aren't going over the line -- it encourages them to stay that way.

Characters uninterested in the accolades of saving damsels in distress can oftentimes be persuaded to join in nevertheless. At the big scale it's simply a no-brainer: when Demons want to destroy the world, that \textit{is} where you keep all your stuff. But even on the small scale there's the fact that threats to human safety are frequently threats to the Vow of Silence (and thus to the very existence of supernatural society). And not a few acts of heroism get \textit{rewards}. And rewards are a lot like wages except that no one expects them to be \textit{taxed}, which is good news for anyone living under society's radar.

While a heroic motivation may seem somewhat uncharacteristic for the genre, it's really not. Not only is it factually true that the heroic motivation is the most common motivation of protagonists in source material, but it's also interestingly true that it is the most common motivation of people playing in After Sundown \textit{games}. Players don't actually get any real power or wealth from the successes of their characters, so it is very understandable when players want to play adventurous and selfless characters -- telling a good story and saving the day are as much reward as the \textit{player} ever gets.

\medskip
\textbf{A Special Note on "Super Heroes"}\\
After Sundown source material includes Comic Book Superheroes. Most notably there is \refwork{Blade}, but let us not forget that the \refwork{League of Extraordinary Gentlemen} is a comic as well, or that much of our present day lore on vampires, zombies, and demons comes from the pages of \refwork{Batman}. The superheroic adventure structure is thus familiar territory for an After Sundown game, and superheroic storytelling tropes are familiar territory for such characters and situations. Episodically saving the town/country/planet from the threat of the week or doggedly hunting down the monster of the week is perfectly acceptable as a campaign goal.

The place where "horror" ends and "super heroes" begin is oftentimes not all that clear. Historically, horror comics have waxed and waned in popularity just as heroes in tights have, but they've been around since the beginning. And the overlap has been severe. Swamp Thing has had crossovers with capes and crossovers with conventional slasher fare. There have been numerous iterations of Wolverine facing zombies or vampires (they are amongst the few things he can use his claws on without getting an M rating), and so on. The point is that if you think of your characters in After Sundown as super heroes who have fangs, this is \textit{not} a betrayal of source material, it is an affirmation of it.

It might seem like the players would have to be part of a "heroic" organization like the Stellar Oracles to undergo such a campaign arc, but that's completely not the case. The super hero plotline is well preserved in \refwork{Buffy the Vampire Slayer} (where the Daziban or The Black Hand could easily stand in for the Watcher's Council), \refwork{Angel} (you can use The Hollow Ones or The Ulmi as fifth season Wolfram \& Heart), and \refwork{Hellboy} (you can use the Storm Lords or Ash Walkers as The Agency).
\medskip

The one place where heroism really falls apart is if one or more players choose to play characters who cannot be reconciled with heroism. This can totally happen, as there is literally nothing in a Vampire's job description that prevents them from being a serial rapist or cannibal or whatever -- indeed it takes a fair amount of work on their part to \textit{not} do that. A fair amount of work that protagonists in stories are very likely to put in, but there are players who don't want to. If one player wants to be a hero and another wants to explore the "dark side" you seriously should not run the campaign at all. It will not go well, and you should save yourself some grief and have one or both players play something else.

%%%%%%%%%%%%%%%%%%%%%%%%%
\subsection{Wealth and Material Comfort}
%%%%%%%%%%%%%%%%%%%%%%%%%
\tagline{\ldots{}but your blood won't pay my bills / I need money / that's what I want.}

Mortal society drives us ever faster along the hedonistic treadmill by ever dangling new niceties, new conveniences, and new shows of conspicuous consumption for one to take part in. And the supernatural creatures of the world are not \textit{immune} to those motivations. Whether from a fear of not having enough pie (as eloquently and disturbingly described by Scarlett O'Hara), a desire to roll around in pie as an exercise of pure id (as demonstrated time and time again by Cookie Monster), a competitive urge to simply \textit{have} more pie than potential rivals (see almost any antagonist in a Disney venture), or merely the thrill of achievement (Remy from Ratatouille), getting pie is a powerful motivator. Perhaps the most useful factor of this set of character motivations is that it is easily judged and easily incorporated into other goals.

More is rather trivially compared with less, so it is generally rather obvious and \textit{easy to predict} when a character's actions will be in accordance with the overall goal of getting a Mercedes full of cheerleaders. This means that the MC can easily throw a bone to the character to get them interested in a plot hook (just append "and there's a reward" to any storyline). And perhaps more importantly still, it means that the \textit{other players} can rather easily make concessions to such a character's motivations when they are crafting player-generated plots. All they have to do is add "Step Three: Profit" or "And you can do whatever you want with your share" to whatever the plan was going to be and the wealth-motivated character is "in."

%%%%%%%%%%%%%%%%%%%%%%%%%%%%%%%%%%%%%%%%%%%%%%%%%%
\section{Jobs, Missions, and Quests}
%%%%%%%%%%%%%%%%%%%%%%%%%%%%%%%%%%%%%%%%%%%%%%%%%%
\tagline{We're here to see a man about a thing. He'll know what it's about.}

One of the motifs that works best for getting a group of diverse characters to work together in a single story is to introduce the story element of a \textit{job}. A complex, difficult, or merely remote task that several characters are contracted to perform. This gives things clear direction and bypasses many "what do we do next?" style arguments for as long as the job hasn't been completed yet.

%%%%%%%%%%%%%%%%%%%%%%%%%
\subsection{Bug Hunt}
%%%%%%%%%%%%%%%%%%%%%%%%%
\tagline{Are we looking at a stand-up fight or a bug hunt?}

The Bug Hunt is so venerable that indeed it was the \textit{assumed} mission for all Role Playing Games back when the default adventure was to go into a dungeon and clear it of dragons. The basic premise is that some area or building is \textit{full of monsters} and the player characters are wanted to \textit{clear them out}. Bug Hunts generally at least start out like the beginning of \refwork{Aliens} or the end of \refwork{Evil Dead 2}, and they can provide a good time to those players whose characters specialize in brutally murdering things they don't like. But they can become more complicated at any time. Were the people who wanted you to clear out the monsters completely honest about the contents? Are there innocents or valuable property that the characters are supposed to protect while taking out the monsters? Are the characters supposed to take one or more of the monsters back alive (or in the case of zombies: \textit{animate})? Is there a bigger problem that is creating the monster infestation in the first place, such as a gate to another world or an evil sorcerer?

While the proximal goal of a bug hunt is combat driven, they are great starting points for suspense and character driven role playing. The hunters can become the hunted, and the mission itself may be more than it seems from a number of angles. In After Sundown, it is usually best to use non-playable supernaturals as the nemeses for a bug hunt game: it's kind of weird to be hunting "werewolves" when one of the players may themselves be a werewolf. But the same basic plot structure takes over when taking out a renegade band of vampires that is threatening the Vow of Silence or attempting to conquer the worlds or something.

%%%%%%%%%%%%%%%%%%%%%%%%%
\subsection{Courier Run}
%%%%%%%%%%%%%%%%%%%%%%%%%
\tagline{I need this to be delivered to Mount Doom.}

The Courier Run is at its core merely an obstacle course. The characters are asked to go from where they are to somewhere else, possibly by way of a series of intermediary locations, possibly any way they want to. The catch of course is that the characters are intended to take a MacGuffin with them. Maybe it's a valuable painting, or a secret letter to a Covenant Bishop. Sometimes it's even alive or otherwise hard to transport. Maybe it's a Mantrap being taken in for study, or a suitcase with a woman in it. Save for the fact that some MacGuffins are physically harder to move without attracting attention than others, it doesn't much matter.

The key to the mission and the reason it's the foundation of some of Western literature's greatest works, is that there is no real limit to how much stuff can be "in the way." Burdened as the player characters are with whatever the MacGuffin is, it is generally not unreasonable that enemies who don't want the MacGuffin delivered can get ahead of the PCs again and again. Nor is it unreasonable that there would be obstacles and enemies already there along any particular route the player characters choose to take. We're looking at classic works like \refwork{The Odyssey} and \refwork{Lord of the Rings} as well as contemporary works like \refwork{The Transporter}.

%%%%%%%%%%%%%%%%%%%%%%%%%
\subsection{Fetch Quest}
%%%%%%%%%%%%%%%%%%%%%%%%%
\tagline{Lots of things are valuable, but there's no replacement for this.}

The Fetch Quest is a time honored time waster in which the player characters are asked to go somewhere, get something, and bring it back. While you will sometimes see this in video games where the players are literally just walking to point B and back, all the memorable ones involve going somewhere that is guarded, dangerous, hidden, or in some other way hard to get stuff out of. Basically, a Fetch Quest is like a Courier Run where you don't even start with the MacGuffin. 

Fetch Quests can very easily escalate into dungeon crawls if the thing being fetched is under heavy guard or lock-n-key (or both). They can also become Reconnaissance if the actual location of the MacGuffin isn't known. But you're probably going to see a lot of the same kinds of plot twists as a Courier Run -- hostile interference, missing MacGuffins, and so on. Remember that in a Fetch Quest, it is very likely that someone wants the MacGuffin to stay where it is, and it is also likely that someone thinks that they have a legitimate claim with the law to enforce that desire.

%%%%%%%%%%%%%%%%%%%%%%%%%
\subsection{Reconnaissance}
%%%%%%%%%%%%%%%%%%%%%%%%%
\tagline{In Auto-Recon 2, you get to have a boat.}

The reconnaissance mission is the simple directive to go to point A or track down person B, find out what's going on, and escape to report about it. This can be a simple stealth, talking, or even research task; or it can be a bit more complicated if that's what is desired. In fact, since the entire point of the job is to find out what's going on, the mission can extremely plausibly turn into any other kind of mission. The player characters are essentially going into a black box mission. They don't know what's going on behind the curtain and then they go behind the curtain and then\ldots{} anything you want.

A common method of spicing up recon work is to have the characters uncover something that is "time critical" such as a bomb plot, a scheduled human sacrifice, a growing zombie army or something else that the team is encouraged to interfere with immediately upon arrival rather than going back and giving a report. A good example of this is pretty much any of the Indiana Jones movies. Another good trope is to have enemies found during the recon which then come after the player characters. Most haunted house stories operate on roughly this premise, where once the characters get in the new mission becomes escaping with their lives.

%%%%%%%%%%%%%%%%%%%%%%%%%
\subsection{Rescue Mission}
%%%%%%%%%%%%%%%%%%%%%%%%%
\tagline{Honestly? I don't care what you do or who you do it to. I want to see my daughter again.}

Rescuing people (and in some cases animals or objects of art) can be one of the clearest objectives available. The thing that the target is being rescued from need not be a hostage situation, sometimes it's just a dangerous situation or even a loveless marriage or exploitive contract. There are plenty of ways to spice or gum up such a scenario. The most obvious of course is by having people who don't want you to rescue the target having more power at their disposal than expected. But you can also have additional groups competing to "rescue" the target or even have the target \textit{not want to be rescued}.

The best part of a rescue mission from the standpoint of an MC is that it both has a definite end \textit{and} leaves a clear path for additional stories. Once a target has been rescued (or not) and brought to a safe and agreed upon location (or not), the characters' interest and job is demonstrably over. On the other hand, you've just introduced some NPCs who demonstrably have enemies and the PCs have crossed swords with those enemies. So rescue missions can serve well in \textit{either} a one shot role or as a stepping stone to further stories.

Remember also that the inherent goal of a rescue mission is to get the target out safely. Thus, while it is often an option to go in guns ablaze, there is legitimately nothing in the job description that would necessitate using physical violence at all. Stealth, social subtleties, and arcane magic can all be highly important in a rescue mission. Indeed, some of the \textit{best} rescue missions from history involve absolutely no one on either side dying. 

%%%%%%%%%%%%%%%%%%%%%%%%%
\subsection{Wet Work}
%%%%%%%%%%%%%%%%%%%%%%%%%
\tagline{The second kill is considerably easier than the first.}

Sometimes the entire goal of a job is just to straight up murder someone. Sometimes the reasons for doing this are totally noble, and there are a lot of those reasons to go around in After Sundown. Sometimes it's just a purely mercenary deal. Sometimes you get to have a little bit of both. African warlords, Columbian drug bosses, and ancient vampire royalty all often control fairly large amounts of resources, and it is not inconceivable that someone might be willing to pay another person some resources in order to free those resources up, completely aside from their relative worth as moral agents.

Just walking up to some guy at Pizza Hog and stabbing them in the face isn't much of an \textit{adventure}, so generally speaking it's best to send the would-be assassins after people who have compounds full of guards, ancient castles filled with magical traps, or undisclosed locations that are possibly in other worlds. The actual act of killing a dude is kind of an anticlimax in that basically you ram something sharp into their chest and they stop moving. So it's advisable for most Wet Work assignments to have the adventure be \textit{getting there} through secrecy and obstacles rather than the final fight itself.
