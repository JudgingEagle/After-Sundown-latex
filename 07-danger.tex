%%%%%%%%%%%%%%%%%%%%%%%%%%%%%%%%%%%%%%%%%%%%%%%%%%
%%%%%%%%%%%%%%%%%%%%%%%%%%%%%%%%%%%%%%%%%%%%%%%%%%
\chapter{Danger}
%%%%%%%%%%%%%%%%%%%%%%%%%%%%%%%%%%%%%%%%%%%%%%%%%%
%%%%%%%%%%%%%%%%%%%%%%%%%%%%%%%%%%%%%%%%%%%%%%%%%%
\tagline{The world would be a safer place if people didn't do dangerous things.}

Action sequences are generally split into 12 second Combat Rounds\index{Combat Round}. That's five combat rounds in a minute. This means that in the inestimable "side scroller" fight in \refwork{Old Boy}, that the fight goes on for 3 rounds before Dae-su Oh spends a round pretending to be unconscious before he goes at it again. The fight between Spiderman and Green Goblin goes: Round 1: Green Goblin throws Peter through a wall, Round 2: Green Goblin throws a bomb through the hole in the wall, Round 3: Green Goblin taunts Peter, Round 4: Green Goblin and Peter exchange punches. Considering how many dice get rolled, such flurries of action can already take a fair amount of time to resolve, so breaking the time down finer than that is generally not worth doing. In fact, when the action gets to even slower parts like cross city chases and boat maneuvers, we pull out of combat time altogether.

Remember that while it is theoretically possible to let loose astounding numbers of bullets in just a second or two, in actual combats it's entirely likely that any particular 12 seconds will go by with no shots at all. Fog of war takes quite a while to roll out.

A note should be made about ranges and distances within After Sundown. The game was designed with Metric distances and speeds in mind, but Imperial distances have been provided also. Anyone familiar with conversion rates will note that the Imperial distances given aren't actually full accurate. They've been rounded so that players can more easily remember them.

%%%%%%%%%%%%%%%%%%%%%%%%%%%%%%%%%%%%%%%%%%%%%%%%%%
\section{Actions and Reactions}
%%%%%%%%%%%%%%%%%%%%%%%%%%%%%%%%%%%%%%%%%%%%%%%%%%

\hspace{\parindent} Normally a character may take one Complex Action\index{Complex Action} or two Simple Actions\index{Simple Action} during their turn. A character may do some relatively large number of Free Actions\index{Free Action}, and like Complex and Simple Actions, all of these are done during the character's turn. Reactions\index{Reactions} and Movement\index{Movement} are done while it is not a character's turn. Interruptions are rare in After Sundown, and normally an action is fully resolved before the next action is declared.

%%%%%%%%%%%%%%%%%%%%%%%%%%%%%%%%%%%%%%%%%%%%%%%%%%
\section{Initiative Order and Passes}
%%%%%%%%%%%%%%%%%%%%%%%%%%%%%%%%%%%%%%%%%%%%%%%%%%
\tagline{"Then I attack, then you attack twice, then I attack, then you attack once?"}

At the beginning of a Combat Round, every player rolls \dicepool{Intuition + Agility}. The number of hits is their Initiative Score\index{Initiative Score}. Characters act in order of Initiative Scores (highest to lowest), and characters whose scores are the same act simultaneously. A character who gets a higher Initiative Score may choose to act \textit{after} characters with a lower Initiative Score if they want. Each character gets one turn. Characters who are entitled to extra Initiative Passes\index{Initiative Passes} (either because they have activated Celerity or because they bought them with Edge) get an additional turn after everyone else has moved. The most number of turns you can have in a round is 4.

%%%%%%%%%%%%%%%%%%%%%%%%%%%%%%%%%%%%%%%%%%%%%%%%%%
\section{Attacking}
%%%%%%%%%%%%%%%%%%%%%%%%%%%%%%%%%%%%%%%%%%%%%%%%%%
\tagline{This is going to hurt you more than me.}

%\begin{table}[htb]
\begin{wraptable}[8]{R}{8cm} \vspace{-1.2cm}
\centering \rowcolors{1}{white}{tan} \caption{Attack Thresholds}
\begin{tabular}{l l l c}
\textbf{Range} & \textbf{Meters} & \textbf{Feet} & \textbf{Threshold}\\
\textbf{A}djacent & <2m & <6ft & Special\\
\textbf{N}ear & 2-5m & 6-16ft & \textbf{1}\\
\textbf{S}hort & 5-20m & 16-65ft & \textbf{2}\\
\textbf{W}ay out & 20-100m & 65-330ft & \textbf{3}\\
\textbf{E}xtreme & 100m-1km & 330-3,300ft & \textbf{4}\\
\textbf{R}emote & 1km+ & 3,301+ & \textbf{5}\\
\end{tabular}
\end{wraptable}

The most common actions in combat are attacking and running away. After all, if \textit{someone} isn't doing one of those things, it can hardly even be considered a combat.

An attack is usually resolved by spending a Simple Action to make the Attack\index{Attacking}. The character makes a test against the threshold. If the requisite number of hits are achieved, the attack hits and the target must soak the amount of damage that the attack inflicts. If the attacking character gets more hits than necessary, the net hits are added to the damage of the attack before the target gets to soak. In most cases, a character's dice pool will be their \dicepool{Strength + Combat} when making a melee attack and \dicepool{Agility + Combat} when making a ranged attack. In the cases when an attack is being made with some kind of sorcery, the dicepool may well be something else entirely and is described in the power's description.

When a character wants to attack something it is more difficult if the target is farther away. The \textit{range}\index{Range} between the target and the attacking character determines the base threshold. However, just because a character has the \textit{accuracy} to strike an opponent out to a specific range does not mean that their weapon is physically capable of reaching that far, or of reaching a target at that range with any accuracy. Most weapons have a maximum range beyond which they cannot be expected to work, and most weapons have a range beyond which they become inaccurate (given in parentheses). Melee attacks of course simply have an absolute limit of their reach. If you want them to go any farther than your arm will take them you have to throw them.

The Threshold to hit something that is Adjacent to you is \textit{zero}. Seriously, it's right next to you. However, if the target is aware and able to resist the attack, the threshold is increased to half the target's Agility+Combat total. Yes, against skilled opposition it is \textit{much} easier to shoot an opponent from 3 meters away where they can't interfere with the shot than it is to shoot them from within arm's reach where they can.

The Threshold can be further modified by circumstances. If the target has cover or its location is suspect (as in the case of illusions or intervening shower curtains), the threshold is increased by 1 or 2. If you're operating beyond the accurate range of your weapon (but still within the maximum range), increase the threshold by 1. If the target is moving quickly, increase the threshold by 1. If the attacking character is moving faster than a careful walk, increase the threshold of a ranged attack (but not a melee attack) by 1.

\textbf{Multiple Attackers:} \index{Attacking!Multiple Attackers}Multiple attackers is only specifically advantageous when the attacking characters are in Close Combat. A victim can only actively resist one more enemy than they have actual Combat Skill. If there are more attackers than that, the target is going to have to allow a certain number of enemies to get free attacks on them. On the other hand, working together with a number of additional attackers \textit{also} requires training. An attacker who is attempting to Rodney King someone suffers a -1 penalty to their attack roll for each additional attack beyond their actual Combat Skill. It is thus entirely reasonable for a number of potential attackers to simply wait outside a beating rather than get in the way.

\textbf{Supine and Prone Enemies:} When a target maintains a low profile by getting on the ground, it is hard to hit them with ranged weaponry -- increasing the threshold to hit them by 1 if they are at any range beyond \textbf{N}ear. On the other hand, hitting actually adjacent enemies who are on the ground is easy, and characters get an extra 2 dice to do that.

%%%%%%%%%%%%%%%%%%%%%%%%%
\subsection{Ranged Weapons} \index{Attacking!Ranged Weapons}
%%%%%%%%%%%%%%%%%%%%%%%%%

%\begin{table}[htb]
\begin{wraptable}[18]{R}{10cm}\vspace{-1.5cm}
\rowcolors{1}{white}{tan} \caption{Ranged Weaponry} \centering
\begin{tabular}{l c c c c}
\textbf{Weapon} & \textbf{Damage} & \textbf{Range} & \textbf{Strength} & \textbf{Size}\\
Light Pistol & 3 & (N)S & 1 & S\\
Heavy Pistol & 4 & (N)W & 2 & S\\
Machine Pistol & 3\textsuperscript{a} & (N)S & 2 & S\\ 
Flare Gun & 1F & (N)W & 2 & S\\
Submachine Gun & 3\textsuperscript{a} & (S)W & 2 & M\\
Crossbow & 3 & (S)W & 2 & L\\
Shotgun & 5 & (N)S & 3 & M\\
Rifle & 5 & (W)E & 3 & L\\
Assault Rifle & 4\textsuperscript{a} & (S)E & 3 & L\\
Auto-Shotgun & 5\textsuperscript{a} & (N)S & 4 & L\\
Machinegun & 6\textsuperscript{a} & (W)R & 5 & L\\
Sniper Rifle & 6 & (E)R & 5 & L\\
Flame Thrower & 4F\textsuperscript{a} & W\textsuperscript{c} & 5 & L\\
Cannon & 7 & (W)R & 8 & H\\
\end{tabular}\\
\textsuperscript{a}: Weapon fires in automatic mode.\\
F: Weapon does fire damage.\\
\textsuperscript{c}: Weapon ignores cover.\\
\end{wraptable}

\hspace{\parindent} Most ranged weapons do an amount of damage that is fixed regardless of how strong the user is. The strength listing on the ranged weaponry table indicates the strength required to use the weapon without penalty in two hands. If the character's actual Strength exceeds that, they may use it without difficulty in one hand, unless it is Large in which case their strength must exceed that by 2 in order to use it successfully in one hand. If the strength value of the weapon exceeds the character's Strength, the threshold to strike a target is increased by the difference, in addition to needing two hands.

\textbf{Size:} \textbf{S}mall weapons can be concealed in a pocket; \textbf{M}edium weapons can be concealed under a coat; \textbf{L}arge weapons can be concealed in a car; \textbf{H}uge weapons do not really fit into cars.

\textbf{Automatic Weapons:} A weapon firing on automatic throws out many bullets in a short period of time. This allows it to be used for suppressive fire, to be fired at multiple enemies who are close together, and makes it more likely to hit something. A weapon fired on automatic gains 3 dice on the attack roll, but the spread of bullets makes fine aiming more difficult -- the increase in threshold for firing at enemies with cover is doubled (basic cover increases threshold by 2, heavy cover increases threshold by \textit{4}). Also the character can't take the Aim action with an automatic weapon, but they can take the Spray-n-Pray action.

\textbf{Bullets:} The amount of bullets fired off in genuine firefights is extremely varied and often frighteningly large. During a 12 second combat round, an assault rifle could easily fire 100 rounds or more (if it even had that many bullets in its magazine or belt-feed). Actually resolving where all of those bullets end up would be far too time consuming to consider doing in most battles. As such, the game doesn't really distinguish between characters squeezing off large numbers of bullets and characters taking hard seconds to line up their targets and fire devastating double-taps into a target's vitals. As such, the game also doesn't bother writing up exactly how many bullets weapons contain. During survival horror segments, ammunition conservation can (and should) be a major concern, but in regular street combat it shouldn't really come up. 

It is also important to note that while the game only models those bullets that have a significant chance of hitting their target, every bullet fired \textit{eventually} hits \textit{something}. Stray bullets do not individually hit people we care about often enough to have such events generated by any combination of numbers on the dice, relegating that very real possibility to the realm of plot devices. However, in cases where the target is very close to other targets that these other targets are very likely to catch a bullet one way or another. If the target is getting cover from a creature and the target is missed by just the difference caused by the cover, then you can assume that the living cover is hit. Similarly, if the target is in or in front of a dense crowd of people, \textit{someone} is getting hit. Deciding \textit{who} is left as a narrative exercise. In such situations, Extras catch stray bullets much more than Luminaries.

%\begin{table}[htb]
\begin{wraptable}[19]{R}{8.3cm}\vspace{-.5cm}
\rowcolors{1}{white}{tan} \caption{Melee Weaponry} \centering
\begin{tabular}{l c c c}
\textbf{Weapon}&\textbf{Damage}&\textbf{Strength}&\textbf{Size}\\
Fist & 0N & 0 & X\\
Bottle & 1N & 1 & S\\
Knife & 2 & 1 & S\\
Hammer & 3 & 1 & S\\
Baseball Bat & 2N & 2 & M\\
Crowbar & 3 & 2 & M\\
Sword & 4 & 2 & M\\
Chair & 2N & 3 & L\\
Axe & 4 & 3 & L\\
Chainsaw & 4 & 4 & L\\
Great Weapon & 5 & 4 & L\\
Sign Post & 5N & 5 & L\\
Fire Hydrant & 5 & 6 & L\\
Lamp Post & 6N & 9 & H\\
Car & 6N & 10 & H\\
\end{tabular}\\
N: Weapon does Normal Damage.
\end{wraptable}

\textbf{Special Ammunition:} Characters in After Sundown will often want to fire bullets that are specifically made of iron, wood, or silver. While these bullets realistically have different characteristics than ones made of copper or lead, that's a level of detail that combat in After Sundown does not actually go to. Special ammunition also exists that is just generally more effective. Whether it's made out of depleted uranium or is special explosive ammunition or whatever, such exotic equipment increases the damage of the gun by 1, costs quite a bit, and puts a real strain on the character's claim to not be a military-grade super villain.

Smooth bore weaponry such as shotguns can be loaded with grains of pretty much anything. Of particular interest to characters in the realm of horror is that they can fill the shells with salt, sand, or live grain in order to suppress Astral, Infernal, or Orphic sorcery respectively. This is quite effective, and the character's \dicepool{Agility + Combat} test can suppress magic out to the range of the weapon.

\textbf{Silencer:} Weaponry enthusiasts will get mad at you for calling them "silencers" because they don't actually reduce the noise of firing a weapon to nothing or even the point of inaudibility. Nevertheless, the "sound suppressor" is called a "silencer" in ordinary conversation, so it is reasonable that characters in the game will refer to it (incorrectly) like that as well. Silencers reduce the sound of using the weapon substantially, to the point that its use will probably not be noticed on the other side of a wall, but this comes at a price. The weapon loses a lot of power, which reduces the Damage by 1 and reduces the accurate range by one category (for example: if you silence a light pistol it has a base damage of 1 and loses accuracy if fired at ranges beyond Adjacent). A silencer also falls apart with use. After more than five shots, a silencer is basically garbage.

%%%%%%%%%%%%%%%%%%%%%%%%%
\subsection{Melee Weapons} \index{Attacking!Melee Weapons}
%%%%%%%%%%%%%%%%%%%%%%%%%

\hspace{\parindent} Melee weaponry differs from ranged weaponry primarily in that it simply goes as far as it will reach and therefore doesn't have a "range" value on the table. Many melee weapons inherently do Normal rather than Lethal damage, and their damage values are followed by an "N". Unlike Ranged Weaponry, which is mostly designed for the purpose, a majority of things that people beat on each other with in hand to hand conflicts are actually improvised weapons -- tools and household items that happen to be at hand when a fight breaks out. The rules for using melee weapons in one hand are the same as for ranged weapons: the character needs to have more Strength than the listed value to use it in one hand, and must exceed it by 2 if the weapon is Large. The threshold to strike a target is increased by the amount the weapon's listed Strength value exceeds the character's Strength if it does.

%%%%%%%%%%%%%%%%%%%%%%%%%
\subsection{Special Attack Actions} \index{Attacking!Special Attacks}
%%%%%%%%%%%%%%%%%%%%%%%%%

\subsubsection{Aim}\hspace{\parindent}  Aiming is the act of taking extra time with a shot in order to make it more accurate. Each Aiming action reduces the effective range between the character and the target by one range category for the next shot. The actual distance does not change, and the threshold is still modified upward if the character is firing from beyond the weapon's accurate range. If the weapon is sufficiently braced that recoil is completely negated and the target does not move substantially from its original position, the character's Aiming can continue to apply on future shots. The first Aim action takes a Simple Action, and each further Aim action takes a progressively longer time frame (1 Round, 1 Minute, 5 Minutes, and finally 20 minutes of preparation to reduce a Remote target to be effectively Adjacent). These subsequent Aim actions all apply to the same attack, but a character can't actually benefit from more subsequent Aim actions than they have actual skill rating in Combat or Rigging. Aiming at targets beyond \textbf{S}hort range requires a scope.

\subsubsection{Suppressive Fire}\hspace{\parindent}  Suppressive Fire is the act of firing a bunch of bullets near a piece of cover that a target (or targets) are hiding behind. Since Suppressive Fire is actually fired at a place where the targets are not, it has no chance of hitting them. What it \textit{can} do is seriously threaten anyone who breaks cover. If during the following round any potential target comes out of cover (even for purposes of popping out to take a better shot), they are subject to an attack as if they had no cover at all. Suppressive Fire is very effective game mechanically with automatic weaponry because autofire has improvements in accuracy but suffers penalties against cover. This is not an accident, as suppressive fire is frequently and effectively used in the real world with weapons that have a high rate of fire. Suppressive Fire is a Complex Action and covers the entire round. Characters with multiple Initiative Passes can do other things while suppressing an area.

\subsubsection{Spray-n-Pray}\hspace{\parindent}  Automatic weaponry can be walked across areas, firing off bullets seemingly at random. This is neither advisable nor safe, but it \textit{can} totally hit people with bullets and kill them and stuff. Which for people with little skill with firearms is not necessarily a bad deal. Spray-n-Pray differs substantially from most actions in After Sundown because the character's skills and attributes don't really get used. Instead, the character nominates an arc and rolls just 4 dice total (you can think of this as having a virtual minimal Agility of 1 and the 3 dice for autofire) against each potential target in the area. Spray-n-Pray ignores threshold modifiers from target speed or poor visibility, but is otherwise a normal ranged attack. Spray-n-Pray is not compatible with Aiming, and thus it will likely not hit anyone behind heavy cover. Spray-n-Pray is a Complex Action, and the threshold to strike a target is 1 if range is Short or less and 3 if it is Way Out. While it is nominally possible for a bullet to impact a target more than 100m away, the chances of this happening are so remote as to be discounted. Without Edge, it just won't happen. Targets farther than Way Out are Threshold 5 to hit with Spray-n-Pray.

\subsubsection{Abduct}\hspace{\parindent}  Characters can grab people and carry them off. This is done fairly frequently in horror movies and is an essential part of the genre. To Abduct someone, the character takes a Complex Action to make an unarmed melee attack (\dicepool{Strength + Combat}) against the victim. If the character gets as many net hits as the target's Strength, the victim is scooped up and possibly thrown over the character's shoulder or tucked under their arm. The victim can scream if they want, but if the character's net hits \textit{exceed} the target's Strength, they can't even scream because their mouth is effectively covered. Once abducted, a victim may attempt to escape on their next round, and may further attempt to escape one other time.

\subsubsection{Disarm}\hspace{\parindent}  Characters can grab items that other characters are carrying. In melee, this can be one by making an attack with a threshold equal to the target's Agility. If the target is \textit{stronger} than the character trying to snatch the item, apply a negative dicepool modifier equal to the difference in Strength scores. If the character attempts to do this with bare hands, they will end up with the object in question in their possession if they get any net hits, but this is also quite dangerous and the character will be denied their defense in melee until their next turn. If the disarm is attempted with a weapon, or an unarmed Disarm attempt succeeds and gets no net hits, the item goes flying or falls to the ground or something as befits the situation. Attacking a held object at range is very difficult, such items are usually small and the threshold is increased by the target's own close combat defense.

\subsubsection{Locking On}\hspace{\parindent}  Characters can prepare a weapon to be deployed. Maybe they hold a knife to the target's throat, maybe they point a gun at the target's back. This takes the time of a normal attack, but no actual attack is made. At any point in the future, the character can make their attack Reactively. If the character becomes distracted, moves faster than a Slow Search or makes another attack (including another Lock On attempt), the old Lock On is lost. This is how the game handles "stick ups", fencing, hostage threatening, and patient sniping.

\subsubsection{Feint}\hspace{\parindent}  A character can attempt to distract or confuse an opponent, making an opening in a battle. Feinting is a Simple Action, and if it succeeds the target loses any "Lock On" they have and does not have the benefit of their Agility or Combat skill for purposes of setting thresholds on the character's attack against them if they follow it up with an attack of some kind (meaning the base threshold to strike them is likely zero). The character makes a \dicepool{Charisma + Expression} or \dicepool{Agility + Tactics} test and the target makes an \dicepool{Intuition + Empathy} test. If the character gets \textit{more} hits, the feint succeeds. While normally a character who has Locked On can use their attack reactively if the target (or anyone else) attempts any "funny stuff" -- in the case of a Feint they have to wait to see if the Feint is successful in making them lose the Lock before taking their free shot. If the Feint fails, they can immediately use their attack in revenge however.

%%%%%%%%%%%%%%%%%%%%%%%%%%%%%%%%%%%%%%%%%%%%%%%%%%
\section{Moving, Evading, and Escaping}
%%%%%%%%%%%%%%%%%%%%%%%%%%%%%%%%%%%%%%%%%%%%%%%%%%
\tagline{There are places I want to be, places I want to see, far away from here, nowhere that is near.}

%\begin{table}[htb]
\begin{wraptable}[9]{R}{6.5cm} \vspace{-.5cm}
\rowcolors{1}{white}{tan} \caption{Speed Categories} \centering
\begin{tabular}{l r r}
\textbf{Movement} & \textbf{Meters} & \textbf{Feet}\\
\textbf{S}low Search & 4m & 13ft\\
\textbf{C}areful Walk & 10m & 32ft\\
\textbf{O}rdinary Walk & 18m & 60ft\\
\textbf{R}apid Jog & 27m & 90ft\\
\textbf{E}xhausting Run & 60m & 200ft\\
\textbf{D}raining Sprint & 100m & 330ft\\
\end{tabular}
\end{wraptable}

The second most defining action in combat (after attacking) is people endeavoring to not be part of the combat. And while that often involves \textit{hiding} the other possibility is \textit{running}. 12 seconds is actually kind of a long time, and characters can move quite substantial distances in those periods. It's long enough for someone who is very good at that sort of thing to run a hundred meters and then change the magazine on an M-16 (seriously). But if you actually watch horror movies (or even documentaries, not that they are better source material for After Sundown), you'll note that people actually spend many 12 second periods without covering that much ground. In part, this is because the high end of speeds that people are capable of frequently take a fair amount of time to get started, and partly this is because running off at full speed without knowing where you are going is often suicidal.

In general, a character declares their intention to move before any actions are taken, and characters can take their actions as if they or their targets were where they were at any point during the turn. 12 seconds is a long time, so if someone is moving around a corner or into view it is reasonable that some number of bullets went towards them while they were in the open. For exceptions to this, see Taking One For the Team and Diving For Cover (below). Often it won't really matter, but characters with the lowest Initiative Score declare their movement first.

\textbf{Movement Penalties:} \index{Movement!Penalties}Doing anything precision based while moving faster than a \textbf{C}areful Walk is actually pretty difficult. On most actions, characters receive a -2 penalty for an \textbf{O}rdinary Walk, a -3 penalty for a \textbf{R}apid Jog, a -4 penalty for an \textbf{E}xhausting Run, and a -6 penalty for a \textbf{D}raining Sprint. Most people can't do much of anything while sprinting. Close Combat is actually something of an exception, in that momentum is also helpful, so characters receive no penalty for a \textbf{R}apid Jog or \textbf{E}xhausting Run to their melee attacks.

\textbf{Going Faster:} \index{Movement!Sprinting}If a character is just concentrating on moving, which is to say that they spend a Complex Action on moving fast, they may make a \dicepool{Strength + Agility} test to increase their speed by 10\% per hit. And yes, this means that a character who gets 6 hits can sprint at nearly 50 KPH (30MPH). That's very fast, but it's also a lot of hits.

%%%%%%%%%%%%%%%%%%%%%%%%%
\subsection{Escaping from Harm} \index{Movement!Escaping}
%%%%%%%%%%%%%%%%%%%%%%%%%
\tagline{Slashers know shortcuts.}

\textbf{Disengaging:} A character can move out of close combat just as they can move into close combat. However, turning your back on a lunatic with a meat cleaver is \textit{dangerous}. To represent this, a character moving out of Adjacent range with an opponent is not only treated as being adjacent with that opponent for the rest of the round, they are no longer actively resisting, so the threshold to hit them is just \textit{zero} (modified of course by circumstances such as visibility and dodging).

\textbf{Taking One for the Team:} If a character disengages from close combat, \textit{another} character still engaged in close combat with the same opponent can choose to throw themselves into harm's way. This redirects a parting attack to the brave character playing bodyguard (or sacrificial lamb). The threshold to strike the bodyguard is the same as a normal attack against them.

\textbf{Dodging:} If a character is aware of attacks being made against them, they can attempt to move out of the way of those attacks. Heck, even if they don't know where the attacks might come from, they can still move erratically and hopefully throw off potential attacks. Dodging is a Complex Action that is not penalized for any movement up to an Exhausting Run. The character makes an \dicepool{Agility + Combat} or \dicepool{Agility + Athletics} test, the character increases the threshold to hit them with attacks they are aware of by 1 less than their number of hits, and increases the threshold against even attacks they cannot predict by 2 less than their number of hits. The defensive benefits of the Dodge last for one round.

\textbf{Diving For Cover:} With especial urgency, a character can attempt to get behind something solid \textit{before} they get hit with bullets or shrapnel. The character makes an \dicepool{Agility + Athletics} or \dicepool{Agility + Stealth} check, and if they get more hits than the initiative count of their attackers, they get behind cover before the attacks are resolved. Diving for Cover can be announced out of turn, and uses up a Simple Action. Diving for Cover is not penalized for movement speeds at all.

\textbf{Untangling:} If a character is trapped in a net, a set of handcuffs, or the grip of a monster they can attempt to escape by untangling themselves. This is generally a Complex Action. The character makes a \dicepool{Strength + Larceny} or \dicepool{Agility + Larceny} test, and if they succeed they have escaped. Untangling can't be tried an unlimited number of times. If the character tries and fails to escape, they are stuck until they get help or the next longer timeframe passes. The threshold to escape a net is generally 2, the threshold to escape a set of handcuffs is generally 3. The threshold to escape a character having abducted them is equal to the number of hits the abductor achieved when grabbing them in the first place.

%%%%%%%%%%%%%%%%%%%%%%%%%
\subsection{Chases} \index{Movement!Chases}
%%%%%%%%%%%%%%%%%%%%%%%%%
\tagline{Man vs. horse races are interesting. The man wins the six second sprint and the three day ultramarathon and loses everything in between.}

Normally in After Sundown we don't keep track of where each character exactly is at any precise point in a 12 second period. Twelve seconds is not long enough to decide that someone isn't going to answer the phone or go to the bathroom or anything, but in terms of climbing stairs or running around a room in a house it's a fairly long time, and we can assume that people are getting shots off opportunistically. However, it is also true that it is very frequently desirable to model one character running \textit{away} from another. In that case the simultaneity of the two characters moving makes the granularity of the turn sequence really problematic.

\subsubsection{Short Chases}\hspace{\parindent}  During a single combat round, it is often important to know whether a character can get away from another in time to not get struck in the back with a meat cleaver. When combat is occurring and one character is attempting to get away from another, a Short Chase may be called for. If the chasing character wins initiative, they catch the target. If the chasing character is moving faster, they catch the target. If neither of those things are true, the target gets away. If the character who would be losing the chase is unhappy with this result (as they may well be), they may attempt Stunt of some level of awesomeness to change their defeat into victory. Most Stunts are \dicepool{Agility + Athletics}, although it is within the realm of possibility to do \dicepool{Agility + Stealth} based Stunts, depending on the terrain. A chase stunt uses a Simple Action. If the stunt is successful, the other character is then forced to lose the chase or replicate the stunt with a similar level of awesomeness; if the stunt is unsuccessful they are caught, and if it misses its mark by more than one hit the character also falls down or wipes out. If the chasing character attempts to duplicate the stunt and they get the same number of hits, they still win the chase but also must use a Simple Action doing it. If they get more hits than required, they use only a Free Action and still win the chase. If the target of the chase gets away, the chasing character can elect to let them go or go on to a Long Chase.

\subsubsection{Long Chases}\hspace{\parindent}  When characters are performing parkour across the city or driving across town the chase can drag on for quite a while. There are concrete examples of chases that have gone on for hours or days. Even a relatively fast paced and action packed like the pulse pounding muscle car expo up and down the hills of San Francisco in \refwork{Bullit} takes ten minutes or so and is actually quite tedious to attempt to replicate in 12 second combat rounds (48 die rolls \textit{each}? No thanks). So a Long Chase is conducted in rounds of arbitrary length. That is to say, the game absolutely does not specify how long it is between stunts as cars continue to roll down the street or people race across rooftops. Honestly, it's hard to even tell, because most of this will get edited out of the movie anyway. What happens is that each round the chased character has the option of performing a Stunt. If they do so, the pursuer must attempt a stunt of equal craziness or the fleeing character escapes. If the retreating character declines to do a stunt to try to get away, the pursuing character can perform a stunt to catch up. If the stunt succeeds, the chased character has to match the stunt or be caught.

When a character in a long chase performs a Stunt, they can choose any level of awesomeness they want from the Pedestrian "weaving through traffic" (threshold 1) to the Extreme "pulling a U-turn through the grass to run the other way on a frontage road" (threshold 4) all the way to the Super Human "flipping the car over the median and driving off the on-ramp through oncoming traffic" (threshold 6). If they fail their stunt, then they wipe out and are no longer involved in the chase. If both characters succeed at the stunt, then the lead grows if the quarry got more net hits, and it shrinks if the pursuer gets more net hits. If neither character gets more hits, the chase continues. If neither character commits to a stunt, then the lead grows if the quarry is physically moving faster and shrinks if the chasing character is.

\begin{description}
\item[Caught:] If the chasing character narrows the lead below a narrow lead they have caught up to the chased character. This may not be the end of the chase, as it merely means that the chasing and chased characters are close enough to effect each other directly. The pursuing character may have limited options if they are both in vehicles (though they take the opportunity to perform a PIT maneuver if they were confident in their driving skills and didn't mind replacing the front end of their car). Once the quarry has been caught there is at least one round to perform close range maneuvers, but if they take their action to attempt to escape again it can go directly back to a Long Chase.
\item[Narrow Lead:] Most Chases begin at a narrow lead. The pursuing character can perceive the chased character or vehicle with ease. If the target turns or performs a stunt, the pursuing character can see that happen. At this range the stunts are normally \dicepool{Strength or Agility + Athletics} on foot or \dicepool{Intuition + Driving} in a car.
\item[Wide Lead:] If the chased character widens their lead then their path is not always visible to their pursuer, and they can plausibly make an unwitnessed turn and throw their pursuer entirely. At this range, stunts are usually performed with \dicepool{Intuition + Stealth}, but may be done with other attribute/skill combinations if the circumstances warrant.
\item[Escaped:] A chased character who pushes their lead past a wide lead has escaped altogether. They aren't even in a chase scene any more, and if the pursuer wants to catch up to them they'll have to scout them out anew.
\end{description}

%%%%%%%%%%%%%%%%%%%%%%%%%
\subsection{Hide and Seek} \index{Movement!Hiding}
%%%%%%%%%%%%%%%%%%%%%%%%%
\hspace{\parindent} People cower under stairs and hide in closets from approaching slashers in horror movies all the time, and sometimes this even works. When you grab yourself a hiding place, you are putting yourself in a position where anyone who looks hard enough for you \textit{will} find you. There are only so many places for a person to be hiding in the basement, so if the man in the hockey mask or the reanimated Nazis spend long enough searching that basement, they \textit{will} find any person hiding there, and it will go poorly for someone. The gamble with hiding is that in fact, ax wielding psychopaths are busy people and they have shit to do. So if they search the basement for some amount of time without finding anyone, they'll give up and try the garage. The way this works is that the seeker announces how long they are going to look in an area and makes an \dicepool{Intuition + Perception} test. The hider makes an \dicepool{Intuition + Stealth} test to determine how long the base time to find them would be, and that time is divided by the hits on the Seeker's Perception test to determine how long it would actually take them to find the hiding character. If they actually spend enough time looking, the hiding place is uncovered, and if they don't, they don't.

%\begin{table}[htb]
\begin{wraptable}[11]{R}{6.7cm} \vspace{-.45cm}
\rowcolors{1}{white}{tan} \caption{Hide and Seek Times} \centering
\begin{tabular}{l c}
\textbf{Base Time}&\textbf{Seeker is looking in}\\
 3 Days&The Woods\\
 1 Day&The Mountain\\
 5 Hours&The Mall\\
 1 Hour&The Library\\
 20 minutes&The House\\
 5 Minutes&The Basement\\
 1 Minute&The Bedroom\\
 1 Round&The Closet\\
\end{tabular}
\end{wraptable}

The base amount of time needed to search an area depends on how large it is and how much crap there is in it. It is important to note that it is entirely intentional that people who are more with-it and stealthy are harder to find even if they are huddling in the same place. That is how it works in the horror genre. Note also that if more than one character is hiding together, that the character with the \textit{best} Stealth check gets to set the threshold to find them. In the horror genre it is importantly true that masked men find you \textit{faster} when you split up and hide separately than when you stay together. The hide and seek table assumes that the seeker is looking in the right place. If the hider is in the closet but the seeker doesn't know that, they may be searching the whole house.

The table assumes that the seeker is looking for something human sized. Very small things can be very difficult to find and can have longer base hide-and-seek times with similar hiding places. Finding an unsorted book in a library or a literal needle in a haystack could easily take all day. Characters who are hiding can also panic and run (or sneak) from cover if they think that the seeker is dedicated enough. This goes to Chases, though the option often exists to make a Stealth stunt to sneak out while the seeker has their attention elsewhere.

%%%%%%%%%%%%%%%%%%%%%%%%%
\subsection{Driving like a Maniac} \index{Movement!Driving}
%%%%%%%%%%%%%%%%%%%%%%%%%

\hspace{\parindent} Even if you're driving a clunker of a VW Bus from the 1960s that goes 0-60 in 12 seconds, you \textit{still} hit 60 inside of a round. During that round, your clunky car will have cleared more than 160 meters, and next round it will go more than twice that distance. Any car that is already moving is basically unreachable by pedestrians unless they have super speed or are already ahead of the vehicle and willing to do an extreme (or even crazy extreme) stunt to get one chance at jumping on like an action cop. Cars start up in a variable amount of time, but you may have to turn the key several times to start even a well oiled machine if you're panicking or the plot requires it. What this means is that if a vehicle hasn't started already at the beginning of the round, it won't start until every character who is moving to the car's location has done so and been given the opportunity to act.

Vehicles move 3.33 meters per round per KPH (or 17 ft per MPH if your speedometer reads in Imperial). As such, even a modest town speed far exceeds a Draining Sprint (30kph / 18mph). Generally speaking, speed limits are there for a reason, which is that if you go more than 10 KPH faster than that, things become \textit{unsafe}. If you're driving at safe speeds, driving is decidedly "not awesome" and generally threshold 0. Characters who are culturally familiar with cars do not even need to make Driving tests under those circumstances. However, if characters drive at unsafe speeds or in really terrible conditions, tests need to be made. This is why characters in movies will often say that they "can't go out in this storm" even when there is apparently a monster loose in town -- because if they don't have a Driving Skill they literally can't.

Generally, driving over 10 KPH (6mph) faster than the safe speed is threshold 1, 20 KPH (12mph) faster is threshold 2, 40 KPH (25mph) faster is threshold 3, 80 KPH (50mph) faster is threshold 4, and so on. Poor visibility and hazardous driving conditions reduce the safe speed below the speed limit. Really terrible conditions may reduce the safe speed all the way to zero. A character needs to make one driving test between each "place" which is a rather fluid concept of topology. Routes that have more important locations on them to potentially crash or break down at require more rolls. Because it's narrative driving, and that's seriously how it works. People sometimes blow a tire in the middle of nowhere, but no one ever loses a tire in the east of nowhere.

%%%%%%%%%%%%%%%%%%%%%%%%%%%%%%%%%%%%%%%%%%%%%%%%%%
\section{Wounds} \index{Wound Levels}
%%%%%%%%%%%%%%%%%%%%%%%%%%%%%%%%%%%%%%%%%%%%%%%%%%
\tagline{OK, that hurt.}

When a character takes damage, they are usually allowed to Soak that damage. This involves rolling a Soak Test\index{Soak Test} (normally Strength), with the hits subtracted from incoming damage. If the damage is soaked to zero, the character takes no perceptible damage. If, after the damage is soaked, it is still greater than zero, then the character suffers a wound. Some number of boxes will be filled up. All characters and objects have 10 boxes on their condition monitor. And when those boxes are filled in, they are marked depending on the type of damage it is:
\begin{description*}
\item[Normal:] If the wound is a Normal Wound, draw a single diagonal line between the lower left of the box and the upper right of the box. Like this: [\texttt{/}]
\item[Lethal:] If the wound is a Lethal Wound, draw two diagonal lines that cross in the box. Like this: [\texttt{X}]
\item[Aggravated:] If the wound is an Aggravated Wound, draw two diagonal lines that cross in the box and run a horizontal line through that. Like this: [\sout{\texttt{X}}]
\end{description*}

%\begin{table}[htb]
\begin{wraptable}[11]{R}{8cm} \vspace{-.6cm}
\rowcolors{1}{white}{tan} \caption{Wound Severity} \centering
\begin{tabular}{l l l}
\textbf{Net} & \textbf{Wound Name} & \textbf{Boxes Filled}\\
1 &\textbf{P}etty & 1 Box\\
2 &\textbf{O}rdinary & 3 Boxes\\
3 &\textbf{S}erious & 6 Boxes\\
4 &\textbf{I}ncapacitating & All 10 Boxes\\
5 &\textbf{T}erminal & All 10 Boxes*\\
6 &\textbf{T}erminal & All 10 Boxes*\\
7+ &\textbf{D}eath& NA\\
\end{tabular}\\
* -- Also, you are probably going to die.
\end{wraptable}

When all 10 of a character's boxes fill up with any kind of mark they are Incapacitated\index{Incapacitated}, but they do not necessarily die. If more Lethal \textit{or} Normal wounds are inflicted when all the boxes are filled in and there are any boxes only filled in as Normal, draw an extra diagonal line through an appropriate number of them to make them Lethal wound boxes. Similarly, Aggravated wounds displace lesser wounds if the track is already full. For ease of accounting, the game arranges all wounds in the order of Aggravated, Lethal, Normal in the character's wound boxes. This can be achieved with simplicity by treating the forward and backward slash of the Lethal wound as separate and placing each one on the first line it fits. You can do the same with the horizontal mark on the Aggravated Wound.

In the case that a character is Incapacitated and at least one of the boxes filled in only with Normal damage, they are in no immediate danger (from their wounds, being incapacitated in a place where you just \textit{took} a wound implies a certain level of urgency in most cases). If however every box is filled in with a line that goes from the lower right to the upper left ([\texttt{\textbackslash}]), the character's condition has a chance of degrading -- sending them spiraling into death, especially if they do not receive medical care. In general, intervention can stabilize such a character if administered within an hour of the injury, and sometimes characters will stabilize anyway. Note that in most cases, when a box gains the downward sloping slash it will gain it as part of a Lethal or Aggravated wound and actually look this [\texttt{X}] or this [\texttt{\sout{X}}]. The different lines also heal separately, so it is entirely possible to be left with just a line from the upper left to the lower right ([\texttt{\textbackslash}]) even though no wound actually makes a mark that looks like that when it is inflicted.

A Terminal Wound is much like an Incapacitating Wound in that it fills up all of the character's wound boxes. The difference is that it is also an emergent threat to the character's life. A character who suffers a Terminal Wound will need to be stabilized within about five minutes or -- barring a miracle -- they will die. And yes, that includes Normal Wounds. A rubber bullet or a boxer's punch is entirely capable of stopping a heart, and then someone had better be on hand with CPR or the curtain is coming down.

\textbf{Wound Penalties:} \index{Wound Penalties}When a character has been \textit{recently} damaged, their ability to act is extremely impaired. The character's actions take a penalty equal to the number of boxes on their condition monitor that have a line running through them from the lower left to the upper right ([\texttt{/}]). This includes any Normal ([\texttt{/}]), Lethal ([\texttt{X}]) or Aggravated ([\texttt{\sout{X}}] wound boxes. Characters with a Willpower in excess of 2 reduce wound penalties (to a minimum of zero) by the amount their Willpower exceeds 2. So for example, a character with a Willpower of 4 can ignore the wound penalties imposed by 2 of their boxes. A character with \linkpower{Indominability} ignores wound penalties completely.

\textbf{Petty Wounds:} A scratch. Will stop bleeding in a minute, heal completely in a day. May still get infected/transmit poison.

\textbf{Ordinary Wounds:} These are painful and physically debilitating, but they generally heal up without much more than a discoloration at the spot, though they will take many days to do so.

\textbf{Serious Wounds:} People who see a serious wound inflicted generally wince, and the agony and pure physical hampering it causes will generally strike mortal terror into the sane.

\textbf{Incapacitating Wounds:} An incapacitating wound is called that because it incapacitates the victim. They may or may not lose consciousness, but they will be unable to stand even if they are able to keep their eyes open. If a character receives medical care or rest, an Incapacitating Wound can be reduced to a Serious Wound.

\textbf{Terminal Wounds:} Your ass is dying. Not dead, but nothing short of a miracle and a pair of lightning paddles is going to save you now. If you are saved, you still have an Incapacitating Wound.

\textbf{Death:} A character who receives 7 unsoaked damage from a single attack simply dies outright. Feel free to describe graphic injuries such as exploding heads and bodies torn in half.

%%%%%%%%%%%%%%%%%%%%%%%%%%%%%%%%%%%%%%%%%%%%%%%%%%
\section{Healing and Death}
%%%%%%%%%%%%%%%%%%%%%%%%%%%%%%%%%%%%%%%%%%%%%%%%%%
\tagline{Good as new\\
\ldots{}or not\ldots{}}

When characters are injured, there are two possibilities: they can either get better or not. Medically speaking, injured tissue can resolve into healed or regenerated tissue, it can resolve into a scar, or it can die. This is actually much farther in-depth than we care to go into. When a character is given time to heal, they reduce the number of boxes that are filled in with a type of damage by 2. However, if there is at least one box remaining with that kind of damage it is \textit{worsened} to the next type (this doesn't happen for Aggravated Wounds, which are already the worst type). So for example, if a character has 3 boxes with Normal Wounds, after 20 minutes of healing, it would be one box with a Normal and Lethal slash in it. If a character is running around being the Action Man, they don't heal at all. Healing doesn't really happen until the character takes some time off from ninja flips and car chases. The exception is characters who have suffered incapacitating or terminal injuries, who start the cycle of healing and possible death right away.

%\begin{table}[htb]
\begin{wraptable}{R}{7.3cm} \vspace{-.3cm}
\rowcolors{1}{white}{tan} \caption{Wound Notation} \centering
\begin{tabular}{l c c}
\textbf{Wound}&\textbf{Symbol}&\textbf{Healing Time}\\
Normal&\textbf{[/]}&20 minutes\\
Lethal&\textbf{[\textbackslash]}&1 day\\
Aggravated&\textbf{\sout{[ ]}}&3 Days\\
\end{tabular}
\end{wraptable}

\textbf{Sample Healing:}\\
\texttt{\textbf{[/] [/] [/] [/] [/] [/] [/] [/] [ ] [ ]}}\\
20 minutes later:\\
\texttt{\textbf{[X] [/] [/] [/] [/] [/] [ ] [ ] [ ] [ ]}}\\
40 minutes later:\\
\texttt{\textbf{[X] [X] [/] [/] [ ] [ ] [ ] [ ] [ ] [ ]}}\\
60 minutes later:\\
\texttt{\textbf{[X] [X] [\textbackslash] [ ] [ ] [ ] [ ] [ ] [ ] [ ]}}\\
80 minutes later:\\
\texttt{\textbf{[\textbackslash] [\textbackslash] [\textbackslash] [ ] [ ] [ ] [ ] [ ] [ ] [ ]}}\\
1 day later:\\
\texttt{\textbf{\sout{[\textbackslash]} [ ] [ ] [ ] [ ] [ ] [ ] [ ] [ ] [ ]}}\\
2 days later:\\
\texttt{\textbf{\sout{[ ]} [ ] [ ] [ ] [ ] [ ] [ ] [ ] [ ] [ ]}}\\
4 days later:\\
\texttt{\textbf{[ ] [ ] [ ] [ ] [ ] [ ] [ ] [ ] [ ] [ ]}}

%%%%%%%%%%%%%%%%%%%%%%%%%
\subsection{Incapacitating and Terminal Wounds}
%%%%%%%%%%%%%%%%%%%%%%%%%

\hspace{\parindent} If a character's entire track is filled, the character's healing timeframe is increased by one category until they have at least one box that has no wounds in it. This means that a character whose track fills up with Normal Wounds will be unconscious for an hour unless someone delivers first aid in the meantime.

If a character has a Terminal Wound, they are dead in 5 minutes unless they spend an Edge or they get 3 hits on a Healing Test, or someone successfully applies first aid. At 1 or 2 hits on a healing or first aid test, the timeframe of death from a Terminal Wound is increased (to 20 minutes, or 1 hour respectively), which may allow a character to get to a place where a better shot at a final Healing Test can be made (or medical personnel to get to them, giving their skill to the final Healing Test). Even though a character with a Terminal Wound is going to be dead before a sufficient timeframe would pass to provide a healing test, a character is always allowed one Healing Test just before they die.

A character who is incapacitated\index{Incapacitated} or terminally wounded can also have their death hastened with further injury. When the character's wound boxes are filled with Lethal damage, additional injuries accumulate towards a threshold of outright killing the target. An injury that would itself be incapacitating or terminal accumulates two towards this death threshold, and an injury that would be Serious or less accumulates only one. A character's Death Threshold\index{Death Threshold} is generally equal to their Willpower (characters with Lure of Destruction have a higher Death Threshold). And yes, this means that once someone is already on death's door, it takes quite a few extra bullets to push them through it -- it's not like they are going to go \textit{into} shock at that point.

%%%%%%%%%%%%%%%%%%%%%%%%%
\subsection{The Healing Test} \index{Healing Test}
%%%%%%%%%%%%%%%%%%%%%%%%%

\hspace{\parindent} When a character is healing, they are entitled to a healing test to prevent lethal or aggravated damage from accumulating. The healing test is \dicepool{Edge + Survival}, and is modified by the conditions the character is in while they are healing. If the character gets a hit, the round of healing goes by without generating any extra wounds. If another character is on hand to provide medical care, that character's \dicepool{Logic + Medicine} test works together with the injured character's \dicepool{Edge + Survival} test as a Teamwork test. Which counts as the main and which counts as the assistant depends on who has the bigger dice pool.

The healing test is especially important for characters who suffer Incapacitating Lethal or Aggravated injuries, because the first time normal wounds heal and generate a Lethal slash when the track is already full will push them into a terminal wound. Note that this really means that if an Extra takes an Incapacitating Wound from a knife or a bullet that they will always go Terminal in an hour unless medical personnel come and save them (they will also be able to crawl around at that point, perhaps able to give some dramatic piece of information before they die).

%%%%%%%%%%%%%%%%%%%%%%%%%
\subsection{First Aid}
%%%%%%%%%%%%%%%%%%%%%%%%%

\begin{wraptable}[7]{R}{7.2cm} \vspace{-1.2cm}
\rowcolors{1}{white}{tan} \caption{Healing Conditions} \centering
\begin{tabular}{l c}
\textbf{Conditions are\ldots{}}&\textbf{Healing Modifier}\\
Dangerous&-2\\
Distressing&+0\\
Restful&+1\\
Sanitary&+2\\
Awesome&+4\\
\end{tabular}
\end{wraptable}

\hspace{\parindent} When a character is trained in first aid, they can reduce the amount of damage on someone substantially. This is pretty much a one-time deal and has to be performed within the "golden hour" of the injury (an hour from when the patient was injured, not an hour from first medical contact). First aid is a \dicepool{Logic + Medicine} test. Removing 2 Normal slashes is threshold 1, also removing 2 Lethal slashes is threshold 2, and further removing 2 Aggravated lines is threshold 3. Additional hits remove an additional 1 slash of each type. So a five hit First Aid test would remove 4 Normal, Lethal, and Aggravated lines from the victim's wound box.

If the victim has a Terminal Wound, it is instead Threshold 3 to stabilize them and reduce them to a normal Incapacitated state. Additional hits remove wounds from boxes as normal. But perhaps the biggest advantage of First Aid is the \textit{time} it takes to make the first healing test during medical treatment -- which is one time frame shorter than normal for the left-most injury type in the wound track (1 minute to rouse someone from Normal Damage, 5 minutes to treat Lethal Damage, and 20 minutes to treat Aggravated Damage).

%%%%%%%%%%%%%%%%%%%%%%%%%
\subsection{Being Dead} \index{Death}
%%%%%%%%%%%%%%%%%%%%%%%%%
\tagline{OK, \textbf{now} we go through the pockets looking for loose change.}

Characters in horror die from time to time. Sometimes a character will end up dying several times, because After Sundown posits ghastly life after death and even has magical ways to restore life to the living. When a character dies, all of their powers that cost Power Points to activate or which otherwise last for a scene end. In addition, anything they have going that could be dispelled is dispelled with their death. Very importantly, a character's power schedule (if any) does not change just because they are dead. A Strigoi can still be fed blood to regain Power Points and a Frankenstein can still be recharged while a corpse. This usually only matters if they have the \linkpower{Restoration} Power, but there are other ways to raise the dead, and it might be important that they rise up with a refilled power reserve on hand.

The dead may also come back as Ghosts\index{Ghost}, whether they are Luminaries or Extras. In After Sundown, no one becomes a Ghost until they've been dead for three settings of the sun. There are movies in which the spirit is left standing there as the body hits the floor, and there are movies in which the spirit wakes up having been dead and buried for a couple days. After Sundown's assumed setting is the latter.

Sometimes a player may want to continue their character as a Wraith rather than start a new character or have their character restored to life. This is a workable storyline, but it is fraught with peril because Wraiths are not normally considered playable (being constantly insubstantial while outside the Gloom). If the players agree, the character can be converted into a Wraith, with their Powers traded out for the basic Wraith abilities, starting what is in essence a new Origin Story. Being a Wraith should not be a license to print superpowers, so the new Powers that Wraiths get should replace some of the abilities they already had rather than just piling more on.

%%%%%%%%%%%%%%%%%%%%%%%%%%%%%%%%%%%%%%%%%%%%%%%%%%
\section{Armor}
%%%%%%%%%%%%%%%%%%%%%%%%%%%%%%%%%%%%%%%%%%%%%%%%%%
\tagline{Some people wear their armor on the inside, as some sort of metaphorical refusal to emotionally connect with others.\\
I find that kevlar vests are much more effective at stopping bullets.}

%\begin{table}[htb]
\begin{wraptable}[11]{R}{7.6cm} \vspace{-.7cm}
\rowcolors{1}{white}{tan} \caption{Armor Types} \centering
\begin{tabular}{l c c c}
\textbf{Armor} & \textbf{Ballistics} & \textbf{Melee} & \textbf{Heat}\\
Leather Jacket&1&1&1\\
Kevlar Vest&2&1&0\\
Flak Vest&4&3&1\\
Chain Mail&2&6&2\\
Plate Mail&3&6&2\\
Riot Gear&6&6&2\\
Hazmat Suit&1&2&4\\
Fire PPEs&2&3&6\\
\end{tabular}
\end{wraptable}

Characters will sometimes wear armor. Armor makes it less likely that a character will die when placed into a deadly situation. Armor is generally heavier than normal clothes, and wearing it is tiring. However, there are characters in After Sundown who have Patience of the Mountain and literally do not get tired. And they often wear armor all the time, even to sleep. It should be noted however that most of the world has a level of civilization that makes wearing armor problematic socially.

Armor has a different rating against different kinds of attacks. The Ballistics rating applies against guns and explosives and the like. The Melee rating applies against knives and claws in addition to baseball bats and large thrown objects like rocks and cars. And finally, the Heat value applies against getting set on fire and also cold. The rating of armor adds additional soak dice\index{Soak Test}. In addition, a character with an armor rating can buy a number of hits on soak tests equal to their armor rating at the rate of 1 hit per three dice. So a character with 2 Armor against an attack could set aside 3 soak dice for one automatic hit or 6 soak dice for 2 automatic hits.

Leather Jackets, Kevlar Vests, and Flak Vests don't cover the character's whole body, and an attacker who knows about them can attack the unarmored bits of their target by voluntarily increasing their to-hit threshold by 1. For armor that truly covers the whole body, such as chain and plate mail, riot gear, and hazmat suits, this is not an option. Remember that outr\'{e} armor like that is often rather hard to explain socially, so even characters who \textit{could} wear such comfortably usually don't.

%%%%%%%%%%%%%%%%%%%%%%%%%%%%%%%%%%%%%%%%%%%%%%%%%%
\section{Various Hazards}
%%%%%%%%%%%%%%%%%%%%%%%%%%%%%%%%%%%%%%%%%%%%%%%%%%
\tagline{Holy crap, that looks dangerous.}

%%%%%%%%%%%%%%%%%%%%%%%%%
\subsection{Explosions}
%%%%%%%%%%%%%%%%%%%%%%%%%
\tagline{Can we make an explosion explode?}

%\begin{table}[htb]
\begin{wraptable}[11]{R}{10cm}\vspace{-.5cm}
\rowcolors{1}{white}{tan} \caption{Explosives} \centering
\begin{tabular}{l c c c}
\textbf{Weapon}&\textbf{Damage}&\textbf{Radius}&\textbf{Size} \\
Flashbang & 1 & 1m & S \\
Concussion Grenade & 6N & 2m & S \\
Hand Grenade & 6 & 6m & S \\
Plastic Explosive Charge & 8 & 50 cm & S \\
Molotov Cocktail & 2F & 1m & M \\
Land Mine & 3 & 1m & M \\
Car Gas Tank & 6F & 2m & L \\
\end{tabular}
\end{wraptable}

Sometimes things blow up, and in the movies things blow up even more. Even when we step outside the action genre, it is a recognized fact that objects in general in any media -- including cooperative horror storytelling -- are substantially more \textit{explosive} than they are in real life. Explosions physically expand in roughly spherical paths (barring the use of shaped charges), and thus in a general sort of way an explosion can be expected to become weaker as per the square of the distance from the point of origin. Except of course it's actually much more complicated than that, because there's gravity and air resistance, and shrapnel pieces that fly a lot more like bullets, and so on. More massive shrapnel flies farther as it loses less power to air resistance, and compression waves pass through denser media better than air. And so on. Game mechanically this is abstracted out into an explosion's \textit{Damage} (which is how much damage targets in the first area have to soak), and the same explosion's \textit{Radius} (which is how far that area extends). An explosion does damage to targets that are outside its Radius, but substantially less. A target that is farther away from the explosion than the Radius takes 2 less damage if they are within the Radius \textit{of} the Radius. And this continues until the Damage reaches zero. In effect, the explosion is modeled in the game as an onion where each band has a thickness of the Radius and does 2 less damage than the band before it. In a nod to the truly epic destruction caused by explosives that are actually adjacent to the target, an explosion will inflict 2 \textit{extra} damage if the explosive is actually touching the target when it goes off.

An Explosion isn't normally affected by net hits on an attack roll, and inflict precisely the same damage if they are thrown perfectly as if they are detonated when simply dropped and forgotten. An exception to that is if an explosion is specifically placed to cause maximum damage. A character's \dicepool{Logic + Sabotage} check can increase the damage bonus for a point blank explosion. Explosions are affected strongly by cover, and their damage ratings are reduced by the coverage and its toughness. Explosions do not necessarily inflict fire damage, even though they do act by burning. Unless otherwise noted, the primary damage is flying shrapnel and concussive force. Explosives can be wrapped in silver, wood chips, or steel in order to make that shrapnel into something that is especially effective against certain supernatural creatures.

%%%%%%%%%%%%%%%%%%%%%%%%%
\subsection{Throwing Things} \index{Attacking!Throwing}
%%%%%%%%%%%%%%%%%%%%%%%%%
\tagline{As soon as the first ape threw their first rock, primates started being a threat to leopards instead of just the other way around.}

Characters will probably end up throwing a lot of stuff in an After Sundown campaign. Explosives, buckets of water, and even just plain chairs. It is well known that you can start a fight pretty much anywhere by throwing a chair in a crowd. When you're throwing something, you get bonuses to your attack roll if the thing is big. It is simply easier to connect with a target if you are throwing a whole car than if you are throwing a sharpened playing card. In general, a Small object (like a bottle or a hungamunga) provides +1 die, a medium object (like a tomahawk or a bucket of water) provides 2 extra dice, a large object (like a chair or a person) provides 4 extra dice, and a huge object (like a car or an altar) provides 6 extra dice. A character can throw things that they can lift but not effectively wield in melee because of their Strength being exceeded, but not very far. Such objects are only accurate out to Adjacent range and have a maximum range of Near. Items that the character can wield can be thrown out to Short Range

Damage inflicted by thrown weapons is usually pretty disappointing. Unless it's designed as a throwing weapon (or randomly shaped like something that is such as a bowling ball), the base damage of such thrown items are only going to be 0 or 1, depending upon hardness, sharpness, and density. Any object that is thrown into someone that is too heavy for them to use will knock them down however. And if it's too heavy for them to lift, they may become trapped under it. So when a character throws a molotov cocktail at an opponent, the damage from throwing the bottle into them is usually pretty inconsequential. The whole "catching on fire" thing is pretty keen though.

%%%%%%%%%%%%%%%%%%%%%%%%%
\subsection{Damage Over Time} \label{subsection:Damage-Over-Time}
%%%%%%%%%%%%%%%%%%%%%%%%%
\tagline{Are you still on fire? You should stop that eventually.}

When a character is soaked in acid, freezing to death, on fire, or otherwise subjected to a damaging situation that is ongoing, we call this DOT or Damage Over Time. While it \textit{could} be modeled as a series of tiny attacks that had a remote chance of doing damage each second, that is far fiddlier than the people actually playing the game need to deal with. DOT damage is added one wound box at a time. A DOT effect has a "delay" number, and that number determines how much time passes between filling in one wound box and the next. When a DOT is introduced to a character, they may make a Resistance Test against it, and the hits on that test are \textit{added} to the Delay Number rather than \textit{subtracting} from the actual damage done. Being covered in acid for long enough is liable to be a problem for anyone. DOT's will continue filling in a wound box on schedule until they end. For external sources of damage, that generally means removing the noxious stimulus, while something like an injected poison usually has an amount of time it will persist based on how much was injected (this time could be cut shorter with things like antidotes or diuretics). Damage from the DOT cannot be healed naturally until the DOT ends, though damage from other sources can be.

\begin{table}[htb] \rowcolors{1}{white}{tan}
\caption{Damage Over Time} \centering
\begin{tabular}{c l l}
\textbf{Delay}&\textbf{Time Between Damage Boxes}&\textbf{Example DOT Source} \\
0 & 3 seconds (Each Initiative Pass) & Falling into hissing green goo \\
1 & 1 Round & Engulfed in billowing flames \\
2 & 2 Rounds & Unable to breathe \\
3 & 5 Rounds (1 minute) & Bitten by deadly serpent \\
4 & 2 Minutes & Exposed to killer frost \\
5 & 5 Minutes & Inhaling noxious smoke \\
6 & 15 Minutes & Watching an M. Night Shyamalan movie \\
7 & 30 Minutes & Staring at a glowing radioactive rock \\
8 & 1 Hour & Inadequate protection from brutal cold \\
9+ & \multicolumn{2}{l}{2 Hours, x2 Time Each Additional Delay Number (4 Hours, 8 hours, etc\ldots{})} \\
\end{tabular}
\end{table}

%%%%%%%%%%%%%%%%%%%%%%%%%
\subsection{Falling}\label{subsection:Falling}\index{Falling}
%%%%%%%%%%%%%%%%%%%%%%%%%

%\begin{table}[htb]
\begin{wraptable}[8]{R}{6cm} \vspace{-.7cm}
\rowcolors{1}{white}{tan} \caption{Falling Damage} \centering
\begin{tabular}{l c}
\textbf{Distance}&\textbf{Damage} \\
\textbf{P}etty (0-2m) & 1N \\
\textbf{O}rdinary (3-4m) & 2N \\
\textbf{S}erious (5-6m) & 3 \\
\textbf{I}ncapacitating (7-10m) & 4 \\
\textbf{T}erminal (11+ m) & 5 \\
\end{tabular}
\end{wraptable}

Characters will fall from time to time. And falling substantial distances actually can take quite an amount of time. However, in a 12 second combat round a character could fall over 600 meters -- so for practical purposes it's usually best to simply have characters hit the ground after having just one Simple Action to try to do something about their situation. It is also true that "The bigger they are, the harder they fall."  That's not just a trite saying that He Man gives before tripping giant robots, it's physical reality. Larger creatures have more mass proportional to their surface area and accelerate at the same speed, truly mice and ants can survive being dropped from any height and elephants can't even jump without breaking their bones on the way down. Game mechanically this truth is handled by preventing characters from using Strength or Armor to soak falling damage, and by having larger creatures take additional damage from falls. Characters can soak damage from falls by performing \dicepool{Agility + Athletics} stunts with a threshold of 1 (net hits soak damage, but the first hit does not). Magical benefits for soaking damage do apply (as they make the character tougher relative to their mass rather than adding additional mass), so a character gains the benefits of Fortitude.

If the character falls onto a hard or sharp surface, increase damage by 1 or more. If the falling creature is large, increase the damage by 1 or more. If the falling creature is small, reduce the damage by 1 or more.

%%%%%%%%%%%%%%%%%%%%%%%%%
\subsection{Electrocution}\label{subsection:Electrocution}\index{Electrocution}
%%%%%%%%%%%%%%%%%%%%%%%%%

%\begin{table}[htb]
\begin{wraptable}[8]{R}{9.1cm}\vspace{-.5cm}
\rowcolors{1}{white}{tan} \caption{Electric Damage} \centering
\begin{tabular}{l c}
\textbf{Shock} & \textbf{Damage} \\
\textbf{O}rdinary (Wall Socket)  & 2N \\
\textbf{S}erious (Electric Fence) & 3N \\
\textbf{I}ncapacitating (High Powered Taser) & 4N \\
\textbf{T}erminal (Lightning Strike)  & 5 \\
\end{tabular}
\end{wraptable}

Electricity damage is something of a oddity. Electricity flows through the path of least resistance, and it inflicts damage based on the resistance of the path it flows through. Thus, you can defend yourself from electricity by covering yourself in high resistance insulation (because it will redirect electricity away from your body to another path) \textit{or} by covering yourself with low resistance conductive mesh (because it will \textit{create} a preferred path through the mesh and away from your organs). From the standpoint of the game, a character who is protected by especially conductive or non-conductive material is \textit{immune} to electrical shocks. Electricity is inherently unpredictable, whenever someone is electrocuted, roll a die -- if it comes up a hit, increase the Damage by 1. Net hits on attacks with electrical outputs are not added to the damage of an electric shock.

%%%%%%%%%%%%%%%%%%%%%%%%%
\subsection{Poison}\label{subsection:Poison}\index{Poison}
%%%%%%%%%%%%%%%%%%%%%%%%%

Poison has a progressive effect that affects the target more as time goes on. Functionally this means that poison is \textit{much slower} than bullets or chainsaws. The way this is handled is as a DOT. However, not all poisons inflict actual damage, many come with special effects in addition to or instead of filling in wound boxes. Poisons normally only accumulate effects for a certain amount of time based on the original dose. When a victim is exposed to additional doses before the first has run its course, use the current timer for how long the poison will last and reduce the Delay by 1 if the timer is less than half over, or leave the Delay number alone and reset the termination timer if it has run more than half of its course. Characters with \linkpower{Patience of the Mountains} or \linkpower{Bite of the Serpent} are immune to poisons. If a character is given anti-venom or some similar treatment, the character gains additional resistance dice to increase the Delay and/or the termination counter is reduced in length (depending on whether it works by clearing the chemical from the victim's system or neutralizing the effects). Note that the dosages are all the "normal" dosages, which for "recreational" poisons (like street drugs) are actually very small. The assumption is that street drugs are being snorted unless otherwise indicated. Poisons my be more or less effective if administered by another route (injecting pepper spray would be all kinds of fatal). The secondary effects of a poison kick in as soon as one damage box is filled in (or would be filled in for non damaging poisons), and end when the Timer runs out. Damage, whether Normal or Lethal, remains until healed.

\begin{table}[htb] \rowcolors{1}{white}{tan}
\caption{Poisons} \raggedright
\begin{tabular}{l c c p{8cm}}
\textbf{Poison} & \textbf{Delay} & \textbf{Timer} & \textbf{Notes}\\
Tear Gas & 2N & 5 rounds & Provides a "dose" for each round of exposure.\\
Pepper Spray & 0N & 2 rounds & \\
Tranq Dart & 0N & 3 rounds & Fatigue\\
Rat Poison & 5 & 1 hour & Ingested.\\
Uranium & 18 & 3 Months & Provides a "dose" for each five minutes of exposure.\\
"Euphoric" & 2\textsuperscript{0} & 1 Hour & Amnesia and Overstimulation\\
"Hallucinogenic" & 2\textsuperscript{0} & 6 Hours & Amnesia and Delusion\\
"Paralytic" & 0\textsuperscript{0} & 10 Minutes & Paralysis\\
"Soporific" & 1\textsuperscript{0} & 1 Hour & Sleep\\
"Toxic" & 0 & 10 Rounds & Agony\\
Meth & 3\textsuperscript{0} & 4 Hours & Stimulation and Overstimulation\\
Opium & 7N & 3 Hours & Anesthetic and Fatigue\\
Cocaine & 5\textsuperscript{0} & 20 Minutes & Anesthetic and Delusion\\
Alcohol & 9N & 1 Hour & Ingested. Repeated dosing can cause Amnesia, Anesthetic, or Delusion\\
\end{tabular}\\
\textbullet{}Poisons in Quotes are the magical poisons available with \linkpower{Bite of the Serpent}. At the character's option, the delay can be decreased by the character's Potency (minimum of 0, after the Soak roll).\\
\textbullet{}When a Poison has a secondary effect, that effect generally lasts for 10 minutes to an hour.\\
\textsuperscript{0}: This Poison doesn't actually do any damage, the damage level is just there so that secondary effects occur. At the MC's option, overdoses may still be fatal if the virtual damage level rises to Terminal.
\end{table}

%%%%%%%%%%%%%%%%%%%%%%%%%%%%%%%%%%%%%%%%%%%%%%%%%%
\section{Temporary Conditions}
%%%%%%%%%%%%%%%%%%%%%%%%%%%%%%%%%%%%%%%%%%%%%%%%%%
\tagline{Just as water has no constant shape, so there are no constant conditions.}

The following is a sample set of conditions that might transiently affect a character in After Sundown games. These conditions are generic cases, it is entirely possible for a character to be "more Delusional" or whatever, in which case the raw numerics of the effect should be increased.

\textbf{Agony:} \index{Agony}The victim is in incredible wracking pain. Their Wound penalties are calculated as if they had two more boxes filled in than they do with Normal marks, up to a maximum of all boxes filled.

\textbf{Amnesia:} \index{Amnesia}Drinking to the point of blacking out will cause a man to lose an evening, and in game terms when a character is not going to remember things they are operating under Amnesia. In the real world there are many ways to get to this state, most of which involve chemicals. When a character is operating under Amnesia they suffer a -2 dicepool penalty to actions and the threshold to resist acting impulsively (including succumbing to Frenzy) is increased by 1. Also remember that the character \textit{won't} remember anything, so it can sometimes be a good narrative tool to skip ahead in the story and then go back and roleplay those events later as a flashback.

\textbf{Anesthetic:} \index{Anesthetic}The character is neither bothered by, nor aware of, noxious stimuli. That means that pain doesn't bother them, but they also miss things that an unaffected person would notice. The character calculates Wound penalties as if two less boxes were filled in, but the character is at -2 dice on Perception or Empathy tests.

\textbf{Delusion:} \index{Delusion}When a character is afflicted by Delusion they respond to things in an irrational fashion. Sometimes this can be well articulated as in the case where a character hallucinates tiny shapes moving in their peripheral vision they'll jump at shadows and, as they acclimatize themselves to their situation, \textit{ignore} actual moving objects. And that just looks disconcerting to other people. Sometimes a character's delusions will be harder to articulate, but they will be no less disconcerting to those around them. A character suffering from Delusions is opposed on all Social tests by 3 dice. And yes, they can end up with negative hits. 

\textbf{Fatigue:} \index{Fatigue}A character who is fatigued has a great deal of difficulty exerting themselves. They cannot perform an Exhausting Run or Draining Sprint. Also their Strength is reduced by 1. And yes, that means that their Soak is reduced. When the body's reserves are exhausted, they are more vulnerable.

\textbf{Overstimulation:} \index{Overstimulation}When a character's sensory input is greater than their ability to handle that sensory input, it can be disorienting and paralyzing. Usually this comes from being exposed to really bright lights or loud noises, such as those produced by a flashbang grenade. But it can also come from within by having a character's senses sensitized (such as from atropine, ecstasy, or the bite of a Strigoi). In any case, when a character is Overstimulated their Initiative is reduced by 2 and their physical actions suffer a -2 dicepool penalty. The character also needs to make an \dicepool{Intuition + Perception} test with a threshold of 2 to even target specific things in the overstimulated sense. A character who is primed to be overstimulated (by a poison or by enhancing their own senses) is not penalized until they are actually confronted with brightish lights or equivalent stimuli.

\textbf{Paralysis:} \index{Paralysis}A fully paralyzed character cannot move. They may or may not be able to move their eyes or blink, depending upon which is more horrible, but they cannot move their arms or even turn their head. The victim's Agility is zero. Paralysis often goes away gradually, with a victim regaining their Agility one point at a time.

\textbf{Stimulation:} \index{Stimulation}The character is filled with vigor and energy. Any fatigue or tiredness they feel is postponed until the end of the effect (at which point they will also \textit{become} fatigued even if they weren't before). The character is incapable of resting during this period.

%%%%%%%%%%%%%%%%%%%%%%%%%%%%%%%%%%%%%%%%%%%%%%%%%%
\section{Wind and Water}
%%%%%%%%%%%%%%%%%%%%%%%%%%%%%%%%%%%%%%%%%%%%%%%%%%
\tagline{It was a dark and stormy night\ldots{}}

It is important to note that weather in the realm of horror is a fair bit worse than the weather of our own world. Just as nearly every horror movie begins with some driving rain and a crash of thunder to set the mood, the night sky of the stories you tell with After Sundown will pelt the earth with lightning and rain all the time. It's atmospheric, and it helps to enforce the feeling of isolation that so many variations of scary stories require.

\textbf{Rain:} \index{Rain}Water falls from the sky. It's dreary, it's uncomfortable, and it makes it hard to see things that are far away. It also makes things very wet, which is why witches wear wide brimmed hats. But the important things from the standpoint of \textit{game mechanics} is the fact that it makes things wet and makes it hard to see.

\textbf{Fog:} \index{Fog}When there's fog in the air it reduces contrast and obscures. It scatters light sources and can paradoxically cause people to be blinded by glare if they shine lights trying to see better. But mostly it saves us money on sets, because far away things become indistinct and altogether invisible. As fog becomes thicker, the distance one can see things clearly is reduced, and the distance one can see anything at all is reduced as well. Bright lights get scattered in fog and ca produce glare. It is entirely possible for a fog to have reduced visibility \textit{and} overstimulation at the same time. Dust and smoke clouds reduce visibility based on their thickness just as fog does, but they are usually composed of light absorbent particles and do not create glare when light is shed within them.

\textbf{Wind:} \index{Wind}Air blows around. It carries one's words away before they have had a chance to impart their intended convictions, ruining apologies and love confessions both. As in the real world, the relative strength of wind in After Sundown is represented on the Beaufort Windforce Scale. However, the bottom end of that scale doesn't make any difference (whether the "leaves are in motion" in a gentle breeze or "not" in an actual dead calm, characters can leave books on park benches without fear that they will fly open). As such, the minimum value of wind strength in the game is 3 -- even when the story is taking place inside and such. That's a little awkward, but it beats the alternative of not being able to use genuine meteorological data in the game.

\begin{table}[htb]
%\begin{wraptable}[9]{R}{13cm} \vspace{-.4cm}
\rowcolors{1}{white}{tan} \caption{Effects of Rain} \centering
\begin{tabular}{l l l}
\textbf{Type of Rain} & \textbf{Impeded Visibility} & \textbf{You get wet unless\ldots{}}\\
Fine Mist & \textbf{R}emote &\ldots{}you move around a bit.\\
Adequate Drizzle& \textbf{E}xtreme&\ldots{}you have a hat.\\
Continuous Shower & \textbf{W}ay Out&\ldots{}you have an umbrella.\\
Tumultuous Rain* & \textbf{S}hort&\ldots{}you are wearing rain gear.\\
Uncompromising Deluge* & \textbf{N}ear&\ldots{}you have a diving suit.\\
\end{tabular}\\
* -- The Sun is effectively blocked by the clouds.
\end{table}
\begin{table}[htb] \rowcolors{1}{white}{tan}
\caption{Effects of Fog} \centering
\begin{tabular}{l l l}
\textbf{Fog Thickness}&\textbf{Impeded Visibility}&\textbf{Limit of Vision}\\
Thin&200 meters&\textbf{R}emote (1200 meters)\\
Light&100 meters&\textbf{E}xtreme (600 meters)\\
Medium&20 meters&\textbf{E}xtreme (300 meters)\\
Thick*&10 meters&\textbf{W}ay Out (100 meters)\\
Pea Soup*&1 meter&2 meters\\
\end{tabular}\\
* -- The Sun is effectively blocked by the Fog.
\end{table}

\begin{table}[htb]\hspace{-.1cm}
\rowcolors{1}{white}{tan} \caption{Effects of Wind} \centering
\begin{tabular}{c l p{5cm} p{5cm}}
\textbf{Strength}&\textbf{It's Called\ldots{}}&\textbf{You see\ldots{}}&\textbf{Game Effects}\\
\textbf{3}&Gentle Breeze&\ldots{}leaves sway.&None.\\
\textbf{4}&Moderate Breeze&\ldots{}dust and loose paper kicked up.&Dusty areas gain thin fog. \\
\textbf{5}&Fresh Breeze&\ldots{}whole branches sway.&\\
\textbf{6}&Strong Breeze&\ldots{}an empty garbage can fall over.&Whistling wind and scattered falling objects obscure sounds.\\
\textbf{7}&High Wind&\ldots{}upper floors in tall buildings shift.&Walking against the wind is like moving in difficult ground.\\
\textbf{8}&Gale&\ldots{}campfires blown out.&Driving Conditions difficult. Movement on foot difficult. Thrown objects penalized.\\
\textbf{9}&Strong Gale&\ldots{}a tree blow down.&Staying upright out of cover is a Hard stunt. (\dicepool{Strength + Athletics or Survival}, Threshold 3)\\
\textbf{10}&Storm&\ldots{}roof tiles peel up and clatter.&\\
\textbf{11}*&Violent Storm&\ldots{}roof tiles fly off of buildings.&Characters can't make themselves heard if they try.\\
\textbf{12}*&Hurricane&\ldots{}some windows breaking.&\\
\textbf{13}*&Hurricane (2)&\ldots{}a mobile home rolled over and over.&Staying upright out of cover is a Crazy Extreme stunt. (\dicepool{Strength + Athletics or Survival}, Threshold 5)\\
\textbf{14}*&Hurricane (3)&\ldots{}a dog flying away &\\
\textbf{15}*&Hurricane (4)&\ldots{}a twig embed itself into a tree lengthwise.&\\
\textbf{16}&Hurricane (5)&\ldots{}Dorothy's house fly away.&Characters out of cover resist a Strength d6 of Normal Damage.\\
\end{tabular}\\
* -- Victory of Typhon creates wind at strength 11 with 2 hits, 12 at 3, 13 at 4, 14 at 5, and strength 15 with 6 hits.
\end{table}
