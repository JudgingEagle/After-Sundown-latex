%%%%%%%%%%%%%%%%%%%%%%%%%%%%%%%%%%%%%%%%%%%%%%%%%%
%%%%%%%%%%%%%%%%%%%%%%%%%%%%%%%%%%%%%%%%%%%%%%%%%%
\chapter{Magic}
%%%%%%%%%%%%%%%%%%%%%%%%%%%%%%%%%%%%%%%%%%%%%%%%%%
%%%%%%%%%%%%%%%%%%%%%%%%%%%%%%%%%%%%%%%%%%%%%%%%%%

%%%%%%%%%%%%%%%%%%%%%%%%%%%%%%%%%%%%%%%%%%%%%%%%%%
\section{The Limits of Magic}
%%%%%%%%%%%%%%%%%%%%%%%%%%%%%%%%%%%%%%%%%%%%%%%%%%
\tagline{"She made me into a newt!"}

Magic in a told story does not need to have \textit{explicit} limits defined for it because it has \textit{implicit} limits of whatever it is that magic happens to do in the story. One can assume that many of the things that magic never did in the narrative were actually outside of its capabilities for one reason or another and that it all worked out somehow. A novel or a movie does not have to tell you that an evil magic car \textit{can't} turn into a giant robot, the fact that it \textit{doesn't} is sufficient for the purposes of the medium. However in a \textit{role playing game} this is absolutely not the case. Since magic is going to be used in creative, goal oriented ways by multiple story contributors (which is a nice way of saying "people are going to push magic as far as it will go into unintended directions in an effort to gain personal advantage"), it is imperative that what magic is specifically capable (and by extension \textit{not} capable) of doing be codified.

Magic in After Sundown comes in several flavors. The first and most obvious kind is "Inherent Magic". This is the stuff that supernatural creatures can do just because of what they are. A golem doesn't need to  know anything special to be able to lift and throw a car, it just does it. The fact that it is a golem gives it the inherent magic to have supernaturally powerful strength. All supernatural creatures have some form of inherent magic and they get more of it as time goes on. The next kind of magic to discuss is "Sorcery". This is a type of magic that is explicitly learned and comes from elsewhere. 

Every supernatural type has some inherent magic associated with it. Even though a Transhuman may have been granted all of their power from a mystic ritual that looks suspiciously like Sorcery, once they have attained that status they are able to use many of their abilities without having to remember mystical formulas or speak arcane words. The Invisible Man can fade from view without "doing" anything, and it is this point which makes his signature Invisibility power an inherent rather than sorcerous one. These magical abilities come in distinct groups called Universal Powers. On the other hand, Dr. West is a Khaibit and his signature power to raise the dead requires mystical effort on his part, and that makes it a sorcerous rather than inherent power. 

Thematically speaking, sorcery is magical knowledge. Thus, much of it is newly researched and much of it can be discovered from ancient texts. Practically speaking, this doesn't really matter. There is plenty of powerful magic that has been lost and plenty more that has yet to be discovered. Sorcerous magic falls into one of three categories, regardless of whether it is new or old. These are Astral, Infernal, and Orphic. Supernatural creatures have the same power sources, and they are inherently resistant to  sorceries with the same power source as themselves (gaining a +3 bonus to any resistance test they are allowed).

Magic can be detected by various means specific to whether it is Astral, Infernal, or Orphic. And when creatures who are personally one of those types use Universal Powers they can be detected as if they were using magic of that type. So if a Frankenstein (an Orphic creature) uses Devastation to lift a car it is an Orphic Magical Action and can be detected by the browning of nearby leaves; while an Android (an Infernal creature) doing the same thing would be an Infernal Magical Action that could be detected by the clouding of clear water.

%%%%%%%%%%%%%%%%%%%%%%%%%%%%%%%%%%%%%%%%%%%%%%%%%%
\section{Detecting and Countering Magic}
%%%%%%%%%%%%%%%%%%%%%%%%%%%%%%%%%%%%%%%%%%%%%%%%%%
\tagline{"No way. No way nohow."}

Sorcery and witchcraft doesn't "just happen" it requires a great deal of evil knowledge and a fair amount of effort. And indeed that exertion can be detected and the efforts unraveled. Those in the know can detect and track the use of spells and sorcery by their effects on certain elements of the world. And indeed, you can also counter dark magic by carrying around bags of stuff and throwing handfuls of the contents at the source.

%%%%%%%%%%%%%%%%%%%%%%%%%
\subsection{Dowsing} \index{Dowsing}
%%%%%%%%%%%%%%%%%%%%%%%%%

\hspace{\parindent} Magic has definite effects on the physical world, and can be detected through careful observation of its effects. When a creature activates a discipline, characters with the right equipment and knowhow can spot the power surge with a Dowsing Test. A Dowsing Test can always be \dicepool{Intuition + Perception}, but dowsing for Astral Magic can also be \dicepool{Logic + Operations}, dowsing for Infernal Magic can also be \dicepool{Logic + Survival}, and dowsing for Orphic Magic can also be \dicepool{Charisma + Animal Ken}. In general, the threshold to notice magic at normal range is 2, and the threshold to give a good directionality to where the magic was used is 4. Mostly, Basic Powers are detectable to \textbf{W}ay Out Range, Advanced Powers or the expenditure of 3 or more power points are detectable to \textbf{E}xtreme Range, and Elder Powers or the expenditure of 5 power points are detectable out to \textbf{R}emote Range. The expenditure of more than 5 power points at once is detectable out to as many kilometers.

\begin{description}
\item[Clean Water] Clean water is pure and healthy and brings joy and solace. It is very much inimical to the magic of the Dark Reflection, which causes clear water to be come darkened momentarily as if it was tainted with soot. An experienced douser can track the strength and direction of the use of Infernal sorcery by the darkness, apparent direction, and persistence of the image of taint in otherwise clear water.

\item[Flowers] Orphic Sorcery is bad for you. Like polonium or something. But for big creatures it's really not something you'd notice without years of exposure, leaving the really observable effects to the very most fragile of lifeforms -- those who would soon die in any case. And while one \textit{could} tote around a bag full of mayflies or the like, most people in the know who want to track necromancers choose to use potted plants. A flower that blooms and dies every day is of course ideal, as it has a high responsiveness and gives good directionality.

\item[Magnets] Astral Sorcery has a noticeable, if weird, affect on magnetic fields. You can track Astral Sorcery and gauge its power with a lodestone. Natural magnetite reacts more strongly than an electro-magnet for whatever reason, so experienced geomancers seriously carry a black rock on a string when they want to find dream sorcerers.
\end{description}

%%%%%%%%%%%%%%%%%%%%%%%%%
\subsection{Dispelling} \index{Dispelling}
%%%%%%%%%%%%%%%%%%%%%%%%%

\hspace{\parindent} Throwing a handful of whatever appropriate powder at a source of magic is a Simple Action, and involves making an \dicepool{Intuition + Rigging} or \dicepool{Agility + Athletics} test, reducing the effective hits of the original casting by the hits gained. Once used to counter magic the powder is completely spoiled, you can't sweep it up and use it again. While this is technically sorcery itself, it can be taught to any Luminary or supernatural creature in moments -- you literally just reach into the bag throw a handful of the stuff at the magic with intent to nullify and it generally works pretty well. Any power that is dispelled completely is suppressed, and cannot be reactivated for the rest of the scene.

\begin{description}
\item[Salt] Salt draws water into itself and preserves food. Things treated with salt remain clean and non-poisonous long after other objects blacken and stink with putrescence. Thrown at Astral magic it draws the wetness and poison of the sorcery into itself, dampening it.

\item[Sand] The fires of Limbo burn ceaselessly, but they \textit{are} still fires. Clean sand thrown upon them douses them -- cutting off the source of wicked Infernal sorcery.

\item[Seeds] Representing the promise of new life and the growth of great strength from humble origins, the seed is a key ritual component in practically every mortal magical tradition ever devised. And indeed, throwing it on Orphic magic results in the nullification of both.
\end{description}

%%%%%%%%%%%%%%%%%%%%%%%%%%%%%%%%%%%%%%%%%%%%%%%%%%
\section{Magical Groups}
%%%%%%%%%%%%%%%%%%%%%%%%%%%%%%%%%%%%%%%%%%%%%%%%%%

\hspace{\parindent} Magical powers are sorted into groups, and each group is either Universal and Sorcerous. Universal powers are mystically expanded abilities (which can be mental, physical, or social), while Sorcerous powers represent magical knowledge that allows the character to do new things with evil magic. The distinction is pretty subtle, but basically Sorceries can be learned, taught, found in a book, and detected, while a Universal ability "just happens". Sorcerous groups are divided into Astral, Infernal, and Orphic Sorceries. They are in some manner connected to one or another otherworld, and their use can be detected and countered based on that connection.

The magical groups are fairly broad and often times represent many different paths to power that are associated game mechanically because that's playable. Nevertheless, to represent the different approaches that magic responds to; every group with a dicepool to activate presents two different Skills that they can be activated with.

\paragraph{Power Levels:} There are three levels of power within each group: \textit{Basic}, \textit{Advanced}, and \textit{Elder}. A character must have at least one Basic power in a group before they can learn an Advanced power of that group. A character must have at least one Advanced power in a group before they can learn an Elder power of that group. Abilities within a group are otherwise unordered. A character can learn any Basic power and then learn another Basic power, or skip to learn an Advanced power. Standard characters begin play with only Basic and Advanced powers. A Devotion is a special Advanced ability that belongs to two different group and requires that a character have at least one Basic power from both in order to be learned.

\paragraph{Protean Powers:} Some powers are Protean abilities: ones that change the character's form. A character who knows any Protean secrets can return to their real form with a Complex Action and a power point even if they have been transformed by hostile magic. Furthermore, if a character has multiple Protean abilities, they may activate more than one simultaneously, and the entire transformation takes the action of the longest transformation. So if the character has two different abilities that both take a Complex Action, they may activate both as a single Complex Action.

%%%%%%%%%%%%%%%%%%%%%%%%%%%%%%%%%%%%%%%%%%%%%%%%%%
\section{Universal Powers}
%%%%%%%%%%%%%%%%%%%%%%%%%%%%%%%%%%%%%%%%%%%%%%%%%%
\tagline{"This is what we are, and this is what you are."}

There are many things that magic can do in horror, and one of the most salient things for After Sundown is that it allows supernatural creatures to do amazing things. Universal powers are those that owe no special allegiance to any world or cult. They require no magic words or special gestures. They are not "spells" in any meaningful sense and one cannot muster a counterspell against them. Most importantly of all, they can be spontaneously developed. A character who has the requisite card can simply acquire a Universal Power without any access to special training or magical research books.

%%%%%%%%%%%%%%%%%%%%%%%%%
\subsection{Authority}
%%%%%%%%%%%%%%%%%%%%%%%%%
\tagline{"Look me in the eyes and tell me that again."}

Authority is the power to impress one's Will directly upon another. Authority requires eye contact to function, although only cursorily. A character who is closing their eyes or wearing mirror shades cannot be controlled with Authority (unless the user has \linkpower{Will to Power}). A character who is actively attempting to avoid seeing a potential dominator's eyes may be able to do so, depending upon the actions in question. Activating Authority is generally an opposed test with the target defending with Willpower. When a character is compelled to do something which is against their ideology (generally including following suicidal orders), they may spend an action dithering. They at this point lose an entire round of actions as they have their internal struggle, making an additional resistance check to attempt to shake off the effects. If they again fail to resist the effects, they carry out the order even if it is something they are ideologically opposed to. Most uses of Authority require instructions to be given to the victim in a manner that they can comprehend. Usually this requires verbal orders to be given in a language they understand; but sign language, written directions, and even silent telepathic commands can suffice. Influences from a character with Basic Authority can be disrupted as if it were a Sorcery with a power source identical to the character using it. The demands of a character with Advanced Authority require killing the user or \linkpower{Purify the Mind} to undo, and a character with Elder Authority has their powers persist even after they are dead.

%%%
\subsubsection{Basic Powers}
%%%

\powerentry{Command} The character gives an instantaneous verbal command, which the victim will follow to the best of their ability. The action itself must be essentially instantaneous, using up no more than a single Complex Action. If it is of an open ended nature such as "lie down", then the victim will spend one Complex Action doing it, and then they are free to do as they please and wonder why they did that. Using Command is a Complex Action, and requires a successful opposed test using either \dicepool{Willpower + Intimidation} or \dicepool{Logic + Persuasion} against Willpower test.

\powerentry{Mesmerism} The character can hypnotize a victim, allowing them to give extended commands that will be obeyed. Commands must be stressed and repeated, with each major instruction taking approximately a minute to convey. During the period of hypnotism, the victim stands there like an obviously hypnotized drone, but behaves seemingly normally when they go off to carry out their instructions. Any sudden sensory stimulation (such as a nearby gunshot or someone shaking the victim) during the Mesmerism breaks the spell. Using Mesmerism is an opposed, extended test pitting the character's \dicepool{Logic + Persuasion} or \dicepool{Willpower + Intimidation} against the target's Willpower. The expected time to bring someone under Mesmerism so that instructions can begin is one minute.

\powerentry{Suggestion} The character can offer a suggestion that the victim will then immediately themselves suggest as if it were their own idea. Whether this idea is followed up upon or discarded as "a bad idea" depends entirely upon how they feel about the idea once they've said it to themselves. The immediate victim does not remember being fed the suggestion in the first place, but the scene may look odd to onlookers. Using a Suggestion is a Simple Action and requires an opposed \dicepool{Logic + Tactics} or \dicepool{Willpower + Expression} vs. Willpower test.

%%%
\subsubsection{Advanced Powers}
%%%

\powerentry{Cloud Memory} Cloud Memory allows the character to erase or alter the memories of a victim. A single use of Cloud Memory can alter about 5 minutes of memory, but net hits increase the time frame. Using Cloud Memory is a Complex Action, and requires a successful opposed \dicepool{Willpower + Expression} or \dicepool{Logic + Artisan} against Willpower test. A Luminary who is presented with proof that their memories are incorrect may spend an Edge to remember. Extras just have to resolve those discrepancies somehow.

\powerentry{Conditioning} Using extended mind controlling techniques, a victim's will is broken and they are transformed into a servant of the character's. Breaking someone's mind in this manner is an extended task that has an expected time frame of 1 day per point of the victim's Willpower, with a dicepool of \dicepool{Willpower + Intimidation} or \dicepool{Logic + Tactics}. Conditioning is Hard and has a threshold of 3. Conditioning supernatural creatures with higher Potencies than the conditioner is even more difficult, and the difference is added to the threshold in that case. Once a victim has been thralled, they follow all orders their new master gives them as if they had been imparted with the force of a successful Mesmerism. At the time of conditioning, the character may choose to alter the ideologies and demeanor of their victim (this is often done to make the giving of orders that would be against the original nature easier to do). Using Conditioning requires an Edge, and it fails if the process is interrupted long enough that it is not continued between when the sun rises or sets and the next time it sets or rises.

%%%
\subsubsection{Elder Powers}
%%%

\powerentry{Possession} Using Possession, the character can transfer their mind and spirit into the body of another. While the character is possessing a victim, their original body is inert and vulnerable, but they have full control of the other's body. They use the victim's Strength and Agility, but their own Social, Mental, and Special attributes as well as their own skills. They may use their own powers in the body of another, and cannot activate any of the victim's powers (though already activated powers will continue to operate and run out their duration normally). Activating Possession is a Complex Action, and requires a successful opposed \dicepool{Willpower + Empathy} or \dicepool{Logic + Persuasion} against Willpower test. The character returns to their own body as soon as the sun rises or sets, the body they are in is knocked unconscious, their Possession power is suppressed by any means, or they spend a Complex Action to return to their own body. The character can only actually be in one body at a time.

\powerentry{Mob Mind} A character with Mob Mind may affect a number of characters with their Authority powers simultaneously. The character's dicepool is reduced by 1 for every doubling of the number of victims. Note that while the character can attack more than target, there's still only one of them, so if they try to possess multiple targets they can still only end up in one body. The character does not need to maintain eye contact with every target, but must make eye contact at least transiently with every target during the 12 second round that the power is activated in (unless they have \linkpower{Will to Power}).


%%%%%%%%%%%%%%%%%%%%%%%%%
\subsection{Celerity}
%%%%%%%%%%%%%%%%%%%%%%%%%
\tagline{There she goes again.}

Celerity is the power to move with the astonishing speed of the supernatural. Celerity must be consciously activated at a cost of three power points, and its effects last for one scene. Distinct from other Powers, the effects of Celerity are \textit{cumulative}. The discipline as a whole is activated or not during a scene, and may be activated as a Reflexive Action while rolling initiative. When activated, the character gains a +2 bonus to Initiative Tests. With Advanced Celerity this bonus increases to +4, and with Elder Celerity this bonus increases to +8. During the initiative phase, a character with Celerity may choose to reduce the speed that they will move at to a more normal velocity. The degree of speed utilized cannot be changed again until next initiative phase. In general, a character whose Celerity more than doubles their speed in any dimension would constitute a breach of the Vow of Silence.

%%%
\subsubsection{Basic Powers}
%%%

\powerentry{Quickness} While Celerity is active, the character gains an extra initiative pass during any confrontation turn.

\powerentry{Nimble Feet} While Celerity is active, the character is able to walk and run at stupendous speeds. The character's personal movement rate is quadrupled. If the character has Advanced Celerity, the speed increase is itself increased to six times. If the character has Elder Celerity the movement increase is eight times. In addition, the character ignores penalties for acting while moving over difficult surfaces (although they may be slowed down by them as normal).


%%%
\subsubsection{Advanced Powers}
%%%

\powerentry{Alacrity} While Celerity is active, the character gains an extra initiative pass during any confrontation turn. This is cumulative with Quickness (for a total of 2 extra IPs if both are known).

\powerentry{Quicken Sight} While Celerity is active, the character may perceive and derive meaning from fast moving objects. They may follow a specific card in a shuffled deck, read a sign on a fast moving train, or gauge the trajectories of bullets in flight. They gain a +4 bonus to Dodge Tests, and negates any bonuses opponents would glean from multiple attacks during an Initiative Pass. If they have Elder Celerity, the bonus increases to 6 dice.


%%%
\subsubsection{Elder Powers}
%%%

\powerentry{Blur} While Celerity is active, the character gains an extra initiative pass during any confrontation turn. This is cumulative with Alacrity and Quickness (for a total of 3 extra IPs if all are known).

\powerentry{Rapid Thought} While Celerity is active, the character is able to consider their situation and their surroundings carefully as if they had little or no time pressures. All penalties for splitting one's attentions between two or more activities (such as wielding two pistols or picking a lock while dangling from a rope) are canceled. The character ignores all penalties for a "rushed job". And finally, the character always wins initiative against enemies, and must roll only if another character also has Rapid Thought.


%%%%%%%%%%%%%%%%%%%%%%%%%
\subsection{Clout}
%%%%%%%%%%%%%%%%%%%%%%%%%
\tagline{I have the power.}

Clout is the power to draw upon the great strength of the supernatural in order to perform feats of literal strength. Those who can draw upon their magical power to augment their physical prowess in this manner find themselves becoming stronger even without drawing directly upon their magical might. A character with Clout gains a +1 bonus to their Strength score. If they have Advanced Clout the bonus increases to +2, and if they have Elder Clout the bonus increases to +3. Clout is an odd case as regards the Vow of Silence because while it is accompanied by no flaming runes or crawling shadows, demonstrations of incredible strength strain credibility and perhaps worse they \textit{draw comments} from observers.

%%%
\subsubsection{Basic Powers}
%%%

\powerentry{Clinging} The character can hold themselves onto vertical surfaces and even ceilings for extended periods of time without special equipment or apparent effort. They can easily move up or down sheer surfaces at the rate of a steady walk, and scuttle across ceilings at the rate of a careful walk without danger or exertion. When such a character attempts to pull Wuxia or Spiderman inspired stunts involving running on walls or brachiating over a crowd they find this much easier, because they can support their weight merely by putting a hand against a wall. For practical purposes, this means that such stunts are much less extreme \textit{for them} than they appear to outsiders. The thresholds are generally reduced by 2 (a crazy extreme stunt would be merely a professional stunt from their perspective).

\powerentry{Vigor} The character may spend a power point as a Free Action to increase their Strength by 1 for the remainder of the scene. This may be activated more than once, and its effects stack. While using Vigor itself has no visible effects, the \textit{use} of vastly increased strength may be obviously unnatural to onlookers. The maximum number of power points a character can spend in a scene on Vigor is three plus their Potency.

%%%
\subsubsection{Advanced Powers}
%%%

\powerentry{Devastation} The character's Strength can emanate from many places at once along an object being touched, allowing them to lift very large, awkward, and even fragile things without issue. It also allows them to have a blow from their hand or foot repeated many times across a wall or floor, causing it to shatter (which is where the power gets its name). This kind of contact telekinesis is best represented in super hero comics -- where characters routinely lift cars by the ends without parts falling off or the ground underneath them giving way. The source of force can be directed out across a distance of 2 meters per Potency, and the character can effectively lift or smash something that they could accomplish with 5 identical friends. With Elder Clout, the number of identical friends increases to 9.

When used as an attack, the character can strike multiple enemies in close combat without penalty or attack a single opponent as if they outnumbered them. A character with Devastation can use any weapon they can use at all in one hand with no penalty.

\powerentry{Giant Size} The character can grow extremely large. By spending four power points and a Complex Action, they can expand to a muscular 3.5 meters in height. This is a Protean power. While in Giant Size, the character has an additional 6 points of Strength, they gain a point of armor, and the base damage of any weapon they use increases by 2 (assuming that it is allowed to grow with them). All of the things that a character wants to grow along with them grow along with them while they are being carried by them, and anything they aren't carrying or that they wish to leave normal size stays normal size.
Some creatures have Giant Size permanently on, and they don't have to pay power points for it (but they can't turn it off and objects do not resize in their hands).

%%%
\subsubsection{Elder Powers}
%%%

\powerentry{Earthquake} By spending three power points, the character can do something stupidly powerful (and generally destructive) with their Strength. A blow reverberates across the ground like a meteor strike, shaking and crushing things out to up to a hundred meters per Potency from their person.  A creature or object struck with the full force of ground zero of this strike is likely obliterated -- the melee attack at the center of this is a formidable (even ludicrous) Damage 12. The character can attempt to restrict the power into doing something useful such as boring out a tunnel, stacking logs, or hurling debris out of a collapsed building. This kind of Popeye-like activity uses an \dicepool{Agility + Athletics} or \dicepool{Logic + Rigging} test to determine its accuracy.

\powerentry{Force Field} The character can project force some distance away from their person. This allows them to stop bullets aimed at compatriots, strangle people from a distance, and even hover by "holding themselves up". Their Strength can be projected reflexively out to 3 meters from their person, and any attacks that target someone or something within or through that area may get blocked by the force field -- meaning that they have to contend with being soaked by the character's Strength before being resolved (yes, this means that the character can effectively use their Strength \textit{twice} when soaking bullets fired at their own person). Their Strength can be used actively as a normal action on things within line of sight. Force Field can be disrupted as if it were a Sorcery with a power source identical to the character using it. The Force Field has an equivalent of 3 hits for this purpose.



%%%%%%%%%%%%%%%%%%%%%%%%%
\subsection{Discernment}
%%%%%%%%%%%%%%%%%%%%%%%%%
\tagline{"You see what you want to see. You hear what you want to hear."}

Discernment is the power to have unusual and enhanced senses. A character with Discernment can perceive what others cannot. A character with Discernment gains a +2 bonus on Perception tests. This bonus increases to +4 if they have Advanced Discernment, and +6 if they have Elder Discernment.

%%%
\subsubsection{Basic Powers}
%%%

\powerentry{Supernatural Senses} A character with supernatural senses can enhance their perceptions well beyond human norms. Vision can become telescopic or even function in total darkness; hearing can become acute enough to hear heart beats; smell can become powerful enough to track at a brisk walk; and so on. Bringing one's senses into the realm of magical badassery also makes one vulnerable to intense sensory input. Bright lights can blind, strong smells can overpower, and loud noises can deafen. While a sense is enhanced, the threshold to resist overstimulation is increased by 2. Activating or deactivating Supernatural Senses is a Free Action.

\powerentry{Aura Perception} A character using Aura perception can bleed their perceptions into other worlds and see magical forces and beings in other worlds. Activating or deactivating Aura Perception is a Simple Action. While active, a character can perceive creatures in the Shallows of one otherworld (chosen when the power is activated) if they are currently in the Mortal World. If they are in the Deeps or Shallows of another world, activating Aura Perception will allow the character to detect creatures in the other version of that same world. A character with active Aura Perception can also see clear halos around active Sorceries, supernatural creatures, magic items, and Luminaries (this property may even be why Luminaries are called that). A character can attempt to identify the type of a supernatural creature, magic item, or sorcery with an \dicepool{Intuition + Empathy} or \dicepool{Logic + Research} test, with a threshold of 5 \textit{minus} the Potency of the target (or the creator of the target). \linkpower{Fictional Self} changes what is seen with Aura Perception.

\powerentry{Sensory Damper} A character with sensory damper is protected from harmful stimuli such as glare, pepper spray, and loud noises. They are able to dial down their sensations to the point of unobtrusiveness. This can be effectively instantaneous in the case of bright flashes, and highly selective in the case of filtering out noise while listening to a conversation across the room. Such a character can stay conscious regardless of wound level (ignoring pain sufficient to knock them out), but it does not affect wound penalties at all.


%%%
\subsubsection{Advanced Powers}
%%%

\powerentry{Psychometry} Psychometry allows a character to view past events by touching and concentrating upon items or people that were involved in those events. The event in question must be described (although that description may well be entirely conversational such as "What happened here?" -- in the case of a brutal crime scene, the context makes that description sufficient), and must have taken place within the last month. The dicepool for Psychometry is \dicepool{Logic + Research} or \dicepool{Intuition + Empathy}. Using Psychometry requires one power point, and the expected amount of time is one minute. Seeing farther back into time can be done by increasing the expected amount of time by one step and increasing the cost by 1 power point for each time increment the viewing window is increased.

\powerentry{Telepathy} A character with Telepathy can communicate mentally with people within line of sight. Voluntary telepathic messages can also be sent and received between people the character knows the name of and who they have touched (regardless of line of sight) so long as they are within a number of kilometers equal to the character's Potency. The contents of an unwilling mind can be read, but only with physical contact and difficulty. Telepathy is always on, but mind reading is a Resisted Extended Action. Mind reading uses either \dicepool{Willpower + Intimidation} or \dicepool{Intuition + Empathy} versus Willpower. If the target understands telepathy and takes effort to obscure details and engages in counter thought, they can use \dicepool{Willpower + Intimidation} or \dicepool{Intuition + Background} (some appropriate background) for resistance. Telepathy is covert except to someone receiving a telepathic communication.

\begin{table}[htb]
\rowcolors{1}{white}{tan} \caption{Telepathy Results and Range} \centering
\begin{tabular}{l l}
\textbf{Net Hits} & \textbf{Thought is\ldots{}}\\
\textbf{1} & On the surface\\
\textbf{2} & Important\\
\textbf{3} & Unimportant to the target\\
\textbf{5} & Something the target does not know they know.\\
\textbf{Base Time} & \textbf{Thought is\ldots{}}\\
1 minute & A simple fact.\\
5 minutes & A mental image.\\
20 minutes & A short narrative.\\
1 hour & An involved explanation.\\
5 hours & The whole story.\\
1 day & Their life story.\\
\end{tabular}
\end{table}

%%%
\subsubsection{Elder Powers}
%%%

\powerentry{Divination} Divination allows a character to ask questions about reality and the future and get actual (if often vague and cryptic) answers. Using Divination takes an hour and costs one Edge.  The Dicepool is either \dicepool{Logic + Research} or \dicepool{Charisma + Bureaucracy}. Divination can be disrupted as if it were a Sorcery with a power source identical to the character using it.

\powerentry{Dimensional Translocation} Activating Dimensional Translocation sends a character and everything they are carrying into the Shallows of another world of their choice (Limbo, Maya, or Mictlan). The character can also return to the material world using this ability, but only from the point they last entered the other world. Crossing either way requires a Complex Action and a power point. Dimensional Translocation can be disrupted as if it were a Sorcery with a power source identical to the character using it (though this requires a Lock On). It requires 3 hits to counter.


%%%%%%%%%%%%%%%%%%%%%%%%%
\subsection{Fortitude}
%%%%%%%%%%%%%%%%%%%%%%%%%
\tagline{Endure. And in enduring, grow strong.}

Fortitude is the power to resist destruction. Some powers of Fortitude are continuously in operation, while others must be activated with power points. What makes Fortitude special is that it does not require the \textit{active} choice to activate it. A character may activate their Fortitude powers passively while unconscious, or in some cases even while \textit{dead}. A character with Fortitude is generally resilient even when their powers are not being activated, gaining a bonus on Physical Resistance Tests of +2 dice. If they have Advanced Fortitude this bonus increases to +4, and if they have Elder Fortitude it increases to +6.

%%%
\subsubsection{Basic Powers}
%%%

\powerentry{Patience of the Mountains} The character does not need to eat, drink, or breathe. They persist night after night as the mountains and valleys do. Never aging or decaying. Such a character can hold any position no matter how awkward without cramping or moving. They are immune to poison. This ability is continuous. The character is still able to learn and change, and indeed Leviathan classically become substantially less human as time moves on.

\powerentry{Revive the Flesh} The character can heal their wounds by drawing upon their magical power. By spending a power point, the character's wounds suture themselves, restoring their body to its original condition without mark or scar. Each power point heals one box of Lethal or two boxes of Normal damage. Wounds healed in this manner are gone in one round. Any wounds short of death can be healed in this manner. Aggravated damage is harder to heal, and takes two power points and an hour per box. You have to pay for the most severe mark type in each box, and then the box is cleared of all marks (moving other damage to the left, if it matters). Declaring the use of this power is a Free action, and then the actual healing of each box takes the listed time.


%%%
\subsubsection{Advanced Powers}
%%%

\powerentry{Restoration} Death is no longer an insurmountable obstacle. While the character is dead, they may spend 2 power points plus an additional power point per point of Potency they possess and four hours to reduce the amount of damage on themselves to one less than Lethal.  Restoration cannot be performed when the character's body has been impaled with something that inflicts aggravated damage or there is gross separation of the head and heart until it is removed and someone else invests power into the reassembled corpse. If the character stays dead for a whole month, any remaining power points they have are lost and they do not gain power points with their normal power schedule (even if that schedule is continuous). If the character lacks the power points to use this discipline, they may yet get a  chance if the vast majority of their body is placed together and invested with sufficient Power by others (for example: with \linkpower{Gift of Health}). Restoration can be disrupted as if it were a Sorcery with a power source identical to the character using it. For purposes of being countered, Restoration has the equivalent of 3 hits.

\powerentry{Indominability} Wounds do not hamper the character. The character suffers no wound penalties and does not go unconscious from injury before they die. The character is also capable of straining themselves severely without noticeable effect: they can carry home anything they can lift. This discipline is passive and usually covert.


%%%
\subsubsection{Elder Powers}
%%%

\powerentry{Endless Persistence} By spending an Edge, the character becomes literally invulnerable for a brief period of time. For one round per Potency, the character ignores all damage, whether aggravated or not, even from a creature's supernatural weakness. This discipline can be activated reactively when damage would be sustained.

\powerentry{Skin of Night} This passive discipline converts all aggravated damage to lethal damage and damage that would normally be lethal to normal damage.



%%%%%%%%%%%%%%%%%%%%%%%%%
\subsection{Magnetism}
%%%%%%%%%%%%%%%%%%%%%%%%%
\tagline{Alright everyone! Let's hear it\ldots{} for me!}

Magnetism is the power to affect others with the otherworldly charisma of the supernatural, either to attract or repel. Characters with Magnetism are especially adept at making an impression and getting people to like or fear them, and gain a +1 bonus on all Socialization tests based on Charisma or Willpower. At Advanced, this bonus increases to +2, and at Elder it increases to +3. A character using Magnetism makes a large impression, and anyone asking about them later will get a similar bonus to any Socialization tests to find out information about them.

%%%
\subsubsection{Basic Powers}
%%%

\powerentry{Attract} The character can "turn on the charm" and grab someone's undivided attention. The character makes a \dicepool{Willpower + Expression} or \dicepool{Charisma + Tactics} test, and the number of hits determine how many peoples' attention can be so grabbed. Activating Attract takes a Complex Action and lasts for an entire scene. Attract does not guarantee that someone will agree with you, nor does it guarantee that they will sleep with you, but it does guarantee that the target will not ignore you or discount the importance of what you say. Attracted onlookers do not actively resist the next Adjacent attack if it comes from anyone else (meaning the threshold to hit them is zero).

\powerentry{Repel} The character can become extremely frightening and intimidating, inspiring fear and shame in onlookers. Using the Repel is a Simple Action and requires a \dicepool{Willpower + Intimidation} or \dicepool{Strength + Tactics} check opposed by the victim's Willpower or Strength. An affected victim runs away or cowers in terror for at least a number of rounds equal to the net hits. Thereafter, an affected victim is shaken up for the remainder of the scene and suffers a -2 die morale penalty on actions. Repel can be disrupted as if it were a Sorcery with a power source identical to the character using it.

%%%
\subsubsection{Advanced Powers}
%%%

\powerentry{Dismissal} The character can wrap themselves in the air of unapproachability, making aggression against them or even refusal of their demands almost unthinkable. By spending three power points, the character's dismissive demeanor takes hold until the end of the scene. The character makes a \dicepool{Charisma + Tactics} or \dicepool{Willpower + Intimidation} check, and anyone who wishes to summon the nerve to act against them must generate an equal number of hits on a \dicepool{Willpower + Intimidation} or \dicepool{Willpower + Survival} test. Failure to do so results in a round lost to dithering. The character's orders are also extremely likely to be obeyed (especially if they involve moving away from the issuer), and the hits are added as a bonus dicepool on any Intimidation or Tactics tests to command or demand.

\powerentry{Summons} The character can send a brief telepathic message (no longer than a twitter post) to someone whose name they know so long as that person is in the same world and no more than 10 kilometers away per point of the character's Potency. The target can then send back a brief reply. If the character so chooses, they may also demand the presence of the target by making an opposed \dicepool{Charisma + Bureaucracy} or \dicepool{Charisma + Empathy} vs. the target's \dicepool{Logic}. If successful, the target becomes aware of where the character basically is, and must attempt to figure out how to get there themselves. This compulsion lasts until the next time the sun rises or sets. Issuing a Summons (whether or not the compulsion for a personal appearance is added) costs two power points and requires a Complex Action. A Summons can be disrupted as if it were a Sorcery with a power source identical to the character using it. The disruption can be leveled at the summoning character \textit{or} the target.

%%%
\subsubsection{Elder Powers}
%%%

\powerentry{Depolarize} The character can reduce those who hear their words to frothing lunacy. The character spends four power points and begins speaking. With an expected time of 10 minutes and a Threshold equal to each potential target's Willpower, a listening victim becomes a raving fanatic, their Willpower reduced to zero until the sun next rises or sets. Depolarize carries as far as the character's voice does, even over telephones or television broadcasts. Depolarize uses \dicepool{Charisma + Persuasion} or \dicepool{Charisma + Bureaucracy}. Additional hits reduce the required timeframe to format minds.

\powerentry{Siren Song} The weak willed are drawn to the character like moths to flame. By spending seven power points, the character can let out a song that instills a compulsion in everyone within a radius up to one kilometer per Potency to come to where the character is. Dangers are ignored, and tasks previously engaged in are abandoned. The character makes a \dicepool{Willpower + Expression} or \dicepool{Charisma + Persuasion} test, and the threshold to affect any target is its Willpower. The Siren Song can be disrupted as if it were a Sorcery with a power source identical to the character using it, but a handful of salt (or whatever) frees just one victim.


%%%%%%%%%%%%%%%%%%%%%%%%%
\subsection{Veil}
%%%%%%%%%%%%%%%%%%%%%%%%%
\tagline{\ldots{}Now you don't.}

Veil is the ability to draw upon one's magical nature to hide things from view. Veil does not affect cameras or other objective traces of a creature's passing, merely prevents observers (even indirect observers) from \textit{noticing} what is there. Normally, Veil can only take effect while the target is not being observed. If observation is continuous, the observers will not see any change. Onlookers do not normally have any say in what they see as presented to them by Veil any more than they have any choice to not see things that are actually there. Characters who have Discernment powers active or who carefully search the area that the character is in have a chance to perceive through it by making an \dicepool{Intuition + Perception} test against the number of hits made to activate the power. Anything a character covered by Veil carries is likewise covered by Veil and anything that the character stops carrying will cease being covered by Veil. An onlooker who notices an object pass into or out of Veil pierces the Veil altogether. Veil is inherently multisensory, onlookers are just as fooled if they close their eyes and listen or try to smell the character as they are relying entirely on their eyes.

%%%
\subsubsection{Basic Powers}
%%%

\powerentry{Hide From Notice} While active, the character is not noticed so long as they don't do anything incredibly obvious to give themselves away. Activating Hide From Notice is a Simple Action and requires an \dicepool{Agility + Stealth} or \dicepool{Intuition + Survival} test. Hide From Notice can be disrupted as if it were a Sorcery with a power source identical to the character using it.

\powerentry{Mask of a Thousand Faces} While active, those who meet the character will treat them as if they were a different person. The character may choose the appearance (including clothing and carried items) freely, but taking any action that would be impossible for the facade allows onlookers to see through the illusion. For example, if a character uses the Mask to appear as a person who had no gun and then \textit{fires} their gun, people would see them as they really are. Activating Mask of a Thousand Faces is a Simple Action and requires an \dicepool{Agility + Stealth} or \dicepool{Charisma + Larceny} test. Mask of a Thousand Faces can be disrupted as if it were a Sorcery with a power source identical to the character using it.


%%%
\subsubsection{Advanced Powers}
%%%

\powerentry{Lost and Found} By spending a power point, a carried object can continue being covered by Veil after it leaves physical contact with the character. The character makes an \dicepool{Agility + Stealth} or \dicepool{Intuition + Larceny} test and the Veil remains affecting the object for an hour (time frame increases with additional hits). Lost and Found can be disrupted as if it were a Sorcery with a power source identical to the character using it.

\powerentry{Hide in Plain Sight} By spending a power point, the character may activate other powers of Veil while being observed. All onlookers are entitled to a resistance check as if they had been carefully searching the area, but if they fail to notice the discrepancy, their mind will fill in vague details that excuse the character's disappearance.


%%%
\subsubsection{Elder Powers}
%%%

\powerentry{Host the Masquerade} A character with Host the Masquerade can allow their Veil powers to be used by others, so long as they stay within 20 meters per point of Potency of the character. Characters so veiled and veiling can still perceive each other in the same way that a self-veiled character can see themselves.

\powerentry{Fictional Self} A character with Fictional Self can fool even magical detection. Making a successful \dicepool{Charisma + Stealth} or \dicepool{Intuition + Medicine} test,  Fictional Self character can appear to feel, think, or be whatever they want others to perceive under mechanical or magical investigation (such as a lie detector or aura reading) unless the investigator gets more hits activating their power or device. Changing one's Fictional Self is a Complex Action that costs a power point, but continuing to have fake thoughts and emotions pass that are consistent with one's false existence requires no action at all.


%%%%%%%%%%%%%%%%%%%%%%%%%%%%%%%%%%%%%%%%%%%%%%%%%%
\section{Sorceries}
%%%%%%%%%%%%%%%%%%%%%%%%%%%%%%%%%%%%%%%%%%%%%%%%%%
\hspace{\parindent} Sorcerous disciplines are different from physical Powers in many ways. While they still retain the essential framework of Basic, Advanced, and Elder versions, Sorcery doesn't "just happen". It can be potentially \textit{countered} because it takes the form of actual manipulations of magical energies associated with one of the three mystical worlds being used and directed. Sorcerous disciplines work poorly when used against creatures with the same power source as their power source world (defender gets +3 on their Resistance Test, if any), and they have means by which they can be held at bay depending on the type of magic that they are.

%%%%%%%%%%%%%%%%%%%%%%%%%%%%%%%%%%%%%%%%%%%%%%%%%%
\section{Astral Sorceries}
%%%%%%%%%%%%%%%%%%%%%%%%%%%%%%%%%%%%%%%%%%%%%%%%%%

\hspace{\parindent} The mercurial rain of the Deep Maya has an origin that few can even speculate upon. But whether The Dreamlands represent an intrusion of our subconscious minds into the realm of the physical or the intrusion of an alien realm into our sleeping thoughts, the fact remains that it is a source of power that those who delve deeply into its mysteries can tap. Astral Sorceries are spells and powers that tap into that strange reality. Magic from Maya has a tendency to be as untamed and inhuman as the implacable beasts and plants that inhabit it.

Astral Sorcery disturbs magnetic fields, including that of the Earth itself (at least on a local level), and a skilled augur can track and judge the strength of the magics of the dreamlands by carefully observing a lodestone.  Astral magic "feels wet" to those who feel its wrath, and indeed it can be countered by judiciously throwing clean salt on it. Salt used in this manner becomes caked and discolored like it had absorbed dirty water.

%%%%%%%%%%%%%%%%%%%%%%%%%
\subsection{Call of the Wild}
%%%%%%%%%%%%%%%%%%%%%%%%%
\tagline{"Be what you want to be until you don't any more."}

Call of the Wild is the sorcerous path of attuning one's self to wild animals, hence the name. This attunement provides great benefits when dealing with animals even in a nonmagical manner. The character gains a +2 bonus on Animal Ken. This bonus increases to +4 with Advanced, and +6 with Elder Call of the Wild. These bonuses are increased by 50\% when dealing with creatures that the character can personally transform into with Beast Form (so for example a character with Beast Form (Bat) would gain a +3, +6, or +9 bonus on Animal Ken checks to train bats).

%%%
\subsubsection{Basic Powers}
%%%

\powerentry{Beast Form} The character can transform into an animal by spending a power point and taking a Complex Action.  The type of animal transformed into is chosen when the ability is learned. When Beast Form is gained as a fixed ability from a character's supernatural type, the form of the beast is often predefined (for example, a Nezumi becomes a rat). This ability can be learned multiple times, and each time the character can choose one more new form than the time before (two new forms with the second learning, for a total of 3). No beast form can be much larger than a human, or smaller than a mouse. Some animals are stronger or more agile than a human, but the total bonus to these attributes never exceeds +2. Many animals are \textit{much} weaker than a human. For example: a Rat Form always has a strength reduced to 1 regardless of the original character's Strength score, and gains a +2 bonus to Agility. This ability is a Protean power. For purposes of being countered, Beast Form has 3 hits.

\powerentry{Tongue of Beasts} The character can speak to and understand the speech of beasts. This is invaluable in training and information gathering. But remember that animals do not gain any special understanding of their environment even though they are able to convey their thoughts to the character. Honestly, dogs speak kind of like the dogs from \refwork{Up!} and rats and cats are no better. Most non-mammalian beasts are downright disappointing conversationalists (with exceptions made for some wicked smart non-mammals like crows and octopuses). Tongue of Beasts is passive and covert, and cannot be dispelled.

%%%
\subsubsection{Advanced Powers}
%%%

\powerentry{The Beckoning} The character emits some suitably animalistic noise and calls all creatures of a specific type to their location. The character can only choose one type of animal at a time, and the power reaches out to 100 meters per point of Potency. Using The Beckoning requires three power points and a Complex Action. The number of beasts that come is dependent upon how many of the type of creature that are within the area, and the urgency with which they come and the degree of control the character has over their actions upon their arrival is based on the number of hits achieved on a \dicepool{Strength + Survival} or \dicepool{Charisma + Empathy} test. When dealing with large or predatory animals, the range of The Beckoning increases to 500 meters per potency. When calling large predatory animals, the range extends to 5 kilometers per point of potency.

\powerentry{Transformation} The character can transform a victim into a beast by spending a power point and a Complex Action. This is a humiliating ordeal, and as the target doubtless has little familiarity with their new form is generally quite awkward and difficult for the victim (over and above being transformed into a frog or pig). The character makes a \dicepool{Strength + Empathy} or \dicepool{Willpower + Survival} test against the target's Intuition. The victim becomes a beast of the character's choice until the next sunrise or sunset. Net hits increase the time frame of a victim's transformation. This ability is a Protean power. A character can't be transformed into an animal much larger than their original size, but they can be transformed into something much smaller.

%%%
\subsubsection{Elder Powers}
%%%

\powerentry{Songs in the Dark} The character can create horrible monsters out of ordinary animals. By spending six power points, any mundane beast can be invested with enough astral power as to be transformed into a monster. The investment takes an hour and the transformation itself takes a day. The character makes a \dicepool{Charisma + Survival} or \dicepool{Willpower + Medicine} test. Making a Behemoth is hard (Threshold 3), while making a Chimera is crazy extreme (Threshold 5). The Giant Animal is created under the character's loose control. A character can control a number of such animals equal to their Willpower.

\powerentry{Soul Investment} The character can place a portion of their soul into another being, and subsequently take the target's body over when the investing character's current body goes unconscious or dies. The character must touch the target for a Complex Action, spend two power points, and make an opposed \dicepool{Willpower + Empathy} or \dicepool{Strength + Survival} test against the target's Willpower. A character can have portions of their soul hidden in a number of creatures equal to their Intuition, and they must choose only one of their victims to hop to when the time comes. While controlling a body, use the Strength and Agility of the host, but all other attributes and all skills of the character. The character does not lose control of a host by \textit{sleeping}, but they do automatically hop again (losing their soul investment in that victim) if they are actually knocked unconscious. The character never loses their effective Soul Investment in their original body, but if they are ever forced to hop and there are no living bodies with remaining Soul Investment, the character dies.


%%%%%%%%%%%%%%%%%%%%%%%%%
\subsection{Chasing the Storm}
%%%%%%%%%%%%%%%%%%%%%%%%%
\tagline{Ye elves of hills, brooks, standing lakes and groves,\\
And ye that on the sands with printless foot\\
Do chase the ebbing Neptune and do fly him\\
When he comes back.}

Chasing the Storm is the astral sorcery dedicated to understanding and controlling the weather. A character with Chasing the Storm knows what the weather should be for a week in advance (which means that they likewise know when weather sorcery is employed), and they gain a +1 dicepool bonus on Survival tests. With Advanced Chasing the Storm, a character knows the upcoming weather a month in advance and gains a +2 bonus. With Elder Chasing the Storm, the character's predictive capabilities extend for a year and a day and they gain a +3 dicepool bonus on Survival.

%%%
\subsubsection{Basic Powers}
%%%

\powerentry{Howling Winds} The character can make winds rise or soften and blow in a direction of their choice. The character can spend a Complex Action to make an \dicepool{Agility + Rigging} or \dicepool{Logic + Operations} test to increase or decrease the Strength of the winds by the number of hits.

\powerentry{Rising Mists} The character can fill areas with thickening fog that obscures vision and makes things seem really spooky. By spending a Complex Action, the character can fill up their immediate vicinity with mists, extending up to 3 meters per Potency from their person. By spending two power points, they can create an actual weather pattern, with mists creeping in out to a kilometer per Potency from their person. The character makes an \dicepool{Agility + Rigging} or \dicepool{Logic + Operations} test to decrease visibility through the mists:

\begin{table}[htb]
\rowcolors{1}{white}{tan} \caption{Rising Mists Concealment} \centering
\begin{tabular}{c l}
\textbf{Hits}&\textbf{Distance to Total Concealment}\\
\textbf{1}& 6 meters\\
\textbf{2}& 3 meters\\
\textbf{3}& 1 meter\\
\textbf{4}& 50 centimeters\\
\textbf{5}& 20 centimeters\\
\textbf{6}& 5 centimeters\\
\end{tabular}
\end{table}

%%%
\subsubsection{Advanced Powers}
%%%

\powerentry{Lightning Strike} A cloudy sky can be split by a lightning bolt crashing to Earth. By spending a Complex Action, the character can direct a lightning bolt to strike an individual or object. The lightning bolt cannot be dodged, its threshold to hit the target at any range is 2, and it is a Damage 5 weapon. It is powered by an \dicepool{Agility + Rigging} or \dicepool{Logic + Electronics} test. Once unleashed, the bolt of lightning is real electricity, and can be defended against with conductive materials in classic Franklin or Tesla fashion.

\powerentry{Tumultuous Rain} The character can pull clouds into a clear sky and rain from clouds. It takes a base amount of time of an hour to bring rain from a clear sky, and net hits on an \dicepool{Agility + Rigging} or \dicepool{Logic + Operations} test can reduce the timeframe. The more it is "naturally" clouded, the easier this is, and the character gains a +2 bonus if there are already clouds \textit{somewhere} in the sky, raising to a +6 bonus if there is already some tiny amount of rain coming down. The weather can be affected out to 5 kilometers per Potency of the character. Once established, a weather pattern persists until natural forces blow it away. This costs 3 power points.

%%%
\subsubsection{Elder Powers}
%%%

\powerentry{Form of Mist} The character can transform themselves into a fine mist as a Free action by spending a power point. This is a Protean ability. While in mist form, a character's body is essentially impervious to physical attacks, save those that are made with a weapon that the character is vulnerable to. Everything the character is carrying when they transform transforms with them, and other creatures thus transformed are essentially helpless until the character becomes solid again. While in mist form the character can fly, but they move no faster than a normal walk. For purposes of being dispelled, Form of Mist has 3 hits.

\powerentry{Victory of Typhon} By spending eight power points, the character can create a "tropical depression" at their location; and subsequently direct its progress. This requires an hour and a power point. The character makes an \dicepool{Agility + Rigging} or \dicepool{Logic + Operations} test, and the massive storm takes form and is unleashed in a direction of the character's choosing. Creating a Tropical Storm is a Professional Task (threshold 2), creating a Category 2 Hurricane is Crazy Extreme (threshold 4), and creating a Category 4 Hurricane is threshold 6. Remember that your storm may in fact be called a Typhoon or a Cyclone, depending upon its ocean of origin.


%%%%%%%%%%%%%%%%%%%%%%%%%
\subsection{Coil of Thorns}
%%%%%%%%%%%%%%%%%%%%%%%%%
\tagline{Leaves drop in Autumn not because of the shortening of the day, but because of the lengthening of the night.}

The Coil of Thorns is the Astral Sorcery dedicated to understanding and manipulating plant life. Characters who practice this magic are able to do much with practically any plant matter. The character's Artisan and Expression skills gain a +2 bonus when working in primarily plant matter media. This bonus increases to +4 with knowledge of Advanced Coil of Thorns, and to +6 with Elder Coil of Thorns. Characters with Coil of Thorns can also make something edible and even delicious out of literally any plant matter. It's not at all obvious how it's done, but their redwood pasta is not bad at all. This magic is inextricably linked in the minds of many supernaturals with the Evil Plants, and indeed any practitioner can speak with Evil Plants of any kind. It has even been suggested that Dryad Witches are actually Pod spies. Most Coil spells are quite powerful but quite time consuming to use. A Coil of Thorns laboratory is called a "Kitchen" and can indeed be created with materials purchasable at a farmer's market.

%%%
\subsubsection{Basic Powers}
%%%

\powerentry{Bitter Fruit} The character can make powerful medicines and poisons out of ordinary fruits and roots. By spending a power point and an hour working in a kitchen, the character's result can be a balm or a curse to those who consume it with a strength equal to the number of hits on a \dicepool{Logic + Medicine} or \dicepool{Intuition + Survival} test. Such products can be made into clearly medicinal pastes and the like, or covertly disguised as fresh foods such as shiny apples. Doses with other effects can also be produced, but if the requisite number of hits is not achieved the resulting product is merely a standard damaging poison with a strength equal to the actual hits achieved.

\begin{table}[htb]
\rowcolors{1}{white}{tan} \caption{Bitter Fruit Poisons} \centering
\begin{tabular}{c l l}
\textbf{Hits}&\textbf{Poison}&\textbf{Effect}\\
\textbf{1}&Anti-toxin& Reduces the power of toxins by its hits.\\
\textbf{3}&Sleep Drought& Consuming victim falls into a deep sleep.\\
\textbf{5}&Hypnotic&Consuming victim enters a highly suggestible state for an hour.\\
\end{tabular}
\end{table}
With Advanced and Elder Coil of Thorns comes the learning of additional recipes such as making powerful acids, paralytics, or explosives. The seed that transforms a Luminary into a Dryad has a threshold of 6 to brew.

\powerentry{Grass Rope} The character can have plant matter grow at tremendous speed according to their will and grasp their enemies with wooden fastness. The character spends a Complex Action and makes a \dicepool{Logic + Rigging} or \dicepool{Intuition + Survival} test, and the leafy strands lash out with a Strength and Agility equal to the number of hits.


%%%
\subsubsection{Advanced Powers}
%%%

\powerentry{Mind Root} The character plants a seed in a victim that slowly roots into the unfortunate's brain and leaves them an emotionless pawn. Creating the Mind Root takes four power points and an hour in the kitchen, and it takes an hour or two for the tendrils to work their way into the victim (making it essentially worthless against a target that is not willing, bound, or sleeping). The Mind Root uses the character's \dicepool{Logic + Medicine} or \dicepool{Intuition + Survival} against the victim's Strength.

\powerentry{Puppetry} The character can command plants to perform actions and have the plants actually perform them at a reasonable speed. By spending a Complex Action, the character's plant minions will move about with distinctly non-plantlike mobility for one round -- attaining an effective Agility equal to the character's Intuition or Logic. The character can also give long term commands to plants, which they will go about performing at their normal speeds (often centimeters a day). Commanding sapient (and presumably evil) plants is more difficult, and requires the character to best them in \dicepool{Logic + Medicine} or \dicepool{Logic + Rigging} against their Willpower to control one of them for a number of rounds equal to the number of net hits. Otherwise Puppetry extends to any number of plants within 10 meters of the character per point of Potency.


%%%
\subsubsection{Elder Powers}
%%%

\powerentry{Abomination} The character can work in a kitchen for an hour and spend seven power points to grow an Evil Plant from inert plant material and evil magic. The product is a harmless if strange looking potted plant that will grow into a horror over the following night. The character makes a \dicepool{Logic + Medicine} or \dicepool{Intuition + Survival} test to get the process started. Making a Man Trap is threshold 3, making a Triffid is threshold 5, and making a Pod is threshold 7. Creating these abominations bestows no special ability to control them. 

\powerentry{Seeds of Destruction} Time destroys all things according to mortals, but even amongst supernaturals it is known that the work of your hands will eventually be claimed by the legacy of nature. Using this ability, the character accelerates this process, causing the progression of seasons and flora to rip things to shreds. Roots shatter stone and flesh alike and tiny pieces blow away in pollen laden dust. This is a damage 3 attack that assaults each target in a cone that is up to 100 meters in length per point of Potency. No inanimate object is treated as larger than small, as bigger things are torn apart from every direction and even the inside. Targets are attacked from inside themselves, so the range is considered Adjacent regardless of physical distance. The character must spend a Complex Action and a power point, and uses \dicepool{Logic + Rigging} or \dicepool{Intuition + Survival}, increasing damage with hits. This power can be used on a much smaller scale without spending a power point, but it only projects 2 meters per point of Potency.


%%%%%%%%%%%%%%%%%%%%%%%%%
\subsection{Trail of Tears}
%%%%%%%%%%%%%%%%%%%%%%%%%
\tagline{Bitterest of all is not the sorrow but to have one's sorrows ignored.}

The Trail of Tears is a magical discipline which harvests the power of misery. In the Deep Astral there is a liquid that is literally created by the fact that sadness exists. This material is called the Tears of Maya, even though in a very real way it is the psychic residue of human sorrow from the Material World. Those who follow the Trail are incredibly good at finding the clouds in life. They gain a +2 bonus to argue against any proposal, fact, or course of action. At Advanced, this negativity increases to +4, and at Elder the ability to speak against things rises to a full +6 bonus.

%%%
\subsubsection{Basic Powers}
%%%

\powerentry{Curse of Failure} Using astral sorcery, the character curses a victim to have luck abandon them at a later date. As a Simple Action, the character attempts a \dicepool{Willpower + Sabotage} or \dicepool{Willpower + Rigging} check against the target's Willpower. If successful, evil magic hangs like a dark cloud of misery over the target's head. At any later date, the character may reactively end this curse to force the victim to reroll all hits on a test. The Curse of Failure is totally covert, all the character actually does is look at the victim funny and the sorcery has no overt special effects (although it can be detected as normal Astral magic).

\powerentry{Pain Drops} Tiny drops fall at the target, dampening not their skin but their pride. The character spends a Simple Action to direct Tears of Maya to fall at a target, and makes an \dicepool{Agility + Combat} or \dicepool{Willpower + Rigging} test to make a standard ranged attack. Pain Drops have a damage rating equal to the character's Willpower, and are soaked with Willpower rather than Strength. Pain Drops cause only illusory damage, and while they cause wound penalties and even incapacitation, they do not actually kill or wrap around into lethal damage. Virtual damage from Pain Drops fades in an hour.

%%%
\subsubsection{Advanced Powers}
%%%

\powerentry{Dark Night of the Soul} By spending a Complex Action belittling the target, the character can provoke a Despair Frenzy in them. The character makes an opposed \dicepool{Willpower + Sabotage} or \dicepool{Willpower + Intimidate} test against the target's Willpower, and if successful the victim enters an immediate Despair Frenzy (regardless of their Master Passion, if any). This ability is largely covert, as it does not appear that the character did anything other than behave in a rude and perhaps senselessly cruel fashion towards the victim.

\powerentry{Water Prison} With a Complex Action and a power point, water can be manipulated into three dimensional shapes that hold their form and are strong as steel. Actually capturing someone with the titular prison made of water requires a ranged combat test using \dicepool{Agility + Combat} or \dicepool{Willpower + Rigging}. Grasping water has a Strength equal to the character's Willpower.

%%%
\subsubsection{Elder Powers}
%%%

\powerentry{Astral Projection} Activating Astral Projection sends a character and possibly their friends into the deep or shallow Dreamlands or back to the material world. Anything the character carries as well as a circle of hand holding creatures up to the character's Willpower in number (and everything \textit{they} are carrying as well) shifts across the worlds with the character. Astral Projection requires a Complex Action and a power point. For purposes of dispelling, Astral Projection has 3 hits.

\powerentry{Object of Envy} The character can reach into a target's mind and create some physical thing that the target deeply desires. This can be something as simple as a sandwich or as complex as a whole person. This costs seven power points regardless of how elaborate or abstractly valuable it is. Electronics created in this way don't function super well because they are made out of evil magic even more so than a modern operating system normally is. A person created in this way is always an Extra with no edge stat or supernatural powers. They are otherwise able to pass as the actual person they are based upon (if any), though they are preternaturally disposed towards evil and cruelty and are under the control of the character who made them. The character makes a \dicepool{Charisma + Expression} test to set the quality of the mimicry of a person; a \dicepool{Logic + Electronics} test to set the quality of created electronics; and an \dicepool{Intuition + Artisan} test to set the quality of most other things. An Object of Envy so exactly duplicates a desired thing that it can set off a Greed or Loneliness frenzy in the target from whom the desire was plucked (in general, the number of hits on the ability generates the threshold to resist the frenzy). A targeted creature can try to keep their desires to themselves, making a \dicepool{Willpower + Sabotage} test as a reactive action -- if the target gets at least as many hits as the character, the Object of Envy never takes form. When an Object of Envy is destroyed, it reverts into tears. Creating an Object of Envy takes a minute.


%%%%%%%%%%%%%%%%%%%%%%%%%
\subsection{Veil of Morpheus}
%%%%%%%%%%%%%%%%%%%%%%%%%
\tagline{When you dream, dream of me.}

The veil between the waking world and the world of sleep is both vast and unfathomable. And yet to those who scry deeply into Maya can see past it, and in time they learn to reach across and pull things from one side to the other. A character who practices the Veil of Morpheus is able to perceive objects and creatures in the Shallow Dreamlands, though not to touch them and is in no danger from them physically. With Advanced Veil of Morpheus, the character can perceive their own surroundings while they are asleep, and are able to wake themselves at will. And when a character has Elder Veil of Morpheus, they know inherently when creatures are dreaming within a kilometer of themselves. Note that in a city, they will often be aware of a \textit{lot} of sleepers.

%%%
\subsubsection{Basic Powers}
%%%

\powerentry{Enchanted Slumber} Using astral sorceries the character makes people supernaturally sleepy. The compulsion to sleep is virtually impossible to overcome for people who are already asleep, and is at best a minor inconvenience for those whose adrenaline is pumping. Activating it is potentially covert, in that the target is not especially aware of anything other than a feeling of sleepiness. The character however produces faintly glowing dust from their hands when used, and expends a Complex Action doing it. Using Enchanted Slumber requires an opposed \dicepool{Charisma + Expression} or \dicepool{Logic + Medicine} vs. Intuition test. An affected target that is already asleep will not awaken for at least an hour (longer with net hits) even in the face of loud noises or shaking. An affected target who is not asleep will go to sleep at the first opportunity and then stay asleep for at least an hour (longer with net hits).

\powerentry{Dream Vision} The character can send messages and images into the dreams of another creature. The dream visions can go an unlimited distance. The character must know the name of the intended recipients, the targets must all be asleep and the character needs to meditate on the sending. The base timeframe to send visions is an hour, but net hits on a \dicepool{Charisma + Expression} or \dicepool{Logic + Research} test can reduce that.


%%%
\subsubsection{Advanced Powers}
%%%

\powerentry{Denial of Privacy} The character views into the dreams of a sleeping target. The character spends a Complex Action attuning themselves to the dreams of a target. The target must be asleep and within 100 meters per hit on a \dicepool{Charisma + Empathy} or \dicepool{Logic + Medicine} test. While so attuned, the character can make themselves appear in the dream or simply voyeuristicly  watch the proceedings.

\powerentry{Horrid Reality} The stuff of dreams becomes the stuff of reality. The character spends a Complex Action and suddenly the physical effects of events that take place within a dreamer's dream reality draw their toll on their body. While the effects of a sexy or merely surreal dream may not even be immediately obvious upon waking, the effects of a nightmare can be catastrophically fatal. The character need only know the name of a potential victim and their location while dreaming, the ability works at any distance. The victim can fight back against the invader with whatever is in their dream, but unless they \textit{also} have Horrid Reality, the only thing they will get from defeating their attacker is to knock them out of their dream until the following sunset. For purposes of dispelling, Horrid Reality has 3 hits.

%%%
\subsubsection{Elder Powers}
%%%

\powerentry{Dreamscape} The character shapes the Deep Dreamlands to be as they want it to be. The character spends an Edge and the dreamlands of Maya begin to adjust themselves to the substance and reality that the character wishes them to contain. This transformation takes a base amount of time of one day, but they can make a \dicepool{Charisma + Expression} or \dicepool{Logic + Operations} test, and reduce the timeframe with net hits. Things created in the dreamlands are like Hollywood sets. While an object might glitter like gold, it's just a prop.

\powerentry{Dreamstep} The character can move seamlessly between the Shallow Dreamlands and the Material World at will. Crossing the threshold between the Shallow Maya and the physical world is a Simple Action and the character can take anything they can carry with them. For purposes of dispelling, Dreamstep has 3 hits.


%%%%%%%%%%%%%%%%%%%%%%%%%%%%%%%%%%%%%%%%%%%%%%%%%%
\section{Infernal Sorceries}
%%%%%%%%%%%%%%%%%%%%%%%%%%%%%%%%%%%%%%%%%%%%%%%%%%

\hspace{\parindent} The Dark Reflection is a horrible place from which few things escape. But its reach and grasp can be felt far outside the ashen pits of Limbo itself. Infernal Sorceries are those magics that manipulate and harness the taint of that parched landscape and turn them to one's own advantage.

The magic from Limbo is a lot like fire. It is also quite specifically malicious and it corrupts things it touches. When clean water is nearby, it discolors slightly as if ash had fallen into it. A skilled exorcist can track the power and location of magic from the Dark Reflection by consulting distilled water in a clear container. When Infernal Sorceries are cast they can be doused by throwing sand on them. Reasonably clean sand must be used -- ground up silicon dioxide works. By the time the sand hits the floor it has been replaced by white and black sand.

%%%%%%%%%%%%%%%%%%%%%%%%%
\subsection{Descent of Entropy}
%%%%%%%%%%%%%%%%%%%%%%%%%
\tagline{Everything ends the same way: it no longer matters and no longer persists.}

The fires of Limbo have long dulled and all there is left is misery and ash. Those who delve into the knowledge of entropy learn grim ends of human nature and command the power of wickedness to compel disease agents to action. To those who follow this path, the inevitable unraveling brought about by time appears entirely normal and expected. A student of the Descent of Entropy gains a +2 bonus to diagnose an ailment, reconstruct an event from evidence, or repair an object that has fallen apart. These bonuses increase to +4 at Advanced and +6 with Elder Descent of Entropy.

%%%
\subsubsection{Basic Powers}
%%%

\powerentry{Abyss of the Body} The character carries hideous diseases that can be foisted upon others with a touch, a bite, or a kiss. A character with Abyss of the Body is not themselves affected by disease, and they can carry all manners of normal plagues and pass them normally. In addition, each character who has Abyss of the Body can voluntarily transmit a purely magical disease. The disease they have at their disposal to use voluntarily is chosen when the discipline is learned. A character infected by additional magical diseases is still immune, but has no control of its transmission. Abyss of the Body can be learned multiple times to control multiple diseases. When spreading a disease voluntarily the character achieves physical contact and uses \dicepool{Strength + Survival} or \dicepool{Charisma + Medicine} against the victim's Strength, with net hits reducing the disease's DOT Delay. The spreading of disease can be dispelled with sand, but the immunity to disease of the character is passive, covert, and impervious to dispelling.


\begin{table}[htb]
\rowcolors{1}{white}{tan} \caption{Abyss of the Body Diseases} \centering
\begin{tabular}{l l l}
\textbf{Disease} & \hyperref[subsection:Damage-Over-Time]{\textbf{Delay}} & \textbf{Effects}\\
Chud & 5 & Victim dissolves into white goo.\\
Cooties & 3\textsuperscript{0}  & Victim driven mad with lust.\\
Doom Flies & 8 & Victim eaten up by maggots, explodes into swarm of flies.\\
Rage Virus & 3 & Victim transforms into Soulless.\\
Thing Infection & 11 & Victim fills up with roots, Triffid bursts out of their husk.\\
Z-Virus & No Damage & When Victim dies, it rises as a Shambler.\\
\end{tabular}
\end{table}

\powerentry{Light of Ennui} The character becomes a beacon of apathy and despair, foiling attempts by those caught in the metaphorical glow to care. The character makes a \dicepool{Charisma + Artisan} or \dicepool{Charisma + Empathy} test against the Willpowers of everyone within \textbf{S}hort range. Everyone affected is unable to muster an emotional response to much of anything. Frenzy doesn't happen, and all appeals to emotion and even magically created emotions fail unless the speaking character can muster more hits than the Light of Ennui. Lighting or dousing the Light of Ennui is a free action. The Light of Ennui sheds literal light in the Dark Reflection, and is enough to see by.

%%%
\subsubsection{Advanced Powers}
%%%

\powerentry{Aura of Decay} When the Aura of Decay is activated, things fall apart into grime and dust. This can be directed somewhat, but generally anything within a few meters of the Character will degrade as if left unattended for days and weeks in mere moments. Objects that are literally on the person of someone will not be devastated by decay unless the character focuses in on them and marks them for destruction. The character can focus in and make a \dicepool{Strength + Artisan} or \dicepool{Charisma + Medicine} test against a target's Intuition to corrode a specific person's equipment as a Complex Action. Activating or deactivating the Aura of Decay is a Simple Action. Normally it is far too slow to affect a fast moving object, but by spending a power point, the speed of corrosion can be enhanced to the point where it will tear items out of the air before the reach the character. Disposing of an incoming thrown piece of wood is easy (threshold 1), stopping bullets is hard (threshold 3), and corroding one's self a path through  being otherwise run over by a truck is crazy extreme (threshold 5).

\powerentry{Contradiction} The character can tempt a victim into doing something they would never do, something wildly against their nature. The character can make an opposed \dicepool{Charisma + Artisan} or \dicepool{Charisma + Bureaucracy} test against the victim's Willpower. This takes conversational time, and is absolutely powerless to convince someone to do something that they are merely moderately opposed to. Contradiction is covert.

%%%
\subsubsection{Elder Powers}
%%%

\powerentry{Howl of the Abyss} A scream of unutterable anguish permeates the Material World and the Dark Reflection for a round. The fires of Limbo belch into the material world. Smoke and ash blows through cracks in the world. And Mirror Goblins spew forth in a brutal wave. The character spends five power points and a Complex Action to makes a \dicepool{Charisma + Artisan} or \dicepool{Charisma + Bureaucracy} test. A number of Mirror Goblins show up equal to the square of the number of hits. Any other creatures within a few meters of the character's location in either the Shallow or Deep Limbo may attempt to escape to the Material World as well.

\powerentry{Wind of Pestilence} Black winds carry ghastly plagues that doom everyone they touch. Everything the shadowy air caresses is exposed to a magical ailment of the character's choice. The winds come out as either a cone that extends for a hundred meters per point of Potency or as a circular maelstrom that is twenty meters in radius per point of Potency at the character's option. The character spends four power points and Complex Action to make either a \dicepool{Strength + Artisan} or \dicepool{Charisma + Medicine} test. Potential victims may resist with their Strength, and the character can nominate one target per hit to be avoided by the gale.


%%%%%%%%%%%%%%%%%%%%%%%%%
\subsection{Names of the Blasphemies}
%%%%%%%%%%%%%%%%%%%%%%%%%
\tagline{It is hubris of the first order to name that which is not known.}

There are words that should not be spoken. Mostly, these are the names of ancient unspeakable evils. And the \textit{reason} these evils are unspeakable is because they will hear you if you speak about them. Upon learning each ability from this discipline, the character is able to choose for themselves a True Name. Abilities that function based upon the character's name only respond to that name, no matter what they refer to themselves as. A character with Basic Names of the Blasphemies can see creatures and objects in the Shallow Limbo, although they do not physically interact with them. With Advanced, the character knows the name of anyone they see an accurate picture of, and with Elder they know the name of anyone they see (regardless of disguises).

%%%
\subsubsection{Basic Powers}
%%%

\powerentry{Learn the Heart's Pain} By concentrating for a Simple Action, the character can know the limits of a person's psyche. The character makes a \dicepool{Charisma + Empathy} or \dicepool{Intuition + Larceny} test, gaining knowledge of the subject as follows:


\begin{table}[htb]
\rowcolors{1}{white}{tan} \caption{Learn The Heart's Pain Results} \centering
\begin{tabular}{c l}
\textbf{Hits} & \textbf{Knowledge Gained}\\
 \textbf{1} & The target's Master Passion, if any (humans usually have no Master Passion)\\
 \textbf{2} & The target's Driving Passion\\
 \textbf{3} & General details of the target (such as career and name)\\
 \textbf{4} & Specific details of the target's hopes and dreams\\
 \textbf{5} & Intimate details of the target's life that the target barely even cares about.\\
\end{tabular}
\end{table}

Using this power is covert. It can only be used once per scene on each other creature.

\powerentry{Poison Heart} The character fills the target's heart with a litany of lies. By spending a Complex Action, the character makes an opposed \dicepool{Charisma + Empathy} or \dicepool{Intuition + Larceny} test against the target's Willpower and if successful causes the target to feel intense feelings of betrayal and disappointment. If the character knows the target's Master Passion, they can incite a Frenzy. If the character knows the target's Driving Passion, they can make them bitterly sullen and full of the feelings of failure (suffering a morale penalty to skill tests until the end of the scene). And if the names of any of the target's friends and loved ones are known, the target can be made to distrust them. Strength of any frenzies, depressions, or misgivings are the number of net hits. This ability is covert.


%%%
\subsubsection{Advanced Powers}
%%%

\powerentry{Bind the Name} By knowing a creature's name, the character can bind them in Limbo, making it even more improbable that they could escape. This binding costs five power points and takes a Complex Action, but can work at any distance. The character makes an Intuition + Empathy or Charisma + Bureaucracy test and the threshold to escape the Dark Reflection is increased for the target by the number of hits. Only the most powerful Name Bindings applies to a target if more than one are applied. The Name Binding can be dispelled at the point in space it was literally invoked, not at the bound target.

\powerentry{Banishment} The character hurls a victim they can see into the Dark Reflection. The character spends a Complex Action and a power point, and makes an opposed \dicepool{Intuition + Larceny} or \dicepool{Charisma + Bureaucracy} test against the target's Intuition to send them into the Dark Reflection. If three net hits are made, the target can be sent into the Deep Limbo. If the character has Elder Names of the Blasphemies, this ability can be used on several targets at once, targeting up to one creature per point of Charisma.


%%%
\subsubsection{Elder Powers}
%%%

\powerentry{The Truest Name} Any time any creature speaks the character's name anywhere in the world, the character hears it and the next snippet of conversation the creature utters (usually amounting to a twitter feed, more or less). If the name is spoken three times in succession, the nearest mirror in the material world to the speaking creature becomes a portal to the nearest mirror in the dark reflection to the character. The character can make a \dicepool{Intuition + Stealth} or \dicepool{Charisma + Bureaucracy} test to reduce the threshold to cross that portal by the number of hits. This ability is covert and continuous. The creation of a mirror portal can be dispelled, but the rest of the effects cannot.

\powerentry{The Great Unbinding} The character summons a Demon. It is a professional grade task to summon one of the Asura (threshold 2), it is totally extreme to summon an Akuma (threshold 4), and it is flat super human to summon one of the Ifrit (threshold 6). If the character knows the name of a specific entity, they may attempt to summon that one in particular to the material world, and the threshold is reduced by 1. If a proposed target's escape threshold is higher than the summoning threshold, use that threshold instead of the base. The Great Unbinding uses \dicepool{Intuition + Empathy} or \dicepool{Charisma + Bureaucracy}. It takes an hour, and can also be used to spring any specific creature whose name is known from the shallow or deep Dark Reflection. A creature that for whatever reason does not wish to go is permitted a Social Resistance test.


%%%%%%%%%%%%%%%%%%%%%%%%%
\subsection{Progress of Glass}
%%%%%%%%%%%%%%%%%%%%%%%%%
\tagline{"We do not always see where we go. But we always go."}

The Progress of Glass is a path of Infernal sorcery that deeply investigates the connections between existence and the Dark Reflection. Followers of the Progress of Glass are able to handle different perspectives very easily -- some would say that they have already driven themselves mad. In any case, spending so long gazing into mirrors has left them able to write in mirror writing or use reflections for targeting without penalty. Also they end up being very good at driving. A user of the Progress of Glass gains a +2 bonus on Driving tests, a bonus that increases to +4 with Advanced and +6 with Elder Progress of Glass.

%%%
\subsubsection{Basic Powers}
%%%

\powerentry{Distant Reflection} The character can peer into a reflective surface and see what is reflected off of another reflective surface elsewhere. The character must be within 100 meters per point of Potency of both surfaces, and must have a pretty good idea of where both surfaces are (and yeah, they have to be able to see at least one of them). Using Distant Reflection requires a \dicepool{Logic + Operations} check or an \dicepool{Intuition + Perception} check. The expected time to get the vision is an hour, but net hits reduce the timeframe.

\powerentry{Deny the Gauntlet} The character can remove the difficulties in moving across gateways into and out of the Dark Reflection. The character spends a Simple Action and makes a \dicepool{Logic + Operations} or \dicepool{Intuition + Stealth} test and the gauntlet is removed for a specific gateway for a Round. Net hits increase the timeframe. The gauntlet is not hampered for purposes of keeping creatures held in Limbo by Name Bindings.


%%%
\subsubsection{Advanced Powers}
%%%

\powerentry{Mirror Pocket} The character can create and subsequently access a virtual space within a reflective surface. Things can be stored in and retrieved from the mirror pocket with relatively little fuss. This looks like a cartoon activity: the character's had passes into the surface causing it to ripple slightly and then some object is left within or drawn out -- in either case demonstrably having more volume than the mirror possibly could. Creating a pocket costs two power points, but putting things in or taking things out is free. The character makes a \dicepool{Logic + Operations} or \dicepool{Intuition + Stealth} check and the mirror pocket has a depth of 20 cm per hit. If the character has Elder Progress of Glass, the depth is 50 cm per hit. A character can maintain multiple mirror pockets in different reflective surfaces, but not in the same one.

\powerentry{Rain of Glass} Shards of blackened, jagged glass rain down and tear the flesh from the bones of everyone in the area. A Rain of Glass can be sent out as a cone up to 30 meters in length per Potency, or fly out in all directions to cover a cylinder centered on the character with a radius of 10 meters per Potency. Using this ability costs a power point and takes a Complex Action. It does a base damage of 3, which is increased with the hits on either an \dicepool{Agility + Combat} or \dicepool{Logic + Operations} check and the damage is resisted like a normal explosion (including the additional protective effects of cover).


%%%
\subsubsection{Elder Powers}
%%%

\powerentry{Doppelg\"{a}nger} A reflection is given life and given the task of supplanting the original person. The character makes a \dicepool{Logic + Expression} or \dicepool{Intuition + Perception} check, spends a Complex Action and seven power points and then the Doppelg\"{a}nger takes form by spending a Complex Action crawling out of the mirrored surface upon which the reflection was cast. It then sets about attempting to murder the original and then take over their life. The Doppelg\"{a}nger is wicked and generally under the command of the character's. When a victim has a Doppelg\"{a}nger created of them, neither they nor the duplicate cast a reflection until one of them is dead. The double has the base attributes of the target, and the degree of mimicry is set by the number of hits:


\begin{table}[htb]
\rowcolors{1}{white}{tan} \caption{Doppelg\"{a}nger Results} \centering
\begin{tabular}{c p{14cm}}
\textbf{Hits} & \textbf{Effects}\\
\textbf{1} & Extra, no supernatural powers, memories, or even coherent speech.\\
\textbf{2} & Can speak like the victim.\\
\textbf{3} & Has an Edge stat if the victim does.\\
\textbf{4} & Has the memories of the target.\\
\textbf{5} & Is a starting supernatural of the type of the victim if the victim is supernatural.\\
\textbf{6} & Has a Potency of the victim or the character (whichever is less), has disciplines that the target has.\\
\end{tabular}
\end{table}

The Doppelg\"{a}nger can be suppressed or dispelled before it successfully crawls out of the mirror, but once it does so it is a "real" person (albeit one having no reflection), and cannot be dispelled with sand. Its presence still offends clear water as Infernal sorcery is want to do, but even that effect fades if the Doppelg\"{a}nger succeeds in slaying its original and gains a reflection of its own.

\powerentry{The Smoking Mirror} The character controls the vertical and the horizontal. The character spends a Complex Action and three power points so they can determine what is seen and heard for the rest of the scene. The area covered is limited to Line of Sight, and extends out to 30 meters per point of Potency. The character makes a \dicepool{Stealth + Intuition} or \dicepool{Logic + Operations} check and observers can only see through the illusion if they get an equal number of hits on an \dicepool{Intuition + Perception} test.



%%%%%%%%%%%%%%%%%%%%%%%%%
\subsection{Song of Swarms}
%%%%%%%%%%%%%%%%%%%%%%%%%
\tagline{"I'd like to help, but I'm covered in bees."}

Insects are kind of creepy at the best of times, but in the horror genre they actually are a source of wickedness and danger. This is borne out in After Sundown, achieved in no small part by Infernal Sorcery. The Song of the Swarm is that sorcery and it has a great deal of internal synergy, many of the abilities within it make other abilities more useful. Being able to send insects where you want them is more useful if you can see what the insects see, and so on and so forth. Having Song of the Swarm allows you to treat crustaceans as if they were smart animals like parrots or dogs for the purposes of Animal Ken. Insect bites to creatures inside insect swarms are usually not much to report on, causing just Damage 1 strikes even en mass, although these strikes do not require a to-hit roll and are not dodged. Anything special that happens with the character's unarmed attacks apply with their insects (especially of note is the Lure of Destruction and the Descent of Entropy).

%%%
\subsubsection{Basic Powers}
%%%

\powerentry{Body Colony} The character's body has bugs growing inside it. In addition to being really gross, they can also fly out of the character's body and crawl back in without causing any lasting damage. The bugs growing in the character's body do pretty much what the character wills them to do, so they can follow much more complex routes than a normal bug of their type. The bugs from a character can swarm one person at a time or carry very small objects. They take an entire round to crawl into or out of the character's body, but they take no actions to direct. The Body Colony is passive and cannot be dispelled.

\powerentry{Small Witness} The character can see through the eyes of insects in their vicinity. The character can see through the eyes of a bug that is within a meter of their body, and can continue to see through its eyes even if the character or the bug moves away, so long as they remain within a kilometer per Potency. A character can maintain a number of small witnesses equal to their Intuition score simultaneously and suffers no penalties from distraction. The character can make their own Perception tests through the senses of the bugs. Small Witness has 3 hits for purposes of being dispelled. It is covert.


%%%
\subsubsection{Advanced Powers}
%%%

\powerentry{Magnify the Swarm} When there are insects in the area, the character can draw upon the magic of Limbo to cause a great number of comparable insects to wriggle into reality. The character spends a Complex Action and a power point, then makes a Charisma + Animal Ken or Willpower + Survival test and every hit allows them to pretty much swarm everything and everyone in a 10m cube. If the character has Elder Song of Swarms, the area may be increased to a 100m cube per hit.

\powerentry{Swarm Body} The character's body breaks into a cloud of insects of roughly the same mass as the character. This is a Protean discipline. The deaths of individual bugs do not meaningfully affect the character, but the deaths of very large numbers may seriously hurt them. When they reform, every 10\% of the original swarm unable to make it translate to a lethal wound. The Swarm body can be dispelled as if it had 3 hits, and however many insects are on hand at the time will coalesce into the original character (with accompanying wounds if not all the insects are available for the forced change). When the Swarm Body ends, whether voluntarily or through being dispelled, the character can elect to materialize anywhere that some of their bugs are, but any bugs unable to reach their new position within 12 seconds die (and the character suffers proportional wounds).

%%%
\subsubsection{Elder Powers}
%%%

\powerentry{Gold and Honey} The insects of the character's can and do build things on their behalf. Bugs perform strange alchemy, devouring things and transforming the material itself into virtually anything that is desired. The character can make a \dicepool{Charisma + Animal Ken} or \dicepool{Logic + Artisan} test to make things of astonishing complexity, and a \dicepool{Logic + Electronics} test to modify or assemble electronics. Gold and Honey is a reflexive action, and the swarm works at the speed of a group of competent contractors. The bug can also "assemble" things into piles of dust if desired, in order to burrow through walls, destroy valuables, or simply to hide evidence. The character can make a \dicepool{Charisma + Sabotage} check to speed such processes.

\powerentry{Riot} The swarms of humanity are as quick to anger as those of ants or bees. The character can incite violence and destruction on the part of normal people, erupting entire neighborhoods into senseless or even slightly directed anger and devastation. A riot generally requires about an hour to cook off, but the character can make a \dicepool{Willpower + Intimidate} or \dicepool{Charisma + Sabotage} check to reduce the timeframe. The waves of bitterness and anger extend out to up to 200 meters per point of Potency, and are no less effective in highly populated areas. Indeed, a "riot" in a dense city is generally far more effective because there are so many more humans doing the rioting. Inciting the Riot is covert, but rampaging people are generally pretty obvious.

%%%%%%%%%%%%%%%%%%%%%%%%%
\subsection{Walk of Flame}
%%%%%%%%%%%%%%%%%%%%%%%%%
\tagline{"People running around. Skin on fire. It's beautiful."}

Fire is likely the dividing line between man and beast. And yet while it is undeniable that fire provides light and warmth, protection from the monsters of the darkness and the forging of bricks and steel -- that's mostly a facet of \textit{technology}. Magical fire is a terrifying thing that rages at the limits of control and burns things down to ash. Firestarters in After Sundown are of course based largely on Carrie, and it is unsurprising therefore that fire magic spends a lot more time spreading panic and a lot less time smelting iron than controlled industrial fires do. The Walk of Flame draws strength from the Dark Reflection. A character with Walk of Flame gains a +2 bonus on Sabotage tests. This bonus increases to +4 with Advanced Walk of Flame, and +6 with Elder.

%%%
\subsubsection{Basic Powers}
%%%

\powerentry{Fire Walking} The character hardens themselves against the flames and no longer burns. This does not protect them from shrapnel propelled by an explosion, but does protect them from any wounds caused by heat itself. Use of this power is entirely reflexive. Fire Walking is passive and covert, and cannot be dispelled.

\powerentry{Hand of Flame} The character can conjure up flame in their hand, varying it in intensity from a lighter flame raising from an extended finger to a 30 cm inferno in their palm. The fire does not burn the hand it sprouts out of (the waves of heat move away), but the user is otherwise not protected from the fire or fires they start. Calling, changing the intensity, or extinguishing of the flame is a Simple action that costs nothing. The fire can inflict as much as the character's Logic attribute in fire damage. Hand of Flame is a melee attack, but does not actually get additional damage for net hits.

%%%
\subsubsection{Advanced Powers}
%%%

\powerentry{Fire Starter} The character concentrates momentarily and something bursts into flame. Alternately, the character may have a bolt or ball of flame fly from some part of their body to explode at the target. Either way, the flames inflict the character's Logic attribute in damage, and are resolved as a ranged attack targeted with \dicepool{Agility + Combat} or \dicepool{Logic + Research}. The attack becomes inaccurate outside of \textbf{S}hort range, and has a maximum range of \textbf{W}ay Out, or \textbf{E}xtreme with Elder Walk of Flame. Damage is increased with net hits as normal, but due to the sheer size of the fire compared to a bullet cover is especially helpful for the target -- as if the attack itself were an explosion affecting only them. Using this discipline is a Simple Action.

\powerentry{Flames of Panic} By spending a power point, the character sends waves of terror into crowds of people. This magic affects every creature except themselves the character can literally see (which means that they can shape its effects by voluntarily closing one eye, for example). The character rolls \dicepool{Logic + Intimidation} or \dicepool{Agility + Survival}, and each affected creature makes a Willpower Resistance Test. Targets are panicked and stampede around irrationally for one round for each net hit of the caster.

%%%
\subsubsection{Elder Powers}
%%%

\powerentry{Hell Storm} By spending four power points and a Complex Action, the character can fill a tremendous area with hellish fire. The flames inflict a base damage equal to the character's Logic attribute, and are resolved as an explosion with a base radius of fifty meters per point of Potency. The explosion can be targeted at a specific place or person as a ranged attack using \dicepool{Logic + Research} or \dicepool{Agility + Combat}. Remember that the damage is increased by 2 for whoever is actually unlucky enough to be at ground zero.

\powerentry{Scorch the Gateway} By spending three power points and a Complex Action, the character can enhance a fire magically such that it leaves a gateway from the mortal world to the Dark Reflection (and back) in its wake. By concentrating on the fire the character can attune the resultant portal to one specific creature for each Complex Action spent concentrating (until the character runs out of names or the fire burns out and the way manifests). The character may then make a \dicepool{Logic + Research + Potency} or \dicepool{Willpower + Intimidate + Potency} roll, and all attuned creatures get a bonus die to escape through the gate for each hit.


%%%%%%%%%%%%%%%%%%%%%%%%%%%%%%%%%%%%%%%%%%%%%%%%%%
\section{Orphic Sorceries}
%%%%%%%%%%%%%%%%%%%%%%%%%%%%%%%%%%%%%%%%%%%%%%%%%%

\hspace{\parindent} Orphic Sorceries are collections of spells and occult knowledge that draw their arcane power from Mictlan. The Gloom provides power that is dark, timeless, destructive, and really very frightening to the living. And for good reason, it truly does act as a window into one's mortality. The land of the dead calls inexorably to all living creatures, and sooner or later it will claim them all, and when Orphic Sorceries are used, the draw of death becomes even stronger. This effect is not particularly noticeable for large creatures like humans or even dogs -- the cells on the outside of your body are in a constant state of death and rebirth anyway, and even the heaviest exposure to the Gloom is unlikely to give you more than a mild frost burn. But for the very tiny and ephemeral creatures whose lives are over in a day anyway, death magic signals the end. A skilled occultist can use a bag of mayflies or the like to spot the moment that death magic is used, and to gauge the strength, distance, and direction towards the source of The Gloom.

Magic of Mictlan is highly antithetical to seeds, and you can counter Orphic Sorceries by throwing grains at them. Rice, wheat, barley, or maize kernels work equally well, but they have to be live grains. Remember that many commercial food products are neutralized with radiation or flash heating to keep them from going bad.

%%%%%%%%%%%%%%%%%%%%%%%%%
\subsection{Lure of Destruction}
%%%%%%%%%%%%%%%%%%%%%%%%%
\tagline{Lately just gazing into the Abyss has lost its thrill. I'm pissing into it.}

There is something distinctly alluring about the end. Not just that we cannot help but eventually reach the conclusion, but that indeed there is always a part of us that \textit{wants} to. Practitioners of the Lure of Destruction are very well acquainted with doom in all its forms and their Death Threshold increases by 2. With Advanced Lure of Destruction, the Death Threshold increases by 4, and with Elder Lure of Destruction it increases by 6. Characters with Lure of Destruction also get a bonus to Healing tests that is equal in size.

%%%
\subsubsection{Basic Powers}
%%%

\powerentry{Bite of the Serpent} The character's bite is toxic. Upon gaining this ability, the character must choose whether their poison is continuous or optional. A character whose poison is optional will have some kind of clear and potentially Vow of Silence breaking physical manifestation of their poison "turning on", while a character whose poison is continuous has no outward sign that their bite carries unusual properties. For example: a Strigoi has retractable serpent fangs, and they are perfectly capable of retracting them and kissing people without anything special happening; while a Soulless has a mouth that appears to just be a normal human mouth but their saliva is a deadly venom. A character with Bite of the Serpent is immune to poisons. A character who takes Bite of the Serpent multiple times may change the poison they secrete at will between available choices. The poison can be countered as if it had 3 hits, but the character's immunity to poison is covert, passive, and cannot be dispelled.


\begin{table}[htb]
\rowcolors{1}{white}{tan} \caption{Bite of the Serpent Poisons} \centering
\begin{tabular}{l l l}
\textbf{Poison}& \textbf{Effect} & \textbf{Notes}\\
 Euphoric & Victim dazed and anesthetized, addictive & This is the poison that Strigoi get.\\
 Hallucinogenic & Victim goes crazy, addictive&\\
 Paralytic & Victim immobilized & This is the poison that Triffids get\\
 Soporific & Victim fatigued & \\
 Toxic & Victim suffers damage & This is the poison that Soulless get\\
\end{tabular}
\end{table}

\powerentry{Touch of Darkness} The character's physical attacks inflict aggravated damage. The character can also scratch and bite into metal without hurting themselves. Touch of Darkness cannot be dispelled.

%%%
\subsubsection{Advanced Powers}
%%%

\powerentry{Glimpse of the Abyss} The character can show a bit of the inevitable doom that awaits us all to a group of targets. The character makes an opposed \dicepool{Strength + Athletics} or \dicepool{Willpower + Expression} test against each victim's Willpower. Every victim who is affected is stunned for a turn with awe and despair. Even Doom loses its impact, and a target who is affected will be harder to affect in the same evening -- apply a 1 hit penalty to the character for each time a particular victim has been assaulted. Using Glimpse of the Abyss takes a Complex Action, and extends to Line of Sight.
\powerentry{Withering} The character can accelerate the rush to death of those around them. By spending a Simple Action, the character can weaken a foe until they cannot even stand. The character makes an \dicepool{Agility + Combat} or \dicepool{Strength + Athletics} test and the victim's Strength is reduced by the number of hits until the end of the scene. Multiple Witherings do not stack, but a victim whose Strength is reduced to zero collapses.

%%%
\subsubsection{Elder Powers}
%%%

\powerentry{Death Knell} The character shuts off access to The Gloom. All creatures with a Potency of zero and an Orphic power source (such as shamblers) simply cease to be supernatural creatures (or indeed, creatures at all). The character makes an \dicepool{Agility + Combat} or \dicepool{Strength + Athletics} test and all Orphic creatures (possibly including the character) lose that many power points. Creatures without enough power points to lose take aggravated damage levels equal to the difference. All Shadow Gates close, and all Ghosts are deported to the Deep Gloom. The power takes a Complex Action and a power point, and it extends out to 100 meters per point of Potency.

\powerentry{Shadow Gate} By spending five power points, the character creates a Shadow Gate, a rift in space that allows the jealous power of Mictlan to rush through into the material world. Depending on the size, creatures may be able to move back and forth through it. The strength of the gate is equal to the hits on a \dicepool{Strength + Athletics} test or a \dicepool{Willpower + Expression} test. The Shadow Gate requires a Complex Action and a power point to open.


%%%%%%%%%%%%%%%%%%%%%%%%%
\subsection{Necromancy}
%%%%%%%%%%%%%%%%%%%%%%%%%
\tagline{"Everything ends and everything dies. That is not a good thing or a bad thing. It's simply inevitable. More than any other truth, the end cannot be eternally avoided."}

Necromancy is the incredibly creepy sorcerous path of handling the dead. Necromancers can see Ghosts and things that are in Mictlan without actually putting themselves into those areas and the potential harm's way that could entail. Characters with Advanced Necromancy can intuitively sense what items constitute a Wraith's fetters and where the fetters of a viewed Wraith might lay. A character with Elder Necromancy can feel the presence and knows the goals of all Poltergeists and Shadow Gates within 100 kilometers of their position. Ghosts intuitively know that a Necromancer can sense them, and they react accordingly.

%%%
\subsubsection{Basic Powers}
%%%

\powerentry{Compel Spirits} The character can send Ghosts from Mictlan into the material world or vice versa. By spending a Complex Action and making an opposed \dicepool{Willpower + Empathy} or \dicepool{Logic + Bureaucracy} test against the Ghost's Willpower, it can be sent to the Material, the Shallow Gloom, or the Deep Gloom, at the necromancer's option.

\powerentry{Summon Spirit} By naming a dead person or holding up a part of their body, the character can draw their Ghost to themselves unless it has been bottled, destroyed, or subsumed into a Poltergeist. The character makes a \dicepool{Logic + Medicine} or \dicepool{Logic + Operations} check, with a difficulty of how exactly they can describe the Ghost they are looking for.

\begin{table}[htb]
\rowcolors{1}{white}{tan} \caption{Summon Spirit Results} \centering
\begin{tabular}{c l}
\textbf{Threshold} & \textbf{Description}\\
\textbf{1}& "My brother Mike."\\
\textbf{2}& "My friend Todd."\\
\textbf{3}& "The owner of this body."\\
\textbf{4}& "That missing girl Karen."\\
\textbf{5}& "The guy who was killed last night."\\
\textbf{6}& -Any dead person known only from books.-\\
\end{tabular}
\end{table}

%%%
\subsubsection{Advanced Powers}
%%%

\powerentry{Nightcry} The character screams a wail of pain and despair into the deepest portions of Mictlan, and draws a Poltergeist to their position. This requires four power points and a Complex Action. By making a \dicepool{Willpower + Empathy} or \dicepool{Logic + Bureaucracy} test against the Poltergeist's Willpower, the character can compel it to begin its rain of destruction upon a target of their choice. What else the Poltergeist does from being in the material world is totally up to it.

\powerentry{Reanimate} The character makes a body into a Shambler or Soulless. A Shambler comes into existence under the control of the necromancer, and a Soulless does not. If the body was once a Luminary, their spirit is drawn back into their body and they become a(n uncontrolled) Revenant. A necromancer can only maintain control of 2 Shamblers per point of Willpower they have. If they make more Shamblers than that, they lose control of some of their Shamblers at random at some inconvenient time in the next day or so. Reanimating the dead is hard (Threshold 3), costs a power point per corpse, and has an expected time of 1 Day (net hits reduce the time normally). It is a \dicepool{Logic + Medicine} or \dicepool{Logic + Operations} check. Remember that Shamblers don't have the capacity to follow complex orders. Reanimate can only be dispelled before the Zombie has actually risen. After that, the Zombie will have to be dealt with normally.

%%%
\subsubsection{Elder Powers}
%%%

\powerentry{Resurrection} The character can return life to someone who is dead. In order to work, the character must have access to the target's ghost and an "appropriate" body. The target's own body is appropriate, as is any other body of the same sex and build. Resurrecting the dead is hard (Threshold 3), and requires a \dicepool{Logic + Medicine} or \dicepool{Willpower + Empathy} check with an expected time of 1 day (net hits reduce the time normally). Resurrection costs seven power points. Resurrecting a Revenant restores them to humanity.

\powerentry{Zombie Mastery} The character has no particular limit for how many Shamblers they can create and retain control of with Reanimate (if they have that ability). In addition, the character can take control of Zombies of any type (Shamblers, Soulless, or even Revenants). This takes a Complex Action and costs a power point. The character makes a \dicepool{Logic + Operations} or \dicepool{Willpower + Tactics} test and takes command of a number of Zombies equal to the hits. Revenants are allowed to resist with Willpower. Commanded Zombies reel and cease brain eating while shuffling about as if struck until given orders.


%%%%%%%%%%%%%%%%%%%%%%%%%
\subsection{Play of Shadows}
%%%%%%%%%%%%%%%%%%%%%%%%%
\tagline{"Spooky. Seriously."}

Play of Shadows is a sorcerous discipline that governs and depends upon shadows. Darkness is in no short supply in something called "After Sundown" and areas with no light at all can be thought of as being like some singular giant shadow. Even extremely bright lights cause objects to cast shadows that are really quite noticeable. As such, in a general sort of way, the fact that Play of Shadows requires shadows to function is pretty much a formality. However, if characters are in the process of sky diving or are in the areas of exceptional \textit{ambient} light, the powers of Play of Shadows can seriously be neutralized. Having Play of Shadows makes a character sneakier and spookier, giving them a +1 bonus to Stealth and Intimidate. This bonus increases to +2 with Advanced, and +3 with Elder Play of Shadows.

%%%
\subsubsection{Basic Powers}
%%%

\powerentry{Eyes of the Night} The character can see and hear out of distant shadows. The character can see in darkness at any time (unlike Supernatural Senses, this use does not render the character susceptible to glare), and by spending a power point they can draw their senses from a pool of darkness  that is within 100 meters per Potency. The character makes an \dicepool{Intuition + Empathy} or \dicepool{Intuition + Perception} check, with a threshold based on how accurately they can describe the shadow they wish to peer out of. Eyes of the Night are covert.

\powerentry{Shadow Casting} The character's lighting appears to have been done by professional special effects technicians. They can manipulate shadows and to a lesser extent light as well. This gives a +2 bonus to any attempt to become the center of attention or to escape unnoticed. This bonus increases to +4 with Advanced Play of Shadows and +6 with Elder Play of Shadows. The Shadow Casting can be dispelled as a spell with 3 hits.

%%%
\subsubsection{Advanced Powers}
%%%

\powerentry{Cloak of Shadow} The character can wrap light around themselves completely, becoming completely transparent along with everything they are carrying. The character still casts a normal shadow and cannot alter their own shadow while the cloak is in place. Invoking the Cloak of Shadow takes a Complex Action and a Power Point, and requires an \dicepool{Intuition + Empathy} or \dicepool{Intuition + Stealth} check. The cloak lasts 5 minutes, and net hits increase the timeframe. The cloak works on anything that uses light -- including cameras -- but has no effect on sounds or smells. The character can dismiss their own Cloak of Shadow at any time as a Simple Action.

\powerentry{Solid Darkness} The character can spend a power point to fashion steel hard tendrils of solid shadow and use them to grasp, carry and tear. The darkness can extend out to a meter per Potency from its source Shadow, and has Strength equal to the Character's Intuition. The source shadow must be within line of sight of the character and the origin of the shadows can move along continuously shadowed paths at the rate of a careful walk. Solid Darkness is completely silent, and can rather easily grab someone by complete surprise. Directing the tendrils of shadow is a Simple Action. Solid Darkness can be dispelled as if it had 3 hits. Solid Darkness vanishes the next time the sun rises. The Solid Darkness can grab or disarm using its own Strength and the Character's Combat skill.

%%%
\subsubsection{Elder Powers}
%%%

\powerentry{Shadow Walk} The character can step into one shadow and come out of another one that they can perceive. The transportation itself is a Simple Action that costs 1 Power to activate. No intervening space is used, and nothing can block the movement. The character may take anything and anyone they or their shadow tendrils can carry. Shadow Walk can be dispelled as if it had 3 hits.

\powerentry{Shadow Body} The character can transform into an intangible shadow form. A body of shadow can pass harmlessly through physical objects save for those of a material they are vulnerable to. Entering or leaving the Shadow Body is a Complex Action. Assuming the Shadow Body takes a power point. Being made out of pure shadow makes it very easy to hide in areas which have any dimness worth mentioning. But it's also super hard to explain from a Vow of Silence point of view. Shadow Body is a Protean Power. Shadow Body can be dispelled as if it had 3 hits.


%%%%%%%%%%%%%%%%%%%%%%%%%
\subsection{Path of Blood}
%%%%%%%%%%%%%%%%%%%%%%%%%
\tagline{Given time, blood does become thicker than water.}

The Path of Blood delves into the intricacies of life and death from a very liquid and visceral standpoint. Deep familiarity of this sort with blood gives advantages. The character gains a +1 bonus to Medicine and Survival for having this discipline. The bonus increases to +2 if they have Advanced, and +3 if they have Elder Path of Blood. It is important to note that not everything creeping in the night \textit{has} any blood. A character or object without any blood cannot \textit{use} Gift of Health or Blood of Acid, and cannot have Theft of Vitae or Blood Puppets used against them. Golems and Androids generally do not have blood, nor do any kind of Ghosts. However, even really gross blood like that found in Zombies or gooey sap like the stuff that oozes from cuts in Dryads or Evil Plants totally counts.

%%%
\subsubsection{Basic Powers}
%%%

\powerentry{Gift of Health} The character can invest a power point into a sample of their own blood, allowing them to transfer power points to others -- even to characters who cannot normally have power points or who are at their limit of power points. The gift comes at a price, for the character's blood is also horrendously addictive; like freebase cocaine. A character who has Gift of Health and \linkpower{Vigor} or \linkpower{Revive the Flesh} may use these transferred power points to strengthen or heal the subject. The power transfer of Gift of Health can be dispelled as if it had been created with 3 hits, but the addictive properties of having drunk power in that manner does not go away no matter how many rice grains a character might be struck with.

\powerentry{Thaumaturgical Forensics} The character can make accurate pronouncements about biological samples and give weird CSI tirades given access to creepy magical equipment. The character makes a \dicepool{Logic + Medicine} or \dicepool{Logic + Research} check. The kinds of information gained varies by the number of hits. Examples follow of investigating the origin of a lock of hair and investigating the source of a lethal wound:


\begin{table}[htb]
\rowcolors{1}{white}{tan} \caption{Thaumaturgical Forensics Results} \centering
\begin{tabular}{c l l}
\textbf{Hits} & \textbf{The Hair Sample} & \textbf{The Wound}\\
\textbf{1}& "Human Woman. Blond." & "Burns. Nasty Ones."\\
\textbf{2}& "She's still alive." & "These fires burned by magic."\\
\textbf{3}&"She's a Luminary." & "A Firestarter did this at short range."\\
\textbf{4}& "Her name is Susan." & "The Firestarter was a Baali Witch."\\
\textbf{5}& "She is currently in Dresden." & "His name was Karlov."\\
\textbf{6}& "She is thinking about betraying her friend Elizabeth." & "This was a fight over money."\\
\end{tabular}
\end{table}


%%%
\subsubsection{Advanced Powers}
%%%

\powerentry{Blood of Acid} 
The character can bleed like one of the xenomorphs from Aliens. Anyone who is within melee range when something inflicts damage on the character is subject to being sprayed with black caustic fluid. Little droplets of the stuff sent in such a circumstance are sufficient to constitute a damage 3 attack. If the character is conscious, they can actively attempt to get their blood to go onto a specific person and try to stage it up with a \dicepool{Strength + Survival} or \dicepool{Strength + Combat} test. The character can also use their blood in more controlled fashions -- that stuff will burn through the lock on a fire door in a few minutes. Blood of Acid can be voluntarily suppressed, or dispelled as a sorcery with 3 hits.

\powerentry{Theft of Vitae}
The character draws blood and power directly out of a victim. Little spheres of blood fly out of the victim towards the character, who thence absorbs the droplets in some suitably dramatic fashion such as having them pop into the character's mouth or outstretched hand. The character uses a Complex Action and makes an opposed \dicepool{Agility + Larceny} or \dicepool{Logic + Medicine} test against the target's Strength. Every net hit causes the victim an unsoaked Lethal damage level and if they have any power points they lose one of them per net hit. This power does not cost power points, and indeed the character \textit{gains} one power point per power point stolen (up to their normal maximum). If the character has a Feeding Power Schedule, they may gain power points even from victims that don't have any.

%%%
\subsubsection{Elder Powers}
%%%

\powerentry{Crimson Death} Given a sample of blood from a target, the character an send a lethal curse against them that will brutally murder them no matter how far they have traveled. The character burns the blood sample and spends an hour sending evil thoughts into it and spends six power points. The character makes an opposed \dicepool{Logic + Medicine} or \dicepool{Logic + Survival} test against the target's Strength. On a success, the victim snaps into a dozen pieces or more, looking momentarily like they were painted upon a broken window pane. Blood gushes from every crack, and they fall into chunks dead as dry bones.

\powerentry{Blood Puppets} The character can take control of victims by puppeting their blood around. By spending three power points and a Complex Action, the character can turn a number of people into marionettes. The character makes a \dicepool{Willpower + Medicine} or \dicepool{Logic + Operations} check, and takes control of a number of human extras equal to the number of hits for the remainder of the scene.


%%%%%%%%%%%%%%%%%%%%%%%%%
\subsection{Symphony of Silence}
%%%%%%%%%%%%%%%%%%%%%%%%%
\tagline{Silence is Deafening.}

The Symphony of Silence is a set of magical music that brings things to a frozen stop. Supposedly it constitutes portions of the inevitable music that ends all of existence. In order to use any ability from the Symphony of Silence, the character must be capable of making music -- although whether through playing a musical instrument or singing is irrelevant. Learning the Symphony of Silence gives you perfect pitch. A character with Advanced Symphony of Silence can hear every sound uttered within the range of music they make -- effectively giving themselves active sonar. A character with Elder Symphony of Silence knows the location of every source of sound no matter how soft within 40 meters of themselves, giving them a frighteningly effective passive sonar ability.

%%%
\subsubsection{Basic Powers}
%%%

\powerentry{Frozen Note} The character can play a song that sharply reduces the temperature in an area or object. The character makes a \dicepool{Logic + Artisan} or \dicepool{Charisma + Expression} test, and reduces the temperature by 5 degrees per hit. When focused on a living creature, this can be quite damaging. Cold blooded creatures pass out, and warm blooded creatures resist one Normal Damage per hit. This has no adverse effect on undead or inanimate creatures like Vampires and Animates. Temperature dropped in this way returns to normal when exposed to heat sources (such as those contained inside a mammal), but there is nothing preventing the character from maintaining the song round to round to keep the temperature low (continuing to damage living creatures). Multiple songs played in rapid succession do not stack. Frozen Note can be focused out to line of sight and cannot be dodged. Making the Frozen Note is a Simple Action, but it can only target each thing once per round.

\powerentry{Missing Voice} The character can move the apparent origins of sounds. The required music appears to eerily come from empty space, and the character can decide the origin of any other sounds as well. The character makes a \dicepool{Logic + Artisan} or \dicepool{Charisma + Expression} test, and the character gains control of the apparent origin of every sound they are aware of within 3 meters per hit for as long as they play. At Advanced Symphony of Silence, this control extends to 10 meters per hit, and at Elder the control extends to 30 meters per hit. So long as the character continues to make the music, the use of Missing Voice is a free action that can be used once per round.

\powerentry{Silent Toll} The character can suppress all sound up to the volume of the music generated, including the music itself. The noise suppressed equals to local peak amplitude, you don't have to keep track of potential spaces between notes and the like. So long as the character continues to make the music, the use of Silent Toll is a free action that can be used once per round. Silent Toll can be dispelled as if it had 3 hits.

%%%
\subsubsection{Advanced Powers}
%%%

\powerentry{Prison of Ice} The character creates ice sufficient to hold someone in place or build something out of. By spending a power point and a Complex Action, the character can create a cubic meter of ice within 10 meters of themselves. The ice can be in any shape, forms instantly, and can cover someone's wrists, feet, or even mouth. The character can make a \dicepool{Logic + Artisan} or \dicepool{Charisma + Expression} test to improve the workmanship and solidity of the ice. Once created, the ice is real ice and cannot be dispelled.

\powerentry{Death Note} The character plays the song that ends a man. The character spends a Complex Action and makes a ranged attack, using \dicepool{Logic + Artisan} or \dicepool{Agility + Combat} and inflicts lethal damage equal to the character's Charisma attribute. Death Note is resisted with Intuition rather than Strength and ignores armor or its equivalent and cannot be dodged. This ability can only be used if the character has already been playing for a minute or more.


%%%
\subsubsection{Elder Powers}
%%%

\powerentry{Frozen Day} The weather is shifted into bitter cold. The character spends five power points and an expected amount of time of 1 day. The character makes a \dicepool{Logic + Artisan} or \dicepool{Charisma + Expression} test, with net hits reducing the amount of time required. The cold snap extends up to 10 kilometers in every direction per Potency of the character. The character can choose any amount of reduction in temperature, and the thermometers will drop pretty linearly to that extent.

\powerentry{Planar Discord} Travel between the worlds becomes essentially impossible while the song is played and for some amount of time afterward. The area covered can be any size centered on the character, to a limit of 1 kilometer per point of Potency in radius. The character makes a \dicepool{Logic + Artisan} or \dicepool{Charisma + Expression} test to increase the time frame of the discontinuity. The timeframe starts at 1 hour with one hit. Creating the discontinuity takes 1 minute, but additional hits can be used to reduce that timeframe instead of increasing duration. Creating the Planar Discord costs 4 power points.



%%%%%%%%%%%%%%%%%%%%%%%%%%%%%%%%%%%%%%%%%%%%%%%%%%
\section{Devotions}
%%%%%%%%%%%%%%%%%%%%%%%%%%%%%%%%%%%%%%%%%%%%%%%%%%

\hspace{\parindent} Devotions are special abilities that you can only gain access to if you have two other inherent disciplines. They are considered to be an Advanced power, but they have the requirement of you having a Basic ability in \textbf{two} different Power categories rather than one. A Devotion does not inherently allow a character to progress to any Elder abilities. At least, it \textit{probably} doesn't -- rumors abound. All devotions are themselves considered to be inherent disciplines.

\powerentry{Adaptive Resilience} (Celerity and Fortitude)
The character can make themselves highly resistant to a threat that they are exposed to. Whenever the character makes a resistance test (including a Soak Test) against something, they can reactively spend a power point and gain 3 extra bonus hits on all future resistance tests against that type of thing (this does not apply to the resistance test that triggers the power in the first place). The adaptive resistance continues to function until it is again activated against a new threat.


\powerentry{Betrayal of the Tongue} (Authority and Magnetism)
An affected target cannot speak lies. They do not have to speak at all, but if they do, the complete truth as they understand it will pour forth from them on any subject they attempt to hold discourse upon. Placing this curse on someone requires a power point and a Complex Action as well as a \dicepool{Logic + Research} or \dicepool{Charisma + Operations} vs. Intuition test. If successful, the betrayal lasts one hour, with net hits increasing the time frame. The power itself is totally covert.

\powerentry{Blind the Senses} (Discernment and Veil)
With a Simple Action, the character can remove a sense from a subject. The character makes an \dicepool{Intuition + Stealth} or \dicepool{Logic + Larceny} test against the target's Intuition. With at least one net hit, the subject loses the use of the sense for 5 minutes. Additional net hits increase the timeframe of the blinding (or deafening, or whatever). This action is covert, it is not obvious to anyone else that anything has happened (although it does detect as magic to dowsing). 

\powerentry{Burrowing} (Veil and Clout)
By spending a power point, the character can tunnel into earth or stone. It is up to the character whether they leave a hole behind themselves. The depth that a character can descend is equal to the number of hits on a \dicepool{Strength + Survival} or \dicepool{Strength + Larceny} test in meters. The character can emerge without spending power. Burrowing can be repeated in order to excavate tunnels in a hurry, and Troglodytes often use it for this purpose. If no hole is left behind them, the ground will simply have a slight discoloration, the actual form of the terrain will be unharmed. Such spots radiate magic of the type of the creature hidden inside them.

\powerentry{Chain of Eyes} (Discernment and Magnetism)
The character can choose two or more creatures they can perceive and allow one or more of the chosen creatures to perceive what one or more other chosen creatures can perceive. With a Simple Action and a power point, the character can add a creature they can perceive to the chain. However many creatures are in the chain, the character can decide which creatures in the chain can voyeuristically leech off the senses of which other creatures in the chain. The character can change that list as a Free Action at any time. Creatures unused to receiving sense data directly into their mind by magic power may become confused. Activating the Chain of Eyes requires an \dicepool{Intuition + Operations} or \dicepool{Intuition + Tactics} test, one hit establishes it with a base timeframe of 10 minutes, and net hits can increase the timeframe.

\powerentry{Cleanse the Body} (Fortitude and Magnetism)
The character may spend power points to heal wounds of others. One Complex Action and a power point will heal two Normal or one Lethal wound. An aggravated wound takes 10 minutes and 2 power points to heal. This power requires that the character be within \textbf{S}hort range and is eerie to see. Once the character has healed one Lethal wound (or two Normal wounds), they can choose to  heal more wounds on the same target with a Free Action each round.

\powerentry{Desire Reflection} (Veil and Magnetism)
By spending a power point, the character can appear as whoever or whatever someone else would respond most favorably to until the end of the scene. This is subject to the normal limits of Veil, and thus without having Vanish From the Mind's Eye the character must content themselves with activating it whilst unobserved and then appearing as the object of desire for the first person who sees them. \textit{With} Vanish From the Mind's Eye, the character may select any person they can perceive and appear as the object of their desire. This ability uses \dicepool{Charisma + Empathy} or \dicepool{Intuition + Larceny}, and may be enough to provoke a Loneliness or Greed Frenzy.

\powerentry{Empty Body} (Discernment and Fortitude)
As much of a curse as a power, the character become intangible (but not invisible). Things pass through the character unless they are made out of a substance that bypasses the resistances of the character, they are created by magic, or the character's powers are nullified generally. For example, a Bagheera with Empty Body could lift a silver fork or be shot by a silver bullet, or burned by the Walk of Flame, but could not open or be held back by a wooden door. Empty Body can be gained a second time, allowing the character to turn it on and off by spending five minutes and three power points. Characters with Empty Body are always solid in the deeps of the world associated with their power source. While the character is insubstantial they can walk on any surface or no surface at all. Moving up or down vertically can be done, but is more work than walking along the floor or hovering at one level -- it's kind of like climbing steep stairs.

\powerentry{Fa\c{c}ade of Nonchalance} (Authority and Veil)
When activated, the fa\c{c}ade prevents extras in the area from \textit{noticing} that events going on around them are in any way out of the ordinary. By spending a power point and a free action, everyone in the vicinity behaves as if things are totally normal for the rest of the scene even if the actual events are completely outside their life experience or moral limits. The character makes a \dicepool{Charisma + Larceny} or \dicepool{Intuition + Survival} test against each onlooker's Intuition. Taking actual damage will snap a victim out of their complacency. The Fa\c{c}ade has to be used \textit{before} the shit hits the fan, because it won't cause people who have already freaked out to calm down.

\powerentry{The Familiar Stranger} (Celerity and Veil)
When active, the character appears to each onlooker as if they were someone the onlooker expected to encounter, someone who "belongs" in the scene. This need not, and often will not be the same person for each onlooker. This power is subject to the normal limits of Veil, save that the character need not be at all familiar with any of the visages they assume.

\powerentry{Flesh of Marble} (Fortitude and Clout)
The character can transform their skin into something very hard. It is not necessarily marble, or even stone: some creatures turn into metal or magically hardened wood. Some creatures get skin that is all craggy or jointed (like The Thing or Colossus from Marvel Comics), and others simply turn into what are apparently living statues. Flesh of Marble is a Protean Discipline. Activating it costs a Simple Action and two power points and lasts for a scene or until dismissed. While active, the character gains 4 bonus dice when Soaking physical damage and their Strength increases by 2. Electricity does not harm a character having invoked Flesh of Marble, either because they are too conductive (if metallic) or too insulated (if stone). 

\powerentry{Flight} (Clout and Magnetism)
The character can \textit{fly}. Like they were a superhero. For 1 power point, the character can fly for the duration of a scene so long as they maintain a speed of at least an Ordinary Walk. If the character wishes to be able to slow down to a Careful Walk or hover (not move at all) without falling out of the sky, they have to spend another power point. Taking off or landing in cramped areas may require an \dicepool{Agility + Athletics} stunt. A flying character moves 3 to 5 times as fast as one walking or running along the ground, and flying is not particularly tiring unless they actually make a Draining Sprint. Flight can be dispelled as a sorcery with 3 hits. A character whose flight has been dispelled falls.

\powerentry{Holistic Ventriloquism} (Fortitude and Veil)
The character can have the sounds and images generated by their presence shift in apparent origin by up to 2 meters. For creatures viewing them, this power can be seen through by the same criteria as any use of Veil, but it fools cameras and recording devices completely. In many cases, the character can become completely invisible and avoid detection by cameras altogether by shifting their image into a wall or the ground.

\powerentry{Phantasmagoria} (Discernment and Authority)
By spending two power points and a Complex Action, the character can make a magical illusion that fools recording devices and the senses of creatures. The character can control the vertical, the horizontal, and the audio within the area. The illusion extends to a column with a 3m radius and height per point of Potency, and it can be moved at any speed out to an area in line of sight by spending a Simple Action per round. The Illusion's contents can be changed to apparently react to events reactively so long as the character can see their own illusion and cares. The realistic nature (or not) of the illusion is determined by an \dicepool{Intuition + Artisan} or \dicepool{Charisma + Expression} test. Characters can spot the weirdness with an appropriate skill (usually \dicepool{Intuition + Empathy} or \dicepool{Intuition + Perception}). The Illusion normally ends at the end of the scene, but for another power point, the character can leave an Illusion standing until the next sunrise, but binding an Illusion like this makes it unable to be moved (though it can still react to apparent stimuli so long as it remains in its defined area).

\powerentry{Purify the Mind} (Authority and Fortitude)
The character can heal madness in others. The purification requires continuous hand-to-face contact,  the expenditure of 2 power points, and has an expected timeframe of 10 minutes. The character makes a \dicepool{Willpower + Tactics} or \dicepool{Logic + Empathy} test with a difficulty threshold equal to the extremitude of the mental problem. A mere life damaging phobia requires professional intervention (Threshold 2), while a life destroying insanity is crazy extreme (Threshold 4). Undoing mental damage and memory edits from magical sources always has a Threshold equal to the number of hits gained to create the effect in the first place. If the character cures the affliction they know what it was and what it was about, so they can say something explanatory and insightful if they want. If the character fails, they get nothing but some nightmarish imagery and vague clues.

\powerentry{Shifting Sands} (Celerity and Magnetism)
The character can rewind time and change their actions accordingly. The sands of time are heavy and small, and the character can only rewind a split second of time. As such, crucial events whose effects are not immediately obvious to the character have already escaped their grasp long before any calamity is apparent. As a reflexive action, the character can spend a power point to shift their position slightly (less than a meter) if they are immediately aware of some kind of danger (such as the floor giving way or being hit with a bullet) or other disaster. The character can use this to dodge a bullet or to be hit by a bullet that was intended for someone else, but the timing is so fine that they must make this choice before the soak roll is made. If they redirect an attack to themselves, they are struck with no net hits, regardless of whether they would have been easier or harder to strike than the intended target. This can also be used socially to prevent major faux pas, though it is useless if the character merely gradually got on the nerves of someone else. If the character gets no hits at all on a social test, they may spend a power point to re roll all the dice. If they get even one hit, then whatever mistakes were made are too far in the past when they become apparent and this power cannot be used.

\powerentry{Shorten the Fuse} (Celerity and Authority)
The character can weaken the resolve and composure of the subject, sending them into a frenzy of destructiveness or a panic as the case may be. The character uses a Complex Action to make an opposed \dicepool{Willpower + Survival} or \dicepool{Willpower + Intimidation} check against the subject's Willpower. If successful, the target loses their cool completely and enters a Fear or Rage frenzy (whichever is most appropriate to the circumstances). The net hits are the strength of the Frenzy.

\powerentry{Telekinesis} (Discernment and Clout)
The character can manipulate things at a distance. Things are moved with a Strength equal to the character's Willpower. Telekinesis is only slightly more accurate than a grabber hand, and does not grant the amazingly devastating effects of leverage you might expect. With intense concentration, telekinesis can be made to perform tiny acts of fine manipulation -- like writing something. When used in this manner, telekinesis is tortuously slow and not very strong (Strength 1). Think of dramatic spirit writing scenes from a movie of your choice and how long it takes to write something short like "Murder".

\powerentry{Tracking Echoes of the Muse} (Discernment and Celerity)
Whenever the character is exposed to sensory stimuli, they can spend a power point to know where it came from. So if they hear a phrase, even over a telephone or from a recording, they can know where in the world it was spoken. By seeing a picture, they can know where it was painted. And so on.

\powerentry{War Form} (Clout and Celerity)
By spending 4 power points and a Complex Action, the character transforms into a monstrous beast. The specifics of the monstrous beast that they become are chosen and fixed when this discipline is gained. A character's War Form is about a 50\% larger than their normal form, and it has claws, or teeth or spines or something. These constitute a 4 damage natural weapon that inflicts Lethal damage. While in War Form, the character's Strength is increased by 3 and their Agility is increased by 2. When a character goes into Frenzy they can (and often do) transform into War Form immediately. This Frenzy-induced transformation takes only a Simple Action and does not cost any power points. A Frenzy transformation discount cannot be saved for later -- it must be the character's first action upon entering Frenzy to count. War Form is a Protean Discipline.

\powerentry{Will to Power} (Authority and Clout)
The character can ignore the eye contact restriction on Authority, allowing them to use the discipline on plants and ghosts and people wearing mirror shades.
